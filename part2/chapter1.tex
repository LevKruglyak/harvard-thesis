\chapter{Bundles and Their Symmetries}\label{chapter:bundles_symmetries}

So far, we've seen several non-obvious historical insights which greatly advanced our abilities to reason about complicated mathematical spaces. The next big idea we'll discuss is that of a \emph{bundle} -- a way to equip a manifold with hidden symmetries.

\begin{insight}
  We can gain deep insights into the structure of a manifold by viewing it as a projection of a higher-dimensional structure with extra symmetry.
\end{insight}

\todo{todo a lot of this}

One such idea was the idea that spaces have \emph{intrinsic} geometric and topological properties -- independent of their embeddings in some ambient affine space. Without an ambient space to inherit structure from, the mathematical spaces must be equipped with additional data which describes their structure and geometry. 
For manifolds, this structure comes in the form of an atlas of coordinate charts which locally describe how pieces of affine space can be ``glued'' together to form the space. With some smoothness conditions on the transition functions between overlapping charts, we get a smooth structure and can do much of differential topology.
\vspace{1em}

When a manifold $M$ is equipped with a smooth structure, a natural object to consider is the \emph{tangent bundle} $TM$. The tangent bundle comes equipped with a projection map $\pi : TM \to M$ which associates the vector space of tangent vectors at a point to that point itself. This bundle forms the basis for many constructions in differential topology and geometry. For instance, the notion of a \emph{vector field} on a manifold can be described as a \emph{section} of the tangent bundle, i.e. a map $\xi : M \to TM$ which satisfies $\pi \circ \xi = \id_M$. This allows us to construct a very simple diffeomorphism invariant:

\begin{definition}
For any smooth manifold $M$, let's set 
\[
  q(M) = \begin{cases}1 & M\textrm{ has a non-zero vector field},\\ 0 & \textrm{ otherwise.}\end{cases}
\]
\end{definition}

If $f : M \to N$ is a diffeomorphism of manifolds, then the differential map $df_p : T_p M \to T_{f(p)} N$ is an isomorphism of vector spaces. This implies that any non-zero vector field on $M$ induces a non-zero vector field on $N$ and vice-versa. Thus, $q$ is the same for diffeomorphic manifolds!
It's not the best topological invariant, but at the very least it can detect the difference between a sphere and a torus.\footnote{At least up to diffeomorphism -- if we allow \emph{continuous} vector fields here, we get a more general homeomorphism invariant.} 
On the torus $S^1\times S^1$, we can construct a non-zero vector field (see \cref{fig:torus_non_vanishing_vector_field}) so $q(S^1\times S^1)=1$. For the sphere $S^2$ on the other hand, there can't be a non-zero vector field so $q(S^2)=0$ -- this is the exact statement of the famous ``Hairy Ball Theorem'' (see \cref{theorem:hairy_ball}).

\begin{theorem}[Hairy Ball]\label{theorem:hairy_ball}
  For any $n>0$, we have $q(S^n) = n\mod 2$. 
\end{theorem}

In some sense, the extra ``room'' provided by the tangent bundle allowed us to measure the topological structure 


\begin{wrapfigure}{r}{0.5\textwidth}
	\centering
	\begin{lkx_diagram}{graphics/diagrams/torus_non_vanishing_vector_field/base.png}
	\end{lkx_diagram}
	\caption{A non-zero vector field on a torus.}\label{fig:torus_non_vanishing_vector_field}
\end{wrapfigure}

\vspace{1em}


% \[
%   g = g_{\mu\nu} \frac{\partial}{\partial x^\mu}\otimes \frac{\partial}{\partial x^{\nu}}
% \]
%
This brings us to the another main topological insight -- viewing manifolds as 
%
% we will explore thoroughly in this chapter.

{\color{red}\lipsum[1-2]}

\section{Vector Bundles}

The simplest example of bundle structure is 

\section{Fiber Bundles}

\section{Principal \texorpdfstring{$G$}{G}-Bundles}

\cite{milnor1963groups}

\section{Classifying Spaces}\label{sec:classifying_spaces}
