\chapter{Vector Bundles}\label{chap:vector-bundles}

\begin{epigraph}{10em}{}
\end{epigraph}

\section{Structure Groups}

Given a rank $k$ vector bundle $\xi : E \to B$ over $\F$, there is an associated \defn{frame bundle} $\pi : \B E \to B$ with fibers $\B_p E = \{ b : \F^k \to E_p \mid b\textrm{ is an isomorphism}\}$ consisting of bases of the vector bundle $E_p$ at each point $p\in B$. Note that frame bundle is a principal $\GL_k\F$-bundle, since there is a right change of basis action given by precomposition on $\F^k$.
In the reverse direction, the action of $\GL_k\F$ on $\F^k$, the associated bundle $\B E\times_{\GL_k\F} \F^k$ is isomorphic to the original vector bundle $E$. In other words, there is a bijective correspondence
\begin{equation}\label{eq:principal-GL-bundle-vector-bundle}
	\left\{\parbox{7.5em}{$\F$-vector bundles of rank $k$ over $X$}\right\}
	\quad\iff\quad
	\left\{\parbox{7em}{principal $\GL_k\F$ bundles over $X$}\right\}
\end{equation}
Whenever we have a Lie group $G$ and homomorphism $\rho : G \to \GL_k\F$, there is a group action of $G$ on $\F^k$ by letting $\rho(g)$ act on $\F^k$ for each $g\in G$. As in the case of $\GL_k \F$, the associated bundle $\B E\times_G \F^k$ is a vector bundle. An isomorphism
\begin{equation}\label{eq:structure-on-a-vector-bundle}
	\lkxfunc{\varphi}{E}{\B E\times_G \F^k}
\end{equation}
is said to be a \defn{$(G,\rho)$-structure}[structure on a vector bundle] on $E$. When $\rho$ is injective, a $(G,\rho)$-structure is said to be a \defn{reduction}[reduction of the structure group] of the structure group to $G$.
Note that every vector bundle comes with a canonical $(\GL_k \F, \id)$-structure.
For real vector bundles, some common structure groups are:
\begin{itemize}
	\item A reduction to $\GL^+_k\R$ corresponds to an orientation of the bundle.
	\item A reduction to $\O_k$ corresponds to a Riemannian structure, i.e. an inner product on the bundle. Every real vector bundle admits such a structure, although not canonically.
	\item A further reduction to $\SO_k$ corresponds to an orientation and a Riemannian structure.
	\item A reduction to $\GL_{k}\C\subset \GL_{2k}\R$ corresponds to a complex structure on a rank $2k$ real vector bundle, i.e. an endomorphism $J$ with $J^2=-\id$. This is equivalent to the data of a complex vector bundle.
	\item A further reduction to $\U_k\subset \GL_k\C$ corresponds to a Hermitian structure on the complex vector bundle, i.e. a Hermitian inner product on the bundle.
\end{itemize}

\begin{remark}
	For those who prefer working with charts, there is a definition of reducing the structure group in these terms.

	Given an $\F$-vector bundle $\pi : E \to B$, and local trivializations $\varphi_\alpha : E|_{U_\alpha} \to U_{\alpha}\times \F^k$ for some open cover $\{U_\alpha\}$, we have a family of transition functions
	\[
		\lkxfunc{g_{\alpha\beta}}{U_\alpha\cap U_\beta}{\GL_k \F}{x}{\varphi_\alpha\circ \varphi_{\beta}^{-1}|_{\{x\}\times \R^k}}
	\]
	For some subgroup $G\subset \GL_k\F$, if we can choose local trivializations such that the resulting transition functions $g_{\alpha\beta}$ take values entirely in $G$, we say that the structure group of $E$ can be reduced to $G$. 
\end{remark}

\begin{remark}
	\todo{Ehrlagen program}
\end{remark}

\section{Universal Bundles}

\begin{proposition}\label{prop:homotopy-invariance-vector-bundle}
	There is a natural
\end{proposition}

\begin{theorem}\label{thm:classifying-space}
\end{theorem}
	Generally, if $G$ is a Lie group there is a natural isomorphism of contravariant functors
	\[
		[-, \BB G] \lkxto \Bun_G(-)
	\]
	where $[-, \BB G]$ is the set of homotopy classes of maps to a space $\BB G$ and $\Bun_G(-)$ is the set of isomorphism classes of principal $G$-bundles over a given space.
	The space $\BB G$ is known as the \defn{classifying space}\footnote{The classifying space is rarely a manifold, and is usually an infinite dimensional CW complex.} of $G$ and this space comes equipped with a \defn{universal bundle} $\zeta : \EE G \to \BB G$. With this universal bundle, the natural isomorphism is easy to describe. Under the appropriate topological restrictions, a map $\tau : X \to \BB G$ gives us a pullback bundle $\tau^*\zeta$ over $X$. This is a principal $G$-bundle over $X$ which is entirely determined by the homotopy type of the \defn{classifying map} $\tau$.

	In some sense, the universal bundle $\zeta$ is the ``most twisted $G$-bundle''. Pullbacks of bundles generally ``dilute'' the twistedness of a bundle -- for instance, the splitting principle allows any complex vector bundle to be pulled back to a direct sum of complex line bundles. It would stand to reason that every bundle is the pullback of a more twisted bundle, and the limit of this process is the universal bundle $\zeta$ over the classifying space. See Chapter IV of \cite{botttu1982differential} for a wonderful exposition on the topic.


\section{Vector Bundles on Spheres}

When the base manifold is a sphere, there is a useful construction which allows for a completely homotopy theoretic classification of vector bundles.

Suppose $\xi : E \to S^m$ is a vector bundle. We can decompose the sphere $S^m$ into hemispheres $S^m=D_+^m\cup D_-^m$, and these disks intersect at an equator $D_+^m\cap D_-^m=S^{m-1}\subset S^m$ -- a sphere one dimension lower. The bundle $\xi$ can then be trivialized on the hemispheres since they are contractible. Let's denote these trivializations
\[
	\lkxfunc{\varphi_+}{E|_{D^m_+}}{D^m_+\times \F^m}
	\quad\textrm{and}\quad
	\lkxfunc{\varphi_-}{E|_{D^m_-}}{D_-^m \times \F^m.}\]
Expanding their equatorial intersection by a tubular neighborhood and choosing transition functions, we get a diffeomorphism 
	$\psi$ in the commutative diagram
\[\begin{tikzcd}
		{S^{m-1}\times \R^m} && {S^{m-1}\times \F^m} \\
		& {E|_{S^{m-1}}}
		\arrow["\psi", from=1-1, to=1-3]
		\arrow["{\varphi_+|_{S^{m-1}}}", from=2-2, to=1-1]
		\arrow["{\varphi_-|_{S^{m-1}}}"', from=2-2, to=1-3]
	\end{tikzcd}\]
which is constant on the first factor, and linear in the second factor. For each point $p\in S^{m-1}$, the diffeomorphism $\psi$ gives a linear function $\tau_p : \R^m \to \F^m$. These linear maps are the ``change of coordinate'' transformations between the fibers on the boundaries of $D_+^m$ and $D_-^m$ as depicted in \cref{fig:clutching-construction}.
\begin{figure}[ht]
	\centering
	\import{diagrams}{clutching-construction.pdf_tex}
	\caption{Getting a map $\tau : S^{m-1}\to \GL_m$ from a vector bundle over $S^{m}$.}\label{fig:clutching-construction}
\end{figure}

Altogether, this family of linear transformations is indexed by the equator $S^{m-1}$, and this gives us a smooth map $\tau : S^{m-1}\to \GL_m \F$. It follows that the homotopy type of $\tau$ is only dependent on the isomorphism type of the bundle $\xi$ since any vector bundle isomorphism induces a homotopy by \cref{prop:homotopy-invariance-vector-bundle}. In other words, we have a map
\[
	\lkxfunc{}{\Vect_m(S^m)}{\pi_{m-1}(\GL_m \F).}
\]
sending a vector bundle $\xi$ to the its associated homotopy class $\tau\in \pi_{m-1}(\GL_m \F)$.
The construction works in the opposite direction as well. Whenever we have a homotopy class $\tau\in \pi_{m-1}(\GL_m \F)$, we can form a bundle $\xi_\tau : E_\tau \to S^{m}$ by letting
\[
	E_\tau = (D_+^m\times \F^m)\cup_h (D_-^{m}\times \F^m),
\]
where $h(x,y)=(x,\tau(x)y)$ is the glueing map.

\begin{theorem}
\end{theorem}

\begin{theorem}
	The clutching construction gives bijections
	\[
		\begin{aligned}
			\lkxfunc{}{\pi_{m-1}(\O_m)}{\Vect_m(S^m)} \\
			\lkxfunc{}{\pi_{m-1}(\SO_m)}{\Vect_m^+(S^m)}
		\end{aligned}
	\]
	between homotopy groups of the orthogonal groups and isomorphism classes of vector bundles.
\end{theorem}
\begin{proof}
	\todo{prove}
\end{proof}

\subsection{Euler Number of Vector Bundles On Spheres}

A similar argument holds when we restrict to oriented bundles, i.e. vector bundles with structure group $\SO_m$.

Under this correspondence the Euler number of a vector bundle can be considered as a group homomorphism
\[
	\lkxfunc{e}{\pi_{m-1}(\SO_m)}{\Z}{\tau}{e(\xi_\tau)}
\]
\begin{theorem}\label{thm:clutching-construction-euler-number}
	Let $p : \SO_{2m} \to \SO_{2m}/\SO_{2m-1}= S^{2m-1}$ be the projection map of the special orthogonal group to the sphere. Then the following diagram commutes,
	\[\begin{tikzcd}
			{\pi_{2m-1}(\SO_{2m})} & \Z \\
			{\pi_{2m-1}(S^{2m-1})}
			\arrow["e", from=1-1, to=1-2]
			\arrow["{p_*}"', from=1-1, to=2-1]
			\arrow[from=2-1, to=1-2]
		\end{tikzcd}\]
	where $\pi_{2m-1}(S^{2m-1})\to \Z$ is the degree isomorphism.
\end{theorem}
\begin{proof}
	\todo{prove}
\end{proof}

\begin{corollary}\label{cor:expressible-euler-numbers-spheres}
	The image of $e : \pi_{2m-1}(\SO_{2m})\to\Z$ is $2\Z$.
\end{corollary}
\begin{proof}
	\todo{prove}
\end{proof}

\begin{remark}
	The clutching construction can be viewed as a special case of \cref{thm:classifying-space}.
	\todo{It can be shown that (under suitable topological restrictions)} there is a homotopy equivalence $\Omega \BB G \simeq G$ where $\Omega$ denotes the loop space operator in homotopy theory. Indeed from a homotopy theory perspective, the classifying space is a ``delooping'' of the group $G$. For spheres, the loop space suspension adjunction gives us natural isomorphisms
	\[
		\begin{aligned}
			\Bun_{G}(S^m) & \cong [S^m, \BB G]            \\
			              & \cong [\Sigma S^{m-1}, \BB G] \\
			              & \cong [S^{m-1}, \Omega\BB G]  \\
			              & \cong [S^{m-1}, G]
			\cong \pi_{m-1}(G).
		\end{aligned}
	\]
	This is a generalized way of understanding the clutching construction.
\end{remark}
