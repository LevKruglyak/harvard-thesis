\section{Morse Theory}\label{sec:morse-theory}

In this is that of Morse theory, a technique connecting the structure of a manifold to the behavior of a (nice) smooth function from the manifold to the real numbers. 
By investigating the discrete set of critical points of such a function, questions of attaching and removing high-dimensional handles is simplified to questions of points on the one-dimensional real line. For this thesis, the main application of Morse theory will be in the proof of the $h$-cobordism theorem (\cref{thm:h-cobordism}) which massively simplifies the problem of finding and classifying exotic spheres. While a fully comprehensive introduction to Morse theory is outside of the scope of this thesis, we'll include a basic overview for completeness. A great classical introduction to Morse theory can be found in Milnor's book on the subject \cite{milnor1963morse}.

If $f : M \to \R$ is a smooth function on a manifold $M$, the points $p\in M$ where the differential $df_p : \T_p M \to \T_{f(p)} \R = \R$ vanish are known as \defn{critical points}, and their images in $\R$ are called \defn{critical values}. In terms of local coordinates $\{x^1,\ldots, x^n\}$ at $p$, this means that
\begin{equation}
	\frac{\partial f}{\partial x^1}=\cdots=\frac{\partial f}{\partial x^n} = 0.
\end{equation}
A critical point $p\in M$ is said to be \defn{non-degenerate}[non-degenerate point] if the matrix
\begin{equation}
	\everymath={\displaystyle}
	\renewcommand*{\arraystretch}{2}
	H_f(p) = \begin{pmatrix}
		\frac{\partial^2 f}{\partial x^1\partial x^1} & \cdots &
		\frac{\partial^2 f}{\partial x^1\partial x^n}                   \\
		\vdots                                        & \ddots & \vdots \\
		\frac{\partial^2 f}{\partial x^n\partial x^1} & \cdots &
		\frac{\partial^2 f}{\partial x^n\partial x^n}                   \\
	\end{pmatrix}(p)
\end{equation}
is invertible at $p$. This is called the \defn{Hessian matrix} of $f$ at $p$, and in this formulation depends on our chosen coordinate system.
There is a coordinate independent way to define the Hessian as a symmetric bilinear form on $\T_p M$ which makes the coordinate invariance of the condition of non-degeneracy manifestly apparent.

\begin{definition}
	The \defn{index}[index of a function] of $f$ at a point $p$ is the maximal dimension of a subspace on which $H_f(p)$ is negative definite, i.e. it is the dimension of $\{v\in \R^n \mid v^\intercal H_f(p) v < 0\}.$
\end{definition}

The index of a function at a point essentially describes the ``shape'' of the function out of a list of finitely many possible shapes. Remember, the index of a function on an $n$-dimensional manifold must be an integer between $0$ and $n$. For instance, in the case of a surface there are three possible shapes -- when both coordinates curve up we get a bowl facing up, when one curves up and one curves down we get a saddle, and when both coordinates curve down we get a bowl facing down. These shapes correspond to indices of $0$, $1$, and $2$ respectively.
\begin{figure}[ht]
	\centering
	\import{diagrams}{placeholder.pdf_tex}
	\caption{An upward bowl, saddle, and downward bowl.}
\end{figure}

The fundamental lemma of Morse theory makes rigorous this notion of a manifold having a shape dictated by a real-valued function -- there is always a local coordinate system which puts the function into a standard form depending on the index.

\begin{lemma}[Morse Lemma]\label{lemma:morse}
	Let $p$ be a non-degenerate critical point of $f$. There is a local coordinate system $(y^1,\ldots, y^n)$ at $p$ such that
	\begin{equation}
		f(y^1,\ldots, y^n)=f(0)-\left[(y^1)^2 + \cdots + (y^{\ell})^2\right] + \left[(y^{\ell + 1})^2 + \cdots + (y^n)^2\right].
	\end{equation}
	where $\ell$ is the index of $f$ at $p$.
\end{lemma}
\begin{proof}
	See Lemma~2.2 of \cite{milnor1963morse}.
\end{proof}

Inspired by this lemma, we call a function $f : M \to \R$ a \defn{Morse function} if all critical points are non-degenerate.

\begin{corollary}
	Non-degenerate critical points are isolated.
\end{corollary}

For a brief demonstration of the power of the Morse lemma, we prove Reeb's theorem, a Morse theoretic criterion for a compact manifold to be homeomorphic to a sphere. Throughout the thesis, we'll usually defer to the more powerful $h$-cobordism theorem to prove that a manifold is homoemorphic to a sphere.
However, it is useful to not always rush for the flamethrower when trying to kill a fly -- a simple swatter might do the trick. We will see a direct application of Reeb's theorem in \cref{sec:milnor-spheres}.

\begin{theorem}[Reeb]\label{thm:reeb}
	If $M$ is a compact manifold and $f$ is a Morse function with exactly $2$ critical points, then $M$ is homeomorphic to a sphere.
\end{theorem}
\begin{proof}
	Firstly, by compactness of $M$ we can find a global minimum $f(x_0)$ and global maximum $f(x_1)$ for some distinct points $x_0$ and $x_1$ (otherwise the function would be constant and not a Morse function with $2$ critical points). We can normalize the function $f$ to have $f(x_0)=0$ and $f(x_1)=1$ without loss of generality. It follows that $x_0$ is a non-degenerate critical point of index 0 and $x_1$ is a non-degenerate critical point of index $n$.
	By the Morse lemma (\ref{lemma:morse}), there is a neighborhood $x_0\in U_0$ with local coordinates $\{y^1,\ldots, y^n\}$ such that
	\begin{equation}
		f(y^1,\ldots, y^n) = (y^1)^2 + \cdots + (y^n)^2.
	\end{equation}
	This gives a Riemannian metric $(dy^1)^2+\cdots+(dy^n)^2$ on $U$ which can be extended to all of $M$ by partitions of unity.

	Given a Riemmanian structure, there is a gradient operator $\nabla : \Omega^0 \to \X(M)$ sending smooth functions to the space of vector fields $\X(M)$.
	In our case, the vector field $\nabla f$ is non-zero everywhere except for at $x_0$ and $x_1$. We thus have a normalized vector field $\nabla f/\|\nabla f\|^2$
	defined everywhere except for at $x_0$ and $x_1$. Let $\varphi_t : M \to M$ be the global flow corresponding to this vector field, i.e. the unique solution to the differential equation
	\begin{equation}
		\left.\frac{d\varphi_t(p)}{dt}\right|_{t=0} = \frac{\nabla f(p)}{\|\nabla f(p)\|^2}
	\end{equation}
	for all $p\in M\setminus \{x_0,x_1\}$. By the chain rule, it follows that
	\begin{equation}
		\frac{d(f\circ \varphi_t(p))}{dt}=\left\langle \frac{d\varphi_t(p)}{dt}, \nabla f\right\rangle = \left\langle \frac{\nabla f}{\|\nabla f\|^2}, \nabla f\right\rangle=1.
	\end{equation}
	In particular, this implies that $f\circ \varphi_t(p) = f(p)+t$. This implies that all of the level sets $f^{-1}(c)$ are diffeomorphic for $c\in (0,1)$. In particular, any disk surrounding $x_0$ can be extended to $f^{-1}[0,1/2]$. The same thing happens for $x_1$ with $f^{-1}[1/2,1]$, so we can write $M$ as the identification of two disks along their boundary. Up to homeomorphism, this is a sphere.
\end{proof}

% \begin{figure}[ht]
% 	\import{diagrams}{placeholder.pdf_tex}
% 	\caption{The homeomorphism constructed in Reeb's theorem}.
% \end{figure}

\begin{remark}
	The reason we only get a homoemorphism and not a diffeomorphism in the proof of \cref{thm:reeb} is because the two hemispheres are identified along the boundary and not along collar neighborhoods.
\end{remark}

The basic ideas used in the proof of Reeb's theorem can be radically generalized.

\begin{definition}
	For any $a\in \R$, the \defn{level set} of a smooth function $f : M \to \R$ is the set
	\begin{equation}
		M_a = f^{-1}(-\infty, a] = \{ p\in M \mid f(p)\leq a\}.
	\end{equation}
\end{definition}

For generalizations, we defer a detailed discussion of Morse theory to the wonderful sources \cite{milnor1963morse}, Chapter 1 of \cite{scorpan2005wild}, or Chapter 8 of \cite{hirsch1976differential}.

% \begin{theorem}
% 	Let $f$ be a Morse function on a compact manifold. The following are true:
% 	\begin{enumerate}[(a)]
% 		\item Suppose $f^{-1}[a,b]$ contains no critical points of $f$ for real numbers $a<b$. Then $M_a$ is diffeomorphic to $M_b$, and $M_a$ is a deformation retract of $M_b$.
% 		\item Let $p$ be a non-degenerate critical point of index $\ell$. Letting $c=f(p)$, suppose that $f^{-1}[c-\epsilon, c+\epsilon]$ is compact and contains no critical points of $f$ aside from $p$. For sufficiently small $\epsilon$, the level set $M^{c+\epsilon}$ has the homotopy type of $M^{c-\epsilon}$ with an $\ell$-cell attached.
% 		\item $M$ has the homotopy type of a CW-complex with a cell in each dimension $\ell$ for each critical point of index $\ell$.
% 	\end{enumerate}
% \end{theorem}
% \begin{proof}
% 	These are Theorems 3.1, 3.2, and 3.3 from \cite{milnor1963morse}.
% \end{proof}
%
% Together, these 
%
% \begin{figure}[ht]
% 	\import{diagrams}{placeholder.pdf_tex}
% 	\caption{Decomposition of a torus by a Morse function.}
% \end{figure}
