\chapter{Appendices}

\section{Morse Theory}\label{sec:morse-theory}

In this is that of Morse theory, a technique connecting the structure of a manifold to the behavior of a (nice) smooth function from the manifold to the real numbers. 
By investigating the discrete set of critical points of such a function, questions of attaching and removing high-dimensional handles is simplified to questions of points on the one-dimensional real line. For this thesis, the main application of Morse theory will be in the proof of the $h$-cobordism theorem (\cref{thm:h-cobordism}) which massively simplifies the problem of finding and classifying exotic spheres. While a fully comprehensive introduction to Morse theory is outside of the scope of this thesis, we'll include a basic overview for completeness. A great classical introduction to Morse theory can be found in Milnor's book on the subject \cite{milnor1963morse}.

If $f : M \to \R$ is a smooth function on a manifold $M$, the points $p\in M$ where the differential $df_p : \T_p M \to \T_{f(p)} \R = \R$ vanish are known as \defn{critical points}, and their images in $\R$ are called \defn{critical values}. In terms of local coordinates $\{x^1,\ldots, x^n\}$ at $p$, this means that
\begin{equation}
	\frac{\partial f}{\partial x^1}=\cdots=\frac{\partial f}{\partial x^n} = 0.
\end{equation}
A critical point $p\in M$ is said to be \defn{non-degenerate}[non-degenerate point] if the matrix
\begin{equation}
	\everymath={\displaystyle}
	\renewcommand*{\arraystretch}{2}
	H_f(p) = \begin{pmatrix}
		\frac{\partial^2 f}{\partial x^1\partial x^1} & \cdots &
		\frac{\partial^2 f}{\partial x^1\partial x^n}                   \\
		\vdots                                        & \ddots & \vdots \\
		\frac{\partial^2 f}{\partial x^n\partial x^1} & \cdots &
		\frac{\partial^2 f}{\partial x^n\partial x^n}                   \\
	\end{pmatrix}(p)
\end{equation}
is invertible at $p$. This is called the \defn{Hessian matrix} of $f$ at $p$, and in this formulation depends on our chosen coordinate system.
There is a coordinate independent way to define the Hessian as a symmetric bilinear form on $\T_p M$ which makes the coordinate invariance of the condition of non-degeneracy manifestly apparent.

\begin{definition}
	The \defn{index}[index of a function] of $f$ at a point $p$ is the maximal dimension of a subspace on which $H_f(p)$ is negative definite, i.e. it is the dimension of $\{v\in \R^n \mid v^\intercal H_f(p) v < 0\}.$
\end{definition}

The index of a function at a point essentially describes the ``shape'' of the function out of a list of finitely many possible shapes. Remember, the index of a function on an $n$-dimensional manifold must be an integer between $0$ and $n$. For instance, in the case of a surface there are three possible shapes -- when both coordinates curve up we get a bowl facing up, when one curves up and one curves down we get a saddle, and when both coordinates curve down we get a bowl facing down. These shapes correspond to indices of $0$, $1$, and $2$ respectively.
\begin{figure}[ht]
	\centering
	\import{diagrams}{placeholder.pdf_tex}
	\caption{An upward bowl, saddle, and downward bowl.}
\end{figure}

The fundamental lemma of Morse theory makes rigorous this notion of a manifold having a shape dictated by a real-valued function -- there is always a local coordinate system which puts the function into a standard form depending on the index.

\begin{lemma}[Morse Lemma]\label{lemma:morse}
	Let $p$ be a non-degenerate critical point of $f$. There is a local coordinate system $(y^1,\ldots, y^n)$ at $p$ such that
	\begin{equation}
		f(y^1,\ldots, y^n)=f(0)-\left[(y^1)^2 + \cdots + (y^{\ell})^2\right] + \left[(y^{\ell + 1})^2 + \cdots + (y^n)^2\right].
	\end{equation}
	where $\ell$ is the index of $f$ at $p$.
\end{lemma}
\begin{proof}
	See Lemma~2.2 of \cite{milnor1963morse}.
\end{proof}

Inspired by this lemma, we call a function $f : M \to \R$ a \defn{Morse function} if all critical points are non-degenerate.

\begin{corollary}
	Non-degenerate critical points are isolated.
\end{corollary}

For a brief demonstration of the power of the Morse lemma, we prove Reeb's theorem, a Morse theoretic criterion for a compact manifold to be homeomorphic to a sphere. Throughout the thesis, we'll usually defer to the more powerful $h$-cobordism theorem to prove that a manifold is homoemorphic to a sphere.
However, it is useful to not always rush for the flamethrower when trying to kill a fly -- a simple swatter might do the trick. We will see a direct application of Reeb's theorem in \cref{sec:milnor-spheres}.

\begin{theorem}[Reeb]\label{thm:reeb}
	If $M$ is a compact manifold and $f$ is a Morse function with exactly $2$ critical points, then $M$ is homeomorphic to a sphere.
\end{theorem}
\begin{proof}
	Firstly, by compactness of $M$ we can find a global minimum $f(x_0)$ and global maximum $f(x_1)$ for some distinct points $x_0$ and $x_1$ (otherwise the function would be constant and not a Morse function with $2$ critical points). We can normalize the function $f$ to have $f(x_0)=0$ and $f(x_1)=1$ without loss of generality. It follows that $x_0$ is a non-degenerate critical point of index 0 and $x_1$ is a non-degenerate critical point of index $n$.
	By the Morse lemma (\ref{lemma:morse}), there is a neighborhood $x_0\in U_0$ with local coordinates $\{y^1,\ldots, y^n\}$ such that
	\begin{equation}
		f(y^1,\ldots, y^n) = (y^1)^2 + \cdots + (y^n)^2.
	\end{equation}
	This gives a Riemannian metric $(dy^1)^2+\cdots+(dy^n)^2$ on $U$ which can be extended to all of $M$ by partitions of unity.

	Given a Riemmanian structure, there is a gradient operator $\nabla : \Omega^0 \to \X(M)$ sending smooth functions to the space of vector fields $\X(M)$.
	In our case, the vector field $\nabla f$ is non-zero everywhere except for at $x_0$ and $x_1$. We thus have a normalized vector field $\nabla f/\|\nabla f\|^2$
	defined everywhere except for at $x_0$ and $x_1$. Let $\varphi_t : M \to M$ be the global flow corresponding to this vector field, i.e. the unique solution to the differential equation
	\begin{equation}
		\left.\frac{d\varphi_t(p)}{dt}\right|_{t=0} = \frac{\nabla f(p)}{\|\nabla f(p)\|^2}
	\end{equation}
	for all $p\in M\setminus \{x_0,x_1\}$. By the chain rule, it follows that
	\begin{equation}
		\frac{d(f\circ \varphi_t(p))}{dt}=\left\langle \frac{d\varphi_t(p)}{dt}, \nabla f\right\rangle = \left\langle \frac{\nabla f}{\|\nabla f\|^2}, \nabla f\right\rangle=1.
	\end{equation}
	In particular, this implies that $f\circ \varphi_t(p) = f(p)+t$. This implies that all of the level sets $f^{-1}(c)$ are diffeomorphic for $c\in (0,1)$. In particular, any disk surrounding $x_0$ can be extended to $f^{-1}[0,1/2]$. The same thing happens for $x_1$ with $f^{-1}[1/2,1]$, so we can write $M$ as the identification of two disks along their boundary. Up to homeomorphism, this is a sphere.
\end{proof}

% \begin{figure}[ht]
% 	\import{diagrams}{placeholder.pdf_tex}
% 	\caption{The homeomorphism constructed in Reeb's theorem}.
% \end{figure}

\begin{remark}
	The reason we only get a homoemorphism and not a diffeomorphism in the proof of \cref{thm:reeb} is because the two hemispheres are identified along the boundary and not along collar neighborhoods.
\end{remark}

The basic ideas used in the proof of Reeb's theorem can be radically generalized.

\begin{definition}
	For any $a\in \R$, the \defn{level set} of a smooth function $f : M \to \R$ is the set
	\begin{equation}
		M_a = f^{-1}(-\infty, a] = \{ p\in M \mid f(p)\leq a\}.
	\end{equation}
\end{definition}

For generalizations, we defer a detailed discussion of Morse theory to the wonderful sources \cite{milnor1963morse}, Chapter 1 of \cite{scorpan2005wild}, or Chapter 8 of \cite{hirsch1976differential}.

% \begin{theorem}
% 	Let $f$ be a Morse function on a compact manifold. The following are true:
% 	\begin{enumerate}[(a)]
% 		\item Suppose $f^{-1}[a,b]$ contains no critical points of $f$ for real numbers $a<b$. Then $M_a$ is diffeomorphic to $M_b$, and $M_a$ is a deformation retract of $M_b$.
% 		\item Let $p$ be a non-degenerate critical point of index $\ell$. Letting $c=f(p)$, suppose that $f^{-1}[c-\epsilon, c+\epsilon]$ is compact and contains no critical points of $f$ aside from $p$. For sufficiently small $\epsilon$, the level set $M^{c+\epsilon}$ has the homotopy type of $M^{c-\epsilon}$ with an $\ell$-cell attached.
% 		\item $M$ has the homotopy type of a CW-complex with a cell in each dimension $\ell$ for each critical point of index $\ell$.
% 	\end{enumerate}
% \end{theorem}
% \begin{proof}
% 	These are Theorems 3.1, 3.2, and 3.3 from \cite{milnor1963morse}.
% \end{proof}
%
% Together, these 
%
% \begin{figure}[ht]
% 	\import{diagrams}{placeholder.pdf_tex}
% 	\caption{Decomposition of a torus by a Morse function.}
% \end{figure}

% \chapter{Vector Bundles}

\section{Structure Groups}

Given a rank $k$ vector bundle $\xi : E \to B$ over $\F$, there is an associated \defn{frame bundle} $\pi : \B E \to B$ with fibers $\B_p E = \{ b : \F^k \to E_p \mid b\textrm{ is an isomorphism}\}$ consisting of bases of the vector bundle $E_p$ at each point $p\in B$. Note that frame bundle is a principal $\GL_k\F$-bundle, since there is a right change of basis action given by precomposition on $\F^k$.
In the reverse direction, the action of $\GL_k\F$ on $\F^k$, the associated bundle $\B E\times_{\GL_k\F} \F^k$ is isomorphic to the original vector bundle $E$. In other words, there is a bijective correspondence
\begin{equation}\label{eq:principal-GL-bundle-vector-bundle}
	\left\{\parbox{7.5em}{$\F$-vector bundles of rank $k$ over $X$}\right\}
	\quad\iff\quad
	\left\{\parbox{7em}{principal $\GL_k\F$ bundles over $X$}\right\}
\end{equation}
Whenever we have a Lie group $G$ and homomorphism $\rho : G \to \GL_k\F$, there is a group action of $G$ on $\F^k$ by letting $\rho(g)$ act on $\F^k$ for each $g\in G$. As in the case of $\GL_k \F$, the associated bundle $\B E\times_G \F^k$ is a vector bundle. An isomorphism
\begin{equation}\label{eq:structure-on-a-vector-bundle}
	\lkxfunc{\varphi}{E}{\B E\times_G \F^k}
\end{equation}
is said to be a \defn{$(G,\rho)$-structure}[structure on a vector bundle] on $E$. When $\rho$ is injective, a $(G,\rho)$-structure is said to be a \defn{reduction}[reduction of the structure group] of the structure group to $G$.
Note that every vector bundle comes with a canonical $(\GL_k \F, \id)$-structure.
For real vector bundles, the most common structure groups are:
\begin{itemize}
	\item A reduction to $\GL^+_k\R$ corresponds to an orientation of the bundle.
	\item A reduction to $\O_k$ corresponds to a Riemannian structure, i.e. an inner product on the bundle. Every real vector bundle admits such a structure, although not canonically.
	\item A reduction to $\SO_k$ corresponds to an orientation and a Riemannian structure.
	\item A reduction to $\GL_{k}\C\subset \GL_{2k}\R$ corresponds to a complex structure on the real vector bundles, i.e. an endomorphism $J$ with $J^2=-\id$.
\end{itemize}
For complex vector bundles, the most common structure group is:
\begin{itemize}
	\item A reduction to $\U_k\subset \GL_k\C$ corresponds to a Hermitian structure, i.e. a Hermitian inner product on the bundle.
\end{itemize}

\begin{remark}
	For those who prefer working with charts, there is a definition of reducing the structure group in these terms.

	Given an $\F$-vector bundle $\pi : E \to B$, and local trivializations $\varphi_\alpha : E|_{U_\alpha} \to U_{\alpha}\times \F^k$ for some open cover $\{U_\alpha\}$, we have a family of transition functions
	\[
		\lkxfunc{g_{\alpha\beta}}{U_\alpha\cap U_\beta}{\GL_k \F}{x}{\varphi_\alpha\circ \varphi_{\beta}^{-1}|_{\{x\}\times \R^k}}
	\]
	For some subgroup $G\subset \GL_k\F$, if we can choose local trivializations such that the resulting transition functions $g_{\alpha\beta}$ take values entirely in $G$, we say that the structure group of $E$ can be reduced to $G$. 
\end{remark}

\begin{remark}
	\todo{Ehrlagen program}
\end{remark}

\section{Vector Bundles on Spheres}

This is a good time for a brief interlude on vector bundles over spheres.
Vector bundles over a sphere can be classified by the clutching construction. Suppose $\xi : E \to S^m$ is a vector bundle. We can decompose the sphere $S^m$ into hemispheres $S^m=D_+^m\cup D_-^m$, and these disks intersect at the equator $D_+^m\cap D_-^m=S^{m-1}\subset S^m$ -- a sphere one dimension lower. The bundle $\xi$ can be trivialized on the hemispheres since they are contractible, and we denote these trivializations
\[
	\lkxfunc{\varphi_+}{E|_{D^m_+}}{D^m_+\times \R^m}
	\quad\textrm{and}\quad
	\lkxfunc{\varphi_-}{E|_{D^m_-}}{D_-^m \times \R^m.}\]
The trivializations must come with transition functions on their intersection (we might have to expand the intersection a bit so that it is open). The transition function is a diffeomorphism $\psi$ in the commutative diagram
\[\begin{tikzcd}
		{S^{m-1}\times \R^m} && {S^{m-1}\times \R^m} \\
		& {E|_{S^{m-1}}}
		\arrow["\psi", from=1-1, to=1-3]
		\arrow["{\varphi_+|_{S^{m-1}}}", from=2-2, to=1-1]
		\arrow["{\varphi_-|_{S^{m-1}}}"', from=2-2, to=1-3]
	\end{tikzcd}\]
which is constant on the first factor, and linear in the second factor. For each point $p\in S^{m-1}$, the diffeomorphism $\psi$ thus gives a linear function $\tau_p : \R^m \to \R^m$. These linear maps are the ``change of coordinate'' transformations between the fibers on the boundaries of $D_+^m$ and $D_-^m$ (see \cref{fig:clutching-construction}).
\begin{figure}[ht]
	\centering
	\import{diagrams}{clutching-construction.pdf_tex}
	\caption{Getting a map $\tau : S^{m-1}\to \GL_m$ from a vector bundle over $S^{m}$.}\label{fig:clutching-construction}
\end{figure}

Altogether, this family of linear transformations is indexed by the equator $S^{m-1}$, and this gives us a smooth map $\tau : S^{m-1}\to \GL_m$. It follows that the homotopy type of $\tau$ is only dependent on the isomorphism type of the bundle $\xi$ since any vector bundle isomorphism can be shown to induce a homotopy of smooth maps. In order words, we have a map
\[
	\lkxfunc{}{\Vect_m(S^m)}{\pi_{m-1}(\GL_m).}
\]
sending a vector bundle $\xi$ to the its associated homotopy class $\tau\in \pi_{m-1}(\GL_m)$.

The construction works in the opposite direction as well. Whenever we have a homotopy class $\tau\in \pi_{m-1}(\GL_m)$, we can form a bundle $\xi_\tau : E_\tau \to S^{m}$ by letting
\[
	E_\tau = (D_+^m\times D^m)\cup_T (D_-^{m}\times D^m),
\]
where $T(x,y)=(x,\tau(x)y)$ is the glueing map.

\todo{write this}

A similar argument holds when we restrict to oriented bundles, i.e. vector bundles with structure group $\SO_m$.

\begin{theorem}
	The clutching construction gives bijections
	\[
		\begin{aligned}
			\lkxfunc{}{\pi_{m-1}(\O_m)}{\Vect_m(S^m)} \\
			\lkxfunc{}{\pi_{m-1}(\SO_m)}{\Vect_m^+(S^m)}
		\end{aligned}
	\]
	between homotopy groups of the orthogonal groups and isomorphism classes of vector bundles.
\end{theorem}
\begin{proof}
	\todo{prove}
\end{proof}

Under this correspondence the Euler number of a vector bundle can be considered as a group homomorphism
\[
	\lkxfunc{e}{\pi_{m-1}(\SO_m)}{\Z}{\tau}{e(\xi_\tau)}
\]
\begin{theorem}\label{thm:euler-number-of-vector-bundle-over-sphere}
	Let $p : \SO_{2m} \to \SO_{2m}/\SO_{2m-1}= S^{2m-1}$ be the projection map of the special orthogonal group to the sphere. Then the following diagram commutes,
	\[\begin{tikzcd}
			{\pi_{2m-1}(\SO_{2m})} & \Z \\
			{\pi_{2m-1}(S^{2m-1})}
			\arrow["e", from=1-1, to=1-2]
			\arrow["{p_*}"', from=1-1, to=2-1]
			\arrow[from=2-1, to=1-2]
		\end{tikzcd}\]
	where $\pi_{2m-1}(S^{2m-1})\to \Z$ is the degree isomorphism.
\end{theorem}
\begin{proof}
	\todo{prove}
\end{proof}

\begin{corollary}\label{cor:expressible-euler-numbers-spheres}
	The image of $e : \pi_{2m-1}(\SO_{2m})\to\Z$ is $2\Z$.
\end{corollary}
\begin{proof}
	\todo{prove}
\end{proof}

\begin{remark}
	The clutching construction is a special case of a more general classification of principal $G$-bundles. Generally, if $G$ is a Lie group there is a natural isomorphism of contravariant functors
	\[
		[-, \BB G] \lkxto \Bun_G(-)
	\]
	where $[-, \BB G]$ is the set of homotopy classes of maps to a space $\BB G$ and $\Bun_G(-)$ is the set of isomorphism classes of principal $G$-bundles over a given space.
	The space $\BB G$ is known as the \defn{classifying space}\footnote{The classifying space is rarely a manifold, and is usually an infinite dimensional CW complex.} of $G$ and this space comes equipped with a \defn{universal bundle} $\zeta : \EE G \to \BB G$. With this universal bundle, the natural isomorphism is easy to describe. Under the appropriate topological restrictions, a map $\tau : X \to \BB G$ gives us a pullback bundle $\tau^*\zeta$ over $X$. This is a principal $G$-bundle over $X$ which is entirely determined by the homotopy type of the \defn{classifying map} $\tau$.

	In some sense, the universal bundle $\zeta$ is the ``most twisted $G$-bundle''. Pullbacks of bundles generally ``dilute'' the twistedness of a bundle -- for instance, the splitting principle allows any complex vector bundle to be pulled back to a direct sum of complex line bundles. It would stand to reason that every bundle is the pullback of a more twisted bundle, and the limit of this process is the universal bundle $\zeta$ over the classifying space. See Chapter IV of \cite{botttu1982differential} for a wonderful exposition on the topic.

	It can be shown that (under suitable topological restrictions) there is a homotopy equivalence $\Omega \BB G \simeq G$ where $\Omega$ denotes the loop space operator in homotopy theory. Indeed from a homotopy theory perspective, the classifying space is a ``delooping'' of the group $G$. For spheres, the loop space suspension adjunction gives us natural isomorphisms
	\[
		\begin{aligned}
			\Bun_{G}(S^m) & \cong [S^m, \BB G]            \\
			              & \cong [\Sigma S^{m-1}, \BB G] \\
			              & \cong [S^{m-1}, \Omega\BB G]  \\
			              & \cong [S^{m-1}, G]
			\cong \pi_{m-1}(G).
		\end{aligned}
	\]
	This is a generalized way of understanding the clutching construction.
\end{remark}

% \subsection{Lattices and Integral Bilinear Forms}
%
% A common source of symmetric bilinear forms over the integers arise from lattices in an inner product space, such as Euclidean or Lorentzian\footnote{i.e. equipped with a form of signature $(1,n-1)$.} space.
% If $(V,\langle -,-\rangle)$ is an inner product space, a lattice $\Lambda\subset V$ inherits the bilinear form $\langle -,-\rangle$. However, this form is generally not integer valued. 
%
% However, some specially constructed lattices do inherit an integer valued bilinear form. Even rarer are the unimodular lattices, i.e. lattices with unimodular inner product. Non-trivial examples of such unimodular lattices appear sporadically, and their algebraic and geometric complexity can be mapped to topological complexity via the intersection form. In particular, we will show in \cref{sec:plumbing} that a lattice with an even form always appears as the intersection form of some manifold. In this way, the ``exotic'' lattices discussed in this section map neatly to the direct construction of exotic spheres.
%
% \begin{definition} 
% 	In Euclidean space $\R^\ell$ with basis $\{e_1,\ldots, e_\ell\}$ consider the lattice
% \[
% 	\Gamma^\ell = \span\left\{\frac{1}{2}(e_1+\cdots + e_\ell), e_i+e_j \mid i<j\right\}.
% \]
% \end{definition}
%
% When $4\mid \ell$, the Euclidean inner product restricts to an integral bilinear form on this lattice since
% \[
% 	\begin{aligned}
% 		\left\langle \frac{1}{2}(e_1+\cdots+e_\ell), \frac{1}{2}(e_1+\cdots+e_\ell)\right\rangle & = \frac{\ell}{4},                                      \\
% 		\left\langle \frac{1}{2}(e_1+\cdots+e_\ell), e_i+e_j\right\rangle                        & = 1,                                                   \\
% 		\langle e_i+e_j, e_p +e_q\rangle                                                         & = \delta_{i,p}+\delta_{i,q}+\delta_{j,p}+\delta_{j,q},
% 	\end{aligned}
% \]
% where $\delta$ denotes the Kronecker delta. Since the Euclidean inner product is positive-definite, so is the bilinear form of this lattice. When $8\mid\ell$, the bilinear form of this lattice is even, otherwise it is odd. 
%
% \begin{proposition}
% 	There is a lattice isomorphism $\Gamma^8\cong E_8$.
% \end{proposition}
% \begin{proof}
% \end{proof}
%
% \begin{theorem}[Mordell]
% 	The lattice $\Gamma^8\cong E_8$ is the only even unimodular positive-definite lattice of rank $8$.
% \end{theorem}
% \begin{proof}
% 	\todo{cite}
% \end{proof}
%
% \begin{definition}
% \todo{Leech lattice}
% \end{definition}
%
% \todo{kissing number problem}
%
% \subsection{The Arf Invariant}\label{sec:arf-invariant}
%
% For any bilinear form $\langle-,-\rangle$ on an $R$-module.
%
%
% We have talked at length about the signature and its role in classifying symmetric bilinear form. For skew-symmetric matrices, the signature is 
%
% \begin{definition}
% \end{definition}
%
