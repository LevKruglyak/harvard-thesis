\chapter{Blah blah}\label{chapter:blah}

% % Epigraph
% \begin{flushleft}
% 	\emph{Turn it, and turn it, for everything is in it.}\\
% 	\emph{Reflect on it and grow old and gray with it.}\\
% 	\emph{Don't turn from it, for nothing is better.}\\
% 	\emph{According to the labor is the reward.}\\
% 	\rule[0pt]{15em}{0.5pt}\\
% 	\emph{Pirkei Avot 5:22}
% \end{flushleft}
%
% \vspace{-8.59em}
% \begin{cjhebrew}
% 	b*en b*ag b*ag 'Omer;
% 	hApok: b*Ah* waha:pok: b*Ah*\textnormal{,}\\
% 	d*:kol*a' bah*\textnormal{,}
% 	Ubah* t*EhE:zey\textnormal{,}
% 	w:siyb Ub:leh bAh*\textnormal{,}\\
% 	Umin*ah* lo' tAzU`a\textnormal{,}
% 	+sE'eyN l:kA mid*Ah TObAh heymEn*Ah\textnormal{.}\\
% 	b*en he' he' 'Omer;
% 	l:pUm .sa`a:rA' 'Ag:rA'\textnormal{.}\\
% 	\rule[0pt]{15em}{0.5pt}\\
% 	p*ir:qey 'AbOt h\textnormal{:}kb
% \end{cjhebrew}
%
% \vspace{2em}
We wanna talk about \cite{milnor1963groups} and \cite{milnor1956manifolds} in \autoref{chapter:introduction}. 

Later in this paper, we'll talk about \cite{milnor2000exotic} and \cite{dieck2008algebraic}. This will be very useful\footnote{This is a footnote of the document}. \cite{dieudonne2009history}.

\section{Early Results}

\begin{theorem}[Euler Characteristic]
	Given a spherical polyhedron with $V$ vertices, $E$ edges, and $F$ faces, we have the relation:
	\[
			V - E + F = 2.
	\]
\end{theorem}

\begin{theorem}[Gauss-Bonnet]
	Let $M$ be a compact, closed, two-dimensional Riemannian surface, and let $K$ be its Gaussian curvature. Then we have:
	\[
		\int_M K\; dA  = 2\pi \chi(M),
	\]
	where $dA$ is the surface element of $M$.
\end{theorem}

\begin{corollary}
	The sum of angles in a geodesic triangle
\end{corollary}

\begin{theorem}[Chern-Gauss-Bonnet]
	Let $M$ be a compact, closed, orientable, $2n$-dimensional Riemannian manifold, and let $\Omega$ be the curvature form of the Levi-Civita connection. Then,
	\[
		\chi(M) = \int_M e(\Omega)\quad\textrm{where}\quad e(\Omega) = \frac{1}{(2\pi)^n} \Pf(\Omega)
	\]
	is the \defn{Euler class}, and $\Pf$ is the Pfaffian.
\end{theorem}

\section{The Section}

\lipsum[1]

\begin{definition}\label{def:characteristic_class}
	Let $h^*(-)$ be a cohomology theory. An \defn{$h^k$-valued characteristic class}[characteristic class] assigns to each bundle $\xi : E(\xi) \to B$ an element $c(\xi)\in h^k(B)$ such that for a bundle map $\xi \to \eta$ over $f : B \to C$ the naturality property $f^*c(\eta) = c(\xi)$ holds.
\end{definition}

\subsection{The subsection}

This is a \keyword{word}. And this is

\lipsum[2]

\begin{definition}\label{def:first}
	A \defn{group} is a set equipped with an associative operation which has identity and inverse. The \defn{cohomology group} is defined as
	\[
		\H^{2,0}(X; \C) = \{\}
	\]
\end{definition}

\begin{definition}\label{def:pontryagin}
	Given a real vector bundle $E$ over $M$, it's \defn{$k$-th Pontryagin class}[Pontryagin class] $p_k(E)$ is defined as
	\[
		p_k(E) = (-1)^k c_{2k}(E\otimes \C) \in \H^{4k}(M; \Z),
	\]
	where $c_{2k}(E\otimes \C)$ denotes the $2k$-th Chern class of the complexification $E\otimes \C$.
\end{definition}

\begin{lemma}\label{lemma:first}
	This is a lemma.
\end{lemma}

\subsection{The subsection}

\begin{definition}\label{def:second}
	This is a definition, following \cref{lemma:first} and \cref{def:first}.
\end{definition}

\begin{figure}\label{fig:exotic_spheres_table}
	\centering
	\begin{tabular}{|c|c|c|c|c|c|c|c|c|c|c|c|c|c|c|c|}
		\hline
		\textbf{Dimension}
		 & 1 & 2 & 3  & 4                      & 5  & 6  & 7
		 & 8 & 9 & 10 & 11                     & 12 & 13 & 14 & 15    \\
		\hline
		\textbf{$\#$ of Exotic Spheres}
		 & 1 & 1 & 1  & \textbf{\color{red} ?} & 1  & 1  & 28
		 & 2 & 8 & 6  & 992                    & 1  & 3  & 2  & 16256 \\
		\hline
	\end{tabular}
	\caption{Number of exotic spheres in each dimension.}
\end{figure}

\lipsum[3-6]
