\chapter{Characteristic Classes}

There are several perspectives on defining characteristic classes.

\section{Stiefel-Whitney Classes}

Here are some popular cohomology rings:
\[
	\begin{aligned}
		\H^\bullet(\RP^n; \Z_2)&\cong \Z_2[\alpha]/(\alpha^{n+1})\qtq{with}|\alpha|=1 \\
		\H^\bullet(\CP^n; \Z)&\cong \Z[\alpha]/(\alpha^{n+1})\qtq{with}|\alpha|=2 \\
		\H^\bullet(S^n; \Z)&\cong \Z[\alpha]/(\alpha^2)\qtq{with}|\alpha|=n \\
	\end{aligned}
\]

As classifying spaces, we have $\BO_n \simeq \Gr_n = \Gr_n(\R^\infty)$, and cohomology ring
\[
  \H^\bullet(\Gr_n; \Z_2) \cong \Z_2[\w_1, \ldots, \w_n]\qtq{with}|\w_i|=1.
\]
For a rank $n$ vector bundle $\xi : E \to B$, let $B\xi : B \to BO_n$ be the classifying map. Then, we define the \defn{$i$-th Stiefel-Whitney class}[Stiefel-Whitney class] as the pullback:
\[
  \w_i(\xi)=(B\xi)^*(\w_i) \in \H^i(B; \Z_2)
\]
We can extend this definition to $\w_0(\xi)$ by always setting it to be the unit $1\in \H^0(B; \Z_2)$.


Equivalently, the Stiefel-Whitney classes can be defined axiomatically as natural transformations $\w_i^{(k)} : \Vect_k \to \H^i(-; \Z_2)$ satisfying the axioms:
\[
  \w_i^{(k)}(\xi)=\begin{cases}1 & i=0,\\ 0 & i > \rank(\xi),\end{cases}\quad \w_n(\xi\oplus \eta) = \sum_{p+q=n}\w_p(\xi)\smile \w_q(\eta), \quad\w_1(\gamma^1(\RP^1)) = \alpha.
\]
Here $\gamma^1(\RP^1)$ is the canonical line bundle over the circle $\RP^1$, also known as the M\"obius bundle.

\section{The Euler Class}

Let $M$ be a smooth manifold.

Let $\Ufr=\{U_\alpha\}$ be an open cover of $M$. The \defn{\v{C}ech-de Rham complex} is the double complex
\[
	C^*(\Ufr, \Omega^*) = \bigoplus_{p,q\geq 0} C^p(\Ufr, \Omega^q)\qtq{where}C^p(\Ufr, \Omega^q) = \prod_{\alpha_0 < \cdots < \alpha_{p}}\Omega^q(U_{\alpha_0}\cap \cdots \cap U_{\alpha_p}).
\]
This double complex has two differentials. The first, $d : C^p(\Ufr, \Omega^q) \to C^p(\Ufr, \Omega^{q+1})$ is defined in the obvious way. In the other direction, we have the operator:
\[
	\lkxfunc{\delta}{C^p(\Ufr, \Omega^q)}{C^{p+1}(\Ufr, \Omega^q)}{\omega}{\sum_{0\leq i \leq p+1} (-1)^i \omega_{\alpha_0\ldots\hat{\alpha_i}\ldots \alpha_{p+1}}}
\]
where I need to expand the explanation of notation a bit. It follows that $\delta^2= 0$.

\begin{theorem}[Generalized Mayer-Vietoris Sequence]
	For any $q \geq 0$, the cochain complex
	\[
		\begin{array}{rcl}
			0 \lkxto \Omega^q(M) \lkxto[r] C^0(\Ufr, \Omega^q) \lkxto[\delta]
			C^1(\Ufr, \Omega^q) \lkxto[\delta] C^2(\Ufr, \Omega^q) \lkxto[\delta]\cdots
		\end{array}
	\]
	is exact, i.e. it's $\delta$-cohomology vanishes.
\end{theorem}

\begin{proof}
	Given a cochain $\omega\in C^p(\Ufr, \Omega^q)$ which is a $\delta$-cocycle, i.e. $\delta \omega = 0$, we can construct a cochain $\tau\in C^{p-1}(\Ufr, \Omega^q)$ with $\delta\tau = \omega$ by
	\[
		\tau_{\alpha_0\ldots\alpha_{p-1}} = \sum_{\alpha}\rho_\alpha\cdot \omega_{\alpha\alpha_0\ldots\alpha_{p-1}}
	\]
	for some partition of unity $\rho_\alpha$ subordinate to the cover $\Ufr$.
\end{proof}

\begin{corollary}
	Suppose $\{\omega_\alpha\in \Omega^p(U_\alpha)\}$ is a set of $p$-forms,
\end{corollary}

\section{Axiomatic/Functorial Perspective}

\section{Chern-Weil Perspective}

\section{Universal Perspective}

\begin{definition}
	For $n\geq 1$ the universal Chern classes $c_r \in \H^{2r}(\BU_n)$
	are characterized as follows:
	\begin{enumerate}
		\item $c_0 = 1$ and $c_r = 0$ if $r > n$.
		\item For $n=1$, $c_1$ is the canonical generator of $\H^2(\BU_1)\cong \Z$.
		\item Under pullback along the inclusion $\iota : \BU_n \to \BU_{n+1}$, we have $\iota^* c_r^{(n+1)} = c_i^{(n)}$.
		\item Under the inclusion $\iota : \BU_n\times \BU_l \to \BU_{k+l}$ we have
		      \[
			      \iota^* c_r = \sum_{0\leq j \leq r} c_j \smile c_{r-j}.
		      \]
	\end{enumerate}
\end{definition}

\begin{proposition}
	There are isomorphisms
	\[
		\H^\bullet(\BU_n) \lkxisom \Z[c_1, \ldots, c_n].
	\]
	which behave as expected under the inclusions.
\end{proposition}

\section{Homotopy Groups}

Suppose $(B, b_0)$ is a pointed space, $p : E \to B$ is a Serre fibration with fiber $F=E_{b_0}$, and $e_0\in p^{-1}(b_0)$. There is a \defn{connecting homomorphism}
\[
	\partial : \pi_n(B, b_0) \to \pi_{n-1}(F, e_0)
\]
which can be defined in the following way.

Suppose $f : S^n \to B$ is a map. We get a homotopy $\varphi_t : S^{n-1} \to B$ with $\varphi_0$ the inclusion of the equator of $S^n$ into $B$, and $\varphi_1$ the constant map at $b_0$. By the homotopy lifting property of the fibration, this homotopy can be lifted (using the chosen point $e_0\in E$) to a unique homotopy $\widetilde{\varphi} : S^{n-1} \to E$. Let's define
\[
		\partial [f] = [\widetilde{\varphi}_1].
\]
By the construction of the lift, it follows that $\widetilde{\varphi}_1 : S^{n-1} \to F$ so this makes sense as a map.

\begin{theorem}
	There is a long exact sequence:
	\[
		\cdots\lkxto \pi_n(F)\lkxto[i_*]\pi_n(E)\lkxto[p_*]\pi_n(B)\lkxto[\partial]\pi_{n-1}(F)\lkxto\cdots
	\]
\end{theorem}

\section{Obstruction Theory}

\begin{theorem}[Obstruction to Extending a Function]
	Let $(X,A)$ be a relative CW complex and $Y$ a path-connected simple space and let $n\geq 1$. Let $f : X_n \to Y$ and let $\ofr_f \in C^{n+1}(X, A; \pi_n(Y))$ be the associated obstruction cocycle. Then $f|_{X_{n-1}}$ extends to $X_{n+1}$ if and only if $\ofr_f\in H^{n+1}(X, A; \pi_n(Y))$ is zero.
\end{theorem}

\begin{theorem}[Obstruction to Extending a Section]
  Let $\xi : E \to B$ be a fiber bundle of CW complexes with homotopically simple fiber $F$ and simply connected base space $B$. Suppose $s\in \Gamma(\xi; B^{(n)})$ is a section of $\xi$ over the $n$-skeleton $B^{(n)}$, and let $[\ofr_s]\in \H^{n+1}(B;\pi_n(F))$ be the associated obstruction cocycle. Then:
  \[
    \left\{ \parbox{9em}{the section $s$ extends from $B^{(n)}$ to $B^{(n+1)}$}\right\} \quad\iff\quad 
    \left\{\parbox{16em}{the associated obstruction cocycle $[\ofr_s]$ vanishes in $\H^{n+1}(B; \pi_n(F))$}\right\}
  \]
\end{theorem}

\begin{proof}
  See Section~18.5 of \cite{fomenko2009homotopical}.
\end{proof}

\section{Genus of a Multiplicative Sequence}

\begin{definition}
	A sequence of polynomials $K_1,K_2,\ldots \in k[p_1,p_2,\ldots]$ is \defn{multiplicative}[multiplicative sequence] if
	\[
		1 + p_1z + p_2z^2 + \cdots = 
		(1+q_1 z + q_2 z^2 + \cdots)
		(1+r_1 z + r_2 z^2 + \cdots)
	\]
	implies that
	\[
		\sum_j K_j(p_1,p_2,\ldots) z^j = \sum_i K_i(q_1,q_2,\ldots)z^i \sum_k K_k (r_1,r_2,\ldots) z^k.
	\]
\end{definition}

If $Q(z)\in k[\![z]\!]$ is a formal power series with constant term $1$, then we can define

\begin{definition}
	The \defn{$L$-genus} is the genus of the formal power series
	\[
		a
	\]
\end{definition}
