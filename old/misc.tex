\chapter{Miscellaneous Results}

\begin{proposition}\label{prop:splittings_affine}
  Let $A, B, C$ be vector spaces in a short exact sequence 
  \[
    0 \lkxto A \lkxto[f] B \lkxto[g] C\lkxto 0.
  \]
  The space of splittings of this short exact sequence is affine over $\Hom(C, A)$.
\end{proposition}
\begin{proof}
\end{proof}


There are many ways to construct spheres, either topologically, differentially, or geometrically. They are, after all, some of the simplest possible shapes. In low dimensions, these constructions become even more plentiful and varied due to the presence of \emph{exceptional isomorphisms} -- low dimensional mathematical coincidences which do not fit into a larger pattern. One such coincidence is the \emph{Hopf fibration}, a particularly nice way of building up spheres in dimensions $1$, $3$, $7$, or $15$.

\section{The Hopf Fibration}\label{sec:hopf_fibration}

To define the Hopf fibration, let's begin with a real normed division algebra $A$ -- recall that this is an algebra over $\R$, (not necessarily commutative or associative) which admits multiplicative inverses and comes with a multiplicative norm $\|\cdot\| : A \to \R$. Surprisingly, the first algebras which come to mind --
the real numbers $\R$, the complex numbers $\C$, the quaternions $\HH$, and the octonions $\OO$ -- are the only such algebras. This is the statement of the Hurwitz theorem, which uniquely characterizes normed division algebras by their dimension, which is proved to be either be $1$, $2$, $4$, or $8$.

In general, for a normed real vector space $V$, we can consider the space of unit vectors
\[
    \S(V) = \{ x\in V : \|x\| = 1\}.
\]
This is the \defn{unit sphere} in $V$, and admits a natural identification with the $(n-1)$-sphere if $V$ has dimension $n$. If a normed division algebra $A$ is also associative, there is further structure imposed on its unit sphere. The multiplication on $A$ gives $\S(A)$ a group structure, turning the sphere into a Lie group. The associative normed division algebras are $\R$, $\C$, and $\HH$, and their unit spheres can be identified with the Lie groups
\[
    \S(\R)\approx \O_1\approx S^0,\quad \S(\C)\approx \U_1\approx S^1, \quad \S(\HH)\approx \SU_2\approx S^3.
\]
In fact, these are the only spheres which have a Lie group structure.

More generally, the norm on $A$ gives us a norm on $A^k$, since we can combine the component-wise norms using the Euclidean norm:
\[
  \|(x_1, \ldots, x_k)\| = \sqrt{\|x_1\|^2+\cdots+\|x_k\|^2}\quad\quad (x_1,\ldots,x_k)\in A^k.
\]
It thus makes sense to also consider the unit $k$-sphere $\S(A^k)$ in $A^k$, and this can be topologically identified with the $(nk-1)$-sphere $S^{nk-1}$, although we don't get a Lie group structure in this case.

There is another associated construction we can use to construct spheres from the data of a normed division algebra $A$. Let's define the \defn{projective $k$-space} over $A$ by the quotient
\[
  \P^k(A) = \frac{A^{k+1} -\{0\}}{(x_1,\ldots, x_{k+1})\sim \lambda \cdot (x_1,\ldots, x_{k+1})\quad \lambda\in A}.
\]
This space can be thought of as consisting of lines passing through the origin in $A^{k+1}$. 

There is a surjective submersion
\[
  \lkxfunc{}{\S(A^{k+1})}{\P^{k}(A)}{(x_1,\ldots, x_{k+1})}{[x_1 : \cdots : x_{k+1}]}
\]

\section{Cobordism}\label{sec:cobordism}
