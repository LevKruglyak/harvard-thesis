\chapter{Characteristic Classes}\label{chapter:characteristic_classes}

In this chapter, we'll explore the rich theory of \emph{characteristic classes} -- a wide bridge of structures which connect the geometric theory of fiber bundles with the algebraic/topological theory of cohomology.
Using characteristic classes, geometric insights about fiber bundles can be translated into algebraic statements in cohomology, which in turn give us topological insights. In the other direction, algebraic computations in cohomology can be used to get geometric insights about fiber bundles.

Although a comprehensive theory of characteristic classes only began to emerge in the first half of the 20th century, the basic principle was discovered almost a century earlier by Gauss and (independently) by Pierre Bonnet. At the time, the study of the geometry of surfaces was done entirely in affine space -- it wasn't until Gauss's \keyword{Theorema Egregium} (Latin for ``Remarkable Theorem'') that it was realized that the geometry of a surface can be studied entirely in terms of its intrinsic structure, independently of any embedding.
It's not surprising why Gauss called this result remarkable -- the global geometry of a surface can be \emph{entirely} determined by the local data of measuring lengths, angles, and their rates of change.

In the modern mathematical language, this data amounts to an inner product structure\footnote{This inner product structure is often called the \defn{first fundamental form} of the surface.} on the tangent bundle $TM$ of a $2$-dimensional manifold $M$. For an example of geometric structure, we can consider the notion of \keyword{Gaussian curvature}, although its precise definition is outside the scope of this example.
Taken at every point, the curvature of a surface forms a scalar field, usually denoted $K$ -- a function which is positive on ``spherical'' regions and negative on ``saddle-shaped'' regions. Assuming a choice of orientation and letting $dA$ be the surface form, we can consider the differential form $ K\,dA/2\pi \in \Omega^2(M)$.
Surprisingly, when integrating this differential form (something which was defined entirely in terms of the geometry of the tangent bundle) gives the Euler characteristic, a purely topological invariant of the base manifold.

\begin{theorem}[Gauss-Bonnet]\label{thm:gauss_bonnet} For any closed Riemannian $2$-manifold $M$, we have:\[\chi(M) = \int_M K\, \frac{dA}{2\pi}.\]
\end{theorem}

Since the map $\Hdr^n(M) \to \R$ which sends a de Rham cohomology class $[\omega]$ to its integral $\int_M \omega$ is an isomorphism, \cref{thm:gauss_bonnet} implies that the cohomology class of $K\,dA/2\pi$ is independent of the choice of inner product on $M$. Rather, it only depends on the structure of $TM$ as an \emph{oriented vector bundle}.
This invariance might lead one to wonder if it's possible to assign a cohomology class to \emph{any} oriented vector bundle on a manifold -- not just the tangent bundle. The answer turns out to be a resounding yes. Gauss and Bonnet had uncovered a deep fact about bundle structures:
\begin{center}
	\slshape
The cohomology of a space serves as a natural source 

of invariants for bundle structures on that space. 
\end{center}
Such an assignment of cohomology classes to bundle structures is known as a \keyword{characteristic class}. 
Here, the vagueness of ``bundle structure'' is intentional -- there are many categories of bundles we can work in. For our purposes, the category of principal $G$-bundles will suffice.

For this to be a useful notion, any such assignment $c$ should respect the common structure between bundles and cohomology classes. Both bundles and cohomology classes \emph{pull back} under continuous maps of the base space, so it would be nice if characteristic classes allowed us to translate this behavior.
Put into notation, if $\xi$ is a bundle on $Y$ and $f : X \to Y$ is a map, we want:
\[
  f^* c(\xi) = c(f^*\xi).
\]
Such an assignment inherently classifies bundles -- the naturality condition implies that isomorphic bundles must be assigned the same class. From a category theoretic viewpoint, this assignment amounts to a natural transformation of functors:
\begin{definition}
  Let $h^*$ be a cohomology theory. An \defn{$h^k$-valued characteristic class}[characteristic class] is a natural transformation of contravariant functors
  \[
  	\lkxsfunc{c}{\Bun_G}{h^k}
  \]
  where $\Bun_G : \Top \to \Set$ is the contravariant functor sending a space to the set of isomorphism classes of principal $G$-bundles.
\end{definition}

Although we haven't yet seen any concrete examples of characteristic classes in general, this definition alone has some interesting consequences. Recall from our discussion in \cref{sec:classifying_spaces} that the functor $\Bun_G$ is representable, meaning that there is some classifying space $BG$ and natural isomorphism
\[
	\lkxsfunc[isom]{}{\Bun_G}{[-, BG].}
\]

\section{The Cohomology of Fiber Bundles}

\todo{Throughout, let's fix $h^*$ to be the singular cohomology theory with some finitely generated commutative coefficient ring $\Lambda$. Assume all spaces are CW complexes.}

Having seen the versatility of fiber bundles in building spaces, constructing topological invariants, and exposing hidden symmetries, we would like to better understand the cohomology of these structures. More specifically, we would like to know:

\begin{center}
	\slshape
	What is the relationship between the cohomology of the fiber,

	total, and base spaces in a fiber bundle? What aspects of

	the bundle structure control this relationship?
\end{center}

We'll begin our investigation with the simplest possible case of a fiber bundle -- the trivial bundle. Picking some fiber space $F$, set the total space to the product $E=B\times F$. Assuming that the spaces $F, B$ have finitely generated cohomology, we get a short exact sequence\footnote{The ``$+1$'' in the sequence indicates that the homomorphism has degree $1$.}
\[
	\begin{tikzcd}
		0 \arrow[r] & h^*(B)\otimes_\Lambda h^*(F) \arrow[r] & h^*(E) \arrow[r, "+1"] & \Tor_1(h^*(B), h^*(F)) \arrow[r] & 0
	\end{tikzcd}
\]
by the K\"unneth formula for cohomology. In many of the cases we are interested in, we can further assume that $h^*(F)$ is a free $\Lambda$-module in each dimension so that the torsion complex $\Tor_1(h^*(B), h^*(F))$ vanishes. We thus get an isomorphism of $\Lambda$-modules:
\[h^*(B)\otimes_\Lambda h^*(F)\cong h^*(E)\]
This isomorphism is the \keyword{cohomology cross product}, defined as
\[
	\lkxlfunc{}{h^*(B)\otimes_\Lambda h^*(F)}{h^*(E)}{\alpha\otimes \beta}{\alpha\times\beta = \pi_B^*(\alpha)\smile \pi_F^*(\beta)}
\]
where $\pi_B : E \to B$ and $\pi_F : E \to F$ are the projection maps. With the aforementioned assumption that $h^*(F)$ is a free $\Lambda$-module, this isomorphism of $\Lambda$-modules is actually an isomorphism of graded rings (or graded $\Lambda$-algebras).\footnote{To see what can go wrong with the multiplicative structure when the cohomology groups $h^*(F)$ are not free, see Example~3E.6 in \cite{hatcher2002}.
}
Altogether, this shouldn't be too surprising; if $E$ is built up in the simplest way possible as a product of $B$ and $F$, then $h^*(E)$ is the simplest thing it could be -- a graded tensor product of $h^*(B)$ and $h^*(F)$.

We've seen the trivial case. What will happen when the fiber bundle is more complicated? Let's see what data we have to work with. Unlike in the case of the trivial fiber bundle, for a general fiber bundle $p : E \to B$ with fiber $F$ we don't have a canonical map $s : E\to F$ with which to define the ``cross product'' homomorphism.
Not only is there not a canonical choice of such a map, but it might not even exist non-trivially! Consider for instance the \keyword{Hopf fibration} $S^1\to S^3 \to S^2$. Any map $s : S^3\to S^1$ must be null-homotopic since $\pi_3(S^1)=0$, and would thus induce the zero map on cohomology.

Maybe we don't need a continuous map $s : E \to F$, but only a homomorphism $s : h^*(F) \to h^*(E)$.
Any such homomorphism should satisfy $\iota_b^*\circ s = \id_{h^*(F)}$, where $\iota_b : F \to E$ is the fiber inclusion map at a particular point $b\in B$. There are some obstructions to finding such homomrphisms. For starters, the restriction map $\iota_b^* : h^*(E) \to h^*(F)$ is not always surjective -- in the Hopf fibration example, the map $\iota_b^* : h^1(S^3) \to h^1(S^1)$ simply can't be surjective since the domain is zero and the target is $\Lambda$. However, if $\iota_b^*$ were surjective then by freeness of $h^*(F)$ there exist splittings of $\iota_b^*$ which is exactly what we're after.

It turns out that the two conditions we've listed -- freeness of $h^*(F)$ and surjectivity of $\iota_b^*$ -- are all that we need to construct a ``cross product'' homomorphism for a general fiber bundle. Some sources call such fiber bundles \defn{totally non-homologous to zero}. The claim that this homomorphism is a module isomorphism is the statement of the Leray-Hirsch Theorem, a generalization of the K\"unneth Formula for fiber bundles.

\subsection{The Leray-Hirsch Theorem}

\begin{theorem}[Leray-Hirsch]\label{thm:leray-hirsch} Let $p : E \to B$ be a fiber bundle with fiber $F$. Suppose that for any $k\in \Z$ and point $b\in B$:
	\begin{enumerate}
		\item $h^k(F)$ is a free $\Lambda$-module of finite rank,
		\item the restriction $\iota_b^* : h^k(E) \to h^k(F)$ is surjective.
	\end{enumerate}
	Choose a splitting $s : h^*(F) \to h^*(E)$ of the surjection $\iota_b^*$. Then, the linear map
	\[
		\lkxlfunc{}{h^*(F)\otimes_\Lambda h^*(B)}{h^*(E)}{\alpha\otimes \beta}{s(\alpha)\smile p^*(\beta)}
	\]
	is an isomorphism of $\Lambda$-modules.
\end{theorem}

We can do slightly better. The projection $p : E \to B$ gives us a natural scalar multiplication action of $h^*(B)$ on $h^*(E)$ given by
\[
	\lkxlfunc{}{h^*(E)\times h^*(B)}{h^*(E)}{(x, \alpha)}{x\smile p^*(\alpha).}
\]
With this scalar multiplication map, $h^*(E)$ is endowed with the structure of a $h^*(B)$-module, an often more useful perspective than viewing it as just a $\Lambda$-module. It should be clear that the isomorphism in \cref{thm:leray-hirsch} is in fact an isomorphism of $h^*(B)$-modules.


\begin{insight}
	The twisting of an ``oriented'' fiber bundle is entirely encoded in the \textbf{multiplicative} structure of cohomology -- the additive structure is the same as the trivial bundle.
\end{insight}

\subsection{Thom Classes and the Thom Isomorphism}

\section{Characteristic Classes In Terms of Curvature}



Let us now review the general theory of the cohomology of fiber bundles -- this is essential in understanding the topology of the total space of a sphere bundle. Remember, we would like to twist the Hopf bundle but not so much as to cause the total space to stop being homemorphic to a sphere.

We will begin our review with the simplest possible case of a fiber bundle -- the trivial bundle. Picking a fiber $F$, we the total space to the product $E=B\times F$. Assuming that the spaces $F, B$ have finitely generated cohomology, we get a short exact sequence\footnote{The ``$+1$'' in the sequence indicates that the homomorphism has degree $1$.}
\[
	\begin{tikzcd}
		0 \arrow[r] & \H^\bullet(B)\otimes \H^\bullet(F) \arrow[r] & \H^\bullet(E) \arrow[r, "+1"] & \Tor_1(\H^\bullet(B), \H^\bullet(F)) \arrow[r] & 0
	\end{tikzcd}
\]
by the K\"unneth formula for cohomology (Theorem 3.15 in \cite{hatcher2002topology}). As in the case of a sphere, we can further assume that $\H^\bullet(F)$ is a free $\Z$-module in each dimension so that the torsion complex $\Tor_1(\H^\bullet(B), \H^\bullet(F))$ vanishes. We thus get an isomorphism of $\Z$-modules:
\[\H^\bullet(B)\otimes \H^\bullet(F)\cong \H^\bullet(E)\]
\begin{remark}
	Note that this is the tensor product of graded rings.
\end{remark}
The isomorphism is the cohomology cross product, defined as
\[
	\lkxfunc{}{\H^\bullet(B)\otimes \H^\bullet(F)}{\H^\bullet(E)}{\alpha\otimes \beta}{\alpha\times\beta = \pi_B^*(\alpha)\smile \pi_F^*(\beta)}
\]
where $\pi_B : E \to B$ and $\pi_F : E \to F$ are the projection maps. With the aforementioned assumption that $\H^\bullet(F)$ is a free $\Z$-module, this isomorphism of $\Z$-modules is actually an isomorphism of graded rings (or graded $\Z$-algebras).\footnote{To see what can go wrong with the multiplicative structure when the cohomology groups $\H^\bullet(F)$ are not free, see Example~3E.6 in \cite{hatcher2002topology}.
}
Altogether, this shouldn't be too surprising; if $E$ is built up in the simplest way possible as a product of $B$ and $F$, then $\H^\bullet(E)$ is the simplest thing it could be -- a graded tensor product of $\H^\bullet(B)$ and $\H^\bullet(F)$.

Unlike in the case of the trivial fiber bundle, for a general fiber bundle $p : E \to B$ with fiber $F$ we don't have a canonical map $s : E\to F$ with which to define the cross product homomorphism.
Not only is there not a canonical choice of such a map, but it might not even exist non-trivially.

\begin{remark}
For instance, in the complex Hopf bundle $S^1\to S^3 \to S^2$, any map $s : S^3\to S^1$ must be null-homotopic since $\pi_3(S^1)=0$, and would thus induce the zero map on cohomology.
\end{remark}

The two conditions for a cross product homomorphism in a general fiber bundle are freeness of $\H^\bullet(F)$ and surjectivity of the induced map $\iota_b^* : \H^k(E) \to \H^k(F)$ corresponding to the fiber inclusion map $\iota_b : F \to E$ at ever point $b\in B$. Note that the latter condition does not hold for the complex Hopf bundle. 

\begin{theorem}[Leray-Hirsch]\label{thm:leray-hirsch} 
	Let $p : E \to B$ be a fiber bundle with fiber $F$. Suppose that for any $k\in \Z$ and point $b\in B$:
	\begin{enumerate}[(a)]
		\item $\H^k(F)$ is a free $\Z$-module of finite rank,
		\item the restriction $\iota_b^* : \H^k(E) \to \H^k(F)$ is surjective.
	\end{enumerate}
	Choose a splitting $s : \H^\bullet(F) \to \H^\bullet(E)$ of the surjection $\iota_b^*$. Then, the linear map
	\[
		\lkxfunc{}{\H^\bullet(F)\otimes \H^\bullet(B)}{\H^\bullet(E)}{\alpha\otimes \beta}{s(\alpha)\smile p^*(\beta)}
	\]
	is an isomorphism of $\Z$-modules.
\end{theorem}
\begin{proof}
	See Theorem 4D.1 of \cite{hatcher2002topology}.
\end{proof}

This is an immensely useful result in computing cohomology, since many spaces arise as total spaces of bundles. We will make extensive use of it in our construction of universal characteristic classes in \cref{sec:universal-characteristic-classes}.

\todo{Finish constructing the Euler class}

\begin{theorem}[Relative Leray-Hirsch]\label{thm:relative-leray-hirsch}
	Suppose $p : (E,E') \to B$ is a relative fiber bundle with fiber $(F,F')$.
\end{theorem}

\begin{definition}\label{def:euler-class}
\end{definition}

\begin{theorem}[Gysin Sequence]
	Whenever we have a bundle $S^{n-1} \to E\to B$ over a simply-connected base $B$, there is an exact sequence known as the \defn{Gysin sequence} 
	\[
		\cdots \lkxto \H^{k-n}(B)\lkxto[e\smile] \H^k(B)\lkxto[p^*] \H^k(M)\lkxto H^{k-n+1}(B)\lkxto \cdots
	\]
	where $e\in \H^n(B)$ is the Euler class of the fiber bundle and $p$ is the projection $E \to B$.
\end{theorem}
\begin{proof}
	See Section 4.D of \cite{hatcher2002topology} or Theorem 17.9.2 of \cite{dieck2008algebraic}.
\end{proof}


Under this correspondence the Euler number of a vector bundle can be considered as a group homomorphism
\[
	\lkxfunc{e}{\pi_{m-1}(\SO_m)}{\Z}{\tau}{e(\xi_\tau)}
\]
