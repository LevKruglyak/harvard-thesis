\chapter{On the Parallelizability of the Spheres}

Summary of \cite{bott1958parallelizability}

\begin{lemma}
	If the Stiefel-Whitney classes $w_1, w_2, \ldots, w_{4k-1}$ of a principal $\GL_m$-bundle are zero, then the Pontryagin class $p_k$ reduced modulo $4$, is equal to $i_* \omega_{4k}$.
\end{lemma}

For a bundle over $S^{4k}$, this means that $\omega_{4k}$ is zero if and only if $p_k$ is divisible by $4$. Now, if you can prove that $p_k$ is divisible by $(2k-1)!$ it will follow that $w_{4k}$ must be zero whenever $k\geq 3$.

\begin{theorem}
	There exists a $\GL_m$-bundle over $S^n$ with $\omega_n\neq 0$ only if $n = 1,2,4,8$.
\end{theorem}

This is a generalization of Wu, who proved this can only happen when $n=2^k$.

\begin{corollary}
	$\R^n$ possesses a bilinear product operation without zero divisors only for $n$ equal to $1,2,4,$ or $8$.
\end{corollary}

\begin{corollary}
	The sphere $S^n$ is parallelizable only for $n=1,3,$ or $7$.
\end{corollary}

\chapter{A Note on Obstructions and Characteristic Classes}

Summary of \cite{kervaire1959}.

\chapter{Constructing the \texorpdfstring{$J$}{J}-homomorphism}

Sort of summary of \cite{whitehead1942}.

\begin{definition}
 The \defn{reduced join} of pointed spaces $X$ and $Y$ is:
 \[X * Y = \Sigma(X\wedge Y)\]
\end{definition}

For example, $S^n * S^m = S^{m+n+1}$.

\begin{definition}
	The \defn{Hopf construction} gives a map
	\[
		\Map(X\times Y, Z) \lkxto \Map(X * Y, \Sigma Z).
	\]
\end{definition}

The original homomorphism defined geometrically is
\[
		J : \pi_r(\SO_q) \lkxto \pi_{r+q}(S^q)
\]
for $q$ and $r\geq 2$. (Hopf defined it for $q=r+1$.) An element of $\SO_q$ can be interpreted as a map $S^{q-1} \to S^{q-1}$, so an element of $\pi_r(\SO_q)$ can be represented by a map:
\[
	\begin{aligned}
	S^r \times S^{q-1} \lkxto S^{q-1} \quad\overset{\textrm{Hopf construction}}{\implies}\quad 
	 S^{r+q} &\lkxto S^q.\phantom{(S^{q-1})}\\
	 \textcolor{gray}{(S^r * S^{q-1})} &\textcolor{gray}{\lkxto} \textcolor{gray}{\Sigma (S^{q-1})}
\end{aligned}
\]
This last map is the image of the element of $\pi_r(\SO_q)$ in $\pi_{r+q}(S^q)$ by the $J$ homomorphism. In stable homotopy theory, we take the limit as $q\to \infty$ to get the \defn{stable $J$-homomorphism}
\[
		J : \pi_r(\SO) \to \pi_r^S.
\]

\chapter{Groups of Homotopy Spheres I}

Summary of \cite{milnor1963groups}.

\section{Homotopy Spheres are Stably Parallelizable}

Let $M$ be a manifold and let $\underline{\R}^1$ denote a trivial line bundle over $M$.

\begin{definition}
	$M$ will be called \defn{s-parallelizable} if the Whitney sum $TM \oplus \underline{\R}^1$ is  a trivial bundle. The bundle $T^sM = TM\oplus\underline{\R}^1$ will be called the \defn{stable tangent bundle} of $M$.
\end{definition}

This is a stable bundle in the sense \cite{kervaire1959}. More explicitly, $T^sM$ is a representative of the stable class of $TM$ (at least for homotopy spheres), i.e. $\{TM\oplus \underline{\R}^k\}$. In particular, if $TM\oplus \underline{\R}^k$ is trivial, then $T^s M$ is trivial as well. (See Lemma 3.5 in \cite{milnor1963groups}.)

\begin{definition}
	For a Lie group $G$, a \defn{stable principal $G$-bundle}[stable principal bundle] is a principal $G$-bundle over a finite CW complex $K$ with $\pi_{q-1}(G)$ stable for $q \leq \dim K$.
\end{definition}

\begin{theorem}
	Every homotopy sphere is \textsc{s}-parallelizable.
\end{theorem}

\begin{proof}
	Let $\Sigma$ be a homotopy $n$-sphere.

	The only obstruction to the triviality of $T^sM$ is a well-defined cohomology class:
	\[
		\mathfrak{o}_n(\Sigma) \in \H^n(\Sigma; \pi_{n-1}(\SO_{n+1})) = \pi_{n-1}(\SO_{n+1})
	\]
	The coefficient group may be identified with the stable group $\pi_{n-1}(\SO)$, but these stable groups have been computed by Bott in \cite{bott1957}, for $n\geq 2$ we have:
	\begin{center}
		\begin{tabular}{c|cccccccc}
			\textrm{residue class of $n\mod 8$} & 0 & 1 & 2 & 3 & 4 & 5 & 6 & 7\\
			\hline
			$\pi_{n-1}(\SO)$ & $\Z$ & $\Z_2$ & $\Z_2$ & 0 & $\Z$ & 0 & 0 & 0.
		\end{tabular}
	\end{center}
	If $\pi_{n-1}(\SO)$ is zero, we are done. 

	If $\pi_{n-1}(\SO) = \Z$, then $n=4k$. According to \cite{kervairemilnor1960} and \cite{kervaire1959}, some non-zero multiple of the obstruction class $\mathfrak{o}_n(\Sigma)$ can be identified with the Pontryagin class $p_k(T^s M) = p_k(TM)$. \todo{(why?)} But the Hirzebruch signature theorem implies \todo{(why?)} that $p_k(\Sigma)$ is a multiple of the signature $\sigma(\Sigma)$ which is zero since $\H^{2k}(\Sigma)=0$. Thus every homotopy $4k$-sphere is \textsc{s}-parallelizable. 

	Finally, suppose $\pi_{n-1}(\SO)= \Z_2$. It follows from an argument of Rohlin \todo{(what?)} that $J_{n-1}(\mathfrak{o}_n(\Sigma))=0$ where $J_{n-1}$ denotes the Hopf-Whitehead homomorphism
	\[
		\lkxfunc{J_{n-1}}{\pi_{n-1}(\SO_k)}{\pi_{n+k-1}(S^k)}
	\]
	in the stable range $k >n$. But $J_{n-1}$ is injective for $n\equiv 1, 2\mod 8$. This is proven by Adams. \todo{(find)} This means that $\mathfrak{o}_n(\Sigma)=0$.
\end{proof}

Some clarifications on the concept of stable parallelizability.

Let $\xi$ be a $k$-dimensional vector space bundle over an $n$-dimensional complex with $k>n$.
\begin{lemma}
If the Whitney sum of $\xi$ with a trivial bundle $\underline{\R}^r$ is trivial, then $\xi$ itself is trivial.
\end{lemma}

\begin{proof}
	We may assume that $r=1$, and that $\xi$ is oriented. An isomorphism \todo{finish}
\end{proof}

\begin{lemma}
	Let $M$ be an $n$-dimensional submanifold of $S^{n+k}$ with $n<k$. Then $M$ is \textsc{s}-parallelizable if and only if its normal bundle is trivial.
\end{lemma}

\begin{proof}
	Let $\tau, \nu$ denote the tangent and normal bundles of $M$. Then $\tau\oplus \nu$ is trivial hence $(\tau\oplus \underline{\R}^1)\oplus \nu$ is trivial. Applying the previous lemma, the conclusion follows.
\end{proof}

\begin{lemma}
	A connected manifold with non-vacuous boundary is parallelizable if and only if it is parallelizable.
\end{lemma}

\section{Which homotopy spheres bound parallelizable manifolds?}

Define a subgroup $bP_{n+1} \subset \Theta_n$ as follows:

A homotopy $n$-sphere $\Sigma$ is in $bP_{n+1}$ if and only if $\Sigma$ is the boundary of a parallelizable manifold.

\begin{claim}
	This condition depends only on the $h$-cobordism class of $\Sigma$.
\end{claim}

\begin{theorem}
	The quotient group $\Theta_n / bP_{n+1}$ is finite.
\end{theorem}

\begin{proof}
	Let $M$ be an \textsc{s}-parallelizable $n$-manifold. 

	Imbed it as $i : M \to S^{n+k}$ for some $k>n+1$ so that it's normal bundle is trivial.

	For each normal $k$-frame $\varphi$, we get an element of $\pi_{n+k}(S^k) = \pi^s(S^n)$ by the Pontryagin-Thom construction. Let's call the set of these elements (as $\varphi$ is allowed to vary) $p(\Sigma)$. \todo{(elaborate)}

	\todo{add lemmas}

	\begin{lemma}
		There is a homomorphism:
		\[
			\lkxfunc{p'}{\Theta_n}{\pi^s(S^n)/p(S^n)}
		\]
	\end{lemma}

	Furthermore, the kernel of $p'$ contains $h$-cobordism classes of homotopy $n$-spheres which bound parallelizable manifolds \todo{(provide lemma)}, which is exactly $bP_{n+1}$. By the first isomorphism theorem, it follows that $\Theta_n/bP_{n+1}$ is isomorphic to a subgroup of $\pi^s(S^n)$ which is finite.
\end{proof}

An alternative way to describe $p(S^n)$ is in terms of the Hopf-Whitehead homomorphism, indeed $p(S^n) = \coker(J_n)$.
	\begin{center}
		\begin{tabular}{c|cccccccc}
			n & 0 & 1 & 2 & 3 & 4 & 5 & 6 & 7\\
			\hline
			$\pi^s(S^n)$ & $\Z_2$ & $\Z_2$ & $\Z_{24}$ & 0 & $0$ & $\Z_2$ & $\Z_{240}$ & $\Z_2\oplus\Z_2$\\
			$\pi^s(S^n) / p(S^n)$ & $0$ & $\Z_2$ & $0$ & 0 & $0$ & $\Z_2$ & $0$ & $\Z_2$\\
			$\Theta_n/bP_{n+1}$ & 0 & 0 & 0 & 0 & 0 & 0 & 0 & $\Z_2$
		\end{tabular}
	\end{center}

\subsection{Proof that $bP_{2k+1}$ are trivial}
\subsection{Proof that $bP_{2k}$ are finite cyclic for $k\neq 2$}
