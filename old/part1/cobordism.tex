\chapter{Cobordism}

Simply put, cobordism is 

% One of the deep connections between geometry and homotopy theory is the \defn{Thom-Pontryagin construction} -- a wide bridge which allows us to transform classification questions about manifolds into homotopy theoretic questions about objects known as \defn{spectra}[spectrum]. More specifically, we'll see how every smooth manifold with additional geometric structure can be associated to an element of a homotopy group of a corresponding spectrum. As such, this spectrum serves as a kind of ``classifying space'' for manifolds of a given geometry.

Now these correspondences are onto but not generally one-to-one, as non-homeomorphic manifolds are often associated to the same homotopy class. The equivalence relation on manifolds which makes the Thom-Pontryagin map into an isomorphism is known as \defn{cobordism}, deriving from the French word \emph{bord}. (meaning edge, border) Conveniently for us, this equivalence relation of cobordism admits a simple description! While classifying manifolds up to homeomorphism 

% Let's suppose we wanted to develop a 
