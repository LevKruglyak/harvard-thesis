\chapter{Geometric Topology}\label{chap:fundamentals}

% \begin{epigraph}{21em}{Pythagoras}
% 	There is geometry in the humming of the strings,\\
% 	and there is music in the spacing of the spheres
% \end{epigraph}
%

\begin{epigraph}{15em}{Frank Herbert}
	Beginnings are such delicate times.
\end{epigraph}

To begin our journey, we will provide some mathematical essentials which will be used ubiquitously throughout the thesis. These prerequisites can loosely be broken up into three sections. First, in \cref{sec:geometric-topology} we will discuss operations on smooth manifolds in geometric topology -- how to glue manifolds together, cut off handles, attach handles, and so on. We also are concerned with how such operations affect the homology of the manifold. By carefully choosing submanifolds representing homology classes, we will see how to alter the homology of a manifold in a controlled way by means of these operations. These operations form the basic ideas of surgery theory, a topic we will explore in greater depth in \cref{chap:h-cobordism} and \cref{chap:classification}. \todo{mention intersection form}

Altering topology by surgery naturally ties into our next topic of cobordism in \cref{sec:cobordism}, a natural notion of equivalence for manifolds. While cutting off or attaching handles changes the diffeomorphism type of a manifold, such operations do not change its cobordism type. This loose notion of equivalence allows us to fully classify manifolds up to cobordism -- something which is generally impossible for homeomorphism or difeomorphism.\footnote{In high dimensions, classifying manifolds is an \defn{undecidable problem}, meaning there is no finite algorithm that will correctly match a manifold to a list of representatives.}
The determination of manifolds up to cobordism is also one of the entrypoints of homotopy theory into differential topology via the Thom-Pontryagin construction, a massively successful crossover which reduces hard geometric problems to (also admittedly hard) problems of computing homotopy groups. With each flavor of cobordism, a new homotopy problem emerges. Ultimately, it will turn out that most of the computational difficulty in computing the set of smooth structures on a sphere is as a result of these problems in homotopy theory.

\todo{what other sections we need}

\pagebreak
\section{Operations on Smooth Manifolds}\label{sec:geometric-topology}

\begin{convention*}
	We assume all manifolds are smooth and with boundary. A manifold is said to be \defn{closed}[closed manifold] if it is compact and with empty boundary.
\end{convention*}

Whenever a closed manifold $N^k$ is embedded into the interior of a larger ambient manifold $M^n$, there is a short exact sequence of tangent bundles
\begin{equation}
	0 \lkxto \T N\lkxto \T M|_N \lkxto \T M/N \lkxto 0
\end{equation}
where $\T M/N=\T M|_N / \T N$ denotes the \defn{normal bundle} of $N$ in $M$. A \defn{tubular neighborhood} of $N$ is a neighborhood $V\supset N$ in $M$ which is the diffeomorphic image of the set
\begin{equation}
	D(\T M/N) = \{\xi \in \T M/N \mid \|\xi\| <1 \}\subset \T M/N,
\end{equation}
for some inner product structure on $\T M/N$. Here $D(\T M/N)$ is the total space of the $(n-k)$-dimensional disk bundle over $N$ arising from the inner product. We also require the restriction of the diffeomorphism to the image of the zero section in $D(\T M/N)$ to be the identity map to $N$.

A tubular neighborhood is then a way of ``thickening'' the submanifold $N$ as in \cref{fig:tubular-neighborhood}.
The diffeomorphism mapping $D(\T M/N) \to V$ can interpreted as a chart of a tube surrounding $N$ in $M$. Namely, for each point $p\in N$, local trivializations of the bundle $D(\T M/N)$ are maps $\R^k \times D^{n-k}\to U\subset V$ which consist of coordinates $(x^1,\ldots,x^k)$ on $N$ and coordinate $(v^{k+1},\ldots, v^{n})$ on an orthogonal disk to $N$.

\begin{figure}[ht]
	\centering
	\import{diagrams}{placeholder.pdf_tex}
	\caption{A tubular neighborhood of a submanifold.}\label{fig:tubular-neighborhood}
\end{figure}

\begin{remark}
	Note that 
\end{remark}



\begin{theorem}[Tubular Neighborhood Theorem]\label{thm:tubular-neighborhood}
	Every properly embedded submanifold has a tubular neighborhood.
\end{theorem}

\begin{theorem}[Collar Neighborhood Theorem]\label{thm:collar-neighborhood}
	Hello world
\end{theorem}

\todo{check boundary}

\subsection{Joining Two Manifolds Along a Submanifolds}

In the case that we have two manifolds $M_1^n$ and $M_2^n$ with interior embeddings $\iota_1 : N \to M_1$ and $\iota_2 : N\to M_2$ of some closed manifold $N^k$, we can join $M_1$ and $M_2$ along $N$ to get a single manifold which we might denote $M_1\cup_N M_2$.
To construct this manifold, let's start by picking tubular neighborhoods $V_1$ and $V_2$ for $\iota_1(N)\subset M_1$ and $\iota_2(N)\subset M_2$ respectively. We then set the joined manifold to be the quotient
\begin{equation}\label{eq:join-definition}
	M_1\cup_N M_2 = \frac{(M_1\setminus \iota_1(N))\sqcup (M_2\setminus \iota_2(N))}{(t\xi_1, \iota_1(p)) \sim ((1-t)\xi_2, \iota_2(p))}
\end{equation}
for all $p\in N$, $t\in(0,1)$, and unit normal vectors $\xi_1\in \T_p M/\iota_1(N)$, $\xi_2\in \T_p M/\iota_2(N)$ with $\|\xi_1\|=\|\xi_2\|=1$. The resulting smooth manifold is called a \defn{join along a submanifold}\footnote{Some authors refer to this operation as ``\defn{pasting}'' or as the ``\defn{generalized connected sum}'', see for instance Section VI.4 in \cite{kosinski1993differential}.} and is one of the fundamental operations of geometric topology.

\begin{remark}
	The tubular neighborhood ``sewing'' procedure in \cref{eq:join-definition} is extremely important in the category of smooth manifolds. If we were in the category of topological manifolds (assuming there were a notion of tubular neighorhood), we could define the join in a simpler way by removing the tubular neighborhoods from both manifolds and identifying the boundaries of the resulting manifolds by some homeomorphism. At each point of the shared boundary, the join would be locally Euclidean since it has a neighborhood which the glueing of two half-disks to get a full disk.

	However, the lack of a globally consistent ``collar'' by which these manifolds were glued leads to many possible smooth structures on the resulting manifold -- even if we require the boundary glueing map to be an orientation-preserving diffeomorphism. We will actually exploit this lack of unique smooth structure in \cref{sec:twisted-spheres} by constructing exotic spheres as the identification of two disks by an orientation-preserving diffeomorphism of their boundaries.
\end{remark}

\subsection{Connected Sum}\label{sec:connected-sum}

The simplest way to join two manifolds $M_1$ and $M_2$ along a submanifold is by embedding a point in both. The resulting join is an operation known as a \defn{connected sum} of $M_1$ and $M_2$. If the manifolds are connected,

and denoted by $M_1\# M_2$.

\begin{theorem}[Disk Theorem]
	test
\end{theorem}
\begin{proof}
	See Theorem B in \cite{palais1960diffeomorphism}.
\end{proof}

\begin{proposition}\label{prop:connected-sum-operation}
	The connected sum
\end{proposition}

\subsection{Spherical Modifications}

\begin{figure}[ht]
	\import{diagrams}{surgery-on-two-holed-torus.pdf_tex}
	\caption{Removing a }
\end{figure}

\subsection{Joining Manifolds Along Submanifolds of the Boundary}

\pagebreak
\section{Homology Classes and Submanifolds}\label{sec:representing-homology-classes}

Let $f : N\subset X$ be a smooth map from a compact oriented $p$-dimensional manifold $N$ to $X$. The data of an orientation on a closed manifold gives a fundamental class $[N]\in \H_p(N)$ which can be pushed forward along the map $f : N \hookrightarrow X$ to give a homology class $f_* [N]\in \H_p(X)$. This is the homology class associated to a smooth map $N\to X$.
This correspondence behaves nicely with respect to perturbations, \todo{finish}

Suppose $H : N\times [0,1] \to X$ is a homotopy with $H(x,0)=f(x)$ which perturbs the map $f$. For any $\epsilon>0$, the map $f_\epsilon : N \to X$ given by $f_\epsilon(x)=H(x,\epsilon)$ is homotopic to $f$ and hence induces the same map on homology $\H_p(N)\to \H_p(X)$. The homology class associated to a map $f : N \to X$ thus solely depends on the homotopy type of $f$ so there is a map
\begin{equation}
	\lkxfunc{}{[N,X]}{\H_p(X).}
\end{equation}
Letting $N=S^p$, we can see that this is a generalization of the Hurewicz homomorphism which links homotopy groups to homology groups via a map $\pi_p(X) \to \H_p(X)$.
As with the Hurewicz homomorphism, this correspondence is generally not surjective or injective. If a space is $k$-connected for $k\geq 1$ the Hurewicz homomorphism is an isomorphism $\pi_{k+1}(X) \to \H_{k+1}(X)$. Consequently, every homology cycle in $\H_{k+1}(X)$ can at least be represented by a smooth map of a sphere $S^{k+1}$ into $X$. This smooth map might have ``double-points'', i.e. when multiple points of the sphere map to the same point in the image and prevent the smooth map from being an embedding.

\begin{example}
	In the punctured plane $X=\R^2\setminus \{0\}$ with homology $\H_1(X)\cong \Z$, the only homology cycles which can be represented by embedded submanifolds are $0,\pm 1$, by a circle not containing the origin and circles of both orientations surrounding the origin respectively. A smooth map representing a cycle of higher degree would necessarily have a double-point as in \cref{fig:double-point}.
\end{example}

\begin{figure}[ht]
	\centering
	\import{diagrams}{double-point.pdf_tex}
	\caption{A double-point in a smooth map representing $\pm 2\in \H_1(\R^2\setminus\{0\})$.}\label{fig:double-point}
\end{figure}

That being said, many of the manifolds considered \todo{basis by embedded submanifolds}.
many specially constructed manifolds we will consider in this chapter will at least have a basis by embedded submanifolds, and every homology class will admit a representation by a smooth map. For a classical account of some issues which can arise when representing homology classes by smooth maps, see Chapter II of Ren\'e Thom's seminal paper \cite{thom1954}.

\begin{remark}
	In some cases, homology classes can \emph{always} be represented by embedded submanifolds. For instance, if $M$ is a $4$-manifold, there are isomorphisms
	\begin{equation}
		\H^2(M; \Z) \cong [M, K(\Z,2)] \cong [M, \CP^\infty] \cong [M,\CP^2],
	\end{equation}
	where the first is the representability of singular cohomology by the Eilenberg-Maclane spectrum, the second identifies $\CP^\infty$ as a $K(\Z,2)$ space, and the third uses the cellular approximation theorem to slide maps onto the $4$-skeleton. Any cohomology cycle $\omega\in \H^2(M)$ can be represented by a smooth function $f : M \to \CP^2$. If we choose this function to be transverse to $\CP^1\subset \CP^2$, then $f^{-1}(\CP^1)$ is an embedded $2$-dimensional submanifold of $M$ which corresponds to a Poincar\'e dual class to $\omega$. When $X$ is a compact manifold, Poincar\'e duality tells us that all $2$-dimensional homology cycles arise from $2$-dimensional cohomology cycles and can thus be represented by embedded submanifolds. This is one example of the attractiveness of $4$-manifolds as geometric objects of study.
\end{remark}

\todo{give general conditions for representability}

\subsection{Intersections in Differential Topology}\label{sec:differential-topology-intersections}

\begin{theorem}[Preimage Theorem]\label{thm:preimage}
	If $f : N \to X$ is a smooth map transverse to a submanifold $M\subset X$ then $S=f^{-1}(M)\subset N$ is a submanifold with the same codimension in $N$ as $M$ in $X$.
\end{theorem}
\begin{proof}
	See the proof of Theorem~6.30 in \cite{lee2013smooth}.
\end{proof}

\begin{remark}\label{rmk:symmetric-preimage-theorem}
	We can get a symmetric version of this theorem as a straightforward corollary. If we have two transverse maps $f : N\to X$ and $g : M\to X$, then the map $f\times g : N\times M \to X\times X$ is transverse to the diagonal submanifold $\Delta\subset X\times X$. \cref{thm:preimage} will then imply that
	\[
		(f\times g)^{-1}(\Delta) \subset M\times N
	\]
	is a submanifold. When $g$ is an embedding, $(f\times g)^{-1}(\Delta)$ can be projected down onto $M$ to get the preimage $f^{-1}(M)$.
\end{remark}

If the manifolds involved in \cref{thm:preimage} are orientable, this preimage $S$ admits a canonical orientation by the following procedure. First of all, recall that for any embedded manifold $M\subset X$ there is an exact sequence of vector bundles by quotienting
\begin{equation}\label{eq:oriented-intersection-number-1}
	\begin{tikzcd}
		0 & {\T M} & {\T X} & {\T X/M} & 0
		\arrow[from=1-1, to=1-2]
		\arrow[from=1-2, to=1-3]
		\arrow[from=1-3, to=1-4]
		\arrow[from=1-4, to=1-5]
	\end{tikzcd}
\end{equation}
where $\T X/M$ is the normal bundle of $M\subset X$. Using the orientations of $X$ and $M$, we can use this exact sequence to get an orientation of the normal bundle $\T X/M$. At every point $p\in S$ of the preimage, the differential map $df$ connects the sequence \cref{eq:oriented-intersection-number-1} to the normal bundle sequence for the embedding $S\subset N$.
\begin{equation}\label{eq:oriented-intersection-number-2}
	\begin{tikzcd}
		0 & {\T_pS} & {\T_p N} & {\T_p N/S} & 0 \\
		0 & {\T_{f(p)}M} & {\T_{f(p)}X} & {\T_{f(p)}X/M} & 0
		\arrow[from=1-1, to=1-2]
		\arrow[from=1-2, to=1-3]
		\arrow["{df_p}", from=1-2, to=2-2]
		\arrow[from=1-3, to=1-4]
		\arrow["{df_p}", from=1-3, to=2-3]
		\arrow[from=1-4, to=1-5]
		\arrow["{df_p}", from=1-4, to=2-4]
		\arrow[from=2-1, to=2-2]
		\arrow[from=2-2, to=2-3]
		\arrow[from=2-3, to=2-4]
		\arrow[from=2-4, to=2-5]
	\end{tikzcd}
\end{equation}
In this diagram \cref{eq:oriented-intersection-number-2}, the rightmost vertical map is an isomorphism by the transversality of $f$ and $M$. This means that we can pullback the orientation on $\T_{f(p)} X/M$ to $\T_p N/S$. Since $\T_p N$ is oriented, the usual ``2-out-of-3'' rule applied to the top row of \cref{eq:oriented-intersection-number-2} gives an orientation of $\T_p S$. See \cref{fig:preimage-orientation} for an example of this orienting procedure.

\begin{figure}[ht]
	\centering
	\import{diagrams}{preimage-orientation.pdf_tex}
	\medskip
	\caption{Orienting a preimage (assuming a clockwise orientation on $X$ and $N$).}\label{fig:preimage-orientation}
\end{figure}

When $M$ and $N$ have complementary dimensions, the preimage $S=f^{-1}(N)\subset M$ is a compact oriented $0$-dimensional manifold. For each point $p\in S$, we have $\T_p S=0$ so the map $\T_p N\to \T_p N/S$ in \cref{eq:oriented-intersection-number-2} is an isomorphism. The orientation of $N$ gives an orientation of $\T_p N$, and the preimage orientation procedure gives us an orientation of $\T_p N/S$. Now we can define:

\begin{definition}
	The \defn{local (oriented) intersection number}[oriented intersection number (local)] of $f$ and $M$ at $p\in S$ is
	\[
		\#_p^X(f, M) = \begin{cases}
			+1 & \T_p N/S \textrm{ has the same orientation as } \T_p N,     \\
			-1 & \T_p N/S \textrm{ has the opposite orientation to } \T_p N.
		\end{cases}
	\]
\end{definition}
Summing over all of the local intersection numbers gives a global quantity.
\begin{definition}
	The \defn{(oriented) intersection number}[oriented intersection number] of a smooth map $f : N \to X$ intersecting a submanifold $M\subset X$ transversally is
	\[
		\#^X(f, M) = \sum_{p\in S} \#_p^X(f, M) \in \Z.
	\]
\end{definition}

\begin{remark}\label{rmk:symmetric-intersection-number}
	For a more symmetric version of this definition when two smooth maps $f : N \to X$ and $g : M \to X$ intersect transversally, we could take inspiration from \cref{rmk:symmetric-preimage-theorem} and define the oriented intersection number of the smooth maps $f$ and $g$ as
	\[
		\#^X(f,g) = \#^{X\times X}(f\times g, \Delta).
	\]
	This symmetric intersection number is graded commutative in the dimensions of $M$ and $N$, i.e.
	\begin{equation}\label{eq:intersection-number-graded-commutative}
		\#^X(f,g) = (-1)^{\dim M\cdot \dim N} \#^X(g,f)
	\end{equation}
\end{remark}

Just as the property of transversality is stable -- resilient to homotopic perturbations -- so too is the oriented intersection number. This follows as a corollary to a more general theorem.

\begin{theorem}
	If $W$ is a compact oriented manifold with boundary, and $H : W \to X$ is a smooth map, then $\#^X(\partial H, M)=0$. Here, we use the notation $\partial H : \partial W \to X$ to refer to the restriction of $H$ to the boundary of $W$.
\end{theorem}

\begin{corollary}
	If $H : [0,1]\times N \to X$ is a smooth homotopy, then $\#^X(H_0, M) = \#^X(H_1, M)$.
\end{corollary}

Applying the construction of the symmetric oriented intersection number gives a map
\begin{equation}\label{eq:oriented-intersection-number-homotopy}
	\lkxfunc{\#^X}{[N,X]\times [M,X]}{\Z}{f,g}{\#^X(f,g)}
\end{equation}
This is a geometric precursor to the intersection form of a manifold, a central object of study in geometric topology.

\begin{remark}
	If we don't assume orientations, we can still get a homotopy invariant intersection number, however we must reduce mod $2$. In this case, we could simply define
	\[
		\#_2^X(f,M) = |S|\mod 2.
	\]
	This is called the \defn{unoriented intersection number}.
	\todo{elaborate}
\end{remark}

Finally, we'll state a useful result in computer self-intersection numbers -- a way to compute the intersection number of a submanifold with itself.

\begin{theorem}\label{thm:euler-number-self-intersection}
	If $M$ is a closed $m$-dimensional submanifold of a $2m$-dimensional submanifold $X$ then we have
	\[
		\#^X(M, M) = \chi(\T X/M)
	\]
	where $\chi(\T X/M)$ is the Euler number of the normal bundle of $M$.
\end{theorem}
\begin{proof}
	\todo{todo}
\end{proof}

\begin{corollary}\label{thm:euler-number-self-intersection-corollary}
	The Euler number $\chi(\xi)$ of an oriented real vector bundle $\xi : E \to B$ over a compact oriented manifold can be expressed as the intersection number
	\[
		\#^E(z,z) = \chi(\xi)
	\]
	where $z : B \to E$ is the zero section.
\end{corollary}

\todo{add 1.C from \cite{levine1985lectures}}

\pagebreak
\section{The Intersection Form}\label{sec:intersection-form}

One of the fundamental invariants for even-dimensional manifolds is the intersection form, a bilinear form on a lattice which captures the geometric data of submanifold intersections. The lattice in question is the free component of the middle-dimensional singular homology, and the pairing of two homology cycles by the form counts their number of ``intersections''. When the homology cycles of complementary dimension (as in the case of middle-dimensional homology cycles) are represented by smooth immersions, we can perturb them to make them transverse without changing the homology class. If the manifolds are compact, the preimage of their intersection is some finte set of points with orientation -- adding up the orientations of the points gives the oriented intersection number.

This is the geometric interpretation of intersection, and we will explore it in more depth when we construct manifolds with given intersection theory in \cref{sec:plumbing}. For now, we will stick to understanding the algebraic properties of intersections as the adage ``think in terms of intersections, prove in terms of homology'' advises. To start, let us suppose that $M$ is a compact oriented $n$-manifold.
The Poincar\'e-Lefschetz duality gives an isomorphism
\begin{equation}
	\lkxfunc{}{\H^{n-p}(M,\partial M)}{\H_p(M)}{\omega}{\omega\frown [M,\partial M]}
\end{equation}
assuming we have an orientation class $[M,\partial M]\in \H_n(M, \partial M)$ corresponding to the orientation of $M$. Under this duality, the intersection of homology classes is defined as the dual operation to the operation of cup product on cohomology classes. This operation is denoted $\alpha\cdot \beta$ for homology cycles $\alpha\in \H_p(M)$ and $\beta\in \H_q(M)$, and is the top map in \cref{eq:homology-intersection}
\begin{equation}\label{eq:homology-intersection}
	\begin{tikzcd}
		{\H_p(M)\otimes \H_q(M)} & {\H_{n-p-q}(M)} \\
		{\H^{n-p}(M, \partial M)\otimes \H^{n-q}(M,\partial M)} & {\H^{2n-p-q}(M,\partial M)}
		\arrow["\tnsv", from=1-1, to=1-2]
		\arrow[tail reversed, from=1-1, to=2-1]
		\arrow[tail reversed, from=1-2, to=2-2]
		\arrow["\smile", from=2-1, to=2-2]
	\end{tikzcd}
\end{equation}
where the vertical maps are the Poincar\'e-Lefschetz isomorphism. Again, the intuition here should be that $\alpha\cdot \beta$ is the homology class representing the intersection of $\alpha$ and $\beta$ when they are arranged in general ``transverse'' position. Done over homology classes of complementary dimension, the resulting homology intersection class is 0-dimensional and hence pairs with an integer multiple $\ell\cdot [M, \partial M]\in \H_n(M,\partial M)$ of the top-dimensional orientation class. The integer multiple $\ell\in \Z$ is the \defn{intersection number}[intersection number of homology classes] of the cycles $\alpha$ and $\beta$. Removing torsion elements and working in middle dimensional homology so that $\alpha$ and $\beta$ live in the same group, we get an integral bilinear form.

\begin{definition}
	Let $M^{2m}$ be a compact oriented even-dimensional manifold, possibly with boundary. The \defn{intersection form} on middle dimensional homology is the bilinear form
	\begin{equation}
		\lkxfunc{Q_M}{\H_m(M)_{\mathrm{free}}\otimes \H_m(M)_{\mathrm{free}}}{\Z}{a\otimes b}{a \tnsv b}
	\end{equation}
	where we identify $\H_0(M)\cong \Z$ and $\H_m(M)_{\mathrm{free}}$ denotes the free component of $\H_m(M)$ -- i.e. the quotient by the subgroup of torsion elements.
\end{definition}

\begin{remark}
	If $m$ is even then $Q_X$ is a symmetric bilinear form and if $m$ is odd then $Q_X$ is a skew-symmetric bilinear form. This follows from the graded commutativity of the cup product, and hence the intersection pairing. For brevity, we say that $Q_X$ is \defn{$m$-symmetric} in such cases.
\end{remark}

\begin{remark}
	Note that the intersection form is defined entirely topologically, without a requirement of smooth structure. That being said, the existence of a smooth structure on a manifold can lead to noticeable effects on the smooth structure.
\end{remark}

It is often helpful to work with the dual pairing, i.e. the cup product pairing on cohomology, since it can be immediately deduced from the multiplicative structure of the cohomology ring.
\begin{definition}
	The intersection form on cohomology is the bilinear form
	\begin{equation}
		\lkxfunc{Q^M}{\H^m(M,\partial M)_{\mathrm{free}}\otimes \H^m(M,\partial M)_{\mathrm{free}}}{\Z}{\alpha\otimes \beta}{\alpha\smile \beta}
	\end{equation}
	where we identify $\H^n(M,\partial M)\cong \H_0(M)\cong \Z$.
\end{definition}

\begin{remark}
	For manifolds which do not come with an orientation, the intersection form can be extended in homology/cohomology with $\Z/2$ coefficients. We call this form the \defn{unoriented intersection form}, and denote it by $\widetilde{Q}_M$ or $\widetilde{Q}^M$ depending on if we are working with homology or cohomology. In the context of embedded submanifolds, this form captures the number of transverse intersection points modulo 2, otherwise known as the unoriented intersection number.
\end{remark}

\begin{remark} \label{rmk:dual-lattice-intersection-form}
	To see the connection between the intersection form on homology and cohomology, let us recall the universal coefficients theorem for cohomology, which gives us an exact sequence
	\[
		0 \lkxto \Ext^1_\Z(\H_{m-1}(M,\partial M)) \lkxto \H^m(M, \partial M) \lkxto \Hom(\H_m(M, \partial M), \Z) \lkxto 0.
	\]
	This is Theorem 3.2 of \cite{hatcher2002topology}. When working with the intersection form, we only care about the torsion free component of homology and cohomology. Note that the $\Ext$ term maps entirely into the torsion part of cohomology $\H^m(M,\partial M)$ since $\Ext^1(F\oplus T; \Z)\cong T$ whenever $F$ is free and $T$ is torsion. We thus get a canonical isomorphism
	\[
		\H^m(M, \partial M) \lkxisom \Hom(\H_m(M,\partial M), \Z).
	\]
	When $\H_{m}(\partial M)$ and $\H_{m-1}(\partial M)$ are trivial, the middle map in
	\[
		\H_m(\partial M) \lkxto \H_m(M) \lkxto \H_m(M,\partial M) \lkxto \H_{m-1}(\partial M)
	\]
	is an isomorphism, and so we get a canonical isomorphism
	\[
		\H^m(M, \partial M) \lkxisom \Hom(\H_m(M), \Z).
	\]
	In other words, (under suitable topological restrictions of $\partial M$) there is a canonical way to identify the lattice for the cohomology intersection form as the dual of the lattice for the homology intersection form. In particular, the matrix representations of the bilinear forms are inverses of each other.
\end{remark}

\subsection{Basic Examples of the Intersection Form}

We will see many examples of manifolds and their intersection forms throughout this thesis, so for now let's just consider the most basic examples -- complex projective spaces and tori.

\begin{proposition}\label{prop:intersection-form-complex-projective-plane}
	The intersection form for any complex projective plane $\CP^{2m}$ of even complex dimension is given by $Q=(1)$, and the intersection form for complex projective plane $\CP^{2m+1}$ of odd complex dimension is trivial.
\end{proposition}
\begin{proof}
	We will compute the intersection form in cohomology, by \cref{rmk:dual-lattice-intersection-form} we can invert the resulting matrix to get an intersection form on homology.

	Recall that the cohomology ring of complex projective space is given by
	\begin{equation}
		\H^\bullet(\CP^{n}) \approx \Z[\alpha]/(\alpha^{n+1})\quad\textrm{with}\quad |\alpha|=2.
	\end{equation}
	A proof of this can be found in any standard algebraic topology book, for instance Theorem~3.19 in \cite{hatcher2002topology}. We can assume without loss of generality that $\alpha^n\in \H^{2n}(\CP^n)$ is the fundamental class corresponding to the canonical orientation on $\CP^n$.
	Note that since the generating element has degree $2$, the middle-dimensional homology $\H^{2m+1}(\CP^{2m+1})$ is trivial and hence so is the intersection form of $\CP^{2m+1}$.

	When the complex dimension is even, the middle-dimension homology $\H^{2m}(\CP^{2m})$ is generated by $\alpha^m$. Since $\alpha^m\smile \alpha^m=\alpha^{2m}$ is a unit multiple of the fundamental class, we have $Q(\alpha^m, \alpha^m)=1$, completing the proof.
\end{proof}

\begin{figure}[ht]
	\centering
	\import{diagrams}{placeholder.pdf_tex}
	\caption{Intersections of homology classes in a complex projective plane.}\label{fig:geometric-intersection-complex-projective}
\end{figure}

It is illuminating to interpret this result geometrically. Let's begin with the complex vector space $\C^{2m+1}$ equipped with a basis $\{e_0, e_1,\ldots, e_{2m}\}$. Consider the linear subspaces
\begin{equation}
	W = \langle e_0, e_1,\ldots, e_m\rangle \quad\textrm{and}\quad U = \langle e_0, e_{m+1},\ldots, e_{2m}\rangle
\end{equation}
in $\C^{2m+1}$. These complex hyperplanes intersect at a complex line $\langle e_0 \rangle = W\cap U$.

Now, we can pass to the projectivization $\P(\C^{2m+1})=\CP^{2m}$ and realize $W$ and $U$ as embedded submanifolds $\P(W)\approx \CP^m\subset \CP^{2m}$ and $\P(U)\approx \CP^m\subset \CP^{2m}$. Since $W$ and $U$ intersect at a line, their projectivizations $\P(W)$ and $\P(U)$ intersect at a point in $\CP^{2m}$. Furthermore, the intersection is transverse, and descending the orientation on $\C^{2m+1}$ to the embedded submanifolds gives an intersection number of $1$. Both of these embedded submanifolds represent the homology class $a^{2m}\in \H_{2m}(\CP^{2m})$ which is the Poincar\'e dual of the cohomology class $\alpha^{2m}(\CP^{2m})$. We again arrive at $Q(a^{2m}, a^{2m})=1$, although this time through homology intersections.

\begin{proposition}\label{prop:intersection-form-torus}
	The intersection form for a torus $T^{2m}=S^m\times S^m$ is given by matrices
	\begin{equation}\label{eq:hyperbolic-form-torus}
		Q = \begin{pmatrix}0 & 1 \\ 1 & 0\end{pmatrix}
		\textrm{ when }m\textrm{ is even, and }
		Q = \begin{pmatrix}0 & 1 \\ -1 & 0\end{pmatrix}
		\textrm{ when }m\textrm{ is odd.}
	\end{equation}
\end{proposition}
\begin{proof}
	Let us again begin with a cohomology computation. The cohomology of a sphere is
	\begin{equation}
		\H^\bullet(S^n) = \Z[\alpha]/(\alpha^2)\quad\textrm{with}\quad |\alpha| = n,
	\end{equation}
	and by the K\"unneth formula (see Theorem~3.15 in \cite{hatcher2002topology}), we have
	\begin{equation}
		\begin{aligned}
			\H^\bullet(T^{2m})=\H^\bullet(S^m\times S^m)\cong \H^\bullet(S^m)\otimes \H^\bullet(S^m)
			 & \cong \Z[\alpha]/(\alpha^2)\otimes \Z[\beta]/(\beta^2) \\
			 & \cong \Z[\alpha,\beta]/(\alpha^2,\beta^2).
		\end{aligned}
	\end{equation}
	Assuming $\alpha$ and $\beta$ are fundamental classes for the spheres, the fundamental class of the torus is $\alpha\smile \beta$. From this multiplicative structure and fundamental class, we clearly have
	\begin{equation}
		Q(\alpha, \alpha)=0, \quad Q(\beta,\beta)=0, \quad Q(\alpha,\beta)=1,\quad Q(\beta,\alpha)=(-1)^m Q(\alpha,\beta)=(-1)^m.
	\end{equation}
	These give exactly the matrices described in \cref{eq:hyperbolic-form-torus}.
\end{proof}

\begin{figure}[ht]
	\centering
	\import{diagrams}{placeholder.pdf_tex}
	\caption{Intersections of homology classes in a torus.}\label{fig:geometric-intersection-torus} 
\end{figure}

The geometric proof of this claim is analogous. The Poincar\'e duals of $\alpha$ and $\beta$, denoted $a$ and $b$ in $\H_{m}(T^{2m})$, are represented
by the embedded submanifolds $S^m\times \{p\}$ and $\{p\}\times S^m$ for some basepoint $p\in S^m$. Shifting an individual embedded sphere to a disjoint embedding by a path taking $p\mapsto p'$ disjoint shows that the self-intersection numbers of $a$ and $b$ are zero. These are the zeroes along the diagonal of the matrices in \cref{eq:hyperbolic-form-torus}. However, the embedded spheres representing $a$ and $b$ intersect transversally at the point $(p,p)\in T^{2m}$. We choose orientations on $S^m\times \{p\}$ and $\{p\}\times S^m$ so that $a\cdot b=1$, then by graded-commutativity we get $b\cdot a=(-1)^m$.

\begin{remark}
	The symmetric form in \cref{eq:hyperbolic-form-torus} is known as the \defn{hyperbolic form}, denoted by
	\[
		H=\begin{pmatrix} 0 & 1\\ 1 & 0 \end{pmatrix}.
	\]
	The hyperbolic form is a fundamental building block for symmetric bilinear forms over the integers and $\Z/2$.
\end{remark}

\subsection{Properties of the Intersection Form}

\begin{proposition}\label{prop:connected-sum-intersection-form}
	For compact manifolds $M_1^{2m}$ and $M_2^{2m}$, we have
	$Q_{M_1\+M_2} \cong Q_{M_1}\oplus Q_{M_2}.$
\end{proposition}
\begin{proof}
\end{proof}

\begin{proposition}\label{prop:orientation-intersection-form}
	For a compact manifold $M^{2m}$, we have
	$Q_{-M} \cong -Q_{M}$.
\end{proposition}
\begin{proof}
	This is immediate, since the fundamental class changes sign as orientation flips.
\end{proof}

An immediate corollary of \cref{prop:connected-sum-operation} and \cref{prop:connected-sum-intersection-form} is
that the intersection form is a homomorphism of commutative monoids, i.e. sets with a commutative associative binary operation and identity elements. On one side, we have the monoid $\mathcal{M}^{2m}$ of oriented compact $2m$-manifolds under connected sum, and on the other side we have $\mathcal{Q}(\Z)$ of bilinear forms valued in $\Z$ under the operation of direct sum. Similarly, the unoriented intersection form maps the monoid of unoriented compact $2m$-manifolds $\widetilde{\mathcal{M}}^{2m}$ to $\mathcal{Q}(\Z)$.
\begin{equation}\label{eq:monoid-homomorphism-intersection-form}
	\lkxfunc{}{\mathcal{M}^{2m}}{\mathcal{Q}(\Z),}{M}{Q_M,}
	\quad\textrm{and}\quad
	\lkxfunc{}{\widetilde{\mathcal{M}}^{2m}}{\mathcal{Q}(\Z/2),}{M}{\widetilde{Q}_M.}
\end{equation}
The monoidal structure of the intersection form is quite useful throughout geometric topology, especially in classification problem.

\subsection{Classification of Manifolds by Intersection Form}
An illustrative case in low dimensions is the classification of compact (unoriented) surfaces up to homeomorphism. Recall that every compact surface is homeomorphic to exactly one of the following surfaces
\[
	S^2,\quad T^2\#\cdots\# T^2,\quad\textrm{or}\quad \RP^2\#\cdots\# \RP^2,
\]
i.e. it is either a sphere, a torus with some number of holes, or an unorientable surface formed by gluing together M\"obius strips. For instance, a Klein bottle is the connected sum $\RP^2\#\RP^2$.
A standard cominatorial proof of this classification by polygonal presentations can be found in Chapter 6 of \cite{lee2011topological}.

By similar arguments to \cref{prop:intersection-form-complex-projective-plane} and \cref{prop:intersection-form-torus}, the unoriented intersection forms of these generating surfaces are given by
\[
	\widetilde{Q}_{S^2}=(0),\quad \widetilde{Q}_{T^2}=\begin{pmatrix}0 & 1 \\ 1 & 0\end{pmatrix},\quad \textrm{and}\quad\widetilde{Q}_{\RP^2} = (1).
\]
For instance, the intersection form of $T^2\+ \RP^2$ is given by
\[
	\widetilde{Q}_{T^2\+ \RP^2} = H\oplus (1)=
	\begin{pmatrix}
		1 & 0 & 0 \\
		0 & 0 & 1 \\
		0 & 1 & 0
	\end{pmatrix}.
\]
This matrix represents a bilinear form, and so the transformation $Q\mapsto P^\intercal Q P$ does not affect the form. In this case, $Q^\intercal =Q$ and $Q^2=I\mod 2$, the transformation $Q\mapsto Q^\intercal Q Q$ gives the form
\[
	\widetilde{Q}_{T^2\+ \RP^2}
	\lkxto \begin{pmatrix}
		1 & 0 & 0 \\
		0 & 1 & 0 \\
		0 & 0 & 1
	\end{pmatrix} =\oplus^3(1)= \widetilde{Q}_{\RP^2\+\RP^2\+\RP^2}.
\]
As it turns out, the underlying surfaces $T^2\+\RP^2$ and $\RP^2\+\RP^2\+\RP^2$ are homoemorphic. This has an easy geometric interpretation. The operation $T^2\+$ can be thought of as adding a handle, and $\RP^2\+\RP^2\+$ being connected sum with a Klein bottle can be thought of as adding a handle in a twisted manner, i.e. one spout on one side of the surface and the other spout on the other side.

However, if we add a handle to a projective plane $\RP^2$ (a M\"obius band with boundary collapsed), we can move one spout around the twist of $\RP^2$ to get a twisted handle (as depicted in \cref{fig:twisted-handle-to-handle}). Thus, the surfaces $T^2\+\RP^2$ and $\RP^2\+\RP^2\+\RP^2$ are homeomorphic, a geometric fact which was detected in part by the algebraic identity of forms $H\oplus (1)=\oplus^3(1)$ in $\mathcal{Q}(\Z/2)$.

\begin{figure}[ht]
	\centering
	\import{diagrams}{placeholder.pdf_tex}
	\caption{Turning $T^2\+\RP^2$ into $\RP^2\+\RP^2\+\RP^2$.}\label{fig:twisted-handle-to-handle}
\end{figure}

\begin{proposition}
	There is a presentation
	\[\mathcal{Q}_{\mathrm{skew}}(\Z/2) = \langle H, (1) \mid H\oplus (1) = \oplus^3 (1)\rangle\]
	where $\mathcal{Q}_{\mathrm{skew}}(\Z/2)$ is the monoid of skew-symmetric bilinear forms over $\Z/2$.
\end{proposition}
\begin{proof}
	See Chapter III of \cite{milnorhuse1973forms} for a generalized statement and proof.
\end{proof}

Just like $H\oplus (1)= \oplus^3 (1)$ is the defining relation for skew-symmetric bilinear forms over $\Z/2$, so too is $T^2\+ \RP^2 = \RP^2\+\RP^2\+\RP^2$ the defining relation for closed surfaces. This leads to an elegant restatement of the classification theorem for closed surfaces.

\begin{theorem}[Classification of Compact Surfaces]
	Let $\mathcal{S}^2\subset \widetilde{\mathcal{M}}^2$ be the monoid of \textit{closed} unoriented surfaces under connected sum. The unoriented intersection form is an isomorphism of monoids
	\[
		\lkxfunc{\widetilde{Q}}{\mathcal{S}^2}{\mathcal{Q}_{\mathrm{skew}}(\Z/2).}
	\]
\end{theorem}

The classification of compact surfaces by the intersection form is a model result of algebraic topology -- a complete algebraic classification of a class of manifolds. Better yet, simple algebraic manipulations correspond to non-trivial topological equivalences. This is part of why intersection forms are so useful -- algebraic intuition scales far better with dimension than does geometric intuition and so bilinear forms are a much more comfortable setting in which to study higher-dimensional topology. For instance, the classification theorem of Michael Freedman in his 1982 paper \cite{freedman1982manifold} is formulated entirely in terms of the intersection form and an additional $\Z/2$-valued invariant detecting the existence of a smooth structure.

\begin{theorem}[Freedman, 1982] Let $\mathcal{S}^4\subset \mathcal{M}^4$ be the monoid of simply-connected closed topological $4$-manifolds. The intersection form
	\[
		\lkxfunc{Q}{\mathcal{S}^4}{\mathcal{Q}_{\mathrm{sym}}(\Z)}
	\]
	is at most two-to-one, i.e. a symmetric intersection form corresponds to at most two topological $4$-manifolds, one which admits a smooth structure and one which does not.
\end{theorem}
An accessible exposition to this remarkable theorem can be found in \cite{behrens2021discembedding}. We will explore this theorem and its consequences with more depth in \cref{sec:smoothing-obstructions}.


\subsection{Intersection Form Invariants}\label{sec:intersection-form-invariants}

While the complete algebraic data of an intersection form captures a lot of the topological structure of a manifold, it is useful to extract further simpler invariants from the intersection form.

We will work with a general integer lattice $\Lambda$ and $Q$ an integral bilinear form over $\Lambda$, not necessarily the intersection form of some manifold. Recall that under a change of basis matrix $P$, the matrix of the bilinear form transforms as $Q\mapsto P^\intercal QP$. We are therefore looking for quantities which are invariant under such transformations, helping us understand the structure of the monoid $\mathcal{Q}(\Z)$.

\begin{definition}
	The \defn{rank} of $Q$ is simply the dimension of the lattice $\Lambda$.
\end{definition}

When $Q$ is the intersection form of a manifold, its rank is the middle Betti number $\beta_m = \dim \H_m(X)_{\textrm{free}}$ of the manifold. This is the simplest invariant of a bilinear form.

\begin{definition}
	A form is said to be \defn{degenerate}[degenerate bilinear form] if $\det Q=0$ and \defn{non-degenerate}[non-degenerate bilinear form] otherwise.
\end{definition}

An equivalent dual way to view a bilinear form is by the homomorphism
\[
	\lkxfunc{Q^\d}{\Lambda}{\Hom(\Lambda, \Z)}{\alpha}{(\beta\mapsto Q(\alpha,\beta)).}
\]
In this context, a bilinear form non-degenerate if and only if $Q^\d$ is injective. We can refine this notion further.
\begin{definition}
	A bilinear form is said to be \defn{unimodular} if $\det Q=\pm 1$. 
\end{definition}
A bilinear form is unimodular if and only if $Q^\d$ is an isomorphism. The notion of unimodularity for integral bilinear forms is a special case of the notion of a \defn{perfect pairing}. A perfect pairing $V\otimes W \to R$ is a bilinear map such that the dual homomorphism $V \to \Hom(W, R)$ is an isomorphism. These notions of degeneracy and unimodularity are not a terribly useful source of invariants by the following proposition.

\begin{proposition}\label{prop:unimodular-intersection-form}
	 If $M$ is a compact manifold with $\H_m(\partial M)$ and $\H_{m-1}(\partial M)$ trivial, then the intersection form $Q_M$ is unimodular.
\end{proposition}
\begin{proof}
	By the universal theorem argument in \cref{rmk:dual-lattice-intersection-form}, we have an isomorphism
	\begin{equation}\label{eq:dual-lattice-isomorphism}
		\H^m(M,\partial M) \lkxisom \Hom(\H_m(M), \Z).
	\end{equation}
	The resulting pairing $\langle -, -\rangle : \H^m(M,\partial M)\otimes \H_m(M) \to \Z$, known as the Kronecker pairing, is therefore a \defn{perfect}[perfect bilinear form] pairing. Since the intersection form is given by
	\[
		Q_M(a,b) = \langle \mathrm{PD}(a), b\rangle,\quad a,b\in \H_m(M)
	\]
	where $\mathrm{PD} : \H^{m}(M,\partial M) \to \H_m(M)$ is the Poincar\'e-Lefschetz duality isomorphism, it follows that $Q_M^\d$ is the composition of $\mathrm{PD}$ with \cref{eq:dual-lattice-isomorphism} and so is an isomorphism itself.
\end{proof}

\todo{examples}
\begin{example}
\end{example}

\begin{definition}
	Write $Q=P^\intercal D P$ for a diagonal real matrix $D$ ($P$ can be a real matrix), then count the number $n^+$ of positive eigenvalues, $n^0$ the number of zero eigenvalues, and number $n^-$ of negative eigenvalues. The \defn{signature}[signature of a bilinear form] of the bilinear form is the difference between the number of positive and negative eigenvalues $n^+-n^-$.
	The triple $(n^+, n^-, n^0)$ is referred to as the \defn{inertia}[inertia of a bilinear form] of $Q$.
\end{definition}

In the context of intersection forms, the signature is only a useful invariant for $4k$-manifolds, since the intersection form of a $(4k+2)$-manifold is skew-symmetric and thus has signature $0$.

\begin{remark}
	By Sylvester's Law of Inertia, the rank and inertia completely classify symmetric bilinear forms on a vector space over $\R$ or $\Q$. Over the integers $\Z$, the classification of symmetric bilinear forms is significantly more complex.

	classifies bilinear forms on a vector space over $\R$ or $\Q$. The case of bilinear forms over $\Z$ is significantly more complex.
\end{remark}

\begin{definition}
	The \defn{signature}[signature of a manifold] of a $4k$-dimensional manifold $M$ is the signature of its intersection form, and is denoted $\sigma(M)$.
\end{definition}

When $Q$ is the intersection form of a manifold $X$, especially a $4k$-manifold so that the form can have non-zero eigenvalues, the signature of the intersection form is referred to as the \defn{signature}[signature of a manifold] of the manifold, and denoted $\sigma(X)$. This is a topological invariant of fundamental importance, and we will see many of its generalizations and equivalent definitions in \cref{sec:hirzebruch-signature-theorem} and \cref{sec:surgery-invariant}.

\begin{definition}
	If for all non-zero elements $\alpha\in \Lambda$ we have $Q(\alpha,\alpha)>0$, then we say that $Q$ is \defn{positive-definite}[positive-definite bilinear form]. On the contrary, if we have $Q(\alpha,\alpha)<0$ for all non-zero elements $\alpha\in \Lambda$, we say that $Q$ is \defn{negative-definite}[negative-definite bilinear form]. Otherwise, $Q$ is said to be \defn{indefinite}[indefinite bilinear form].
\end{definition}

\begin{definition}
	If for all elements $a$, $Q(a,a)$ even, then $Q$ is said to be \defn{even}[even bilinear form]. Otherwise, we say that $Q$ is \defn{odd}[odd bilinear form].\footnote{Some sources refer to this property as ``type''. Odd forms are said to be \defn{type I}[type I bilinear form] and even forms are to be \defn{type II}[type II bilinear form].}
\end{definition}

\subsection{Bilinear forms from lattices}

A common source of symmetric bilinear forms over the integers arise from lattices in an inner product space such as Euclidean or Lorentzian space. If $(V,\langle -,-\rangle)$ is an inner product space, a lattice $\Lambda\subset V$ inherits the bilinear form $\langle -,-\rangle$. However, this form is generally not integer valued and so isn't of interest to us. Some very specially constructed lattices however do inherit an integer valued bilinear form.

For instance, in Euclidean space $\R^\ell$ with basis $\{e_1,\ldots, e_\ell\}$ consider the lattice
\[
	\Gamma^\ell = \span\left\{\frac{1}{2}(e_1+\cdots + e_\ell), e_i+e_j \mid i<j\right\}.
\]
When $4\mid \ell$, the Euclidean inner product restricts to an integral bilinear form on this lattice since
\[
	\begin{aligned}
		\left\langle \frac{1}{2}(e_1+\cdots+e_\ell), \frac{1}{2}(e_1+\cdots+e_\ell)\right\rangle & = \frac{\ell}{4},                                      \\
		\left\langle \frac{1}{2}(e_1+\cdots+e_\ell), e_i+e_j\right\rangle                        & = 1,                                                   \\
		\langle e_i+e_j, e_p +e_q\rangle                                                         & = \delta_{i,p}+\delta_{i,q}+\delta_{j,p}+\delta_{j,q},
	\end{aligned}
\]
where $\delta$ denotes the Kronecker delta. Since the Euclidean inner product is positive-definite, so is the bilinear form of this lattice. When $8\mid\ell$, the bilinear form of this lattice is even, otherwise it is odd. A special thing happens when $\ell=8$; in this case the lattice becomes unimodular. Geometrically, this means that the fundamental parallelipiped of $\Gamma^8$ has volume $1$. This is rare among lattices, and $\Gamma^8$ is usually called the ${E_8}$ lattice due to its connection to the root system of the exceptional Lie group $\E_8$.

\begin{definition}\label{def:E8-lattice}
	The \defn{${E_8}$ lattice} or \defn{${E_8}$ form} is given by the matrix
	\todo{matrix form}
	% \[/
	% 	E_8=
	% 	\begin{pmatrix}
	% 		2 & 1 &   &   &   &   &   &   \\
	% 		1 & 2 & 1 &   &   &   &   &   \\
	% 		  & 1 & 2 & 1 &   &   &   &   \\
	% 		  &   & 1 & 2 & 1 &   &   &   \\
	% 		  &   &   & 1 & 2 & 1 & 0 & 1 \\
	% 		  &   &   &   & 1 & 2 & 1 & 0 \\
	% 		  &   &   &   & 0 & 1 & 2 & 0 \\
	% 		  &   &   &   & 1 & 0 & 0 & 2 \\
	% 	\end{pmatrix}.
	% \]
\end{definition}

\begin{theorem}[Mordell]
	The lattice $E_8=\Gamma^8$ is the only even unimodular positive-definite lattice of rank $8$.
\end{theorem}
\begin{proof}
	\todo{cite}
\end{proof}

In fact, $E_8$ is the smallest even unimodular positive-definite lattice.
\begin{theorem}
	The signature of an even unimodular bilinear form is divisible by $8$.
\end{theorem}
\begin{proof}
	\todo{introduce characteristic elements}
\end{proof}

The attractive properties of the $E_8$ lattice make it a great source of inspiration for topological constructions, however we'd like to emphasize that it does show up naturally in ``the wild''. For instance:

\begin{example}
	In algebraic geometry, the \defn{Kummer K3 surface} defined by the homogeneous polynomial
	\[
		\textrm{K3} = \left\{ [z_0 : z_1 : z_2 : z_3]\in \CP^3 \mid z_0^4 + z_1^4+z_2^4 + z_3^4=0\right\}
	\]
	has the intersection form $Q_{\textrm{K3}} = E_8\oplus E_8 \oplus^3 H$ of signature 16 and rank 22.
\end{example}

While the classification of definite forms is far more complicated, we already have all of the tools to form a complete list of indefinite forms. First, we can prove that:

\begin{theorem}\label{thm:indefinite-bilinear-forms-isomorphic}
	Two unimodular indefinite bilinear forms are isomorphic if they have the same rank, parity, and signature.
\end{theorem}
\begin{proof}
\end{proof}

We are now equipped to list all of the unimodular indefinite forms.
\begin{proposition}
	Every unimodular indefinite form is isomorphic to either
	\[
		\oplus^p (1)\oplus^q (-1)
		\quad\textrm{or}\quad
		\pm \oplus^r E_8 \oplus^{s>0} H.
	\]
	for some integers $p,q,r,s$. The left side represents all possible odd unimodualr indefinite forms and the right side represents all possible even unimodular indefinite forms.
\end{proposition}
\begin{proof}
	The form $\oplus^p (1)\oplus^q(-1)$ has rank $p+q$ and signature $p-q$, so any rank and signature can be achieved which by \cref{thm:indefinite-bilinear-forms-isomorphic} implies that every odd unimodular \todo{finish}
\end{proof}

\todo{finish}


\medskip
\todo{geometric interpretation of lattices,
	\href{https://math.stackexchange.com/questions/2944104/obtaining-two-holed-torus-as-a-quotient-of-bbb-c}{maybe this article?}}
