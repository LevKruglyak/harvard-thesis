\chapter{Tools of Geometric Topology}\label{chap:fundamentals}

% \begin{epigraph}{21em}{Pythagoras}
% 	There is geometry in the humming of the strings,\\
% 	and there is music in the spacing of the spheres
% \end{epigraph}
%
%
% \begin{epigraph}{15em}{Frank Herbert}
% 	Beginnings are such delicate times.
% \end{epigraph}

\begin{epigraph}{20em}{Sir Michael Atiyah}
	Algebra is the offer made by the devil to the \\
	mathematician. The devil says: ``I will give \\
	you this powerful machine, it will answer any \\
	question you like. All you need to do is give\\
	me your soul: give up geometry.
\end{epigraph}

\noindent
To begin our journey, we will provide some geometric topology essentials which will be used ubiquitously throughout the thesis.

First, in \cref{sec:geometric-topology} we will discuss operations on smooth manifolds -- how to glue manifolds together, cut off handles, attach handles, stitch together manifolds at multiple points, and so on. The two operations most of interest to us are the \defn{connected sum}, which glues together two manifolds of the same dimension, and \defn{surgery}, which adds or removes handles. 
Armed with a powerful set of topological cutlery, we begin to investigate how smooth manifold operations change the topology of the manifold. 

We then discuss how to translate between the algebra of homology and cohomology and the geometry of manifolds and embeddings.

\todo{what other sections we need}

\pagebreak
\section{Smooth Manifolds Operations}\label{sec:geometric-topology}

Cutting and pasting is the category of smooth manifolds is a subtle craft. 
Unlike in the topological manifold category, combining manifolds resembles ``sewing'' more than ``gluing''. In order to ensure that the resulting manifold will have a unique smooth structure, we must identify the spaces in a small region surrounding the submanifold. The extra space provided by the region allows one manifold to ``smoothly curve into'' the other.

\begin{remark} 
	We will see in \cref{sec:twisted-spheres} that exotic spheres can be formed by identifying two disks along \emph{just} their boundary by an orientation preserving diffeomorphism. This always gives an ordinary sphere in the topological category but not in the smooth category.
\end{remark}

We begin with some technical lemmas from differential topology. Whenever a manifold $N^k$ is embedded into an ambient manifold $M^n$, there is a short exact sequence of vector bundles on $N$
\begin{equation}
	0 \lkxto \TT N\lkxto \TT M|_N \lkxto \TT M/N \lkxto 0.
\end{equation}
Here $\TT M/N=\TT M|_N / \TT N$ denotes the \defn{normal bundle} of $N$ in $M$, which has rank $(n-k)$.

A \defn{tubular neighborhood} of $N$ is a neighborhood $\widetilde{N}\supset N$ in $M$ which is the diffeomorphic image of the total space of the normal bundle $\T M/N$ with the diffeomorphism $\tau : \T M/N \to \widetilde{N}$ satisfying:
\begin{enumerate}[(a)]
	\item $\tau$ is the identity on the image of the zero section $s_0 : N \to \T M/N$,
	\item For any point $p\in \partial N$ on the boundary, $\tau(\T_p M/N)\subset \partial M$.
\end{enumerate}


A tubular neighborhood should be thought of as a way of ``thickening'' the submanifold $N$ to have the same dimension as $M$, as depicted in \cref{fig:tubular-neighborhood}. 

\begin{figure}[ht]
	\centering
	\import{diagrams}{placeholder.pdf_tex}
	\caption{Tubular neighborhoods of some submanifolds.}\label{fig:tubular-neighborhood}
\end{figure}

The diffeomorphism $\tau : \T M/N \to \widetilde{N}$ can interpreted as a chart of a ``tube'' surrounding $N$ in $M$. Namely, for each point $p\in N$, local trivializations of the normal bundle $\T M/N$ are maps $\R^k \times \R^{n-k}\to U\subset \widetilde{N}$ which consist of coordinates $(x^1,\ldots,x^k)$ on $N$ and coordinates $(v^{k+1},\ldots, v^{n})$ on an orthogonal plane to $N$.

\begin{remark}
	A common convention is to define a tubular neighborhood as the diffeomorphic image of the associated disk bundle $\D(\T M/N)$. This convention makes explicit the ``tube'' of a  ``tubular neighborhood''. We will refer to such neighborhoods as \defn{closed tubular neighborhoods}[closed tubular neighborhood], since they are the closure of a tubular neighborhood.
\end{remark}

This leads to a basic technical lemma of differential topology.

\begin{theorem}[Tubular Neighborhood Theorem]\label{thm:tubular-neighborhood}
	Every embedded submanifold has a tubular neighborhood.\footnote{This result does not require $M$ to be compact.}
\end{theorem}
\begin{proof}
	See Chapter II of \cite{kosinski1993differential} or Theorem 6.24 of \cite{lee2013smooth} for the case $M=\R^n$.
\end{proof}

\begin{warning}
	In Conventions, we assumed all smooth manifolds to be compact and all submanifolds to be properly embedded and neat. Absent these assumptions, the tubular neighborhood theorem as stated is not true.
\end{warning}

In a similar vein, when we have just a manifold $M^n$, a \defn{collar neighborhood} of its boundary is a neighborhood $\widetilde{\partial M}\supset \partial M$ which is the diffeomorphic image of a trivial ``ray bundle'' $\partial M\times [0,\infty)$. Just as for tubular neighborhoods, we require that the diffeomorphism is the identity on the zero-section $\partial M\times\{0\}$.

\begin{theorem}[Collar Neighborhood Theorem]\label{thm:collar-neighborhood}
	Every manifold with non-empty boundary has a collar neighborhood.
\end{theorem}
\begin{proof}
	See Theorem 9.25 of \cite{lee2013smooth}.
\end{proof}

\begin{proposition}
	If $N\subset M$ is a submanifold, there is a collar neighborhood of $\partial M$ which restricts to a collar neighborhood of $\partial N$ in $N$.
\end{proposition}
\begin{proof}
	See Proposition~2.8.2 in Chapter II of \cite{kosinski1993differential}.
\end{proof}

\begin{figure}[ht]
	\centering
	\import{diagrams}{placeholder.pdf_tex}
	\caption{Collar neighborhoods.}\label{fig:collar-neighborhoods}
\end{figure}

\subsection{Joining Two Manifolds Along a Submanifolds}

We can now address one of the fundamental operations in geometric topology -- the joining of two manifolds by submanifolds.

Let us suppose we have manifolds $M_1^n$ and $M_2^n$ of the same dimension, with embeddings $\iota_1 : N\to M_1$ and $\iota_2 : N\to M_2$ of a manifold $N^k$. Our goal is to construct a joined manifold $M$ which we denote $M_1\cup_N M_2$.
We start by picking tubular neighborhoods $\widetilde{N_i}$ for $\iota_i(N)\subset M_i$ and let $\tau_i: N\times \R^{n-k} \to \widetilde{N_i}$ be the diffeomorphism.
We define $M$ to be the quotient
\begin{equation}\label{eq:join-definition}
	M_1\cup_N M_2 = \frac{M_1\setminus \iota_1(N)\sqcup M_2\setminus \iota_2(N)}{\tau_1(p, t\xi) \sim \tau_2(p, t^{-1}\xi)}
\end{equation}
for all $p\in N$, $t\in(0,\infty)$, and unit vectors $\xi\in S^{n-k-1}\subset \R^{n-k}$. The resulting smooth manifold $M=M_1\cup_N M_2$ is called a \defn{join along a submanifold}\footnote{Some authors refer to this operation as ``\defn{pasting}'' or as the ``\defn{generalized connected sum}'', see for instance Section VI.4 in \cite{kosinski1993differential}.} and is one of the fundamental operations of geometric topology.

\begin{figure}[ht]
	\centering
	\import{diagrams}{placeholder.pdf_tex}
	\caption{A join of two manifolds along a submanifold.}\label{fig:join-along-submanifold}
\end{figure}

\begin{remark}
	As we have remarked, the ``sewing'' procedure in \cref{eq:join-definition} by tubular neighborhoods is vital in the category of smooth manifolds. In the category of topological manifolds\footnote{Assuming, of course, that there is a topological notion of  tubular neighborhoods.}, we could define the join by removing the tubular neighborhoods entirely (rather the just the zero sections) and identifying the resulting boundaries.
	At each point of the shared boundary, the join would be locally Euclidean since it has a neighborhood which the gluing of two half-disks to get a full disk.
\end{remark}

\begin{proposition}\label{prop:join-along-submanifolds-well-defined}
	The join $M$ of two manifolds along a submanifold has a smooth structure that agrees with the smooth structures of $M_1\setminus \iota_1(N)$ and $M_2\setminus \iota_2(N)$ in $M$. 
\end{proposition}
\begin{proof}
	There is a technical but simple proof which involves writing out the transition functions for charts on $M$ arising from charts on $M_1$ and $M_2$ which did not intersect $\iota_1(N)$ or $\iota_2(N)$. The result then follows from the smoothness of $t \mapsto t^{-1}$ on $t\in (0,\infty)$.
\end{proof}

\begin{proposition}\label{prop:join-along-submanifolds-orientation}
	If $M_1$, $M_2$, and $N$ are oriented, with $\iota_1$ orientation-preserving and $\iota_2$ orientation-reversing, then $M$ has an orientation compatible with $M_1\setminus \iota_1(N)$ and $M_2\setminus \iota_2(N)$.
\end{proposition}
\begin{proof}
	This follows since $t\mapsto t^{-1}$ and $\iota_2$ both reverse orientations. We omit the technical details for brevity.
\end{proof}

\begin{theorem}
	Up to orientation-preserving diffeomorphism, the join of manifolds along a submanifold is independent of the choices of tubular neighborhood.
\end{theorem}
\begin{proof}
	\todo{cite}
\end{proof}

\subsection{Connected Sum}\label{sec:connected-sum}

The simplest submanifold by which to join two manifolds is a point.

\begin{definition}
	The \defn{connected sum} $M_1\+M_2$ of two manifolds $M_1$ and $M_2$ is their join along an embedded point.
\end{definition}

Visually, this operation can be thought of as cutting disks out of both manifolds and connecting them with a tube, as depicted in \cref{fig:connected-sum}.

\begin{remark}
	Note that by our assumption of submanifolds preserving boundary, the point along which the manifolds are joined cannot lie on the boundary of either manifold.
\end{remark}

\begin{figure}[ht]
	\centering
	\import{diagrams}{placeholder.pdf_tex}
	\caption{A connected sum of two surfaces.}\label{fig:connected-sum}
\end{figure}

This operation is (up to orientation-preserving orientation) independent of the choice of embedding, justifying the basepoint-free notation of $M_1\+ M_2$.\footnote{As stated in Conventions, we assume that all manifolds are connected. Otherwise the connected sum would only be well-defined assuming a choice of connected component of $M_1$ and $M_2$.}
Proving the independence of connected sum from the choice of basepoint is non-trivial, and follows from a technical result by Richard Palais.

\begin{theorem}[Disk Theorem]\label{thm:disk}
	If $M^n$ is an oriented manifold and $\iota_1, \iota_2 : D^n \to M$ are orientation-preserving disk embeddings, then there is a diffeomorphism $f : M \to M$ such that $\iota_1 = f\circ \iota_2$.
\end{theorem}
\begin{proof}
	See Theorem B in \cite{palais1960diffeomorphism}.
\end{proof}

\begin{corollary}\label{cor:connected-sum-operation}
		The connected sum is well-defined, associative, and commutative up to orientation-preserving diffeomorphism.
\end{corollary}

\begin{corollary}
	For any manifold $M^n$, there is a diffeomorphism $M\+ S^n\cong M$.
\end{corollary}

\begin{corollary}
	$\partial(M_1\+M_2) = \partial M_1\sqcup \partial M_2$.
\end{corollary}

Let us now briefly discuss the effect of connected sum on homology. If we have two oriented manifolds $M_1$ and $M_2$ of dimension $n>1$, their connected sum $M_1\# M_2$ can be decomposed as a union of open sets diffeomorphic to $M_1\setminus\{p\}$ and $M_2\setminus\{p\}$. We denote these open sets $M_1^\circ$ and $M_2^\circ$ respectively. Their intersection is diffeomorphic to a tubular neighborhood of $S^{n-1}$, so by the Mayer-Vietoris sequence, we have an exact sequence
\[
	\H_k(S^{n-1})\lkxto \H_k(M_1^\circ)\oplus \H_k(M_2^\circ)\lkxto[p_k] \H_k(M_1\+M_2)\lkxto \H_{k-1}(S^{n-1})
\]
In lowest dimension $k=0$, we know that $\H_0(M_1\+ M_2)\cong \Z$ since the connected sum is connected. Next, when $k=1$, the edge term $\H_{k-1}(S^{n-1})$ is non-trivial, but the kernel is trivial by a rank argument in the $k=0$ segment and so $p_1$ is also an isomorphism.
When $k$ is not $n-1$, the boundary terms vanish completely and so $p_k$ is also an isomorphism. In the remaining cases, we get the exact sequence
\begin{equation}\label{eq:connected-sum-cohomology}
	\begin{tikzcd}
	0 & {\H_n(M_1^\circ)\oplus \H_n(M_2^\circ)} & {\H_n(M_1\+M_2)} & {\H_{n-1}(S^{n-1})} \\
	0 & {\H_{n-1}(M_1\+M_2)} & {\H_{n-1}(M_1^\circ)\oplus \H_{n-1}(M_2^\circ)}
	\arrow[from=1-1, to=1-2]
	\arrow["p_n", from=1-2, to=1-3]
	\arrow["q", from=1-3, to=1-4]
	\arrow[from=1-4, to=2-3]
	\arrow[from=2-2, to=2-1]
	\arrow["p_{n-1}"',from=2-3, to=2-2]
\end{tikzcd}
\end{equation}
If $M_1$ and $M_2$ are both closed, so is $M_1\+M_2$, and hence $q$ is an isomorphism by a fundamental class argument. This implies that $p_{n-1}$ is an isomorphism, so we conclude:

\begin{proposition}\label{prop:homology-connected-sum-closed}
	If $n>1$ and $M_1$ and $M_2$ are closed oriented manifolds, we have
	\[
		\H_k(M_1\+M_2) \cong \begin{cases}
			\H_k(M_1^\circ)\oplus \H_k(M_2^\circ) & 0 < k < n,\\
			\Z & k=0\textrm{ or }n,\\
			0 & \textrm{otherwise.}
		\end{cases}
	\]
\end{proposition}

\begin{example}
	The compact surface $X_g$ of genus $g$ can be broken down as a $g$-repeated connected sum of the torus $T^2=S^1\times S^1$, i.e. $X_g \cong \+[g] T^2$. It follows that by \cref{prop:homology-connected-sum-closed} that the Betti numbers are $\beta_0=1$, $\beta_1=2g$, and $\beta_2=1$.
\end{example}

The preceding is an example of the trivial case of a join of two manifolds. In general, the situation is not as nice. First of all, the submanifold $N$ might not have a trivial normal sphere bundle so computing the homology of the intersection of $M_1^\circ$ and $M_2^\circ$ becomes more complex. Additionally, in many applications the manifolds involved will not be closed, the homology in dimensions $n-1$ and $n$ behaves differently than in \cref{prop:homology-connected-sum-closed}.

\subsection{Attaching a Handle}

The next important special case of the join operation is the attaching of a handle.

\begin{definition}

\end{definition}

\subsection{Removing a Handle}

\begin{figure}[ht]
	\import{diagrams}{surgery-on-two-holed-torus.pdf_tex}
	\caption{Reducing the genus of a surface by spherical modification}
\end{figure}

\subsection{Joining Manifolds Along Submanifolds of the Boundary}

\pagebreak
\section{Homology Classes and Submanifolds}\label{sec:representing-homology-classes}

Let $f : N^k\to M^n$ be a smooth map from a closed oriented manifold $N^k$ into some manifold $M^k$. An orientation on $N$ determines a fundamental homology class $[N]\in \H_k(N)$ which can be pushed forward along the map $f : w \to M$ to give a homology class $f_* [N]\in \H_k(M)$. 

The correspondence behaves nicely with respect to homotopic perturbations.
Suppose $H : N\times [0,1] \to M$ is a homotopy with $H(x,0)=f(x)$ which perturbs the map $f$. For any $\epsilon>0$, the map $f_\epsilon : N \to M$ given by $f_\epsilon(x)=H(x,\epsilon)$ is homotopic to $f$ and hence induces the same map on homology $\H_k(N)\to \H_k(M)$. The homology class associated to a map $f : N \to M$ thus solely depends on the homotopy type of $f$ so there is a map
\begin{equation}\label{eq:homotopy-class-to-homology-class}
	\lkxfunc{}{[N,M]}{\H_k(M).}
\end{equation}
Letting $N=S^k$, we see that this is a generalization of the Hurewicz homomorphism which links homotopy groups to homology groups via a map $\pi_k(M) \to \H_k(M)$.
As with the Hurewicz homomorphism, the correspondence in \cref{eq:homotopy-class-to-homology-class} is generally not surjective or injective. 

In the case of the Hurewicz homomorphism, if $M$ is $(k-1)$-connected for $k > 1$ the Hurewicz homomorphism is in fact an isomorphism $\pi_{k}(M) \cong \H_{k}(M)$. Consequently, every homology cycle in $\H_{k}(M)$ can at least be represented by an smooth map of a sphere $S^{k}$ into $X$. However, this smooth map need not be an immersion, and even still might have unavoidable ``double-points'' -- multiple points of the sphere mapping to the same point in the image and preventing the smooth map from being an embedding.

\begin{example}
	In the punctured plane $M=\R^2\setminus \{0\}$ with homology $\H_1(M)\cong \Z$, the only homology cycles which can be represented by embedded submanifolds are $0,\pm 1$, by a circle not containing the origin and circles of both orientations surrounding the origin respectively. A smooth map representing a cycle of higher degree would necessarily have a double-point as in \cref{fig:double-point}.
\begin{figure}[ht]
	\centering
	\import{diagrams}{double-point.pdf_tex}
	\caption{A double-point in a smooth map representing $-2\in \H_1(\R^2\setminus\{0\})$.}\label{fig:double-point}
\end{figure}
\end{example}

That being said, many of the manifolds considered \todo{basis by embedded submanifolds}.
many specially constructed manifolds we will consider in this chapter will at least have a basis by embedded submanifolds, and every homology class will admit a representation by a smooth map. For a classical account of some issues which can arise when representing homology classes by smooth maps, see Chapter II of Ren\'e Thom's seminal paper \cite{thom1954}.

\begin{remark}
	In some cases, homology classes can \emph{always} be represented by embedded submanifolds. For instance, if $M$ is a $4$-manifold, there are isomorphisms
	\begin{equation}
		\H^2(M; \Z) \cong [M, K(\Z,2)] \cong [M, \CP^\infty] \cong [M,\CP^2],
	\end{equation}
	where the first is the representability of singular cohomology by the Eilenberg-Maclane spectrum, the second identifies $\CP^\infty$ as a $K(\Z,2)$ space, and the third uses the cellular approximation theorem to slide maps onto the $4$-skeleton. Any cohomology cycle $\omega\in \H^2(M)$ can be represented by a smooth function $f : M \to \CP^2$. If we choose this function to be transverse to $\CP^1\subset \CP^2$, then $f^{-1}(\CP^1)$ is an embedded $2$-dimensional submanifold of $M$ which corresponds to a Poincar\'e dual class to $\omega$. When $X$ is a compact manifold, Poincar\'e duality tells us that all $2$-dimensional homology cycles arise from $2$-dimensional cohomology cycles and can thus be represented by embedded submanifolds. This is one example of the attractiveness of $4$-manifolds as geometric objects of study.
\end{remark}

\todo{give general conditions for representability}

\subsection{Intersections in Differential Topology}\label{sec:differential-topology-intersections}

\begin{theorem}[Preimage Theorem]\label{thm:preimage}
	If $f : N \to X$ is a smooth map transverse to a submanifold $M\subset X$ then $S=f^{-1}(M)\subset N$ is a submanifold with the same codimension in $N$ as $M$ in $X$.
\end{theorem}
\begin{proof}
	See the proof of Theorem~6.30 in \cite{lee2013smooth}.
\end{proof}

\begin{remark}\label{rmk:symmetric-preimage-theorem}
	We can get a symmetric version of this theorem as a straightforward corollary. If we have two transverse maps $f : N\to X$ and $g : M\to X$, then the map $f\times g : N\times M \to X\times X$ is transverse to the diagonal submanifold $\Delta\subset X\times X$. \cref{thm:preimage} will then imply that
	\[
		(f\times g)^{-1}(\Delta) \subset M\times N
	\]
	is a submanifold. When $g$ is an embedding, $(f\times g)^{-1}(\Delta)$ can be projected down onto $M$ to get the preimage $f^{-1}(M)$.
\end{remark}

If the manifolds involved in \cref{thm:preimage} are orientable, this preimage $S$ admits a canonical orientation by the following procedure. First of all, recall that for any embedded manifold $M\subset X$ there is an exact sequence of vector bundles by quotienting
\begin{equation}\label{eq:oriented-intersection-number-1}
	\begin{tikzcd}
		0 & {\T M} & {\T X} & {\T X/M} & 0
		\arrow[from=1-1, to=1-2]
		\arrow[from=1-2, to=1-3]
		\arrow[from=1-3, to=1-4]
		\arrow[from=1-4, to=1-5]
	\end{tikzcd}
\end{equation}
where $\T X/M$ is the normal bundle of $M\subset X$. Using the orientations of $X$ and $M$, we can use this exact sequence to get an orientation of the normal bundle $\T X/M$. At every point $p\in S$ of the preimage, the differential map $df$ connects the sequence \cref{eq:oriented-intersection-number-1} to the normal bundle sequence for the embedding $S\subset N$.
\begin{equation}\label{eq:oriented-intersection-number-2}
	\begin{tikzcd}
		0 & {\T_pS} & {\T_p N} & {\T_p N/S} & 0 \\
		0 & {\T_{f(p)}M} & {\T_{f(p)}X} & {\T_{f(p)}X/M} & 0
		\arrow[from=1-1, to=1-2]
		\arrow[from=1-2, to=1-3]
		\arrow["{df_p}", from=1-2, to=2-2]
		\arrow[from=1-3, to=1-4]
		\arrow["{df_p}", from=1-3, to=2-3]
		\arrow[from=1-4, to=1-5]
		\arrow["{df_p}", from=1-4, to=2-4]
		\arrow[from=2-1, to=2-2]
		\arrow[from=2-2, to=2-3]
		\arrow[from=2-3, to=2-4]
		\arrow[from=2-4, to=2-5]
	\end{tikzcd}
\end{equation}
In this diagram \cref{eq:oriented-intersection-number-2}, the rightmost vertical map is an isomorphism by the transversality of $f$ and $M$. This means that we can pullback the orientation on $\T_{f(p)} X/M$ to $\T_p N/S$. Since $\T_p N$ is oriented, the usual ``2-out-of-3'' rule applied to the top row of \cref{eq:oriented-intersection-number-2} gives an orientation of $\T_p S$. See \cref{fig:preimage-orientation} for an example of this orienting procedure.

\begin{figure}[ht]
	\centering
	\import{diagrams}{preimage-orientation.pdf_tex}
	\medskip
	\caption{Orienting a preimage (assuming a clockwise orientation on $X$ and $N$).}\label{fig:preimage-orientation}
\end{figure}

When $M$ and $N$ have complementary dimensions, the preimage $S=f^{-1}(N)\subset M$ is a compact oriented $0$-dimensional manifold. For each point $p\in S$, we have $\T_p S=0$ so the map $\T_p N\to \T_p N/S$ in \cref{eq:oriented-intersection-number-2} is an isomorphism. The orientation of $N$ gives an orientation of $\T_p N$, and the preimage orientation procedure gives us an orientation of $\T_p N/S$. Now we can define:

\begin{definition}
	The \defn{local (oriented) intersection number}[oriented intersection number (local)] of $f$ and $M$ at $p\in S$ is
	\[
		\#_p^X(f, M) = \begin{cases}
			+1 & \T_p N/S \textrm{ has the same orientation as } \T_p N,     \\
			-1 & \T_p N/S \textrm{ has the opposite orientation to } \T_p N.
		\end{cases}
	\]
\end{definition}
Summing over all of the local intersection numbers gives a global quantity.
\begin{definition}
	The \defn{(oriented) intersection number}[oriented intersection number] of a smooth map $f : N \to X$ intersecting a submanifold $M\subset X$ transversally is
	\[
		\#^X(f, M) = \sum_{p\in S} \#_p^X(f, M) \in \Z.
	\]
\end{definition}

\begin{remark}\label{rmk:symmetric-intersection-number}
	For a more symmetric version of this definition when two smooth maps $f : N \to X$ and $g : M \to X$ intersect transversally, we could take inspiration from \cref{rmk:symmetric-preimage-theorem} and define the oriented intersection number of the smooth maps $f$ and $g$ as
	\[
		\#^X(f,g) = \#^{X\times X}(f\times g, \Delta).
	\]
	This symmetric intersection number is graded commutative in the dimensions of $M$ and $N$, i.e.
	\begin{equation}\label{eq:intersection-number-graded-commutative}
		\#^X(f,g) = (-1)^{\dim M\cdot \dim N} \#^X(g,f)
	\end{equation}
\end{remark}

Just as the property of transversality is stable -- resilient to homotopic perturbations -- so too is the oriented intersection number. This follows as a corollary to a more general theorem.

\begin{theorem}
	If $W$ is a compact oriented manifold with boundary, and $H : W \to X$ is a smooth map, then $\#^X(\partial H, M)=0$. Here, we use the notation $\partial H : \partial W \to X$ to refer to the restriction of $H$ to the boundary of $W$.
\end{theorem}

\begin{corollary}
	If $H : [0,1]\times N \to X$ is a smooth homotopy, then $\#^X(H_0, M) = \#^X(H_1, M)$.
\end{corollary}

Applying the construction of the symmetric oriented intersection number gives a map
\begin{equation}\label{eq:oriented-intersection-number-homotopy}
	\lkxfunc{\#^X}{[N,X]\times [M,X]}{\Z}{f,g}{\#^X(f,g)}
\end{equation}
This is a geometric precursor to the intersection form of a manifold, a central object of study in geometric topology.

\begin{remark}
	If we don't assume orientations, we can still get a homotopy invariant intersection number, however we must reduce mod $2$. In this case, we could simply define
	\[
		\#_2^X(f,M) = |S|\mod 2.
	\]
	This is called the \defn{unoriented intersection number}.
	\todo{elaborate}
\end{remark}

Finally, we'll state a useful result in computer self-intersection numbers -- a way to compute the intersection number of a submanifold with itself.

\begin{theorem}\label{thm:euler-number-self-intersection}
	If $M$ is a closed $m$-dimensional submanifold of a $2m$-dimensional submanifold $X$ then we have
	\[
		\#^X(M, M) = \chi(\T X/M)
	\]
	where $\chi(\T X/M)$ is the Euler number of the normal bundle of $M$.
\end{theorem}
\begin{proof}
	\todo{todo}
\end{proof}

\begin{corollary}\label{thm:euler-number-self-intersection-corollary}
	The Euler number $\chi(\xi)$ of an oriented real vector bundle $\xi : E \to B$ over a compact oriented manifold can be expressed as the intersection number
	\[
		\#^E(z,z) = \chi(\xi)
	\]
	where $z : B \to E$ is the zero section.
\end{corollary}

\todo{add 1.C from \cite{levine1985lectures}}

\pagebreak
\section{The Intersection Form}\label{sec:intersection-form}

One of the fundamental invariants for even-dimensional manifolds is the intersection form, a bilinear form on a lattice which captures the geometric data of submanifold intersections. The lattice in question is the free component of the middle-dimensional singular homology, and the pairing of two homology cycles by the form counts their number of ``intersections''. When the homology cycles of complementary dimension (as in the case of middle-dimensional homology cycles) are represented by smooth immersions, we can perturb them to make them transverse without changing the homology class. If the manifolds are compact, the preimage of their intersection is some finte set of points with orientation -- adding up the orientations of the points gives the oriented intersection number.

This is the geometric interpretation of intersection, and we will explore it in more depth when we construct manifolds with given intersection theory in \cref{sec:plumbing}. For now, we will stick to understanding the algebraic properties of intersections as the adage ``think in terms of intersections, prove in terms of homology'' advises. To start, let us suppose that $M$ is a compact oriented $n$-manifold.
The Poincar\'e-Lefschetz duality gives an isomorphism
\begin{equation}
	\lkxfunc{}{\H^{n-p}(M,\partial M)}{\H_p(M)}{\omega}{\omega\frown [M,\partial M]}
\end{equation}
assuming we have an orientation class $[M,\partial M]\in \H_n(M, \partial M)$ corresponding to the orientation of $M$. Under this duality, the intersection of homology classes is defined as the dual operation to the operation of cup product on cohomology classes. This operation is denoted $\alpha\cdot \beta$ for homology cycles $\alpha\in \H_p(M)$ and $\beta\in \H_q(M)$, and is the top map in \cref{eq:homology-intersection}
\begin{equation}\label{eq:homology-intersection}
	\begin{tikzcd}
		{\H_p(M)\otimes \H_q(M)} & {\H_{n-p-q}(M)} \\
		{\H^{n-p}(M, \partial M)\otimes \H^{n-q}(M,\partial M)} & {\H^{2n-p-q}(M,\partial M)}
		\arrow["\tnsv", from=1-1, to=1-2]
		\arrow[leftrightarrow, from=1-1, to=2-1]
		\arrow[leftrightarrow, from=1-2, to=2-2]
		\arrow["\smile", from=2-1, to=2-2]
	\end{tikzcd}
\end{equation}
where the vertical maps are the Poincar\'e-Lefschetz isomorphism. Again, the intuition here should be that $\alpha\cdot \beta$ is the homology class representing the intersection of $\alpha$ and $\beta$ when they are arranged in general ``transverse'' position. Done over homology classes of complementary dimension, the resulting homology intersection class is 0-dimensional and hence pairs with an integer multiple $\ell\cdot [M, \partial M]\in \H_n(M,\partial M)$ of the top-dimensional orientation class. The integer multiple $\ell\in \Z$ is the \defn{intersection number}[intersection number of homology classes] of the cycles $\alpha$ and $\beta$. Removing torsion elements and working in middle dimensional homology so that $\alpha$ and $\beta$ live in the same group, we get an integral bilinear form.

\begin{definition}
	Let $M^{2m}$ be a compact oriented even-dimensional manifold, possibly with boundary. The \defn{intersection form} on middle dimensional homology is the bilinear form
	\begin{equation}
		\lkxfunc{Q_M}{\H_m(M)_{\mathrm{free}}\otimes \H_m(M)_{\mathrm{free}}}{\Z}{a\otimes b}{a \tnsv b}
	\end{equation}
	where we identify $\H_0(M)\cong \Z$ and $\H_m(M)_{\mathrm{free}}$ denotes the free component of $\H_m(M)$ -- i.e. the quotient by the subgroup of torsion elements.
\end{definition}

\begin{remark}
	If $m$ is even then $Q_X$ is a symmetric bilinear form and if $m$ is odd then $Q_X$ is a skew-symmetric bilinear form. This follows from the graded commutativity of the cup product, and hence the intersection pairing. For brevity, we say that $Q_X$ is \defn{$(-1)^m$-symmetric} in such cases.
\end{remark}

\begin{remark}
	Note that the intersection form is defined entirely topologically, without a requirement of smooth structure. That being said, the existence of a smooth structure on a manifold can lead to noticeable effects on the smooth structure.
\end{remark}

It is often helpful to work with the dual pairing, i.e. the cup product pairing on cohomology, since it can be immediately deduced from the multiplicative structure of the cohomology ring.
\begin{definition}
	The intersection form on cohomology is the bilinear form
	\begin{equation}
		\lkxfunc{Q^M}{\H^m(M,\partial M)_{\mathrm{free}}\otimes \H^m(M,\partial M)_{\mathrm{free}}}{\Z}{\alpha\otimes \beta}{\alpha\smile \beta}
	\end{equation}
	where we identify $\H^n(M,\partial M)\cong \H_0(M)\cong \Z$.
\end{definition}

\begin{remark}
	For manifolds which do not come with an orientation, the intersection form can be extended in homology/cohomology with $\Z/2$ coefficients. We call this form the \defn{unoriented intersection form}, and denote it by $\widetilde{Q}_M$ or $\widetilde{Q}^M$ depending on if we are working with homology or cohomology. In the context of embedded submanifolds, this form captures the number of transverse intersection points modulo 2, otherwise known as the unoriented intersection number.
\end{remark}

\begin{remark} \label{rmk:dual-lattice-intersection-form}
	To see the connection between the intersection form on homology and cohomology, let us recall the universal coefficients theorem for cohomology, which gives us an exact sequence
	\[
		0 \lkxto \Ext^1_\Z(\H_{m-1}(M,\partial M)) \lkxto \H^m(M, \partial M) \lkxto \Hom(\H_m(M, \partial M), \Z) \lkxto 0.
	\]
	This is Theorem 3.2 of \cite{hatcher2002topology}. When working with the intersection form, we only care about the torsion free component of homology and cohomology. Note that the $\Ext$ term maps entirely into the torsion part of cohomology $\H^m(M,\partial M)$ since $\Ext^1(F\oplus T; \Z)\cong T$ whenever $F$ is free and $T$ is torsion. We thus get a canonical isomorphism
	\[
		\H^m(M, \partial M) \lkxisom \Hom(\H_m(M,\partial M), \Z).
	\]
	When $\H_{m}(\partial M)$ and $\H_{m-1}(\partial M)$ are trivial, the middle map in
	\[
		\H_m(\partial M) \lkxto \H_m(M) \lkxto \H_m(M,\partial M) \lkxto \H_{m-1}(\partial M)
	\]
	is an isomorphism, and so we get a canonical isomorphism
	\[
		\H^m(M, \partial M) \lkxisom \Hom(\H_m(M), \Z).
	\]
	In other words, (under suitable topological restrictions of $\partial M$) there is a canonical way to identify the lattice for the cohomology intersection form as the dual of the lattice for the homology intersection form. In particular, the matrix representations of the bilinear forms are inverses of each other.
\end{remark}

\subsection{Basic Examples of the Intersection Form}

We will see many examples of manifolds and their intersection forms throughout this thesis, so for now let's just consider the most basic examples -- complex projective spaces and tori.

\begin{proposition}\label{prop:intersection-form-complex-projective-plane}
	The intersection form for any complex projective plane $\CP^{2m}$ of even complex dimension is given by $Q=(1)$, and the intersection form for complex projective plane $\CP^{2m+1}$ of odd complex dimension is trivial.
\end{proposition}
\begin{proof}
	We will compute the intersection form in cohomology, by \cref{rmk:dual-lattice-intersection-form} we can invert the resulting matrix to get an intersection form on homology.

	Recall that the cohomology ring of complex projective space is given by
	\begin{equation}
		\H^\bullet(\CP^{n}) \approx \Z[\alpha]/(\alpha^{n+1})\quad\textrm{with}\quad |\alpha|=2.
	\end{equation}
	A proof of this can be found in any standard algebraic topology book, for instance Theorem~3.19 in \cite{hatcher2002topology}. We can assume without loss of generality that $\alpha^n\in \H^{2n}(\CP^n)$ is the fundamental class corresponding to the canonical orientation on $\CP^n$.
	Note that since the generating element has degree $2$, the middle-dimensional homology $\H^{2m+1}(\CP^{2m+1})$ is trivial and hence so is the intersection form of $\CP^{2m+1}$.

	When the complex dimension is even, the middle-dimension homology $\H^{2m}(\CP^{2m})$ is generated by $\alpha^m$. Since $\alpha^m\smile \alpha^m=\alpha^{2m}$ is a unit multiple of the fundamental class, we have $Q(\alpha^m, \alpha^m)=1$, completing the proof.
\end{proof}

\begin{figure}[ht]
	\centering
	\import{diagrams}{placeholder.pdf_tex}
	\caption{Intersections of homology classes in a complex projective plane.}\label{fig:geometric-intersection-complex-projective}
\end{figure}

It is illuminating to interpret this result geometrically. Let's begin with the complex vector space $\C^{2m+1}$ equipped with a basis $\{e_0, e_1,\ldots, e_{2m}\}$. Consider the linear subspaces
\begin{equation}
	W = \langle e_0, e_1,\ldots, e_m\rangle \quad\textrm{and}\quad U = \langle e_0, e_{m+1},\ldots, e_{2m}\rangle
\end{equation}
in $\C^{2m+1}$. These complex hyperplanes intersect at a complex line $\langle e_0 \rangle = W\cap U$.

Now, we can pass to the projectivization $\P(\C^{2m+1})=\CP^{2m}$ and realize $W$ and $U$ as embedded submanifolds $\P(W)\approx \CP^m\subset \CP^{2m}$ and $\P(U)\approx \CP^m\subset \CP^{2m}$. Since $W$ and $U$ intersect at a line, their projectivizations $\P(W)$ and $\P(U)$ intersect at a point in $\CP^{2m}$. Furthermore, the intersection is transverse, and descending the orientation on $\C^{2m+1}$ to the embedded submanifolds gives an intersection number of $1$. Both of these embedded submanifolds represent the homology class $a^{2m}\in \H_{2m}(\CP^{2m})$ which is the Poincar\'e dual of the cohomology class $\alpha^{2m}(\CP^{2m})$. We again arrive at $Q(a^{2m}, a^{2m})=1$, although this time through homology intersections.

\begin{proposition}\label{prop:intersection-form-torus}
	The intersection form for a torus $T^{2m}=S^m\times S^m$ is given by matrices
	\begin{equation}\label{eq:hyperbolic-form-torus}
		Q = \begin{pmatrix}0 & 1 \\ 1 & 0\end{pmatrix}
		\textrm{ when }m\textrm{ is even, and }
		Q = \begin{pmatrix}0 & 1 \\ -1 & 0\end{pmatrix}
		\textrm{ when }m\textrm{ is odd.}
	\end{equation}
\end{proposition}
\begin{proof}
	Let us again begin with a cohomology computation. The cohomology of a sphere is
	\begin{equation}
		\H^\bullet(S^n) = \Z[\alpha]/(\alpha^2)\quad\textrm{with}\quad |\alpha| = n,
	\end{equation}
	and by the K\"unneth formula (see Theorem~3.15 in \cite{hatcher2002topology}), we have
	\begin{equation}
		\begin{aligned}
			\H^\bullet(T^{2m})=\H^\bullet(S^m\times S^m)\cong \H^\bullet(S^m)\otimes \H^\bullet(S^m)
			 & \cong \Z[\alpha]/(\alpha^2)\otimes \Z[\beta]/(\beta^2) \\
			 & \cong \Z[\alpha,\beta]/(\alpha^2,\beta^2).
		\end{aligned}
	\end{equation}
	Assuming $\alpha$ and $\beta$ are fundamental classes for the spheres, the fundamental class of the torus is $\alpha\smile \beta$. From this multiplicative structure and fundamental class, we clearly have
	\begin{equation}
		Q(\alpha, \alpha)=0, \quad Q(\beta,\beta)=0, \quad Q(\alpha,\beta)=1,\quad Q(\beta,\alpha)=(-1)^m Q(\alpha,\beta)=(-1)^m.
	\end{equation}
	These give exactly the matrices described in \cref{eq:hyperbolic-form-torus}.
\end{proof}

\begin{figure}[ht]
	\centering
	\import{diagrams}{placeholder.pdf_tex}
	\caption{Intersections of homology classes in a torus.}\label{fig:geometric-intersection-torus} 
\end{figure}

The geometric proof of this claim is analogous. The Poincar\'e duals of $\alpha$ and $\beta$, denoted $a$ and $b$ in $\H_{m}(T^{2m})$, are represented
by the embedded submanifolds $S^m\times \{p\}$ and $\{p\}\times S^m$ for some basepoint $p\in S^m$. Shifting an individual embedded sphere to a disjoint embedding by a path taking $p\mapsto p'$ disjoint shows that the self-intersection numbers of $a$ and $b$ are zero. These are the zeroes along the diagonal of the matrices in \cref{eq:hyperbolic-form-torus}. However, the embedded spheres representing $a$ and $b$ intersect transversally at the point $(p,p)\in T^{2m}$. We choose orientations on $S^m\times \{p\}$ and $\{p\}\times S^m$ so that $a\cdot b=1$, then by graded-commutativity we get $b\cdot a=(-1)^m$.

\begin{remark}
	The symmetric form in \cref{eq:hyperbolic-form-torus} is known as the \defn{hyperbolic form}, denoted by
	\[
		H=\begin{pmatrix} 0 & 1\\ 1 & 0 \end{pmatrix}.
	\]
	The hyperbolic form is a fundamental building block for symmetric bilinear forms over the integers and $\Z/2$.
\end{remark}

\subsection{Properties of the Intersection Form}

\begin{proposition}\label{prop:connected-sum-intersection-form}
	For compact manifolds $M_1^{2m}$ and $M_2^{2m}$, we have
	$Q_{M_1\+M_2} \cong Q_{M_1}\oplus Q_{M_2}.$
\end{proposition}
\begin{proof}
\end{proof}

\begin{proposition}\label{prop:orientation-intersection-form}
	For a compact manifold $M^{2m}$, we have
	$Q_{-M} \cong -Q_{M}$.
\end{proposition}
\begin{proof}
	This is immediate, since the fundamental class changes sign as orientation flips.
\end{proof}

An immediate corollary of \cref{cor:connected-sum-operation} and \cref{prop:connected-sum-intersection-form} is
that the intersection form is a homomorphism of commutative monoids, i.e. sets with a commutative associative binary operation and identity elements. On one side, we have the monoid $\mathcal{M}^{2m}$ of oriented compact $2m$-manifolds under connected sum, and on the other side we have $\mathcal{Q}(\Z)$ of bilinear forms valued in $\Z$ under the operation of direct sum. Similarly, the unoriented intersection form maps the monoid of unoriented compact $2m$-manifolds $\widetilde{\mathcal{M}}^{2m}$ to $\mathcal{Q}(\Z)$.
\begin{equation}\label{eq:monoid-homomorphism-intersection-form}
	\lkxfunc{}{\mathcal{M}^{2m}}{\mathcal{Q}(\Z),}{M}{Q_M,}
	\quad\textrm{and}\quad
	\lkxfunc{}{\widetilde{\mathcal{M}}^{2m}}{\mathcal{Q}(\Z/2),}{M}{\widetilde{Q}_M.}
\end{equation}
The monoidal structure of the intersection form is quite useful throughout geometric topology, especially in classification problems.

\subsection{Classification of Manifolds by Intersection Form}
An illustrative case in low dimensions is the classification of compact (unoriented) surfaces up to homeomorphism. Recall that every compact surface is homeomorphic to exactly one of the following surfaces
\[
	S^2,\quad T^2\#\cdots\# T^2,\quad\textrm{or}\quad \RP^2\#\cdots\# \RP^2,
\]
i.e. it is either a sphere, a torus with some number of holes, or an unorientable surface formed by gluing together M\"obius strips. For instance, a Klein bottle is the connected sum $\RP^2\#\RP^2$.
A standard cominatorial proof of this classification by polygonal presentations can be found in Chapter 6 of \cite{lee2011topological}.

By similar arguments to \cref{prop:intersection-form-complex-projective-plane} and \cref{prop:intersection-form-torus}, the unoriented intersection forms of these generating surfaces are given by
\[
	\widetilde{Q}_{S^2}=(0),\quad \widetilde{Q}_{T^2}=\begin{pmatrix}0 & 1 \\ 1 & 0\end{pmatrix},\quad \textrm{and}\quad\widetilde{Q}_{\RP^2} = (1).
\]
For instance, the intersection form of $T^2\+ \RP^2$ is given by
\[
	\widetilde{Q}_{T^2\+ \RP^2} = H\oplus (1)=
	\begin{pmatrix}
		1 & 0 & 0 \\
		0 & 0 & 1 \\
		0 & 1 & 0
	\end{pmatrix}.
\]
This matrix represents a bilinear form, and so the transformation $Q\mapsto P^\intercal Q P$ does not affect the form. In this case, $Q^\intercal =Q$ and $Q^2=I\mod 2$, the transformation $Q\mapsto Q^\intercal Q Q$ gives the form
\[
	\widetilde{Q}_{T^2\+ \RP^2}
	\lkxto \begin{pmatrix}
		1 & 0 & 0 \\
		0 & 1 & 0 \\
		0 & 0 & 1
	\end{pmatrix} =\oplus^3(1)= \widetilde{Q}_{\RP^2\+\RP^2\+\RP^2}.
\]
As it turns out, the underlying surfaces $T^2\+\RP^2$ and $\RP^2\+\RP^2\+\RP^2$ are homoemorphic. This has an easy geometric interpretation. The operation $T^2\+$ can be thought of as adding a handle, and $\RP^2\+\RP^2\+$ being connected sum with a Klein bottle can be thought of as adding a handle in a twisted manner, i.e. one spout on one side of the surface and the other spout on the other side. Note that there might be a global notion of ``side'' if the manifold is non-orientable, but locally this picture holds.

One such non-orientable case is the projective plane $\RP^2$. If we add a torus handle to $\RP^2$ (a M\"obius band with boundary collapsed), we can move one spout around the twist of $\RP^2$ to get a twisted handle (as depicted in \cref{fig:twisted-handle-to-handle}). Thus, the surfaces $T^2\+\RP^2$ and $\RP^2\+\RP^2\+\RP^2$ are homeomorphic, a geometric fact which was detected in part by the algebraic identity of forms $H\oplus (1)=\oplus^3(1)$ in $\mathcal{Q}(\Z/2)$.

\begin{figure}[ht]
	\centering
	\import{diagrams}{placeholder.pdf_tex}
	\caption{Turning $T^2\+\RP^2$ into $\RP^2\+\RP^2\+\RP^2$.}\label{fig:twisted-handle-to-handle}
\end{figure}

\begin{proposition}
	Let $\mathcal{Q}_{\mathrm{skew}}(\Z/2)$ be the monoid of skew-symmetric bilinear forms over $\Z/2$. There is a presentation
	\[\mathcal{Q}_{\mathrm{skew}}(\Z/2) = \langle H, (1) \mid H\oplus (1) = \oplus^3 (1)\rangle.\]
\end{proposition}
\begin{proof}
	See Chapter III of \cite{milnorhuse1973forms} for a generalized statement and proof.
\end{proof}

Just like $H\oplus (1)= \oplus^3 (1)$ is the defining relation for skew-symmetric bilinear forms over $\Z/2$, so too is $T^2\+ \RP^2 = \RP^2\+\RP^2\+\RP^2$ the defining relation for closed surfaces. This leads to a clean restatement of the classification theorem for closed surfaces.

\begin{theorem}[Classification of Compact Surfaces]
	Let $\mathcal{S}^2\subset \widetilde{\mathcal{M}}^2$ be the monoid of \textit{closed} unoriented surfaces under connected sum. The unoriented intersection form is an isomorphism of monoids
	\[
		\lkxfunc{\widetilde{Q}}{\mathcal{S}^2}{\mathcal{Q}_{\mathrm{skew}}(\Z/2).}
	\]
\end{theorem}

The classification of compact surfaces by the intersection form is a model result of algebraic topology -- a complete algebraic classification of a class of manifolds. Better yet, simple algebraic manipulations correspond to non-trivial topological equivalences. This is part of why intersection forms are so useful -- algebraic intuition scales far better with dimension than does geometric intuition and so bilinear forms are a much more comfortable setting in which to study higher-dimensional topology. For instance, the classification theorem of Michael Freedman in his 1982 paper \cite{freedman1982manifold} is formulated entirely in terms of the intersection form and an additional $\Z/2$-valued invariant detecting the existence of a smooth structure.

\begin{theorem}[Freedman, 1982] Let $\mathcal{S}^4\subset \mathcal{M}^4$ be the monoid of simply-connected closed \emph{topological} $4$-manifolds. The intersection form
	\[
		\lkxfunc{Q}{\mathcal{S}^4}{\mathcal{Q}_{\mathrm{sym}}(\Z)}
	\]
	is at most two-to-one, i.e. a symmetric intersection form corresponds to at most two topological $4$-manifolds, one which admits a smooth structure and one which does not.
\end{theorem}
An accessible exposition to this remarkable theorem can be found in \cite{behrens2021discembedding}. We will explore this theorem and its consequences with more depth in \cref{sec:smoothing-obstructions}.


\subsection{Intersection Form Invariants}\label{sec:intersection-form-invariants}

While the complete algebraic data of an intersection form captures a lot of the topological structure of a manifold, it is useful to extract further simpler invariants from the intersection form.

We will work with a general integer lattice $\Lambda$ and $Q$ an integral bilinear form over $\Lambda$, not necessarily the intersection form of some manifold. Recall that under a change of basis matrix $P$, the matrix of the bilinear form transforms as $Q\mapsto P^\intercal QP$. We are therefore looking for quantities which are invariant under such transformations, helping us understand the structure of the monoid $\mathcal{Q}(\Z)$.

\begin{definition}
	The \defn{rank} of $Q$ is simply the dimension of the lattice $\Lambda$.
\end{definition}

When $Q$ is the intersection form of a manifold, its rank is the middle Betti number $\beta_m = \dim \H_m(X)_{\textrm{free}}$ of the manifold. This is the simplest invariant of a bilinear form.

\begin{definition}
	A form is said to be \defn{degenerate}[degenerate bilinear form] if $\det Q=0$ and \defn{non-degenerate}[non-degenerate bilinear form] otherwise.
\end{definition}

An equivalent dual way to view a bilinear form is by the homomorphism
\[
	\lkxfunc{Q^\d}{\Lambda}{\Hom(\Lambda, \Z)}{\alpha}{(\beta\mapsto Q(\alpha,\beta)).}
\]
In this context, a bilinear form non-degenerate if and only if $Q^\d$ is injective. We can refine this notion further.
\begin{definition}
	A bilinear form is said to be \defn{unimodular} if $\det Q=\pm 1$. 
\end{definition}
A bilinear form is unimodular if and only if $Q^\d$ is an isomorphism. The notion of unimodularity for integral bilinear forms is a special case of the notion of a \defn{perfect pairing}. A perfect pairing $V\otimes W \to R$ is a bilinear map such that the dual homomorphism $V \to \Hom(W, R)$ is an isomorphism. These notions of degeneracy and unimodularity are not a terribly useful source of invariants by the following proposition.

\begin{proposition}\label{prop:unimodular-intersection-form}
	 If $M$ is a compact manifold with $\H_m(\partial M)$ and $\H_{m-1}(\partial M)$ trivial, then the intersection form $Q_M$ is unimodular.
\end{proposition}
\begin{proof}
	By the universal theorem argument in \cref{rmk:dual-lattice-intersection-form}, we have an isomorphism
	\begin{equation}\label{eq:dual-lattice-isomorphism}
		\H^m(M,\partial M) \lkxisom \Hom(\H_m(M), \Z).
	\end{equation}
	The resulting pairing $\langle -, -\rangle : \H^m(M,\partial M)\otimes \H_m(M) \to \Z$, known as the Kronecker pairing, is therefore a \defn{perfect}[perfect bilinear form] pairing. Since the intersection form is given by
	\[
		Q_M(a,b) = \langle \mathrm{PD}(a), b\rangle,\quad a,b\in \H_m(M)
	\]
	where $\mathrm{PD} : \H^{m}(M,\partial M) \to \H_m(M)$ is the Poincar\'e-Lefschetz duality isomorphism, it follows that $Q_M^\d$ is the composition of $\mathrm{PD}$ with \cref{eq:dual-lattice-isomorphism} and so is an isomorphism itself.
\end{proof}

\todo{examples}
\begin{example}
\end{example}

\begin{definition}
	Write $Q=P^\intercal D P$ for a diagonal real matrix $D$ ($P$ can be a real matrix), then count the number $n^+$ of positive eigenvalues, $n^0$ the number of zero eigenvalues, and number $n^-$ of negative eigenvalues. The \defn{signature}[signature of a bilinear form] of the bilinear form is the difference between the number of positive and negative eigenvalues $n^+-n^-$.
	The triple $(n^+, n^-, n^0)$ is referred to as the \defn{inertia}[inertia of a bilinear form] of $Q$.
\end{definition}

In the context of intersection forms, the signature is only a useful invariant for $4k$-manifolds, since the intersection form of a $(4k+2)$-manifold is skew-symmetric and thus has signature $0$.

\begin{remark}
	By Sylvester's Law of Inertia (see \cite{lam2005quadratic}), the rank and inertia completely classify symmetric bilinear forms on a vector space over $\R$ or $\Q$. Over the integers $\Z$, the classification of symmetric bilinear forms is significantly more complex.

	classifies bilinear forms on a vector space over $\R$ or $\Q$. The case of bilinear forms over $\Z$ is significantly more complex.
\end{remark}

\begin{definition}
	The \defn{signature}[signature of a manifold] of a $4k$-dimensional manifold $M$ is the signature of its intersection form, and is denoted $\sigma(M)$.
\end{definition}

When $Q$ is the intersection form of a manifold $X$, especially a $4k$-manifold so that the form can have non-zero eigenvalues, the signature of the intersection form is referred to as the \defn{signature}[signature of a manifold] of the manifold, and denoted $\sigma(X)$. This is a topological invariant of fundamental importance, and we will see many of its generalizations and equivalent definitions in \cref{sec:hirzebruch-signature-theorem} and \cref{sec:surgery-invariant}.

\begin{definition}
	If for all non-zero elements $a\in \Lambda$ we have $Q(a,a)>0$, then we say that $Q$ is \defn{positive-definite}[positive-definite bilinear form]. On the contrary, if we have $Q(a,a)<0$ for all non-zero elements $a\in \Lambda$, we say that $Q$ is \defn{negative-definite}[negative-definite bilinear form]. Otherwise, $Q$ is said to be \defn{indefinite}[indefinite bilinear form].
\end{definition}

\begin{definition}
	If for all elements $a$, $Q(a,a)$ even, then $Q$ is said to be \defn{even}[even bilinear form]. Otherwise, we say that $Q$ is \defn{odd}[odd bilinear form].\footnote{Some sources refer to this property as ``type''. Odd forms are said to be \defn{type I}[type I bilinear form] and even forms are to be \defn{type II}[type II bilinear form].}
\end{definition}

\medskip
\todo{geometric interpretation of lattices,
	\href{https://math.stackexchange.com/questions/2944104/obtaining-two-holed-torus-as-a-quotient-of-bbb-c}{maybe this article?}}

\pagebreak
\section{Morse Theory}\label{sec:morse-theory}

The next tool in our arsenal is that of Morse theory, introduced in \cref{sec:morse-theory}.

\todo{Morse theory is a technique connecting the structure of a manifold to the behavior of a (nice) smooth function from the manifold to the real numbers. By investigating the discrete set of critical points of such a function, questions of attaching and removing high-dimensional handles is simplified to questions of points on the one-dimensional real line. For this thesis, the main application of Morse theory will be in the proof of the $h$-cobordism theorem of \cref{chap:h-cobordism}, however we make occasional use of Morse theory in historical remarks and for explicit constructions of exotic spheres scattered throughout.}

While a fully comprehensive introduction to Morse theory is outside of the scope of this thesis, we'll include a basic overview for completeness. A great classical introduction to Morse theory can be found in Milnor's book on the subject \cite{milnor1963morse}.

If $f : M \to \R$ is a smooth function on a manifold $M$, the points $p\in M$ where the differential $df_p : \T_p M \to \T_{f(p)} \R = \R$ vanish are known as \defn{critical points}, and their images in $\R$ are called \defn{critical values}. In terms of local coordinates $\{x^1,\ldots, x^n\}$ at $p$, this means that
\begin{equation}
	\frac{\partial f}{\partial x^1}=\cdots=\frac{\partial f}{\partial x^n} = 0.
\end{equation}
A critical point $p\in M$ is said to be \defn{non-degenerate}[non-degenerate point] if the matrix
\begin{equation}
	\everymath={\displaystyle}
	\renewcommand*{\arraystretch}{2}
	H_f(p) = \begin{pmatrix}
		\frac{\partial^2 f}{\partial x^1\partial x^1} & \cdots &
		\frac{\partial^2 f}{\partial x^1\partial x^n}                   \\
		\vdots                                        & \ddots & \vdots \\
		\frac{\partial^2 f}{\partial x^n\partial x^1} & \cdots &
		\frac{\partial^2 f}{\partial x^n\partial x^n}                   \\
	\end{pmatrix}(p)
\end{equation}
is invertible at $p$. This is called the \defn{Hessian matrix} of $f$ at $p$, and in this formulation depends on our chosen coordinate system.
There is a coordinate independent way to define the Hessian as a symmetric bilinear form on $\T_p M$ which makes the coordinate invariance of the condition of non-degeneracy manifestly apparent.

\begin{definition}
	The \defn{index}[index of a function] of $f$ at a point $p$ is the maximal dimension of a subspace on which $H_f(p)$ is negative definite, i.e. it is the dimension of $\{v\in \R^n \mid v^\intercal H_f(p) v < 0\}.$
\end{definition}

The index of a function at a point essentially describes the ``shape'' of the function out of a list of finitely many possible shapes. Remember, the index of a function on an $n$-dimensional manifold must be an integer between $0$ and $n$. For instance, in the case of a surface there are three possible shapes -- when both coordinates curve up we get a bowl facing up, when one curves up and one curves down we get a saddle, and when both coordinates curve down we get a bowl facing down. These shapes correspond to indices of $0$, $1$, and $2$ respectively.
\begin{figure}[ht]
	\centering
	\import{diagrams}{placeholder.pdf_tex}
	\caption{An upward bowl, saddle, and downward bowl.}
\end{figure}

The fundamental lemma of Morse theory makes rigorous this notion of a manifold having a shape dictated by a real-valued function -- there is always a local coordinate system which puts the function into a standard form depending on the index.

\begin{lemma}[Morse Lemma]\label{lemma:morse}
	Let $p$ be a non-degenerate critical point of $f$. There is a local coordinate system $(y^1,\ldots, y^n)$ at $p$ such that
	\begin{equation}
		f(y^1,\ldots, y^n)=f(0)-\left[(y^1)^2 + \cdots + (y^{\ell})^2\right] + \left[(y^{\ell + 1})^2 + \cdots + (y^n)^2\right].
	\end{equation}
	where $\ell$ is the index of $f$ at $p$.
\end{lemma}
\begin{proof}
	Let's assume without loss of generality that $f(p)=0$. Given any local coordinate system $(x^1,\ldots, x^n)$ at $p$, we can write
	\begin{equation}
		f(x^1,\ldots, x^n) = \sum_{1\leq j \leq n} x^j g_j(x^1,\ldots, x^n)
	\end{equation}
	where $g_j$ are functions satisfying $g_j(0)=(\partial f / \partial x^j)(0)$.
	This can be achieved by setting $g_j(x_1,\ldots, x_n) = \int_0^1 (\partial f/\partial x^j)(tx_1, \ldots, tx_n)\,dt$. Since $p$ is a critical point

	\todo{basic idea}
\end{proof}

Inspired by this lemma, we call a function $f : M \to \R$ a \defn{Morse function} if all critical points are non-degenerate.

\begin{corollary}
	Non-degenerate critical points are isolated.
\end{corollary}

For a brief demonstration of the power of the Morse lemma, we'll prove Reeb's theorem, a Morse theoretic criterion for a compact manifold to be homeomorphic to a sphere. Throughout the thesis, we'll usually defer to the more powerful $h$-cobordism theorem to prove that a manifold is homoemorphic to a sphere.
However, it is useful to not always rush for the flamethrower when trying to kill a fly -- a simple swatter might do the trick. We'll see a direct application of this lighter theorem in \cref{sec:milnor-spheres}.

\begin{theorem}[Reeb]\label{thm:reeb}
	If $M$ is a compact manifold and $f$ is a Morse function with exactly $2$ critical points, then $M$ is homeomorphic to a sphere.
\end{theorem}
\begin{proof}
	Firstly, by compactness of $M$ we can find a global minimum $f(x_0)$ and global maximum $f(x_1)$ for some distinct points $x_0$ and $x_1$ (otherwise the function would be constant and not a Morse function with $2$ critical points). We can normalize the function $f$ to have $f(x_0)=0$ and $f(x_1)=1$ without loss of generality. It follows that $x_0$ is a non-degenerate critical point of index 0 and $x_1$ is a non-degenerate critical point of index $n$.
	By the Morse lemma (\ref{lemma:morse}), there is a neighborhood $x_0\in U_0$ with local coordinates $\{y^1,\ldots, y^n\}$ such that
	\begin{equation}
		f(y^1,\ldots, y^n) = (y^1)^2 + \cdots + (y^n)^2.
	\end{equation}
	This gives a Riemannian metric $(dy^1)^2+\cdots+(dy^n)^2$ on $U$ which can be extended to all of $M$ by partitions of unity.
	% A Riemannian metric on a manifold $M$ determines an isomorphism $\T^\d M \cong \T M$, and hence an isomorphism $\Omega^1(M)=\Gamma(\T^\d M) \cong \Gamma(\T M)=\X(M)$ between the space of $1$-forms and the space of vector fields. Composing this isomorphism with the exterior derivative $d : \Omega^0(M)\to \Omega^1(M)$ gives the gradient operator $\nabla : \Omega^0(M) \to \X(M)$ sending a function to its vector field. 

	Given a Riemmanian structure, there is a gradient operator $\nabla : \Omega^0 \to \X(M)$ sending functions to vector fields.
	In our case, the vector field $\nabla f$ is non-zero everywhere except for at $x_0$ and $x_1$. We thus have a normalized vector field $\nabla f/\|\nabla f\|^2$
	defined everywhere except for at $x_0$ and $x_1$. Let $\varphi_t : M \to M$ be the global flow corresponding to this vector field, i.e. the unique solution to the differential equation
	\begin{equation}
		\left.\frac{d\varphi_t(p)}{dt}\right|_{t=0} = \frac{\nabla f(p)}{\|\nabla f(p)\|^2}
	\end{equation}
	for all $p\in M\setminus \{x_0,x_1\}$. By the chain rule, it follows that
	\begin{equation}
		\frac{d(f\circ \varphi_t(p))}{dt}=\left\langle \frac{d\varphi_t(p)}{dt}, \nabla f\right\rangle = \left\langle \frac{\nabla f}{\|\nabla f\|^2}, \nabla f\right\rangle=1.
	\end{equation}
	In particular, this implies that $f\circ \varphi_t(p) = f(p)+t$.

	\todo{finish the proof}
\end{proof}

The basic ideas used in the proof of Reeb's theorem can be radically generalized.

\begin{definition}
	For any $a\in \R$, the \defn{level set} of a smooth function $f : M \to \R$ is the set
	\begin{equation}
		M_a = f^{-1}(-\infty, a] = \{ p\in M \mid f(p)\leq a)\}.
	\end{equation}
\end{definition}

\begin{figure}
\end{figure}

\begin{theorem}
	Let $f : M \to \R$ be a smooth function and suppose $f^{-1}[a,b]$ contains no critical points of $f$ for real numbers $a<b$. Then $M_a$ is diffeomorphic to $M_b$. Furthermore $M_a$ is a deformation retract of $M_b$.
\end{theorem}

\begin{theorem}
	Let $f : M \to \R$ be a smooth function and let $p$ be a non-degenerate critical point of index $\ell$. Letting $c=f(p)$, suppose that $f^{-1}[c-\epsilon, c+\epsilon]$ is compact and contains no critical points of $f$ aside from $p$. For sufficiently small $\epsilon$, the level set $M^{c+\epsilon}$ has the homotopy type of $M^{c-\epsilon}$ with an $\ell$-cell attached.
\end{theorem}

\begin{theorem}
	If $f$ is a Morse function with compact level sets (for instance if $M$ is compact), then $M$ has the homotopy type of a CW-complex with a cell in each dimension $\ell$ for each critical point of index $\ell$.
\end{theorem}

\todo{finish}

\begin{theorem}[Morse Inequality]
	Let $M^n$ be a closed manifold with $f : M\to \R$ a Morse function. Let $\beta_i=\rank \H_i(M)$ be the $i$-th Betti number of $M$ and let $c_i$ be the number of critical points of $f$ of index $i$. Then, for every $\ell\in \Z^{\geq 0}$ we have
	\begin{equation}
		\beta_\ell - \beta_{\ell-1} + \beta_{\ell-2} - \cdots +(-1)^\ell \beta_0 \leq c_\ell - c_{\ell-1} + c_{\ell-2} - \cdots + (-1)^\ell c_0.
	\end{equation}
\end{theorem}

\pagebreak
\section{Cobordism}\label{sec:cobordism}

The basic principle of cobordism is to declare two manifolds equivalent if there is a manifold one dimension higher which connects the two manifolds. As an equivalence relation, cobordism is far looser than the notions of homoemorphism or diffeomorphism and so allows for a full classification of manifolds. Many notions in geometry and topology -- for instance characteristic classes -- depend solely on the cobordism type of a manifold, so understanding the structure of cobordism is immensely helpful. 

In this section we only begin to scratch the surface of this rich theory, leaving many details for \cref{chap:classification} where we discuss the Pontryagin-Thom construction.

\begin{remark}
	That the implied compactness assumption throughout the thesis is important here, otherwise any manifold $M$ is trivially the boundary of $M\times [0,\infty)$.
\end{remark}

For now, let us begin with the two simplest types of cobordism. Let us temporarily disband with the assumption made throughout the rest of the thesis that all manifolds need to be connected.

\begin{definition}
	An \defn{unoriented cobordism} between closed $n$-manifolds $M_1$ and $M_2$ is an $(n+1)$-manifold $W$ with $\partial W = M_1\sqcup M_2$. We use $W : M_1\bord M_2$ to refer to the cobordism.
\end{definition}

\begin{definition}
	An \defn{oriented cobordism} between closed oriented $n$-manifolds $M_1$ and $M_2$ is an oriented $(n+1)$-manifold $W$ with $\partial W = M_1\sqcup (-M_2)$. We use the notation to $W : M_1\sobord M_2$ refer to the cobordism.
\end{definition}

If we have two unoriented cobordisms $W : M_1 \bord M_2$ and $W' : M_1'\bord M_2'$, their disjoint union $W\sqcup W'$ is an unoriented cobordism from $M_1\sqcup M_1'$ to $M_2\sqcup M_2'$. Disjoint union is thus a well-defined commutative operation on cobordism classes of manifolds.
\begin{figure}[ht]
	\centering
	\import{diagrams}{unoriented-cobordism-Z2.pdf_tex}
	\caption{An unoriented cobordism $M\times [0,1] : M\sqcup M \bord \varnothing$.}\label{fig:unoriented-cobordism-Z2}
\end{figure}

The identity element is the empty set, and the inverse of a manifold $M$ is the manifold itself, since $M\times [0,1]$ is a cobordism from $M\sqcup M$ to the identity element $\varnothing$ (see \cref{fig:unoriented-cobordism-Z2}). This observation motivates the following definition.

\begin{definition}
The \defn{$k$-th unoriented cobordism group}[unoriented cobordism group] $\Omega^\O_k$ is the abelian group of cobordism classes of closed $k$-manifolds under the disjoint union. 
\end{definition}

% \begin{}
% \end{}

\begin{definition}
	The \defn{$k$-th oriented cobordism group}[oriented cobordism group] $\Omega^\SO_k$ is the abelian group of oriented cobordism classes of closed $k$-manifolds
	under disjoint union. The identity component is the empty set $\varnothing$, and negation is given by reversing orientation. 
\end{definition}

Note that the oriented cobordism group can be thought of as a $\Z$-module, with multiplication action on a closed manifold $M$ given by
\[
	n \cdot M = \begin{cases} M\sqcup \cdots \sqcup M & n > 0,\\ (-M)\sqcup \cdots \sqcup (-M) & n < 0,\\ \emptyset & n=0,\end{cases}
\]
for all $n\in \Z$, where $\sqcup$ is repeated $|n|$ times. Since there is no notion of negation in the unoriented case, the unoriented cobordism group is a $\Z/2$-module.

\begin{figure}[ht]
	\centering
	\import{diagrams}{pair-of-pants.pdf_tex}
	\caption{A cobordism $W$ between $S^1$ and $(S^1\sqcup S^1)$.}\label{fig:pair-of-pants}
\end{figure}

For a simple example of a cobordism between a circle and a disjoint union of circles, see \cref{fig:pair-of-pants}. Note that this cobordism could be made much simpler by removing the handle. Simplifying cobordisms in this way is one of the major applications of surgery theory.

\begin{example}
	There is an isomorphism $\Omega_0\cong \Z/2$. An unoriented 0-dimensional manifold is just a set of points. Any pair of points is cobordant to the empty set by a path connecting them. Since adding pairs of points doesn't change the cobordism type, the number of points modulo 2 determines the cobordism class entirely.
\end{example}

\begin{example}
	There is an isomorphism $\Omega_0^\SO \cong \Z$. An oriented 0-dimensional manifold is still a set of points, however the orientation now equips each point with a ``charge'', we might label them as $+$ or $-$. Note that points of opposite ``charges'' cancel out by a path between them oriented from $-$ to $+$. Given some set of points of various charges, we can always eliminate pairs of opposite charges and are left with a homogeneous set of charge. Adding up all of the pluses or minuses, we get an integer. This integer determines the cobordism class, and is invariant to adding or removing pairs of opposing charge.
\end{example}

\begin{example}
	Both the oriented and unoriented cobordism groups are trivial in dimension 1, since every circle is the boundary of a disk.
\end{example}

In higher dimensions, the classification becomes much more interesting.

\begin{figure}[ht]
	\renewcommand{\arraystretch}{1.2}
	\centering
	\begin{tabular}{r||c|c||c|c}
		$k$ & $\Omega_k$          & generators                                          & $\Omega_k^\SO$ & generators                 \\
		\hline
		$0$ & $\Z/2$              & a point                                             & $\Z$           & a point                    \\
		$1$ & $0$                 &                                                     & $0$            &                            \\
		$2$ & $\Z/2$              & $\RP^2$                                             & $0$            &                            \\
		$3$ & $0$                 &                                                     & $0$            &                            \\
		$4$ & $\Z/2\oplus \Z/2$   & $\RP^4$, $\RP^2\times \RP^2$                        & $\Z$           & $\CP^2$                    \\
		$5$ & $\Z/2$              & $\SU_3/\SO_3$                                       & $\Z/2$         & $\SU_3/\SO_3$              \\
		$6$ & $(\Z/2)^{\oplus 3}$ & $\RP^6$, $\RP^2\times \RP^4$, $(\RP^2)^{\times 3}$, & $0$            &                            \\
		$7$ & $\Z/2$              & $(\SU_3/\SO_3) \times \RP^2$                        & $0$            &                            \\
		$8$ & $(\Z/2)^{\oplus 4}$ & $\RP^8, \RP^6\times \RP^2, \cdots$                  & $\Z\oplus \Z$  & $\CP^4, \CP^2\times \CP^2$ \\
	\end{tabular}
	\medskip
	\caption{Structure of unoriented and oriented cobordism groups.}\label{fig:cobordism-structure-table}
\end{figure}

The structure of \cref{fig:cobordism-structure-table} makes a lot more sense in the context of

\begin{proposition}
	The product of manifolds is a well-defined operation with respect to cobordism.
\end{proposition}
\begin{proof}
	\todo{proof}
\end{proof}

\begin{definition}
	The \defn{oriented cobordism ring} $\Omega^\SO_\bullet$ is the set of oriented cobordism classes of closed manifolds under the operations of disjoint union and product.
\end{definition}

The oriented cobordism ring has a grading by
\[
	\Omega_\bullet^\SO = \bigoplus_{k\geq 0} \Omega^\SO_k.
\]

\begin{theorem}[Thom-Pontryagin]\label{thm:oriented-cobordism-structure}
	There is a ring isomorphism
	\[
		\Omega_\bullet^\SO \otimes \Q \lkxto \Q[x_4, x_8, x_{12}, \ldots]
	\]
	where $x_{4k}$ are cobordism classes representing $\CP^{2k}$.
\end{theorem}

\subsection{The $h$-Cobordism Theorem}
\begin{theorem}[$h$-cobordism]\label{thm:h-cobordism}
	Within some category of manifolds $\mathscr{C}$, if $M$ and $N$ are closed, simply-connected manifolds of dimensions $\geq 5$ and $W : M \hbord N$ is a simply-connected $h$-cobordism, then $W$ is $\mathscr{C}$-isomorphic to the cylinder $M\times [0,1]$. Furthermore, the isomorphism can be chosen to be the identity on $M\times \{0\}$.
\end{theorem}

\begin{example}
	The simple-connectedness assumption of the $h$-cobordism theorem is required, since the spaces $L(7,1)\times S^4$ and $L(7,2)\times S^4$ are $h$-cobordant but not diffeomorphic.
\end{example}

\begin{theorem}
	In the manifold categories $\TOP$ and $\PL$, the generalized Poincar\'e conjecture is true in dimensions $\geq 5$.
\end{theorem}
\begin{proof}
	\todo{cone construction, fails for $\DIFF$ because you can't take a smooth cone.}
	\todo{mention method of engulfing?}
\end{proof}

\begin{corollary}\label{thm:h-cobordism-diffeomorphism}
	In the smooth oriented manifold category, two simply-connected closed manifolds of dimensions $\geq 5$ are $h$-cobordant if and only if they are diffeomorphic (by an orientation preserving diffeomorphism).
\end{corollary}
\begin{proof}
	If $f : M_1 \to M_2$ is a diffeomorphism between manifolds $M_1$ and $M_2$, they are $h$-cobordant by the manifold $W=M_1\times [0,1]\cup_f M_2$, where we glue $M_2$ onto $M_1\times \{1\}$ in $M_1\times [0,1]$ by $f$.
	Conversely, if $W : M_1\sohbord M_2$ is an $h$-cobordism, by the $h$-cobordism theorem (\ref{thm:h-cobordism}) there must be a diffeomorphism $f : W \to M_2$ must map to $M_1\times \{1\}$, this gives a diffeomorphism $M_2 \to M_1$. If we choose $f$ to reverse orientation on $M_1\to M_1\times \{0\}$, then the restriction $f|_{M_2}$ will preserve orientation.
\end{proof}

\pagebreak
\section{Groups of Homotopy Spheres}\label{sec:groups-of-homotopy-spheres}

With the $h$-cobordism theorem and \cref{thm:h-cobordism-diffeomorphism}, we have arrived at our first major simplification to the problem of classifying smooth structures in dimensions $\geq 5$ -- to classify smooth structures on $S^n$, it is enough to classify the $h$-cobordism classes of smooth manifolds which are homeomorphic to $S^n$, and to find smooth manifolds which are homeomorphic to $S^n$,. it suffices to consider smooth manifolds which have the homotopy type of $S^n$.

\begin{definition}
	A \defn{homotopy $n$-sphere}[homotopy sphere] is a smooth manifold which is homotopy equivalent to the sphere $S^n$. By the generalized Poincar\'e conjecture, all homotopy spheres are homeomorphic to spheres.
\end{definition}

\begin{definition}
	Let $\Theta^n$ be the pointed set of $h$-cobordism classes of homotopy $n$-spheres (the basepoint is the ordinary sphere $S^n$).
\end{definition}

So far, this is a fairly general setup. Instead of spheres, we could use any simply-connected base manifold and consider the set of $h$-cobordism classes of manifolds homotopy equivalent to it. This is known as the \defn{surgery structure set}, and we will expand on this briefly in \cref{sec:surgery-theory-in-general}. However, in the case of spheres, there is a special bit of extra data which does not usually generalize -- a group structure under the connected sum operation defined in \cref{sec:connected-sum}.

\begin{theorem}\label{thm:group-of-homotopy-spheres}
	The connected sum turns $\Theta^n$ into a group with identity element $S^n$.
\end{theorem}
\begin{proof}
	We have already proved that the connected sum is well-defined up to diffeomorphism, associative, and commutative up to orientation preserving diffeomorphism.  
	The remaining proof splits as three lemmas -- these are Lemmas 2.2, 2.3, and 2.4 in \cite{milnorkervaire1963groups}.

	\begin{lemma}
		The connected sum is a well-defined operation on $\Theta^n$.
	\end{lemma}
	\begin{proof}
		We will prove something slightly more general. Let $M_1, M_1'$ and $M_2, M_2'$ be closed, simply-connected, and oriented manifolds, and suppose $W_1 : M_1\sohbord M_1'$ and $W_2 : M_2\sohbord M_2'$ are $h$-cobordisms. The goal is to construct an $h$-cobordism $W : (M_1\+M_2)\sohbord (M_1'\+M_2')$.

		\begin{figure}[ht]
			\centering
			\import{diagrams}{placeholder.pdf_tex}
			\caption{A join of two $h$-cobordisms along embedded arcs.}\label{fig:connected-sum-of-h-cobordisms}
		\end{figure}

		Recall that the connected sum $M_1\+M_2$ was defined as the join of $M_1$ and $M_2$ along points $p_1\in M_1$ and $p_2\in M_2$. Let us assume these were the points along which the connected sum $M_1\+M_2$ was taken, and $p_1'\in M_1'$ and $p_2'\in M_2'$ the points along which the connected sum $M_1'\+M_2'$ was taken. Choose smooth paths $\gamma_i : [0,1] \to W_i$ which send $p_i\mapsto p_i'$, and assume without loss of generality that $\gamma_i$ are embeddings which are transverse to the boundary of $\partial W_i$, making it possible to get tubular neighborhoods
		\[
			\lkxfunc{\iota_i}{\R^n\times [0,1]}{W_i.}
		\]
		Let $W$ be the join of $W_1$ and $W_2$ along the image of the arcs $\gamma_i$, as depicted in \cref{fig:connected-sum-of-h-cobordisms}. This is an oriented smooth manifold with boundary
		\[
			\partial W = (M_1\#M_2)\sqcup -(M_1'\+M_2')
		\]
		so $W$ is a cobordism $W : (M_1\+M_2)\sobord (M_1'\+M_2')$. We just need to prove that $W$ is an $h$-cobordism to complete the proof.

		\begin{lemma}\label{lemma:removing-point-homotopy-equivalence-h-cobordism}
			If $W : M \sohbord M'$ is an $h$-cobordism with $M,M'$ compact oriented and simply connected, $p\in M$ is a point, and $T$ is a tubular neighborhood of an arc from $p$ to $p'\in M'$, then the inclusion $M\setminus\{p\} \to W\setminus T$ is a homotopy equivalence.
		\end{lemma}
		\begin{proof}
			The long exact sequence of the inclusion $j : (M,M\setminus\{p\}) \to (W, W\setminus T)$ gives
			\[
				\H_{i+1}(W\setminus T, M\setminus\{p\})\lkxto
				\H_i(M\setminus\{p\}) \lkxto[j_*] \H_i(W\setminus T) \lkxto \H_i(W\setminus T, M\setminus\{p\})
			\]
			Note that by excision, $\H_i(W\setminus T, M\setminus\{p\})\cong \H_i(W,M)$, and since $W$ is an $h$-cobordism, $\H_i(W,M)\cong 0$. Consequently, $j$ induces isomorphisms on homology and since $M,M'$, and $W$ are simply connected, Whitehead's theorem and the Hurewicz isomorphism imply that $j$ is a homotopy equivalence.
		\end{proof}

		With \cref{lemma:removing-point-homotopy-equivalence-h-cobordism} in mind, we use the Mayer-Vietoris sequence
		\[\begin{tikzcd}
				{\H_i(S^{n-1})} & {\H_i(M_1\setminus\{p_1\})\oplus\H_i(M_2\setminus\{p_2\})} & {\H_{i}(M_1\+M_2)} & {\H_{i-1}(S^{n-1})} \\
				{\H_i(S^n)} & {\H_i(W_1\setminus T_1)\oplus\H_i(W_2\setminus T_2)} & {\H_{i}(W)} & {\H_{i-1}(S^n)}
				\arrow[from=1-1, to=1-2]
				\arrow[from=1-1, to=2-1]
				\arrow["f"', from=1-2, to=1-3]
				\arrow["{(j_1)_*\oplus (j_2)_*}", from=1-2, to=2-2]
				\arrow[from=1-3, to=1-4]
				\arrow["j_*", from=1-3, to=2-3]
				\arrow[from=1-4, to=2-4]
				\arrow[from=2-1, to=2-2]
				\arrow["g"', from=2-2, to=2-3]
				\arrow[from=2-3, to=2-4]
			\end{tikzcd}\]
			where $j_i : M_i\setminus \{p_i\} \to W_i\setminus T_i$ and $j : M_1\+ M_2\to W$ are inclusions. Note that $(j_1)_*\oplus (j_2)_*$ is an isomorphism by \cref{lemma:removing-point-homotopy-equivalence-h-cobordism}. When $i$ is not $1,n-1,$ or $n$, it follows immediately that $f$ and $g$ are isomorphisms and so $j$ is an isomorphism. In the remaining cases, a simple application of the  snake lemma does the trick.
			By simple connectedness, Whitehead's theorem, and the Hurewicz isomorphism, it follows that the inclusion of $M_1\+M_2 \to W$ is a homotopy equivalence, completing the proof.
	\end{proof}

	\begin{remark}
		In the previous lemma, the simple-connectedness assumption was not strictly required -- only used to reduce the problem of homotopy equivalence to a homology problem by Whitehead's theorem. It is possible to directly construct an explicit homotopy equivalence with a more careful argument.
	\end{remark}

	Next, we will prove a simple characterization of when a homotopy sphere is equal to the identity element in $\Theta^n$.

	\begin{lemma}\label{lemma:null-h-cobordant-iff-bounds-contractible}
		A simply-connected manifold $M$ is $h$-cobordant to $S^n$ if and only if $M$ bounds a contractible manifold.
	\end{lemma}
	\begin{proof}
		The forward direction is fairly straightforward. If $W : M\sohbord S^n$ is an $h$-cobordism, we can glue a tubular neighborhood of $D^{n+1}$ in $\R^{n+1}$ to $(-S^n)\subset \partial W$ along a collar neighborhood of $-S^n$ in $\partial W$. Any homotopy equivalence of $W$ to $S^n$ can then be extended to a contraction along $D^{n+1}$.

		In the reverse direction, let us assume that $M$ bounds a contractible manifold $W$. Pick any point in the interior of $M$ and remove the interior of an embedded disk $D^{n+1}\subset W$. This gives a simply-connected cobordism $W'=W\setminus \Int(D^{n+1}) : M\sobord (-S^n)$. By the long exact sequence of the inclusion $j : (D^{n+1}, S^n) \to (W,W')$, we have
		\[
			\H_i(W', S^n) \lkxto H_i(S^n)\lkxto[j_*] \H_i(W')\lkxto \H_{i-1}(W',S^n),
		\]
		and by contractability of $W$ the boundary terms vanish and so $S^n \to W'$ is a homotopy equivalence. Then, by the relative Poincar\'e duality isomorphism $\H_k(W',M)\cong \H^{n+1-k}(W', S^n)$ and the long exact sequence of the pair $(W',M)$, it follows that $M\to W'$ is a homotopy equivalence and so $W'$ is in fact an $h$-cobordism. 
	\end{proof}

	\begin{lemma}
		If $M$ is a homotopy sphere, then $M\+(-M)$ bounds a contractible manifold.
	\end{lemma}
	\begin{proof}
		\begin{figure}[ht]
			\centering
			\import{diagrams}{placeholder.pdf_tex}
			\caption{A contractible boundary of $M\+(-M)$.}
		\end{figure}
	\end{proof}


	\noindent This completes the proof.
\end{proof}

\subsection{Framed Manifolds}

There is a final

\begin{definition}
\end{definition}

\begin{definition}
\end{definition}
