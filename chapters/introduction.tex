\begin{flushleft}
	\textsl{There is geometry in the humming of the strings,}\\
	\textsl{and there is music in the spacing of the spheres.}\\
	\rule[0pt]{21em}{0.5pt}\\
	-- \textsc{Pythagoras}\\
	\vspace{2em}
\end{flushleft}

% \begin{conjecture}[Poincar\'e Conjecture]
%  Every closed topological $3$-manifold which is simply connected is homeomorphic to the $3$-sphere $S^3$.
% \end{conjecture}
%
% \todo{define PL, Diff, Top etc, show differences for instance cone construction}
%
% \begin{conjecture}[Generalized Poincar\'e Conjecture]
%   Letting $\mathscr{C}$ be either $\Top$, $\PL$, or $\Diff$, any $\mathscr{C}$-manifold which is homotopy equivalent to the $n$-sphere $S^n$ is also $\mathscr{C}$-isomorphic to $S^n$.
% \end{conjecture}
%
% \begin{theorem}[$h$-cobordism]\label{thm:h-cobordism}
%   Let $n\geq 5$ and $\mathscr{C}$ be either $\Top$, $\PL$, or $\Diff$. If $M$ and $N$ are $\mathscr{C}$-manifolds and $W : M \hbord N$ is an $h$-cobordism between them, then $W$ is $\mathscr{C}$-isomorphic to the cylinder $M\times [0,1]$.
% \end{theorem}
%
%
%
% \begin{definition}
%   A closed oriented (smooth) $n$-manifold $M$ is called a \defn{homotopy sphere} if it has the homotopy type of the $n$-sphere $S^n$. An \defn{exotic sphere} is a homotopy $n$-sphere which is not diffeomorphic to the standard $n$-sphere $S^n$.
% \end{definition}
%
% \subsection*{Why is this complicated?}
%
% Any homotopy sphere is \defn{stably parallelizable} -- meaning the stable isomorphism class of its tangent bundle is trivial.
%
% \begin{theorem}
%   If $\Sigma$ is a homotopy $n$-sphere, then $\T \Sigma \oplus \underline{\R}$ is trivial. 
% \end{theorem}
% \begin{proof}
% \end{proof}
%
% In fact, a much stronger result holds true.
% \begin{theorem}
%   If $\Sigma$ is a homotopy $n$-sphere with $f : S^n \to \Sigma$ the homotopy equivalence, then there is a bundle isomorphism $f^*\T\Sigma \approx \T S^n$.
% \end{theorem}
% \begin{proof}
% \end{proof}
%
% \subsection*{Groups of Homotopy Spheres}
%
% See \cite{milnor1963groups} and \cite{levine1985lectures}
%
% \begin{definition}
%   Let $\Theta_n$ denote the group of diffeomorphism classes homotopy $n$-spheres under the operation of connected sum.
% \end{definition}
%
% \begin{definition}
% \end{definition}
%
% \begin{definition}
%   Let $\bP_{n+1}$ denote the subgroup of $\Theta_n$ of (classes of) homotopy $n$-spheres which bound parallelizable manifolds.
% \end{definition}
%
% \begin{theorem}[Kervaire-Milnor]
%   The group of homotopy $(4k-1)$-spheres bounding parallelizable manifolds is a cyclic group of order:
%   \[
%     |\bP^{4k}| = 2^{2k-2}(2^{2k-1}-1)\varepsilon_k\cdot \mathrm{num}(B_{2k}/4k) 
%   \]
% \end{theorem}
%
% \subsection*{The Kirby-Siebenmann Class}
%
% \begin{theorem}
%   For $n\geq 5$, there is an isomorphism $\pi_n(\pl/\diff)\approx \Theta_n$.
% \end{theorem}
%
% \subsection*{Global Gravitational Anomalies}
%
% See \cite{witten1985global}.
