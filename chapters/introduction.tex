\chapter{Introduction}\label{chap:introduction}

\begin{epigraph}{10em}{}
\end{epigraph}

In the years following John Milnor's unexpected 1956 discovery of a non-standard smooth structure on the seven-dimensional sphere, differential topology began to undergo a whirlwind of growth. Within six years following the initial discovery, Milnor and Michel Kervaire would publish a nearly complete classification of exotic structures on spheres of any dimensions, with some notable exceptions. 
Their quick work on this problem was aided by a decade of advancements in adjacent topics -- the works of Ren\'e Thom and Lev Pontryagin on cobordism, Hirzebruch's signature theorem, Frank Adams' computations of the image of the $J$-homomorphism, Bott's periodicity theorem, Smale's $h$-cobordism theorem, and many other related results. 

\todo{middle}

The goals of this thesis are -- excuse the bad pun -- manifold. Chief among them is the desire for a comprehensive, mostly self-contained exposition to the topic of exotic spheres. While researching the topic, most sources I found either included exotic spheres tangentially as part of a broader theory, or only focused on a sole aspect of their construction or classification.
In this thesis I hope to take a more leisurely route -- not to be afraid to stop and appreciate perspectives or constructions which, while not essential, greatly illuminate the topic.
\medskip

As promised in the title of the thesis, I also hope to emphasize the rich geometry and topology woven throughout the topic. On the geometric side, this involves the way in which lattice geometry shapes the topology of exotic spheres, the connection between the index of elliptic differential operators and exotic sphere invariants, and constructions of exotic spheres as towers of branched coverings of classical knots. Topological perspectives on exotic spheres are more common, but we still make sure to include plenty of diagrams and ensure that all constructions are well-motivated.

The final aspect I would like to focus on is that of computation. \todo{computaion}
