\chapter{Introduction}\label{chap:introduction}

\begin{epigraph}{17em}{Douglas Ravenel}
  My initial inclination was to call this\\
  book ``The Music of the Spheres'', but\\
  I was dissuaded from doing so by my \\
  diligent publisher, who is ever mindful \\
  of the sensibilities of librarians.
\end{epigraph}

In the years following John Milnor's unexpected 1956 discovery of a non-standard smooth structure on the seven-dimensional sphere, the field of differential topology underwent a preriod of tremendous growth. Within six years of the initial discovery, Milnor and Michel Kervaire would publish a nearly complete classification of exotic structures on spheres of any dimensions, with some notable exceptions. 
Their quick work on this problem was aided by a decade of advancements in adjacent topics -- the works of Ren\'e Thom and Lev Pontryagin on cobordism, Hirzebruch's signature theorem, Frank Adams' computations of the image of the $J$-homomorphism, Bott's periodicity theorem, Smale's $h$-cobordism theorem, and many other results. 

\todo{middle, talk about later results}

\begin{center}
  \textsl{What is the set of smooth structures on a sphere?}
\end{center}

The goals of this thesis are -- excuse the bad pun -- manifold. Chief among them is the desire for a comprehensive, mostly self-contained exposition to the topic of exotic spheres. While researching the topic, most sources I found either included exotic spheres tangentially as part of a broader theory, or only focused on a sole aspect of their construction or classification.
In this thesis I hope to take a more leisurely route -- not to be afraid to stop and appreciate perspectives or constructions which, while not essential, greatly illuminate the topic.
As promised in the title of the thesis, I also hope to emphasize the role geometry plays in this topic.
This includes the way in which lattice geometry shapes the topology of exotic spheres, the connection between the index of elliptic differential operators and exotic sphere invariants, and constructions of exotic spheres as towers of branched coverings of classical knots. 

The final aspect I would like to focus on in this thesis is that of computation.
One of the striking things about the set of smooth structures on spheres is how random it appears. Why are there exactly 28 smooth structures in on 7-dimensional spheres, exactly 6 smooth structures in 10 dimensions and 992 smooth structures in 11 dimensions? It's difficult to answer these questions in a fully satisfying or ``intuitive'' way, and we'll often defer hard and tedious homotopy theoretic results for brevity.
That being said, there are plenty of places where computations are doable and illuminating -- for instance we'll show, directly, that there are (at least) 28 smooth structures in 7-dimensions and 992 smooth structures in 11-dimensions. I will also include plenty of Wolfram scripts for cases where the lengths of high-dimensional computations exceed our patience.
\todo{computaion}

\section*{Outline}
