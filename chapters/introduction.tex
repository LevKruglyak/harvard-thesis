\begin{flushleft}
	\textsl{There is geometry in the humming of the strings,}\\
	\textsl{and there is music in the spacing of the spheres.}\\
	\rule[0pt]{21em}{0.5pt}\\
	-- \textsc{Pythagoras}\\
	\vspace{2em}
\end{flushleft}

The basic objects of study in differential topology are smooth manifolds. Loosely speaking, smooth manifolds are spaces which locally look like Euclidean space of a given dimension and contain additional ``smooth structure'' which makes it possible to do calculus on them.

\begin{definition}
\end{definition}

\todo{physics and math}

% A smooth structure gives rise to tangent spaces -- at each point of the manifold there is a notion of infinitesimal direction, and the set of all such infinitesimal directions forms a vector space of tangent vectors. \todo{physics}

While the class of smooth manifolds offers a tempting pasture for the exploration of the shape of space, it is far from the only type of manifold one can define. Another equally valid category in which to study topology of manifolds is $\PL$, or the piecewise linear category. A $\PL$ structure on a manifold consists of 

\section{The Generalized Poincar\'e Conjecture}

\begin{conjecture}[Poincar\'e Conjecture]
 Every closed topological $3$-manifold which is simply connected is homeomorphic to the $3$-sphere $S^3$.
\end{conjecture}

\begin{conjecture}[Generalized Poincar\'e Conjecture]
  Letting $\mathscr{C}$ be either $\Top$, $\PL$, or $\Diff$, any $\mathscr{C}$-manifold which is homotopy equivalent to the $n$-sphere $S^n$ is also $\mathscr{C}$-isomorphic to $S^n$.
\end{conjecture}

\subsection{Smooth Tricks of the Trade}

\begin{theorem}[$h$-cobordism]\label{thm:h-cobordism}
  Let $n\geq 5$ and $\mathscr{C}$ be either $\Top$, $\PL$, or $\Diff$. If $M$ and $N$ are $\mathscr{C}$-manifolds and $W : M \hbord N$ is an $h$-cobordism between them, then $W$ is $\mathscr{C}$-isomorphic to the cylinder $M\times [0,1]$.
\end{theorem}

\todo{Reeb theorem, morse theory, conclude with looking for homotopy spheres.}

% \todo{define PL, Diff, Top etc, show differences for instance cone construction}
%
%
%
%
%
% \begin{definition}
%   A closed oriented (smooth) $n$-manifold $M$ is called a \defn{homotopy sphere} if it has the homotopy type of the $n$-sphere $S^n$. An \defn{exotic sphere} is a homotopy $n$-sphere which is not diffeomorphic to the standard $n$-sphere $S^n$.
% \end{definition}
%
% \subsection*{Why is this complicated?}
%
% Any homotopy sphere is \defn{stably parallelizable} -- meaning the stable isomorphism class of its tangent bundle is trivial.
%
% \begin{theorem}
%   If $\Sigma$ is a homotopy $n$-sphere, then $\T \Sigma \oplus \underline{\R}$ is trivial. 
% \end{theorem}
% \begin{proof}
% \end{proof}
%
% In fact, a much stronger result holds true.
% \begin{theorem}
%   If $\Sigma$ is a homotopy $n$-sphere with $f : S^n \to \Sigma$ the homotopy equivalence, then there is a bundle isomorphism $f^*\T\Sigma \approx \T S^n$.
% \end{theorem}
% \begin{proof}
% \end{proof}
%
% \subsection*{Groups of Homotopy Spheres}
%
% See \cite{milnor1963groups} and \cite{levine1985lectures}
%
% \begin{definition}
%   Let $\Theta_n$ denote the group of diffeomorphism classes homotopy $n$-spheres under the operation of connected sum.
% \end{definition}
%
% \begin{definition}
% \end{definition}
%
% \begin{definition}
%   Let $\bP_{n+1}$ denote the subgroup of $\Theta_n$ of (classes of) homotopy $n$-spheres which bound parallelizable manifolds.
% \end{definition}
%
% \begin{theorem}[Kervaire-Milnor]
%   The group of homotopy $(4k-1)$-spheres bounding parallelizable manifolds is a cyclic group of order:
%   \[
%     |\bP^{4k}| = 2^{2k-2}(2^{2k-1}-1)\varepsilon_k\cdot \mathrm{num}(B_{2k}/4k) 
%   \]
% \end{theorem}
%
% \subsection*{The Kirby-Siebenmann Class}
%
% \begin{theorem}
%   For $n\geq 5$, there is an isomorphism $\pi_n(\pl/\diff)\approx \Theta_n$.
% \end{theorem}
%
% \subsection*{Global Gravitational Anomalies}
%
% See \cite{witten1985global}.
