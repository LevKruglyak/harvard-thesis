\chapter{Introduction}\label{chap:introduction}

\begin{epigraph}{17em}{Douglas Ravenel}
  My initial inclination was to call this\\
  book ``The Music of the Spheres'', but\\
  I was dissuaded from doing so by my \\
  diligent publisher, who is ever mindful \\
  of the sensibilities of librarians.
\end{epigraph}

In the years following John Milnor's unexpected 1956 discovery \cite{milnor1956manifolds} of a non-standard smooth structure on the seven-dimensional sphere, the field of differential topology underwent a period of tremendous growth. Within six years of the initial discovery, Milnor and Michel Kervaire would publish a nearly complete classification of exotic structures on spheres of any dimension in \cite{milnorkervaire1963groups}.


Their quick work on this problem was aided by a decade of advancements in adjacent topics -- the works of Ren\'e Thom \cite{thom1954} and Lev Pontryagin \cite{pontryagin1959homotopy} on cobordism, Hirzebruch's signature theorem \cite{hirzebruch1966methods}, Frank Adams' computations of the image of the $J$-homomorphism, Bott's periodicity theorem \cite{bott1959stable}, Smale's $h$-cobordism theorem \cite{smale1961generalized}, and many other results. 

\subsection*{Outline}

The goals of this thesis are manifold. Central among them is the desire for a comprehensive introduction to the topic of the construction and detection of exotic spheres. While researching for this thesis, many sources I found included exotic spheres either tangentially or emphasized a sole aspect of their construction or classification. In this thesis, we take a leisurely route, stopping and appreciating perspectives or constructions which, while not essential, greatly illuminate the topic. Exotic spheres are a subtle craft, and focusing on multiple perspectives deepens the understanding of any individual one. 
%
% As promised in the title of the thesis, I also hope to emphasize the role geometry plays in this topic.
% This includes the way in which lattice geometry shapes the topology of exotic spheres, the connection between the index of elliptic differential operators and exotic sphere invariants, and constructions of exotic spheres as towers of branched coverings of classical knots. 


%
% The final aspect I would like to focus on in this thesis is that of computation.
% One of the striking things about the set of smooth structures on spheres is how random it appears. Why are there exactly 28 smooth structures in on 7-dimensional spheres, exactly 6 smooth structures in 10 dimensions and 992 smooth structures in 11 dimensions? It's difficult to answer these questions in a fully satisfying or ``intuitive'' way, and we'll often defer hard and tedious homotopy theoretic results for brevity.
% That being said, there are plenty of places where computations are doable and illuminating -- for instance we'll show, directly, that there are (at least) 28 smooth structures in 7-dimensions and 992 smooth structures in 11-dimensions. I will also include plenty of Wolfram scripts for cases where the lengths of high-dimensional computations exceed our patience.

\section*{Conventions}

\subsection*{General}

\begin{itemize}
  \item Definitions of terms will be formatted {\color{blue}\emph{blue and emphasized}}, and hyperlinks will be formatted just blue (e.g. \cref{fig:first}).
  \item We use $\cong$ instead of $\oldcong$ to denote isomorphisms, diffeomorphisms, etc.
\end{itemize}

\subsection*{Algebra}

\begin{itemize}
  \item The group of units in a ring $A$ is denoted $A^\times$.
  \item A \defn{graded ring} is a ring $A$ equipped with a decomposition of its underlying group as $A=\bigoplus_{k\in \Z_{\geq 0}} A[k]$ such that the ring multiplication sends $A[k_1]\times R[k_2] \to A[k_1+k_2]$. Elements belonging to $A[k]$ are said to be \defn{homogeneous of degree $k$}[homogeneous element of a graded ring]. 
  \item A graded ring is said to be \defn{graded-commutative} if for any homogeneous elements $x\in A[k_1]$ and $y\in A[k_2]$, we have
    \[
      x\cdot y = (-1)^{k_1k_2} y\cdot x
    \]
  \item The \defn{completion}[completion of a graded ring] of a graded ring $A$ is the direct product $\widehat{A} = \prod_{k\geq 0} A[k]$.
  \item When expanding an element $a$ in a graded ring or completion of a graded ring as an infinite series, we use the notation
  \[
      a = a_0 + a_1t+a_2t^2+\cdots
  \]
  The variable $t$ should be understood as a notational formal variable.
  \item $A_{(1)}\subset A^\times$ denotes the units with monic leading coefficient (i.e. $a_0=1$).
\end{itemize}

\subsection*{Differential Topology}
\begin{itemize}
  \item A \defn{topological manifold} $M$ is a locally Euclidean second-countable Hausdorff space. \todo{boundary}
  \item A \defn{smooth structure} $\mathscr{S}$ on a topological manifold $M$ is a collection of open charts $\mathscr{S}=\{(U_\alpha, \varphi_\alpha)\}_{\alpha\in I}$ such that the transition functions 
	\begin{equation}\label{eq:transition-function}
		\lkxfunc{g_{\alpha\beta}}{\varphi(U_\alpha\cap U_\beta)}{\R^n\textrm{ or } \R^{n-1}\times [0,\infty)}
	\end{equation}
	are smooth for all $\alpha,\beta\in I$. We require that $\mathscr{S}$ be maximal with respect to this property, i.e. the addition of any chart $(U,\varphi)$ not in $\mathscr{S}$ breaks the smoothness of \cref{eq:transition-function}.
  \item Unless otherwise specified, all manifolds are assumed to be smooth, connected, and possibly with boundary. A \defn{closed}[closed manifold] manifold is a compact manifold with empty boundary.
  \item When introducing a manifold, we often put its dimension as a superscript. For example, ``let $M^n$ be a manifold'' should be read as ``let $M$ be an $n$-dimensional manifold''.

  \item We assume that all submanifolds $N\subset M$ are properly embedded and neat, i.e.
    \vspace{-0.5em}
    \begin{itemize}
      \item the inclusion $\iota : N \to M$ is a proper map,
      \item $\partial N\subset \partial M$,
      \item the boundary $\partial N$ intersects $\partial M$ transversally.
    \end{itemize}
\end{itemize}

\subsection*{Algebraic Topology}
\begin{itemize}
  \item Given a coefficient ring $R$, $\H^i(-; R)$ denotes the singular cohomology with coefficients in $R$, and $\H_i(-;R)$ denotes singular homology with coefficients in $R$. $\H^\bullet(-;R)$ is the singular cohomology ring.
  \item $\HdR$ denotes de-Rham cohomology, and $\Hc$ denotes compactly-supported de-Rham cohomology.
  \item Whenever we refer to a general (co)homology theory $h$, we mean one of the pairs:
    \begin{itemize}
      \item Singular (co)homology with coefficients in $R=\Z, \Z/2, \Z[1/2],$ or $\Q$.
      \item Compactly supported de-Rham cohomology and de-Rham cohomology.
    \end{itemize}
    The homology groups are denoted $h_i(-)$ and the cohomology groups are denoted $h_i(-)$. Note that all pairs 
\end{itemize}

\subsection*{Differential Geometry}
\begin{itemize}
  \item Vector bundles are over a field $\F=\R$ or $\C$, by default assumed to be $\R$.

  \item By default, we denote the total space of a vector bundle with normal math font, i.e. $E \to X$, and denote the bundle itself with calligraphic font, i.e. $\mathcal{E} : E \to X$. 

  \item The tangent bundle of a manifold $M$ is dented $\TT M$, with total space $\T M$.

  \item As with manifolds, a superscript $\mathcal{E}^k$ on a vector bundle denotes its rank. 

  \item Given a Riemannian inner product structure $\langle-,-\rangle$ on a vector bundle $\mathcal{E}^k$, we let $\S(\mathcal{E})$ be the associated sphere bundle (with fibers $S^{k-1}$) and $\D(\mathcal{E})$ the associated disk bundle (with fibers $D^k$). The total spaces of these bundles are
  \[
    \S(E) = \{ \xi\in E \mid \langle \xi, \xi\rangle =1 \}
    \quad\textrm{and}\quad
    \D(E) = \{ \xi\in E \mid \langle \xi, \xi\rangle\leq 1 \}
  \]
  respectively. Note that $\partial \D(E) = \S(E)$ as manifolds.
\end{itemize}
