\begin{flushleft}
	\textsl{A mathematician is a blind man in }\\
	\textsl{a dark room looking for a black cat}\\
	\textsl{which isn’t there.}\\
	\rule[0pt]{15em}{0.5pt}\\
	\textsl{-- Unknown}
	\vspace{2em}
\end{flushleft}

This chapter will build up to the Atiyah-Singer index theorem -- a remarkable gem of 20th century mathematics which unifies \todo{todo}
The index theorem is able to express the ``number'' of solutions to an elliptic system of differential equations on a manifold in terms of the topology of the manifold. This gives subtle integrability (i.e. being integer-valued) conditions for certain combinations of characteristic numbers.

Starting with an introduction to the Hirzebruch signature theorem and $L$-genus, we'll construct (a generalization of) an invariant used by Milnor to detect the first exotic sphere \cite{milnor1956manifolds} in 1956.
\todo{finish this}

\begin{convention*}
	All manifolds are assumed to be oriented. Unless otherwise specified, $k\geq 1$.
\end{convention*}

\section{The Hirzebruch Signature Theorem}\label{sec:hirzebruch-signature-theorem}

Recall that the intersection form of a closed $4k$-manifold $X$ is the symmetric bilinear form on $\H^{2k}(X)$ given by
\[
	\lkxfunc{I_X}{\H^{2k}(X)^{\times 2}}{\R}{\alpha,\beta}{\int_X \alpha\smile \beta.}
\]
Since $I_X$ is a symmetric bilinear form over a finite dimensional real vector space, it is determined up to isomorphism by its signature, i.e. the difference in dimensions of its maximal positive definite and negative subspaces. 

\begin{definition}
	The \defn{signature} $\sigma(X)$ of a closed $4k$-manifold $X$ is the signature of the form $I_X$.
\end{definition}

Given the structure of the cohomology ring, the signature of a manifold is straightforward to calculate. For instance:
\begin{proposition}
	The signature of even-dimensional complex projective planes is $\sigma(\CP^{2m})=1$.
\end{proposition}
\begin{proof}
	The cohomology of $\CP^{2m}$ has the ring structure
	\[
		\H^\bullet(\CP^{2m}) = \R[\alpha]/(\alpha^{2m+1})\quad\textrm{where}\quad |\alpha|=2.
	\]
	Since the middle dimensional cohomology has dimension $\dim \H^{2m}(\CP^{2m}) = 1$ and $\CP^{2m}$ can be oriented so that $\int_{\CP^{2m}} \alpha^{2m} = 1$, it follows that the signature is $1$.
\end{proof}

\begin{remark}
	As a convention, if the dimension of a closed manifold $X$ is not divisible by $4$ we define the signature $\sigma(X)=0$ to be trivial.
\end{remark}

For more topological details on the signature and intersection form, we refer the reader to \cref{sec:signature} for a thorough review.

\subsection*{Harmonic Differential Forms}

There is a analytic interpretation to the 

If $X_1$ and $X_2$


\subsection*{The $L$-Genus}

\begin{definition}
	The \defn{$\bm{L}$-genus} $\{L_k\}$ is the multiplicative sequence of polynomials corresponding to the series $Q(z) = \sqrt{z}/\tanh\sqrt{z}$.
\end{definition}

\begin{theorem}[Hirzebruch Signature Theorem]\label{thm:hirzebruch_signature}
	Let $X$ be a closed (oriented) $4k$-manifold. Then we have
	\[
		\sigma(X) = \int_X L_k(p_1, \cdots, p_k)(X),
	\]
	where $L_k$ is the $L$-genus.
\end{theorem}
\begin{proof}
	It suffices to check this identity for the generators $\CP^{2k}$. The total Pontryagin class of $\CP^{2k}$ is $p(\CP^{2k})=(1+\alpha^2)^{2k+1}$. Using multiplicativity of $L$ and $L(1+z)=\sqrt{z}/\tanh\sqrt{z}$, we have
	\[
		\begin{aligned}
			L(p)(\CP^{2k})
			= L\left((1+\alpha^2)^{2k+1}\right)
			= L(1+\alpha^2)^{2k+1}
			= \left(\alpha/\tanh \alpha\right)^{2k+1}
			\quad\in\H^\bullet(\CP^{2k}).
		\end{aligned}
	\]
	Here, we consider $\alpha/\tanh \alpha$ as a formal power series in $\alpha$ truncated by the relation $\alpha^{2k+1}=0$ in the cohomology ring $\H^\bullet(\CP^{2k})$. The characteristic number $L_k(p_1,\ldots,p_k)[\CP^{2k}]$ is then the coefficient of $\alpha^{2k}$ in the expansion of $(\alpha/\tanh \alpha)^{2k+1}$.
	By elementary complex analysis, this coefficient can be extracted by taking a contour integral around a small $\varepsilon$-circle about the origin in $\C$:
	\[
		\begin{aligned}
			L_k(p_1,\cdots, p_k)[\CP^{2k}]
			 & = \frac{1}{2\pi i}\oint_{S^1_\varepsilon} \frac{dz}{z^{2k+1}} \left(\frac{z}{\tanh z}\right)^{2k+1}
			 &                                                                                                       \\[0.5em]
			 & = \frac{1}{2\pi i}\oint_{S^1_\varepsilon} \frac{dz}{\tanh^{2k+1} z}\quad
			 & u  =\tanh(z),\quad
			du =(1-u^2)dz
			\\[0.5em]
			 & = \frac{1}{2\pi i}\oint_{S^1_\varepsilon} \frac{1}{u^{2k+1}}\cdot\frac{du}{1-u^2}
			 &                                                                                                       \\[0.5em]
			 & = \frac{1}{2\pi i}\oint_{S^1_\varepsilon} \frac{1+u^2+u^4+\cdots}{u^{2k+1}}\,du                       \\[0.5em]
			 & =1.                                                                                                 &
		\end{aligned}
	\]
	Since $\sigma(\CP^{2k})=1$, this completes the proof.
\end{proof}

\section{Milnor's Invariant}\label{sec:milnors-invariant}

\section{The Atiyah-Singer Index Theorem}\label{sec:atiyah-singer-index-theorem}

\begin{theorem}[Atiyah-Singer Index Theorem]\label{thm:atiyah-singer}
	Let $(E,D)$ be an elliptic complex over a closed $n$-manifold $X$. Then the index of $(E,D)$ can be expressed as
	\[
		\ind(E,D) = (-1)^{n(n+1)/2}\int_X \Theta^{-1}\ch \sigma(E)\smile \Td(\T X^\C)
	\]
	where $\Theta : \H^\bullet(X;\Q) \to \H_c^\bullet(\T^\d X;\Q)$ is the Thom isomorphism.
\end{theorem}

\begin{theorem}
	Let $E$ be an oriented vector bundle over $X$. There exists a spin structure on $E$ if and only if the second Stiefel-Whitney class $w_2(E)\in \H^2(X;\Z/2)$ vanishes. In this case, there is a bijective correspondence:
	\[
		\left\{
		\parbox{9em}{
			spin structures on $E$
		}
		\right\}
		\quad\iff\quad
		\H^1(X;\Z/2)
	\]
\end{theorem}

\begin{definition}
\end{definition}

\begin{theorem}\label{thm:Ahat-integrality}
	If $X$ is a closed spin $2k$-manifold, then $\Ahat[X]$ is an integer. Furthermore, if $k$ is odd then $\Ahat[X]$ is an even integer.
\end{theorem}

\section{Fancier Exotic Sphere Invariants}

\begin{definition}
	
\end{definition}
