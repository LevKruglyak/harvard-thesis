\begin{flushleft}
	\textsl{A mathematician is a blind man in }\\
	\textsl{a dark room looking for a black cat}\\
	\textsl{which isn’t there.}\\
	\rule[0pt]{15em}{0.5pt}\\
	\textsl{-- Unknown}
	\vspace{2em}
\end{flushleft}

In \cref{chap:construction}, the plumbing theorem gave us a surjective map
\[
		b_k : \Z \lkxsurj \bP^{4k}
\]
which constructs a homotopy sphere bounding a parallelizable manifold of signature $8t$ for any $t\in \Z$. By the first isomorphism theorem, it follows that $\bP^{4k}\cong \Z/t_k$ for some $t_k\in \Z$. In this chapter, we'll determine this number $t_k$ and hence derive a formula for the number of exotic $(4k-1)$-spheres bounding parallelizable manifolds. 

There are two avenues of exploration here. The first approach is to use theorems from index theory to construct invariants on $\bP^{4k}$ which can distinguish two homotopy spheres by their signatures. While this approach is beautiful and can detect many exotic spheres, it only gives lower bounds on $t_k$ and is unable to achieve a full classification. A better, although harder approach is to directly compute the kernel $\ker b_k$. We'll see how to do this in \cref{sec:milnor-kervaire-theorem}. As there are many insights gleamed in the first approach, let's begin here.

\section{Characteristic Classes}
\todo{should this be here or in appendix?}

\subsection{Stiefel-Whitney Classes}
\subsection{Chern Classes}
\subsection{Pontryagin Classes}

\begin{proposition}\label{prop:pontryagin-class-of-complex-projective-space}
	The total Pontryagin class of complex projective space is given by
	\[
		p(\CP^n) = (1+\alpha^2)^{n+1}\mod \alpha^{n+1},
	\]
	where $\alpha$ is the degree $2$ generator of the cohomology ring $\H^\bullet(\CP^n)\cong \R[\alpha]/(\alpha^{n+1})$.
\end{proposition}

\subsection{Chern-Weil Theory}

\section{The Hirzebruch Signature Theorem}

Let's understand the signature of a manifold on a deeper level, since it is of such fundamental importance to the structure of $\bP^{4k}$. \todo{historical note} It turns out that the signature of a manifold can be completely expressed in terms of the Pontryagin numbers of the manifold, and the subtleties of this connection leads to many non-trivial results about the integrality of rational expressions involving Pontryagin numbers. The first observation about signature which we could make is:

\begin{proposition}
	If $X$ is the boundary of a manifold $W$, then $\sigma(W)=0$.
\end{proposition}
\begin{proof}
\end{proof}

This has an immediate consequence. If $X$ is a disjoint union $X_1\sqcup (-X_2)$ which bounds a manifold $W$, then it follows that $\sigma(X_1\sqcup (-X_2))=\sigma(X_1)-\sigma(X_2)=0$ so that $\sigma(X_1)=\sigma(X_2)$ (It should be fairly clear that the intersection form of a disjoint union splits as a direct sum so the signature is additive). 
This is exactly the notion of oriented cobordism!

\begin{proposition}
	$\sigma(X_1\times X_2) = \sigma(X_1)\times \sigma(X_2)$.
\end{proposition}

\begin{corollary}
	The signature of a manifold is thus a ring homomorphism
	\[
		\lkxfunc{\sigma}{\Omega^\SO_{\bullet}}{\Z.}
	\]
\end{corollary}

Furthermore, this ring homomorphism factors through the rational cobordism ring as a map \[\Omega_\bullet^\SO\otimes \Q \lkxto \Z\] since torsion elements of $\Omega_\bullet^\SO$ must be mapped to $0\in \Z$. Fortunately for us, we already know the structure of $\Omega_\bullet^\SO\otimes \Q$ -- it is a polynomial ring in $\Q$ generated by the cobordism classes of $\CP^{2k}$. Since the signature of $\CP^{2k}$ is $1$ we can, at least in principle, compute the signature of any closed oriented manifold. Given a closed oriented manifold $M$ and a cobordism decomposition of it as a sum of a torsion component $T$ and products of complex projective planes, the signature of $M$ will count the number of homogeneous monomials in this decomposition.
\[
	M\sobord T\sqcup \bigsqcup_{i\in I} \CP^{2k_{i,1}}\times\cdots \times\CP^{2k_{i,\ell_i}}
	\quad\implies\quad
	\sigma(M) = |I|.
\]
The signature isn't the only integer-valued cobordism invariant out there. The Pontryagin numbers of an closed oriented manifold also depend solely on the cobordism type of the underlying manifold. For the complex projective spaces,

\subsection{Multiplicative Sequences and Genera}

\todo{write about the structure of $\Omega_\bullet^\SO$, Thom-Pontryagin construction, multiplicative sequences, genera, etc}

\begin{definition}
	The \defn{$\bm{L}$-genus} $\{L_k\}$ is the multiplicative sequence of polynomials corresponding to the series $Q(z) = \sqrt{z}/\tanh\sqrt{z}$.
\end{definition}

\begin{theorem}[Hirzebruch]\label{thm:hirzebruch-signature-theorem}
	Let $X$ be a closed $4k$-manifold. Then we have
	\[
		\sigma(X) = \int_X L_k(p_1, \cdots, p_k)(X),
	\]
	where $L_k$ is the $L$-genus.
\end{theorem}
\begin{proof}
	It suffices to check this identity for the generators $\CP^{2k}$. The total Pontryagin class of $\CP^{2k}$ is $p(\CP^{2k})=(1+\alpha^2)^{2k+1}$. Using multiplicativity of $L$ and $L(1+z)=\sqrt{z}/\tanh\sqrt{z}$, we have
	\[
		\begin{aligned}
			L(p)(\CP^{2k})
			= L\left((1+\alpha^2)^{2k+1}\right)
			= L(1+\alpha^2)^{2k+1}
			= \left(\alpha/\tanh \alpha\right)^{2k+1}
			\quad\in\H^\bullet(\CP^{2k}).
		\end{aligned}
	\]
	Here, we consider $\alpha/\tanh \alpha$ as a formal power series in $\alpha$ truncated by the relation $\alpha^{2k+1}=0$ in the cohomology ring $\H^\bullet(\CP^{2k})$. The characteristic number $L_k(p_1,\ldots,p_k)[\CP^{2k}]$ is then the coefficient of $\alpha^{2k}$ in the expansion of $(\alpha/\tanh \alpha)^{2k+1}$.
	By elementary complex analysis, this coefficient can be extracted by taking a contour integral around a small $\varepsilon$-circle about the origin in $\C$:
	\[
		\begin{aligned}
			L_k(p_1,\cdots, p_k)[\CP^{2k}]
			 & = \frac{1}{2\pi i}\oint_{S^1_\varepsilon} \frac{dz}{z^{2k+1}} \left(\frac{z}{\tanh z}\right)^{2k+1}
			 &                                                                                                       \\[0.5em]
			 & = \frac{1}{2\pi i}\oint_{S^1_\varepsilon} \frac{dz}{\tanh^{2k+1} z}\quad
			 & u  =\tanh(z),\quad
			du =(1-u^2)dz
			\\[0.5em]
			 & = \frac{1}{2\pi i}\oint_{S^1_\varepsilon} \frac{1}{u^{2k+1}}\cdot\frac{du}{1-u^2}
			 &                                                                                                       \\[0.5em]
			 & = \frac{1}{2\pi i}\oint_{S^1_\varepsilon} \frac{1+u^2+u^4+\cdots}{u^{2k+1}}\,du                       \\[0.5em]
			 & =1.                                                                                                 &
		\end{aligned}
	\]
	Since $\sigma(\CP^{2k})=1$, this completes the proof.
\end{proof}

The Hirzebruch signature theorem is a shining example of a truly remarkable theorem in mathematics -- it gives us easy computational means to uncover highly non-trivial relationships between complicated objects. For us, the most useful consequence of the Hirzebruch signature theorem is that it gives us subtle integrability and divisibility theorems. For instance, with relatively little effort we can calculate a few $L$-genus polynomials to get:
\begin{equation}\label{eq:L-genus}
	\begin{aligned}
		 & L_1 = \frac{p_1}{3},\quad
		 & L_2 = \frac{7p_2 - p_1^2}{45},\quad
		 & L_3 = \frac{62p_3 - 13p_1p_2 + 2p_1^3}{945},\quad\cdots
	\end{aligned}
\end{equation}
Having done this, the expression for the leading coefficient of $L_3$ in \cref{eq:L-genus} immediately implies:
\begin{corollary}
	If $X$ is a closed $12$-manifold with trivial $\H^4(X)$, then $\sigma(X)$ is divisible by $62$.
\end{corollary}
The observation that for such manifolds $X$, the quantity $\sigma(X)/62$ is an integer, let alone equal to $945p_3[X]$, is far from obvious, and yet it pops out immediately from the signature theorem. These subtle relationships are immensely useful in defining invariants capable of detecting exotic spheres as we will see in the following \cref{sec:constructing-exotic-sphere-invariants}.

While we're on the topic of the $L$-genus, let's compute its leading coefficient. Often times, the manifolds which we'll apply the signature theorem to will be connected enough that the lower order Pontryagin classes vanish -- leaving just the leading term.
Luckily for us, the coefficient of this leading term admits a simple description in terms of the \defn{Bernoulli numbers} $B_{2k}$, a sequence of rational numbers which appear ubiquitously throughout topology, homotopy theory, number theory, and many other disciplines. There are many conventions in the literature, but for our purposes we can define them as the terms appearing in the series expansion of $\tanh z$:
\begin{equation}\label{eq:tanh_series}
	\tanh z = z - \frac{z^3}{3} + \frac{2z^5}{15} - \frac{17z^7}{315}+\cdots = \sum_{k\geq 1} (-1)^k\frac{2^{2k}(2^{2k}-1)B_{2k}}{(2k)!}\, z^{2k-1}.
\end{equation}
With this definition, the first few Bernoulli numbers are given:
\begin{equation}\label{eq:bernoulli_numbers}
	B_0 = 1,\quad B_2 = \frac{1}{6},\quad B_4 = \frac{1}{30},\quad B_6=\frac{1}{42},\quad B_{8}=\frac{1}{30},\quad B_{10} = \frac{5}{66},\quad\cdots
\end{equation}

\todo{complexity of the sequence indicates how complicated the geometry can get}

\begin{proposition}\label{prop:leading_coefficient_L_genus}
	The leading coefficient of the $L$-genus is $s_k=2^{2k}(2^{2k-1}-1)B_{2k}/(2k)!$
\end{proposition}
\begin{proof}
	\todo{this proof}
\end{proof}

\section{Constructing Exotic Sphere Invariants}\label{sec:constructing-exotic-sphere-invariants}

Armed with the power of the Hirzebruch signature theorem, let's now try to build some invariants for the homotopy spheres in $\bP^{4k}$.
Many of the invariants discussed thus far in this thesis have been defined entirely out of the intrinsic geometric and topological data of a manifold. Due to the topological simplicity of homotopy spheres, it's difficult to imagine how we could construct such an invariant which can distinguish smooth structure. One of the great ideas of 20th century topology is to pass to a coboundary in situations like this. Namely, if the usual invariants vanish on a manifold $X$, find a manifold $W$ a dimension higher which has $X$ as a boundary (this isn't always possible!). In this case, $W$ is referred to as a \defn{coboundary} of $X$. If we are careful, we can use the topology and geometry of $W$ to construct an invariant which does not depend on the choice of $W$. This is exactly what we will do for exotic spheres in this section. We are in a particularly good position to try this idea since every homotopy sphere in $\bP^{4k}$ necessarily has a coboundary along with a parallelism of the coboundary.

\begin{remark}
	This idea of passing to a coboundary is an example of constructing a \defn{secondary invariant}. When primary invariants, in this case characteristic forms, turn out to be zero, we lift to a case where they are not zero, and use the descent data to measure ``how'' the original invariants vanished. This is a central idea of Chern-Simons theory \cite{chernsimons1974geometricinvariants}, a topic which has found widespread application in constructing quantum field theories. \todo{(?) write more about this?}
\end{remark}

\subsection{Smooth Invariants of Manifolds with Boundary}\label{sec:smooth-invariants-of-manifolds-with-boundary}

Before we can define secondary invariants, we need a good generalization of characteristic numbers to the case of manifolds with boundary. Recall that we need 

\todo{discussion about why topological invariance fails with boundary absent additional restrictions.}

This requires understanding smooth invariants of manifolds with boundary, and this will be the focus of the section.
Throughout let's assume $X$ is a closed $n$-manifold with oriented coboundary $B$.
First of all, we'll note that many \emph{topologically} defined invariants generalize naturally to the case of manifolds with boundary. For instance, the Euler characteristic can be defined as a topological invariant at least for any finite CW complex.
Generalizing the intersection form and correspondingly the signature requires a slight generalization of the Poincar\'e duality theorem:

% \begin{theorem}[Poincar\'e-Lefschetz Duality]
% 	Suppose $X$ is an $n$-dimensional manifold with boundary $\partial X$. Given a fundamental class $[X, \partial X]\in \H^{n}(X, \partial X)$, there is a duality isomorphism
% 	\[
% 		\lkxfunc{}{\H^k(X, \partial X)}{\H_{n-k}(X)}{\omega}{\omega\frown [X,\partial X]}
% 	\]
% 	given by cap product with the fundamental class.
% \end{theorem}
% \begin{proof}
% 	See Theorem~18.6.1 in \cite{dieck2008algebraic}.
% \end{proof}
% As in the case of ordinary Poincar\'e duality, the isomorphism $\H_0(X)\approx \R$ allows us to interpret the cap product as integration when $\omega$ is top dimensional -- there is an isomorphism $\H^m(X)\to \R$ which sends $\omega$ to $\int_X \omega$.
%
% This allows us to define the signature and intersection form of a $4k$-dimensional manifold with boundary.
%
% \begin{definition}
% 	If $X$ is an $4k$-manifold with boundary $\partial X$, the (relative) \defn{intersection form} is the symmetric bilinear form given by
% 	\[
% 		\lkxfunc{I_{X}}{\H^{2k}(X, \partial X)\otimes \H^{2k}(X, \partial X)}
% 		{\R}{\alpha\otimes \beta}{\int_X \alpha\smile \beta.}
% 	\]
% 	The (relative) \defn{signature} $\sigma(X, \partial X)$ of $X$ is the signature of this bilinear form.
% \end{definition}

\begin{convention*}
	Most of the literature simply uses the notation $\sigma(X)$ even when $X$ has a non-empty boundary, and we will adopt this convention outside of this chapter. However, in this chapter, we would like to keep the distinction meaningful for better clarity.
\end{convention*}

While there are no topological constraints on a manifold in order to generalize the signature or Euler characteristic, constraints do appear when generalizing characteristic forms to the relative setting.
Characteristic forms are not a priori relative cohomology classes, so pulling them back to obtain \emph{relative} characteristic forms in order to integrate requires additional assumptions about the topology of the boundary $X$.
For any integer $\ell$, the pair $(B, \partial B) = (B, X)$ gives us a long exact sequence of cohomology groups
\begin{equation}\label{eq:relative_characteristic_classes_exact_sequence}
	\H^{\ell-1}(X) \lkxto \H^{\ell}(B, X) \lkxto[j] \H^{\ell}(B) \lkxto \H^{\ell}(X)
\end{equation}
where $j : \H^{\ell}(B, X) \to \H^{\ell}(B)$ is the induced map of the inclusion $(B,\emptyset) \to (B, X)$. This is an isomorphism if the groups on either side of \cref{eq:relative_characteristic_classes_exact_sequence} are trivial. In this case, we can pullback:

\begin{definition}\label{defn:relative_characteristic_form}
	Suppose that $\H^{\ell}(X)$ and $\H^{\ell-1}(X)$ are trivial. For a characteristic form $c_\ell(B) \in \H^{\ell}(B)$, the \defn{relative characteristic form} is the pullback
	\[
		c_\ell(B, X) = j^{-1} c_\ell(B) \quad\in \H^{\ell}(B, X).
	\]
	The \defn{relative characteristic number} is the integral $c_\ell[B,X]=\int_B c_\ell(B,X)$.
\end{definition}

For instance, we could define relative Pontryagin numbers in this way:

\begin{definition}\label{defn:relative_pontryagin_number}
	Given a polynomial $K\in \Q[x_1,\ldots, x_k]$ satisfying the conditions of \todo{cite}, suppose that $\H^{4i}(X)$ and $\H^{4i-1}(X)$ are trivial for all $i$ for which $K$ has a $x_i$ term.
	In this case, we define the \defn{relative Pontryagin number} to be the integral
	\[
		\begin{aligned}
			K(p_1, \ldots, p_k)[B,X]
			 & = \int_B K(p_1, \ldots, p_k)(B,X)       \\
			 & = \int_B K(j^{-1}p_1, \ldots, j^{-1}p_k)(B).
		\end{aligned}
	\]
\end{definition}

\begin{remark}
	Note that we pullback \emph{before} applying the polynomial $K$. This is because pulling back a top-dimensional form on $B$ is not generally possible since $\H^{n}(X)$ is non-trivial and generated by the fundamental class.
\end{remark}

\begin{remark}
	Note that if $X=\emptyset$, relative characteristic forms and numbers correspond exactly to the non-relative versions since $j$ becomes the identity map.
\end{remark}

We've defined some useful relative invariants -- we now have the relative signature and relative Pontryagin numbers, although the latter comes with some topological restrictions. Our original goal was to use relative invariants of the coboundary $B$ to get an invariant for the boundary $X$. Thus, our next question should be:
\begin{center}
	\textsl{How do relative invariants change with the coboundary?}
\end{center}

There is a elegant trick we can use to help us answer this. If $B_1$ and $B_2$ are coboundaries for $X$, we can form a closed $(n+1)$-manifold $C$ by glueing $B_1$ and $B_2$ along their boundary $X$. There is a unique smooth structure on $C$ which agrees with the smooth structures of $B_1$ and $B_2$, and we can give $C$ the orientation which agrees with the orientation of $B_1$ and therefore with the reverse orientation of $B_2$.

By the Mayer-Vietoris sequence, we have an exact sequence
\[
	\H^{\ell-1}(X)\lkxto \H^{\ell}(C) \lkxto[\mu] \H^{\ell}(B_1)\oplus \H^{\ell}(B_2) \lkxto \H^{2k}(X)
\]
for any $\ell$, where $\mu$ is the map which restricts a form $\omega\in \H^\ell(C)$ to the form $\omega|_{B_1}\oplus \omega|_{B_2}\in \H^{\ell}(B_1)\oplus \H^{\ell}(B_2)$.
The relative version of the exact sequence is of the form
\[
	0 \lkxto \H^{\ell}(C; X) \lkxto[\rho] \H^{\ell}(B_1;X)\oplus \H^{\ell}(B_2;X) \lkxto 0,
\]
so we have an isomorphism $\rho$.
These maps are related neatly by the inclusion isomorphisms in \cref{eq:relative_characteristic_classes_exact_sequence}, and we can use these to form the commutative square:
\begin{equation}\label{eq:closing_coboundaries_square}
	\begin{tikzcd}
		{\H^{\ell}(C,X)} & {\H^{\ell}(B_1,X)\oplus\H^{\ell}(B_2,X)} \\
		{\H^{\ell}(C)} & {\H^{\ell}(B_1)\oplus\H^{\ell}(B_2)}
		\arrow["j_1\oplus j_2"', from=1-2, to=2-2]
		\arrow["\rho"', from=1-1, to=1-2]
		\arrow["j"', from=1-1, to=2-1]
		\arrow["\mu"', from=2-1, to=2-2]
		\arrow["h", from=1-2, to=2-1, dashed]
	\end{tikzcd}
\end{equation}
In the case that $\H^{\ell-1}(X)$ and $\H^\ell(X)$ are trivial, every map in this diagram is an isomorphism. Otherwise, we can only assume that the top map $\rho$ is an isomorphism.
Of particular interest to us is the diagonal map $h = j\circ \rho^{-1}$, which ``glues'' together relative forms on $B_1$ and $B_2$ to a form on $C$.

This glueing map satisfies the naturality properties:

\begin{proposition}\label{prop:variation_naturality_poincare}
	If $\alpha\in \H^{n+1}(B_1, X)$ and $\beta\in \H^{n+1}(B_2,X)$, then we have
	\[
		\int_C h(\alpha\oplus \beta) = \int_{B_1}\alpha - \int_{B_2}\beta.
	\]
\end{proposition}
\begin{proof}
	\todo{do this proof}
\end{proof}

\begin{proposition}\label{prop:variation_naturality_cup}
	If
	$\alpha_i\in \H^{\ell_i}(B_1,X)$ and $\beta_i \in \H^{\ell_i}(B_2,X)$ for $i=1,2$, then we have
	\[
		h(\alpha_1\oplus\beta_1) \smile h(\alpha_2\oplus \beta_2) = h(\alpha_1\smile \alpha_2 \oplus \beta_1\smile \beta_2).
	\]
\end{proposition}
\begin{proof}
	\todo{do this proof}
\end{proof}

As an immediate corollary of these two properties, we now have:
\begin{corollary}
	Suppose $\alpha_i\in \H^{\ell_i}(B_1, X)$ and $\beta_i\in \H^{\ell_i}(B_2,X)$ for $0\leq i < k$, and $K\in \Q[x_1,\ldots,x_k]$ a polynomial with $K(x^{\ell_1}, \ldots, x^{\ell_k})$ homogenous of degree $n+1$. Then, we have
	\[
		\int_C K(h(\alpha_1\oplus \beta_1),\ldots, h(\alpha_k\oplus \beta_k))
		=
		\int_{B_1} K(\alpha_1,\ldots, \alpha_k) - \int_{B_2} K(\beta_1,\ldots, \beta_k).
	\]
\end{corollary}

In particular, this gives us a formula for the change in relative Pontryagin numbers under a change of coboundary:
\begin{proposition}\label{prop:relative_pontryagin_number_variation}
	For $K\in \Q[x_1,\ldots, x_k]$ and $X$ as in \cref{defn:relative_pontryagin_number}, we have
	\begin{equation}\label{eq:relative_pontryagin_number_variation}
		K(p_1,\ldots,p_k)[B_1,X] - K(p_1,\ldots, p_k)[B_2,X] = K(p_1,\ldots,p_k)[C].
	\end{equation}
\end{proposition}

Another corollary relates to the signature. When $n=4k-1$, it makes sense to talk about the intersection forms of $B_1,B_2,$ and $C$. In this case, for forms $\alpha_1,\alpha_2\in \H^{2k}(B_1, X)$ and $\beta_1,\beta_2\in \H^{2k}(B_2,X)$ we can set $\alpha=h(\alpha_1\oplus\alpha_2)$ and $\beta=h(\beta_1\oplus \beta_2)$ in $\H^{2k}(C)$ and get
\[
	\int_{C} \alpha\smile \beta = \int_{B_1} \alpha_1\smile \alpha_2 - \int_{B_2}\beta_1\smile \beta_2.
\]
If $h$ is an isomorphism in dimension $2k$, for instance if $\H^{2k}(X)$ and $\H^{2k-1}(X)$ are trivial, then every element of $\H^{2k}(C)$ admits such a decomposition. This means that under the identification of $\H^{2k}(B_1,X)\oplus \H^{2k}(B_2,X)$ with $\H^{2k}(C)$ by $h$, we have
\[
	I_C = \begin{pmatrix}I_{B_1} & 0 \\ 0 & -I_{B_2}\end{pmatrix}.
\]
In other words, the intersection form of $C$ is the difference of the intersection form of $B_1$ and $B_2$. For the signature, this has the following implication, similar to \cref{prop:relative_pontryagin_number_variation}:
\begin{proposition}\label{prop:signature_variation}
	If $\H^{2k}(X)$ and $\H^{2k-1}(X)$ are trivial, then the signature satisfies the relation
	\begin{equation}\label{eq:signature_variation}
		\sigma(B_1, X) - \sigma(B_2, X) = \sigma(C).
	\end{equation}
\end{proposition}

Overall, from perspective of secondary invariants:
\begin{center}
	\textsl{The change of a secondary invariant with coboundary is expressible in terms}\\
	\textsl{of the invariant applied to a closed manifold.}
\end{center}

\todo{This whole section needs to be rewritten starting here, left over from old draft}

\subsection{Milnor's Invariant}

Let's see what types of invariants can be constructed out of relative Pontryagin numbers and the relative signature. Suppose $X$ is a $7$-dimensional homotopy sphere with $8$-dimensional coboundary $W$. Based on the cohomology, we know
\[
	\begin{aligned}
		\H^3(X)=0,  & \quad \H^4(X)=0 \\
		\H^7(X)=\R, & \quad \H^8(X)=0 \\
		\H^3(X)=0,  & \quad \H^4(X)=0
	\end{aligned}
	\quad\implies\quad
	\begin{aligned}
		 & p_1^2\textrm{ does have a relative generalization}         \\
		 & p_2\textrm{ does not have a relative generalization}       \\
		 & \sigma\textrm{ satisfies \cref{prop:signature_variation}.}
	\end{aligned}
\]
Thus, the two invariants of interest to us are
\[
	p_1^2[W,X]
	\quad\textrm{and}\quad
	\sigma(W, X).
\]
Now, for a \emph{closed} $8$-manifold $W$, rearranging using the Hirzebruch signature theorem gives us the expression:
\begin{equation}\label{eq:7-manifold_rearrangement}
	\sigma(W) = \frac{7p_2[W] - p_1^2[W]}{45}
	\quad\implies\quad
	p_2[W] = \frac{45\sigma(W) + p_1^2[W]}{7}.
\end{equation}
This suggests that there \emph{is} some analogue of the second Pontryagin class for $B$. For manifolds with boundary, we could define the number
\[
	\widetilde{p_2}[W, X] = \frac{45\sigma(W, X) + p_1^2[W, X]}{7}.
\]
This is a \emph{rational} number,
which reduces to the second Pontryagin number $p_2[W]$, an integer, when $X=\emptyset$. How does the quantity change under a change in coboundary, say if $W_1$ and $W_2$ were coboundaries? Letting $C$ be the $8$-manifold obtained by glueing them together, we see that
\[
	\begin{aligned}
		\widetilde{p_2}[W_1,X] - \widetilde{p_2}[W_2,X]
		 & = \frac{45\sigma(W_1,X) + p_1^2[W_1,X]}{7} - \frac{45\sigma(W_2, X) + p_1^2[W_2,X]}{7} \\
		 & =\frac{45\sigma(C) + p_1^2[C]}{7} = p_2[C].
	\end{aligned}
\]
But this last term is just an ordinary Pontryagin number, and hence an integer. While $\widetilde{p_2}$ is a priori a rational number for a given coboundary, it changes by an integer -- namely by the Pontryagin number $p_2[C]$ of a closed manifold.
Taking the fractional part of $\widetilde{p_2}$ thus gives us an invariant of $X$ which is \emph{independent of the coboundary}! 

\begin{definition}.
	Let $X$ be a closed $7$-manifold with $\H^3(X)$ and $\H^4(X)$ trivial.\footnote{Technically, we should add an assumption about $X$ being null-cobordant for a coboundary to exist in the first place, but all oriented $7$-manifolds are null-cobordant so this is unnecessary.} The \defn{Milnor invariant}\footnote{This differs from Milnor's original definition in \cite{milnor1956manifolds} by a factor of $2$ (mod 7) -- there he defined $\lambda=(2p_1^2-\sigma)/7$.} of $X$ is
	\[
		\boxed{\lambda_{\mathrm{milnor}}(X) = \frac{1}{7}\left(3\sigma(W,X)+p_1^2[W,X]\right)\mod 1}
	\]
	for any oriented coboundary $W$ of $X$.
\end{definition}

\begin{remark}
	Based on the results in \cref{sec:smooth-invariants-of-manifolds-with-boundary}, the Milnor invariant is additive in the connected sum, i.e. $\lambda_{\textrm{milnor}}(X_1\# X_2) = \lambda_{\textrm{milnor}}(X_1)+\lambda_{\textrm{milnor}}(X_2)$.
\end{remark}

Applying this to $\bP^8$, we get our first confirmation of the existence of exotic spheres in dimension $7$. For homotopy spheres $X$ in $\bP^8$, all relative Pontryagin numbers of a parallelizable coboundary $W$ will vanish, so $\lambda_{\textrm{milnor}}(X)=(3/7)\sigma(W, X)$. Since the Milnor invariant is additive in connected sum, we get a homomorphism
\[
		\lkxfunc{\lambda_{\textrm{milnor}}}{\bP^8}{\Q/\Z.}
\]
Recall that by the plumbing theorem, $\bP^8$ contains homotopy spheres bound by manifolds of any signature. Precomposing this plumbing map with the Milnor invariant gives
\[
	\begin{array}{rcl}
		\Z \lkxto & \bP^8 \lkxto & \Q/\Z \\
		t \lkxmapsto &X \lkxmapsto & 8t\cdot 3/7 \mod 1
	\end{array}
\]
The image of $\bP^8$ in $\Q/\Z$ thus has order $7$, generated by $1/7$. Taking the kernel, it follows that:

\begin{proposition}
	$\bP^8$ has a subgroup of index $7$, so $7$ divides $|\bP^8|$.
\end{proposition}

\begin{remark}
	\todo{historical note, discuss the Milnor spheres defined as the total space of a bundle, maybe compute the invariants}
\end{remark}

\subsection*{Milnor's Invariant for $(4k-1)$-Manifolds}

The basic ideas outlined for $7$-manifolds should work for $(4k-1)$-manifolds in general, so let's see what happens in this case. If we require that $\H^{2i}(X)$ and $\H^{4i}(X)$ are trivial for all $i<k$, then Poincar\'e duality ensures that $\H^{2i-1}(X)$ and $\H^{4i}(X)$ are trivial as well. \footnote{Just like $7$-dimensional case, the existence of an oriented coboundary follows from the cobordism ring $\Omega^\SO_\bullet$.}
As in this $7$-dimensional case, all but the top-dimensional Pontryagin classes can be generalized to a coboundary $W$.
Now let $W$ be a closed $4k$-manifold. Using the Hirzebruch signature theorem, we can do a rearrangement similar to \cref{eq:7-manifold_rearrangement} to get the expression
\begin{equation}\label{eq:4k-1-manifold_rearrangement}
	\sigma(W) = L_k(p_1, \ldots, p_k)[W]\quad\implies\quad
	p_k[W] = \frac{\sigma(W) - L_k(p_1,\ldots, p_{k-1}, 0)[W]}{s_k}
\end{equation}
where $s_k=L_k(0,\ldots, 0, 1)$ is the coefficient of $x_k$ in $L(x_1,\ldots, x_k)$.
We can isolate $p_k$ in this way since it is top-dimensional, or in other words the $L$-polynomial can be written in the form
\[
	L_k(x_1,\ldots, x_{k-1}, x_k) = s_k\cdot x_k +
	\left\{\parbox{18em}{terms involving lower order Pontryagin classes $p_1,\ldots, p_{k-1}$}\right\}.
\]
Now as before, we can use the rearrangement \cref{eq:4k-1-manifold_rearrangement} to get a rational number
\[
	\widetilde{p_k}[W, X] = \frac{\sigma(W, X) - L_k(p_1,\ldots, p_{k-1}, 0)[W,X]}{s_k}
\]
which acts as a rational generalization of the $k$-th Pontryagin class. By \cref{prop:relative_pontryagin_number_variation} and \cref{prop:signature_variation}, for coboundaries $W_1$ and $W_2$ we get
\[
	\widetilde{p_k}[W_1, X] - \widetilde{p_k}[W_2, X] = p_k[C],
\]
where $C$ is the glueing of $B_1$ with $B_2$. Since the $k$-th Pontryagin number of a closed manifold is an integer, we take the fractional part of $\widetilde{p_k}[B,X]$ and arrive at an easy generalization of the Milnor invariant for a $7$-manifold:

\begin{definition}
	Let $X$ be a closed $(4k-1)$-manifold with $\H^{4i}(X)$ and $\H^{2i}(X)$ trivial for all $i<n$ and coboundary $W$. The \defn{Milnor invariant} of $X$ is
	\[
		\boxed{
			\lambda_{\mathrm{milnor}}(X^{4k-1}) = \frac{1}{s_k}\Big(\sigma(W, X) - L_k(p_1, \ldots, p_{k-1},0)[W,X]\Big)\mod 1
		}
	\]
	where $s_k = L_k(0,\ldots,0,1)$ is the coefficient of the last term in the $L$-polynomial.
\end{definition}

\begin{example}
	The first few such invariants are:
	\[
		\begin{aligned}
			\lambda_{\mathrm{milnor}}(X^7)    
			& = \frac{4}{7}\sigma(W, X) - \frac{1}{7}\cdot p_1^2[W, X]\mod 1\\
			\lambda_{\mathrm{milnor}}(X^{11})    
			&= \frac{15}{62}\sigma(W,X) - \frac{1}{62}\left(2p_1^3-13p_1p_2\right)[W,X]\mod 1\\
			\lambda_{\mathrm{milnor}}(X^{15}) 
			& = \frac{101}{127}\sigma(W, X) - 
			\frac{1}{381}\left(3p_1^4-22p_1^2p_2 + 19p_2^2 + 71p_1p_3\right)[W,X]\mod 1.
		\end{aligned}
	\]
\end{example}

\todo{link wolfram code?}

As with the $7$-dimensional invariant, these expressions imply that $31$ divides $|\bP^{12}|$ and that $127$ divides $|\bP^{16}|$. More generally, the Milnor invariant gives the lower bound:
\begin{proposition}\label{prop:milnor-lower-bound}
	The order of $\bP^{4k}$ is divisible by
	\[
		\denom\left(\frac{8}{s_k}\right)=\denom\left(\frac{(2k)!}{2^{2k-3}(2^{2k-1} -1)B_{2k}}\right).
	\]
\end{proposition}

This isn't a very good lower bound, but it's a start. For instance, it's only able to detect $127$ of the $8128$ homotopy spheres in $\bP^{16}$ (we'll compute this later), and the situation gets worse as $k$ increases in size. While an explicit calculation of the size of $\bP^{16}$ requires a different idea entirely, improving these bounds can be done by constructing better invariants.

Before any generalization takes place, it's helpful to take a bird's eye view of what happened here. We started with the observation that the signature and Pontryagin numbers of a homotopy sphere were trivial, so we lifted to a coboundary. Once on this coboundary, we pick some characateristic number which takes on a restricted set of values for closed manifolds, which represent a ``change of coboundary''. Moding out by the image of such changes, we get an invariant purely of the boundary. To summarize, we have a loose procedure:
\[
	\left\{\parbox{12.5em}{An integrality theorem for a characteristic number of a closed $(n+1)$-manifold}\right\}
	\quad\implies \quad
	\left\{\parbox{12em}{A diffeomorphism invariant for $n$-manifolds}\right\}
\]
In this case, the integrality of the $p_k$ Pontryagin number of a closed $4k$-manifold led us to Milnor's invariant for $(4k-1)$-manifolds. 
If we use a different integrality theorem, would this procedure give us a usefully different invariant? The answer turns out to be yes.

\section{Index Theory}
While careful use of the $L$-genus is sufficient to understand the structure of $\bP^{4k}$, we'll now take a detour to explore some refinements of Milnor's invariant using generalizations of Hirzebruch's signature theorem. A time-pressed reader could skip straight to \cref{sec:milnor-kervaire-theorem}, although we warn that they would miss out on some spectacular mathematics.

\subsection{Harmonic Differential Forms}

\todo{Interpretation of signature as index of a differential operator.}

\subsection{Spin Geometry and the Dirac Operator}

\todo{motivation for spin structure out of factor of first chern class in characteristic series}

\begin{theorem}
	Let $E$ be an oriented vector bundle over $X$. There exists a spin structure on $E$ if and only if the second Stiefel-Whitney class $w_2(E)\in \H^2(X;\Z/2)$ vanishes. In this case, there is a bijective correspondence:
	\[
		\left\{
		\parbox{9em}{
			spin structures on $E$
		}
		\right\}
		\quad\iff\quad
		\H^1(X;\Z/2)
	\]
\end{theorem}

\begin{theorem}\label{thm:Ahat-integrality}
	If $X$ is a closed spin $2k$-manifold, then $\Ahat[X]$ is an integer. Furthermore, if $k$ is odd then $\Ahat[X]$ is an even integer.
\end{theorem}

\subsection{The Eells-Kupier Invariant}\label{sec:eells-kupier-invariant}

\todo{same construction as the milnor invariant, using the Ahat genus instead of top Pontryagin class}

\begin{definition}
	The \defn{Eells-Kupier invariant} of a manifold $\Sigma^7$ is
\end{definition}

\begin{example}
	The first few Eells-Kupier invariants are
	\[
		\begin{aligned}
			\lambda_{\mathrm{ek}}(X) &= \frac{1}{224}\sigma(W,X)-\frac{1}{896}p_1^2[W, X]\mod 1
		\end{aligned}
	\]
\end{example}

\begin{proposition}
	The order of $\bP^{4k}$ is divisible by $2^{2k-2}(2^{2k-1}-1)\varepsilon(k)$.
\end{proposition}

Together with \cref{prop:milnor-lower-bound}, this gives us
\todo{LCM lower bound, some numerical tables?}


\subsection{The Atiyah-Singer Index Theorem}

\todo{no clue if I need to go so general but it is a really cool piece of math and it would be cool to include the invariant I found. This section might become a remark or footnote}

\begin{theorem}[Atiyah-Singer Index Theorem]\label{thm:atiyah-singer}
	Let $(E,D)$ be an elliptic complex over a closed $n$-manifold $X$. Then the index of $(E,D)$ can be expressed as
	\[
		\ind(E,D) = (-1)^{n(n+1)/2}\int_X \Theta^{-1}\ch \sigma(E)\smile \Td(\T X^\C)
	\]
	where $\Theta : \H^\bullet(X;\Q) \to \H_c^\bullet(\T^\d X;\Q)$ is the Thom isomorphism.
\end{theorem}

\subsection{Twisted $\Ahat$-Genus}

\begin{definition*}
	Let $X$ be a closed $23$-manifold with $\H^i(X)=0$ for $i<23$ when $i\equiv 0,3\mod 4$. Then if $X$ has an $11$-connected string coboundary $W$, define: 
	\[
		\mu(X) = \frac{153945\sigma(W,X) + 2591p_2^3[W, X]}{521432801280}\mod 1.
	\]
\end{definition*}

\section{The Milnor-Kervaire Theorem}\label{sec:milnor-kervaire-theorem}

\todo{I'm not sure if I want to introduce the $J$-homomorphism / the results of Adams here, I may defer to the next section on stable homotopy theory.}

\begin{theorem}[Milnor-Kervaire]
	For any $k>1$, $bP^{4k}$ is a cyclic group of order
	\[
	|\bP^{4k}|=2^{2k-2}(2^{2k-1}-1)\varepsilon(k)|\coker =2^{2k-2}(2^{2k-1}-1)\varepsilon(k)\denom\left(\frac{4k}{B_{2k}}\right).
	\]
\end{theorem}

\begin{remark}
	It's interesting to note that the Eells-Kupier invariant is able to detect everything except for $\denom(4k/B_{2k})$. \todo{is this interesting? ask some professors}
\end{remark}
% This chapter will build up to the Atiyah-Singer index theorem -- a remarkable gem of 20th century mathematics which unifies \todo{todo}
% The index theorem is able to express the ``number'' of solutions to an elliptic system of differential equations on a manifold in terms of the topology of the manifold. This gives subtle integrability (i.e. being integer-valued) conditions for certain combinations of characteristic numbers.
%
% Starting with an introduction to the Hirzebruch signature theorem and $L$-genus, we'll construct (a generalization of) an invariant used by Milnor to detect the first exotic sphere \cite{milnor1956manifolds} in 1956.
% \todo{finish this}
%
% \begin{convention*}
% 	All manifolds are assumed to be oriented. Unless otherwise specified, $k\geq 1$.
% \end{convention*}
%
% \section{The Hirzebruch Signature Theorem}\label{sec:hirzebruch-signature-theorem}
%
% Recall that the intersection form of a closed $4k$-manifold $X$ is the symmetric bilinear form on $\H^{2k}(X)$ given by
% \[
% 	\lkxfunc{I_X}{\H^{2k}(X)^{\times 2}}{\R}{\alpha,\beta}{\int_X \alpha\smile \beta.}
% \]
% Since $I_X$ is a symmetric bilinear form over a finite dimensional real vector space, it is determined up to isomorphism by its signature, i.e. the difference in dimensions of its maximal positive definite and negative subspaces. 
%
% \begin{definition}
% 	The \defn{signature} $\sigma(X)$ of a closed $4k$-manifold $X$ is the signature of the form $I_X$.
% \end{definition}
%
% Given the structure of the cohomology ring, the signature of a manifold is straightforward to calculate. For instance:
% \begin{proposition}
% 	The signature of even-dimensional complex projective planes is $\sigma(\CP^{2m})=1$.
% \end{proposition}
% \begin{proof}
% 	The cohomology of $\CP^{2m}$ has the ring structure
% 	\[
% 		\H^\bullet(\CP^{2m}) = \R[\alpha]/(\alpha^{2m+1})\quad\textrm{where}\quad |\alpha|=2.
% 	\]
% 	Since the middle dimensional cohomology has dimension $\dim \H^{2m}(\CP^{2m}) = 1$ and $\CP^{2m}$ can be oriented so that $\int_{\CP^{2m}} \alpha^{2m} = 1$, it follows that the signature is $1$.
% \end{proof}
%
% \begin{remark}
% 	As a convention, if the dimension of a closed manifold $X$ is not divisible by $4$ we define the signature $\sigma(X)=0$ to be trivial.
% \end{remark}
%
% For more topological details on the signature and intersection form, we refer the reader to \cref{sec:signature} for a thorough review.
%
% \subsection*{Harmonic Differential Forms}
%
% There is a analytic interpretation to the 
%
% If $X_1$ and $X_2$
%
%
% \subsection*{The $L$-Genus}
%
%
%
% \section{Milnor's Invariant}\label{sec:milnors-invariant}
%
% \section{The Atiyah-Singer Index Theorem}\label{sec:atiyah-singer-index-theorem}
%
%%
% \section{Fancier Exotic Sphere Invariants}
%
% \begin{definition}
% 	
% \end{definition}
