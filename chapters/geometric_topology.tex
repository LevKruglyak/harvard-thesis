\providecommand{\HC}{\mathcal{M}}

\chapter{Geometric Topology}

\todo{generalized Poincare conjecture, classification problems, simple class of manifolds}

\begin{definition}
  A \defn{highly-connected manifold} is a $(n-1)$-connected closed $2n$-manifold.
\end{definition}

Let $\HC^{2n}$ denote the class of highly-connected manifolds. Note that this condition is purely homotopical.

If $n=1$, let's assume that $R=\Z_2$, otherwise we can choose any commutative coefficient ring.
\begin{proposition}
	A highly connected $2n$-manifold $M$ has cohomology
	\[
		\H^0(M; R) = R,\quad \H^n(M; R) = R\oplus \cdots \oplus R,\quad \H^{2n}(M; R)=R,
	\]
	and zero cohomology in all other dimensions.
\end{proposition}

\begin{proof}
\end{proof}

\todo{The homotopy theory of these manifolds}

\section{Intersection Theory}

\todo{The fundamental invariant}

For now, let's assume all manifolds are compact and closed.
Suppose $X$ is an oriented $n$-manifold, and $Y_1, Y_2$ are oriented $p,q$ submanifolds respectively.
% Up to oriented cobordism, 
Without loss of generality, we can assume that $Y_1\pitchfork Y_2$.

\begin{definition}
  The \defn{oriented intersection number} is $I_X(Y_1, Y_2)$.
\end{definition}

\begin{definition}
\end{definition}

\section{Compact Surfaces}

Let's now focus our attention to the case when $n=1$, i.e. $0$-connected closed $2$-manifolds or simply put -- connected closed surfaces. This class of manifolds admits a wonderfully elegant algebraic description, and showcases many common geometric techniques in a way that is easy to visualize.

Recall that 

Let $\Sigma$ be a connected compact surface. Unlike in the previous cases, we can't assume that $\Sigma$ is simply connected and hence orientable, in fact the only simply connected compact surface is the sphere.


\begin{theorem}[Classification of Compact Surfaces]
	The $\Z_2$ intersection form gives an isomorphism of commutative monoids $\HC^2\cong \UBil(\Z_2)$. 
\end{theorem}

	In particular, there is a presentation of $\HC^2$:
	\[
		\HC^2 = \langle A, B \mid A\oplus A\oplus A = A\oplus B\rangle\quad\textrm{where}\quad A = \begin{bmatrix}1\end{bmatrix}, B = \begin{bmatrix}0&1\\1&0\end{bmatrix}
	\]

\section{Integral Bilinear Forms}

\section{Morse Theory}

\section{Handlebodies}
