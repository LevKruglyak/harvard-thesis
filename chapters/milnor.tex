\chapter{Milnor Manifolds}\label{ch:milnor}

In this chapter, we'll see our first examples of exotic spheres and become familiar with some of the mathematical tools which will become ubiquitous in the later chapters. 
In fact, the construction here was historically the first example of such structure, described by John Milnor in his 1956 paper ``On manifolds homeomorphic to the seven sphere''. (see \cite{milnor1956manifolds}) At the time, Milnor wasn't trying to construct exotic spheres -- nobody had any reason to suspect they even existed since no exotic structures existed in lower dimensions. As with many discoveries across history, the finding of an exotic sphere was a sort of happy accident.

In the 1950s, Milnor was interested in understanding the topology of high dimensional manifolds. There are more than a few ways to get started. For one, you might start with the case of closed $n$-manifolds which are $(n-1)$-connected, i.e. the manifolds that have the homology type of an $n$-sphere. Even more restrictively, we might consider $n$-manifolds that have the exact homotopy type of an $n$-sphere. However, this simplicity of homotopy groups (and correspondingly, cohomology) gives very little data with which to define invariants. Furthermore, classifying such manifolds was tantamount to proving the generalized Poincar\'e conjecture -- something which Milnor did not know how to do at the time.

So let's work in the next simplest case -- for any $n\geq 2$ consider the class of manifolds: 
\[\mathscr{M}^{2n} = \left\{\textsl{closed $2n$-manifolds which are $(n-1)$-connected}\right\}\]
Let's suppose $M$ is a manifold of this type. By the Hurewicz theorem, it follows that for any commutative coefficient ring $R$ we have $\H_k(M; R)=0$ for $0<k<n$ and $\H_n(M; R)\cong \pi_n(M)\otimes R$ by an extra step with the universal coefficients theorem. Since $M$ is an orientable manifold it follows that $\H_{2n}(M; R)=R$, generated by the fundamental class $[M]_R$ corresponding to a chosen $R$-orientation. 
Finally, by Poincar\'e duality (which applies since $M$ is closed and orientable) and the universal coefficients theorem, we have isomorphisms $\H_n(M; R)\cong \H^n(M; R)\cong \Hom(\H_n(M; R), \Z)$. Thus $\H_n(M;R)$ is a free $R$-module, of finite rank because $M$ is compact. Continuing this in other dimensions, we get the symmetry $\H_k(M; R)\cong \H^k(M; R)$.

\begin{proposition}
Any manifold $M$ of $\mathscr{M}^{2n}$ type has cohomology:
\[
  \H^0(M; R) = R,\quad \H^n(M; R) = R\oplus \cdots \oplus R,\quad \H^{2n}(M; R)=R,
\]
with zero cohomology in all other dimensions. 
\end{proposition}

What sort of algebraic invariants can we build out of this cohomological data? One of the wonderful things about cohomology is that it comes with a natural multiplicative structure; the cup product. In particular, this gives us a bilinear form:
\begin{definition}\label{def:intersection-form}
  The \defn{intersection form} of a closed, $R$-orientable $2n$-manifold $M$ is the bilinear form
\[
  \lkxfunc{\omega_M}{\H^n(M;R)^{\otimes 2}}{R}{\alpha, \beta}{\langle\alpha\smile\beta, [M]_R\rangle.}
\]
If $n$ is even, then $\omega_M$ is a symmetric form, otherwise it is skew-symmetric. 
\end{definition}
This intersection form has since turned out to be a fundamental topological invariant for even dimensional manifolds -- even more so for $4k$-manifolds where the symmetry of $\omega_M$ lends itself to particularly nice constructions. For instance in 1982, Michael Freedman proved the following rather remarkable theorem which later won him the Fields medal: (see \cite{freedman1982})

\begin{theorem}
  The homeomorphism type of a manifold of $\mathscr{M}^4$ type is entirely determined by its intersection form and whether or not it admits a PL structure.
  \footnote{This obstruction to a PL-structure is the \defn{Kirby-Siebenmann class}, and should be thought of as an element $\kappa(M)\in \H^4(M;\Z/2)$. \todo{We'll discuss this in ??}}
\end{theorem}


As a CW complex, $M$.
\[\begin{tikzcd}
	{D^{2n}} & M \\
	{S^{2n-1}} & {S^n\vee\cdots\vee S^n}
	\arrow[from=1-1, to=1-2]
	\arrow[hook, from=2-1, to=1-1]
	\arrow["f", from=2-1, to=2-2]
	\arrow[from=2-2, to=1-2]
\end{tikzcd}\]

For a fuller historical account by Milnor of his original discovery of exotic spheres, see \cite{milnor2000exotic}.

\section{Spherical Fiber Bundles}

Let's 

\section{The \texorpdfstring{$\lambda$}{λ}-Invariant}

