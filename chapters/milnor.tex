\chapter{Milnor Manifolds}\label{ch:milnor}

In this chapter, we'll see our first examples of exotic spheres and become familiar with some of the mathematical tools which will become ubiquitous in the later chapters.
In fact, the construction here was historically the first example of such structure, described by John Milnor in his 1956 paper ``On manifolds homeomorphic to the seven sphere''. (see \cite{milnor1956manifolds}) At the time, Milnor wasn't trying to construct exotic spheres -- nobody had any reason to suspect they even existed since no exotic structures existed in lower dimensions. As with many discoveries across history, the finding of an exotic sphere was a sort of happy accident.\footnote{To hear the following historical exposition from the man himself, see \cite{milnor2000exotic}.}

\section{Milnor Manifolds}

In the 1950s, Milnor was interested in understanding the topology of high dimensional manifolds, and there are more than a few ways to get started. For one, we might start with the case of closed $n$-manifolds which are $(n-1)$-connected, i.e. manifolds that have the homotopy type of a sphere. This triviality of lower dimensional homotopy groups (and correspondingly, cohomology) gives us little readily available data with which to define invariants. Furthermore, classifying such manifolds is tantamount to proving the generalized Poincar\'e conjecture -- something which Milnor did not know how to do at the time.

So let's work in the next simplest case -- the class of manifolds:
\[\mathscr{M}^{2n} = \left\{\textsl{closed $2n$-manifolds which are $(n-1)$-connected}\right\}\]
% This class of manifolds is quite nice, and there have been many classification results

\begin{proposition}
	For $n\geq 2$, any manifold $M$ of $\mathscr{M}^{2n}$ type has cohomology:
	\[
		\H^0(M; R) = R,\quad \H^n(M; R) = R\oplus \cdots \oplus R,\quad \H^{2n}(M; R)=R,
	\]
	with zero cohomology in all other dimensions.
\end{proposition}

\begin{proof}
By the Hurewicz theorem, we have $\H_k(M; R)=0$ for $0<k<n$ and $\H_n(M; R)\cong \pi_n(M)\otimes R$ by an extra step with the universal coefficients theorem. Since $M$ is an orientable manifold it follows that $\H_{2n}(M; R)=R$, generated by the fundamental class $[M]_R$ corresponding to a chosen $R$-orientation.
Finally, by Poincar\'e duality (which applies since $M$ is closed and orientable) and the universal coefficients theorem, we have isomorphisms $\H_n(M; R)\cong \H^n(M; R)\cong \Hom(\H_n(M; R), \Z)$. Thus $\H_n(M;R)$ is a free $R$-module, of finite rank because $M$ is compact. Continuing this in other dimensions, we get isomorphisms $\H_k(M; R)\cong \H^k(M; R)$ so applying the earlier computation of the homology we are done.
\end{proof}

The rank of $\H^n(M)$, i.e. the middle Betti number $\beta_n$, is one of the obvious invariants we could consider. Up to homotopy, $M$ can be build up as a CW complex by starting with a point, attaching $\beta_n$ cells of dimension $n$, and then attaching a single $2n$-cell by some attaching map $f \in \pi_{2n-1}(S^n\vee \cdots \vee S^n)$. In other words $M$ is, up to homotopy, the pushout:
\[\begin{tikzcd}
		{D^{2n}} & M \\
		{S^{2n-1}} & {S^n \vee \cdots \vee S^n}
		\arrow[from=1-1, to=1-2]
		\arrow[hook, from=2-1, to=1-1]
		\arrow["f", from=2-1, to=2-2]
		\arrow[from=2-2, to=1-2]
	\end{tikzcd}\]
	When $\beta_n=0$, the manifold is a homotopy sphere -- a case we're not ready to handle just yet. Does anything interesting happen when $\beta_n=1$? Here, the homotopy type of $M$ is determined by an attaching map $f\in \pi_{2n-1}(S^n)$. 

\todo{explain the full classification of $\mathscr{M}^{2n}$}

\section{The Intersection Form}

\todo{link}

What sort of algebraic invariants can we build out of this cohomological data? One of the wonderful things about cohomology is that it comes with a natural multiplicative structure; the cup product. Composing this cup product with the Poincar\'e duality isomorphism $\H^{2n}(M; R) \to R$ gives us a non-degenerate bilinear form:
\begin{definition}\label{def:intersection-form}
	The \defn{intersection form} of a closed, $R$-orientable $2n$-manifold $M$ is the bilinear form
	\[
		\lkxfunc{\omega_M}{\H^n(M;R)\times \H^n(M; R)}{R}{(\alpha,\beta)}{\langle\alpha\smile\beta, [M]_R\rangle.}
	\]
\end{definition}
\begin{note*}
	Unless otherwise specified, we'll assume $R=\Z$ when $M$ is orientable and $R=\Z/2$ otherwise.
\end{note*}

\todo{geometric intuition}

Since the cup product is graded commutative, i.e. $\alpha\smile \beta = (-1)^{|\alpha||\beta|} \beta\smile\alpha$, it follows that $\omega_M$ is symmetric for even $n$, and skew-symmetric for odd $n$. As an invariant, the intersection form turns out to be far more useful in the former case than in the latter.
For our class of manifolds $\mathscr{M}^{2n}$, we can take $R=\Z$ and view $\omega_M$ as a bilinear form on the free $\Z$-module $\H^n(M)$. In this case, \emph{any} skew-symmetric bilinear form on this module can be put into canonical symplectic form
\[
	\omega_M(x_i, x_j) = \begin{cases}\delta_{ij} & \textrm{if }i \leq j, \\ -\delta_{ij} & \textrm{if }i > j\end{cases} \quad\implies\quad \omega_M =\begin{pmatrix}0 & I_\ell \\ -I_\ell & 0\end{pmatrix}
\]
for some free basis $\{x_i\}^\ell_{i=1}$ of $\H^n(M)$. In other words, the only interesting information which the intersection form can tell us about a manifold $M$ of $\mathscr{M}^{2n}$ type is the rank of $\H^n(M)$ -- something which, presumably we would already know.
Things get more interesting in the symmetric case. The classification of symmetric bilinear forms is closely related to the classification of quadratic forms, (one-to-one actually aside from characteristic $2$) which is far from trivial.

\begin{definition}
	The \defn{signature} of an oriented $4n$-manifold is the difference in the number of positive and negative eigenvalues of $\sigma_M$.
\end{definition}

This intersection form has since turned out to be a fundamental topological invariant for even dimensional manifolds -- even more so for $4n$-manifolds where the symmetry of $\omega_M$ lends itself to particularly nice constructions.

For instance, in 1982, Michael Freedman proved the following remarkable classification theorem which later won him the Fields medal: (see \cite{freedman1982})

\begin{theorem}
	The homeomorphism type of a manifold of $\mathscr{M}^4$ type is entirely determined by its intersection form and whether or not it admits a PL structure.\footnote{The obstruction to a PL structure is the \defn{Kirby-Siebenmann class}, an element $\kappa(M)\in \H^4(M;\Z/2)$ which vanishes if $M$ admits a PL structure. \todo{We'll discuss this in ??}}
\end{theorem}

\section{Compact Surfaces}

Let's take a brief \todo{continue}

\begin{definition}
  Let $k$ be a field of characteristic $2$.
  The \defn{Arf invariant} of a quad
\end{definition}

\begin{theorem}[Classification of Compact Surfaces]
	The intersection form is an isomorphism of commutative monoids
	\[
		\mathscr{M}^2\lkxisom \cat{Bil}(\F_2)
	\]
\end{theorem}
Equivalently, the homeomorphism classes of compact connected surfaces without boundary are:
\[
	S^2,\quad T\#\cdots \# T,\quad\textrm{and}\quad\P^2\#\cdots \# \P^2
\]

% \begin{proposition}
%   
% \end{proposition}

\section{Spherical Fiber Bundles}

% In retrospect, looking at 

\section{The \texorpdfstring{$\lambda$}{λ}-Invariant}

