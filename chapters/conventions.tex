\chapter*{Conventions}

\subsection*{General}

\begin{itemize}
  \item Definitions of terms will be formatted {\color{blue}\emph{blue and emphasized}}, and hyperlinks will be formatted just blue (e.g. \cref{fig:first}).
  \item We use $\cong$ instead of $\oldcong$ to denote isomorphisms, diffeomorphisms, etc.
\end{itemize}

\subsection*{Differential Topology}
\begin{itemize}
  \item Unless otherwise specified, all manifolds are assumed to be smooth, connected, and possibly with boundary. A \defn{closed}[closed manifold] manifold is a compact manifold with empty boundary.
  \item When introducing a manifold, we often put its dimension as a superscript. For example, ``let $M^n$ be a manifold'' should be read as ``let $M$ be an $n$-dimensional manifold''.
  \item The reverse orientation of a manifold $M$ is denoted by $\overline{M}$.
  \item We assume that all submanifolds $N\subset M$ are properly embedded and neat, i.e.
    \vspace{-0.5em}
    \begin{itemize}
      \item the inclusion $\iota : N \to M$ is a proper map,
      \item $\partial N\subset \partial M$,
      \item the boundary $\partial N$ intersects $\partial M$ transversally.
    \end{itemize}
\end{itemize}

\subsection*{Differential Geometry}
\begin{itemize}
  \item Vector bundles are taken over the fields $\F=\R$ or $\C$, by default assumed to be $\R$.

  \item The total space of a vector bundle is denoted with normal math font, i.e. $E \to X$, and the bundle itself is denoted with calligraphic font, i.e. $\mathcal{E} : E \to X$. 

  \item The tangent bundle of a manifold $M$ is denoted $\TT M$, with total space $\T M$.

  \item As with manifolds, a superscript $\mathcal{E}^k$ on a vector bundle denotes its rank. 

  \item Given a Riemannian inner product structure $\langle-,-\rangle$ on a vector bundle $\mathcal{E}^k$, we let $\S(\mathcal{E})$ be the associated sphere bundle (with fibers $S^{k-1}$) and $\D(\mathcal{E})$ the associated disk bundle (with fibers $D^k$). The total spaces of these bundles are
  \[
    \S(E) = \{ \xi\in E \mid \langle \xi, \xi\rangle =1 \}
    \quad\textrm{and}\quad
    \D(E) = \{ \xi\in E \mid \langle \xi, \xi\rangle\leq 1 \}
  \]
  respectively. Note that $\partial \D(E) = \S(E)$ as manifolds.
\end{itemize}
