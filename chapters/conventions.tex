\chapter*{Conventions}

\subsection*{General}

\begin{itemize}
  \item Definitions of terms will be formatted {\color{blue}\emph{blue and emphasized}}, and hyperlinks will be formatted just blue (e.g. \cref{fig:triangles}).
  \item We use $\cong$ instead of $\oldcong$ to denote isomorphisms, diffeomorphisms, etc.
\end{itemize}

\subsection*{Algebra}

\begin{itemize}
  \item A \defn{graded ring} is a ring $A$ equipped with a decomposition of its underlying group as $A=\bigoplus_{k\in \Z_{\geq 0}} A[k]$ such that the ring multiplication sends $A[k_1]\times R[k_2] \to A[k_1+k_2]$. Elements belonging to $A[k]$ are said to be \defn{homogeneous of degree $k$}[homogeneous element of a graded ring]. 
  \item A graded ring is said to be \defn{graded-commutative} if for any homogeneous elements $x\in A[k_1]$ and $y\in A[k_2]$, we have
    \[
      x\cdot y = (-1)^{k_1k_2} y\cdot x
    \]
  \item The \defn{completion}[completion of a graded ring] of a graded ring $A$ is the direct product $\widehat{A} = \prod_{k\geq 0} A[k]$.
\end{itemize}

\subsection*{Differential Topology}
\begin{itemize}
  \item A \defn{topological manifold} $M$ is a locally Euclidean second-countable Hausdorff space. \todo{boundary}
  \item A \defn{smooth structure} $\mathscr{S}$ on a topological manifold $M$ is a collection of open charts $\mathscr{S}=\{(U_\alpha, \varphi_\alpha)\}_{\alpha\in I}$ such that the transition functions 
	\begin{equation}\label{eq:transition-function}
		\lkxfunc{g_{\alpha\beta}}{\varphi(U_\alpha\cap U_\beta)}{\R^n\textrm{ or } \R^{n-1}\times [0,\infty)}
	\end{equation}
	are smooth for all $\alpha,\beta\in I$. We require that $\mathscr{S}$ be maximal with respect to this property, i.e. the addition of any chart $(U,\varphi)$ not in $\mathscr{S}$ breaks the smoothness of \cref{eq:transition-function}.
  \item Unless otherwise specified, all manifolds are assumed to be smooth, connected, and possibly with boundary. A \defn{closed}[closed manifold] manifold is a compact manifold with empty boundary.
  \item When introducing a manifold, we often put its dimension as a superscript. For example, ``let $M^n$ be a manifold'' should be read as ``let $M$ be an $n$-dimensional manifold''.

  \item We assume that all submanifolds $N\subset M$ are properly embedded and neat, i.e.
    \vspace{-0.5em}
    \begin{itemize}
      \item the inclusion $\iota : N \to M$ is a proper map,
      \item $\partial N\subset \partial M$,
      \item the boundary $\partial N$ intersects $\partial M$ transversally.
    \end{itemize}
\end{itemize}

\subsection*{Algebraic Topology}
\begin{itemize}
  \item Given a coefficient ring $R$, $\H^i(-; R)$ denotes the singular cohomology with coefficients in $R$, and $\H_i(-;R)$ denotes singular homology with coefficients in $R$. $\H^\bullet(-;R)$ is the singular cohomology ring.
  \item $\HdR$ denotes de-Rham cohomology, and $\Hc$ denotes compactly-supported de-Rham cohomology.
  \item Whenever we refer to a general (co)homology theory $h$, we mean one of the pairs:
    \begin{itemize}
      \item Singular (co)homology with coefficients in $R=\Z, \Z/2, \Z[1/2],$ or $\Q$.
      \item Compactly supported de-Rham cohomology and de-Rham cohomology.
    \end{itemize}
    The homology groups are denoted $h_i(-)$ and the cohomology groups are denoted $h_i(-)$. Note that all pairs 
\end{itemize}

\subsection*{Differential Geometry}
\begin{itemize}
  \item Vector bundles are over a field $\F=\R$ or $\C$, by default assumed to be $\R$.

  \item By default, we denote the total space of a vector bundle with normal math font, i.e. $E \to X$, and denote the bundle itself with calligraphic font, i.e. $\mathcal{E} : E \to X$. 

  \item The tangent bundle of a manifold $M$ is dented $\TT M$, with total space $\T M$.

  \item As with manifolds, a superscript $\mathcal{E}^k$ on a vector bundle denotes its rank. 

  \item Given a Riemannian inner product structure $\langle-,-\rangle$ on a vector bundle $\mathcal{E}^k$, we let $\S(\mathcal{E})$ be the associated sphere bundle (with fibers $S^{k-1}$) and $\D(\mathcal{E})$ the associated disk bundle (with fibers $D^k$). The total spaces of these bundles are
  \[
    \S(E) = \{ \xi\in E \mid \langle \xi, \xi\rangle =1 \}
    \quad\textrm{and}\quad
    \D(E) = \{ \xi\in E \mid \langle \xi, \xi\rangle\leq 1 \}
  \]
  respectively. Note that $\partial \D(E) = \S(E)$ as manifolds.
\end{itemize}
