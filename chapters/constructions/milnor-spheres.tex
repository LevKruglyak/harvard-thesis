\section{Milnor Spheres}\label{sec:milnor-spheres}

We will begin with the classical construction of exotic spheres as total spaces of bundles over a sphere, starting with Milnor's original construction of an exotic 7-sphere in \cite{milnor1956manifolds}.
To see how spheres can arise as total spaces of bundles over a sphere, we will begin by reviewing the Hopf bundles -- bundles which express ordinary spheres of dimensions $1$, $3$, $7$, and $15$ as sphere bundles over $S^1$, $S^2$, $S^4$, and $S^8$ respectively. By ``twisting'' Hopf bundles in sufficiently high dimension, we get exotic spheres, although such a construction only works in these specific dimensions. This construction does admit a partial generalization by means of ``plumbing'' a string of bundles. We discuss this construction in the following \cref{sec:plumbing}.

The simplest Hopf bundle is the real Hopf bundle, which expresses the circle $S^1$ as the total space of an $S^0$ bundle over $S^1$. Topologically, this is the double cover of $S^1$. 

Consider the projection map
\[
	\lkxfunc{\pi_\R^n}{S^n}{\RP^n}{(x_1,\ldots, x_{n+1})}{[x_1:\cdots: x_{n+1}]}
\]
sending a point $p$ on a sphere $S^n\subset \R^{n+1}$ to the line passing through $p$ and the origin. This is a double cover since antipodal points on a sphere lie on the same line -- and in fact $S^n$ is the universal covering space of $\RP^n$ when $n>1$. The fibers of this map are thus the zero-dimensional sphere $S^0$. It is useful to interpret these fibers as the Lie group $\O_1\cong \Z/2$ consisting of norm-preserving linear transformations of a real line.

When $n=1$, there is an isomorphism $\RP^1\approx S^1$ so $\pi_\R^1$ is the $S^0$-bundle
\[
	S^0 \lkxto S^1\lkxto S^1.
\]
This is the \defn{real Hopf bundle}[Hopf bundle (real)].
Note that real Hopf bundle $\pi_\R^1$ can be interpreted as the boundary of the compact M\"obius bundle $M$, which is a $D^1$-bundle over $S^1$.

Advancing from $\R$ to $\C$, we next consider the projection map
\[
	\lkxfunc{\pi_\C^n}{S^{2n+1}}{\CP^n}{(z_1,\ldots, z_{n+1})}{[z_1:\cdots: z_n]}
\]
where we interpret $S^{2n+1}\subset \R^{2n+2}$ as a subset of $\C^{n+1}$. Rather than send a point to a real line, the map $\pi_\C^n$ sends a point $p\in \C^{n+1}$ on the complex unit sphere to the complex plane passing through the origin and $p$. Note that changing the ``phase'' of the point $p$ along the plane keeps the point in the plane. The fibers thus consist of these phase changes, i.e. the Lie group $\U_1=\{e^{i\theta} : \theta\in \R\}$ which is diffeomorphic to $S^1$.

When $n=1$, there is an isomorphism of the Riemann sphere $\CP^1\cong S^2$ by stereographic projection. This gives the us the infamous \defn{complex Hopf bundle} $\pi_\C^1$
\[
	S^1 \lkxto S^3 \lkxto S^2.
\]

\begin{remark*}
	Unlike the real and complex numbers, the ring of quaternions $\HH$ is not a field since it is not commutative. We define the quaternionic projective space $\HP^{n}$ as the quotient of $\HH^{n+1}$ by {\normalfont\emph{right}} multiplication by $\HH$.
\end{remark*}

In the final case of the quaternionic plane $\HH$, we have a projection map
\[
	\lkxfunc{\pi_\HH^n}{S^{4n+3}}{\HP^n}
\]
where we consider $S^{4n+3}\subset \HH^{n+1}$. This bundle with fibers given by the set of unit quaternions $S^3\subset \HH$.
In the case $n=1$, the difeomorphism $\HP^1\cong S^3$ gives the \defn{quaternionic Hopf bundle}[Hopf bundle (quaternionic)]
\[
	S^3 \lkxto S^7 \lkxto S^4.
\]
As before, the fiber $S^3$ should be interpreted as the Lie group $\Sp_1$ of unit quaternions. These act as quaternionic phase shifts.

\begin{remark*}
	There is a fourth Hopf bundle arising from the octonionic projective plane, taking the form $S^7\to S^{15}\to S^8$. A comprehensive introduction can be found in \cite{baez2002octonions}.
\end{remark*}

Now let $\F\in\{\R, \C, \HH\}$, and $d=\dim_\R \F$ be the real dimension. The Hopf bundles can then be denoted $\pi_\F^1 : S^{2d-1} \to S^d$ with fibers $S^{d-1}$.

\begin{proposition}\label{prop:boundary-of-hopf-bundle}
	The space $\FP^2\setminus \Int(D^{2d})$, i.e. a projective plane with an open disk removed, is a $D^d$-bundle over $S^d$ which bounds the Hopf bundle $\pi_\F^1$.
\end{proposition}
\begin{proof}
	First, note that $\FP^2$ can be obtained by identifying $\partial D^{2d}$ with $S^d$ by the Hopf bundle. This follows from a more general CW structure decomposition of $\FP^n$, found in any introductory topology text.
	\begin{lemma}
		There is a CW structure on $\FP^n$ in which the $(kd)$-skeleta are $\FP^k$ and the attaching maps are $\pi_\F^k$.
	\end{lemma}

	To see why $\FP^2\setminus \Int(D^{2d})$ is a disk bundle, note that we can consider $D^{2d}$ as the subset of points $(x,y)\in \F^2$ with $\|x\|^2+\|y\|^2\leq 1$. We can assume that $\Int(D^{2d})$ is embedded into this disk $D^{2d}$ as a disk of radius $1/2$. Under this embedding, $\FP^2\setminus \Int(D^{2d})$ is the quotient of the annulus $A(1/2,1)$ in $\F^2$ along its outside boundary by the Hopf bundle map.

	If we parametrize a line $L$ by $(\alpha t, \beta t)$ for fixed $\alpha,\beta\in S^d\subset \F$ and $t\in \F$, then the intersection of the line with the annulus $L\cap A(1/2,1)$ is diffeomorphic to an annulus in $\F$. Since the entire outer boundary of this annulus is collapsed to a point by the Hopf map, this annulus is diffeomorphic to the disk $D^d$ in $\F$.
\end{proof}

\begin{figure}[ht]
	\centering
	\import{diagrams}{projective-plane-bundle.pdf_tex}
	\caption{$\RP^2\setminus \Int(D^2)$ as a $D^1$-bundle over $S^1$.}\label{fig:mobius-bundle-and-hopf-bundle}
\end{figure}

\begin{remark*}\label{rmk:lie-group-structure-Sd}
	When $d\in \{0,1,3\}$ any bundle with fibers $S^d$ can be interpreted as the boundary of a $D^{d+1}$ bundle. Since each of these spheres carries a Lie group structure\,\footnote{In fact, these are the only spheres which admit a Lie group structure.} which admits an effective action on $\F$, given an $S^d$ bundle $\pi : E \to B$ we can associate an $\F$-vector bundle $\mathcal{E} : E\times_{S^d} \F \to B$ which has $\pi$ as an associated sphere bundle. If we act on $D^d\subset \F$ instead, we get a disk bundle bounding $\pi$.
\end{remark*}

In the coming section, we generalize the Hopf bundles, twisting the bundle just enough so that the total space has a non-standard smooth structure but not so much that it ceases to be homeomorphic to a sphere.
Of course, this will lose some of the elegant and canonical geometric structure of the Hopf bundles.

\subsection{Sphere Bundles and the Euler Class}

Now we depart from the geometric comfort of the Hopf bundles and suppose we had a general oriented sphere bundle of the form
\begin{equation}\label{eq:general-spherical-fibration}
	S^{n-1}\lkxto M \lkxto S^n.
\end{equation}
When will $M$ be homotopy equivalent to a sphere? There are a few ways to proceed here -- for instance we could use the homology Serre spectral sequence and the Hurewicz map in order to compute the homology of $M$ from the data of the attaching map $\delta : \pi_d(S^d) \to \pi_{n-1}(S^{n-1})$. 
A simpler approach more in line with the theme of this thesis involves the Euler class, which we discussed in \cref{sec:euler-class}.

In the case of the bundle of \cref{eq:general-spherical-fibration}, the Gysin sequence gives us exact sequences of the form
\[
	\cdots \lkxto \H^\ell(S^n) \lkxto \H^{\ell}(M) \lkxto \H^{\ell - n+1}(S^n) \lkxto[e\,\smile] \H^n(S^n) \lkxto \cdots
\]
When $\ell$ is not $0,n-1,n,$ or $2n-1$, the terms $\H^\ell(S^n)$ and $\H^{\ell - n+1}(S^n)$ vanish, implying that $\H^\ell(M)$ is trivial. As a connected oriented $(2n-1)$-manifold, we know that $\H^0(M)$ and $\H^{2n-1}(M)$ are both isomorphic to $\Z$. In the problematic case of $\ell = n-1$ or $n$, note that we have the exact sequence
\[
	0 \lkxto \H^{n-1}(M) \lkxto \H^0(S^n) \lkxto[e\,\smile] \H^n(S^n) \lkxto \H^n(M) \lkxto 0
\]
Here $H^0(S^n)$ and $H^n(S^n)$ are both isomorphic to $\Z$, with the Euler class acting as a multiplication map. It follows that terms $\H^{n-1}(M)$ and $\H^n(M)$ vanish if and only if multiplication by the Euler class is an isomorphism. This forces the Euler number to be $\pm 1$.

\begin{proposition}\label{prop:homotopy-type-spherical-bundle}
	The total space of a bundle $S^{n-1} \to M \to S^n$ is a homotopy sphere if and only if the Euler number is $e=\pm 1$.
\end{proposition}

\begin{remark*}
	There is a nice geometric example of the condition $e=\pm 1$ for the Hopf bundle.

	The crux of this is the interpretation of the Euler number as a self-intersection number of a submanifold $N\subset M$.
	In the case of the complex Hopf bundle, \cref{prop:boundary-of-hopf-bundle} tells us that $\CP^2\setminus \Int(D^4)$ is a $D^2$-bundle over $S^3$ which bounds the complex Hopf bundle. Let us call the total space $W=\CP^2\setminus \Int(D^4)$, and denote the Hopf bundle $\pi : S^3 \to S^2$, with $\overline{\pi} : W \to S^2$ the bounding bundle.
	\[
		\begin{tikzcd}
			D^2 & W & \\
			& & S^2\\
			S^1 & S^3 &
			\arrow["\partial"', from=1-1, to=3-1]
			\arrow["\partial"', from=1-2, to=3-2]
			\arrow["\pi"', from=3-2, to=2-3]
			\arrow["\overline{\pi}", from=1-2, to=2-3]
		\end{tikzcd}
	\]
	While $\pi$ certainly does not have a section, $\overline{\pi}$ does -- for one thing, $0\in D^2$ is a fixed point of the $\U_1$ action on $D^2$ so the zero-section is perfectly valid.

	Next, note that removal of the open $4$-disk from $\CP^2$ does not affect the lower dimensional homology groups of $\CP^2\setminus \Int(D^4)$. In particular, the intersection form remains unchanged.
	Furthermore, the embedding by the zero-section of $\overline{\pi}$ of $S^2\subset W$ generates the middle-dimensional homology $\H^2(W)$. The complex line bundle associated to the Hopf bundle $\pi$ is then a tubular neighborhood of $S^2\subset W$ and so the Euler number of the Hopf bundle is exactly the self-intersection of $S^2$ in $W$. However, the intersection form of $\CP^2$ is the same as the intersection form of $W$, namely $(1)$ by \cref{prop:intersection-form-complex-projective-plane}. This means that the Euler class of the Hopf bundle is $\pm 1$ which agrees with \cref{prop:homotopy-type-spherical-bundle}.
\end{remark*}

\subsection{Milnor's Exotic Spheres in 7-Dimensions}

Using \cref{prop:homotopy-type-spherical-bundle}, we now attempt to twist the Hopf bundle while ensuring that the homotopy type of total space remains $S^7$. The Lie group structure of $S^3$ means that oriented $S^3$ bundles are in bijection with oriented rank $4$ vector bundles. 
By the clutching construction (see \cite{hatcher2003ktheory}), oriented rank $4$ vector bundles over $S^4$ are classified by the homotopy group $\pi_3(\SO_4)$ so we begin with its computation.

\begin{proposition}\label{prop:3rd-homotopy-SO4}
	There is an isomorphism $\pi_3(\SO_4)\cong \Z\oplus \Z$.
\end{proposition}
\begin{proof}
	Consider action of $\Sp_1\times \Sp_1$ on the quaternionic plane $\HH$ by conjugation\footnote{This action is important for the construction of Milnor spheres, so choosing $x\mapsto q_1xq_2$ or $x\mapsto q_1xq_2^{-1}$ for the action has downstream effects in the formulas. This is purely a convention, and we use the former one to agree with the original exposition of Milnor in \cite{milnor1956manifolds}.}
	\begin{equation}\label{eq:sp1xsp1-conjugation}
		\lkxfunc{p}{\Sp_1\times \Sp_1}{\Aut(\HH)}{(q_1,q_2)}{(x\mapsto q_1xq_2).}
	\end{equation}
	Since conjugation by quaternions of unit norm preserves the quaternionic norm, this action defines a map $p' : \Sp_1\times \Sp_1\to\SO_4$.
	Note that the kernel of $p'$ is clearly $\{(1,1),(-1,1)\}\cong \Z/2$. Both Lie groups are connected, with $\Sp_1\times \Sp_1$ of dimension $3+3=6$ and $\SO_4$ of dimension $4(4-1)/2=6$ so it follows by a dimension argument that $p'$ must be surjective. Therefore, $p'$ is a covering map and so induces an isomorphism on homotopy groups $\pi_n$ for $n>1$. We thus have a sequence of isomorphisms.
	\[
		\pi_3(\SO_4)\cong \pi_3(\Sp_1\times \Sp_1)\cong \pi_3(\Sp_1)\times \pi_3(\Sp_1) \cong \Z\oplus \Z.
	\]
	This completes the proof.
\end{proof}

Note that the isomorphism $\Z\cong \pi_3(\Sp_1)$ sends an integer $k\in \Z$ to the $k$-power map $\Sp_1\to \Sp_1$ sending $q\mapsto q^k$. Here we identify the usual domain of $S^3$ with $\Sp_1$. By the action in \cref{eq:sp1xsp1-conjugation}, this means that any $(i,j)\in \Z\oplus\Z$ can be identified with the map
\begin{equation}\label{eq:ij-associated-map}
	\lkxfunc{}{\Sp_1}{\SO_4}{q}{(x\mapsto q^ixq^{j})}\in \pi_3(\SO_4).
\end{equation}
\begin{definition}
	For each $(i,j)\in \pi_3(\SO_4)$, the \defn{Milnor manifold $\Sigma_{i,j}^7$}[Milnor manifold (7-dimensional)] is the total space of the bundle associated to the clutching function $(i,j)$. Let $p_{i,j} : \Sigma_{i,j}^7 \to S^4$ be the bundle map.
\end{definition}

To determine when $\Sigma_{i,j}^7$ is homeomorphic to a sphere, we calculate the Euler number of the bundle corresponding to a pair $(i,j)\in \Z\oplus\Z$. We begin with a useful general lemma.

\begin{lemma}\label{lemma:euler-number-clutching-construction}
	For even $m$, let $p : \SO_m \to S^{m-1}$ be the map sending an isometry $T\in \SO_m$ to $T(x)\in S^m$ for a fixed point $x$. As a group homomorphism $\pi_{m-1}(\SO_m)\to \Z$, the Euler number is the map on $\pi_{m-1}$ induced by this map. In other words, the triangle commutes:
	\[
		\begin{tikzcd}
			\pi_{m-1}(\SO_m) & \pi_{m-1}(S^{m-1})\\
			\Z &\\
			\arrow["e"', from=1-1, to=2-1]
			\arrow["p_*", from=1-1, to=1-2]
			\arrow[from=1-2, to=2-1]
		\end{tikzcd}
	\]
\end{lemma}
\begin{proof}
	See Section 1 of \cite{levine1985lectures}.
\end{proof}

\begin{proposition}\label{prop:euler-number-of-milnor-manifold}
	The Euler number of the bundle associated to $(i,j)\in \Z\oplus \Z$ is $i+j$.
\end{proposition}
\begin{proof}
	By \cref{lemma:euler-number-clutching-construction}, the Euler number of a bundle associated to a clutching function in $\pi_3(\SO_4)$ is a group homomorphism
	\[
		\lkxfunc{e}{\pi_3(\SO_4)}{\pi_3(S^3)\cong \Z}
	\]
	induced by the projection map $p : \SO_4 \to S^3$ sending an isometry $T$ to $T(x)$ for some fixed point $x$.
	If we identify $\Sp^1\cong S^3$, and choose the point to be $x=1$, the map $p_*(i,j)$ sends
	\[
		\begin{array}{r@{\;}c@{\;}c@{\;}r@{\;}r@{\;}r}
			\Sp_1 & \lkxto     & \SO_4              & \lkxto     & \Sp^1    \\
			q     & \lkxmapsto & (x\mapsto q^ixq^j) & \lkxmapsto & q^{i+j}.
		\end{array}
	\]
	This map has degree $i+j\in\pi_3(S^3)$, which is the Euler number.
\end{proof}

\begin{proposition}\label{prop:milnor-spheres-homeomorphic-to-spheres}
	When $i+j=\pm 1$, the Milnor manifold $\Sigma_{i,j}^7$ is homeomorphic to $S^7$.
\end{proposition}

This follows immediately from the $h$-cobordism theorem since $\Sigma_{i,j}^7$ has the homotopy type of $S^7$ by \cref{prop:homotopy-type-spherical-bundle}. However at the time of Milnor's construction, the $h$-cobordism had not been proven yet so he used a Morse theoretic argument. To avoid over-reliance on powerful results, we will also directly prove homeomorphism using Reeb's theorem (\cref{thm:reeb}), i.e. we construct a function $f : \Sigma_{i,j}^7 \to \R$ with exactly two non-degenerate critical points.

\begin{proof}[Proof of \cref{prop:milnor-spheres-homeomorphic-to-spheres}]
	Identifying $S^3\subset \HH$ with the set of unit norm quaternions, the clutching function $S^3 \to \Aut(\HH)$ corresponding to $(i,j)$ is $v\mapsto u^ivu^j$. Consider the charts
	\[
		U_0 = \{ [u_0:1] \mid u_0\in \HH\}\quad\textrm{and}\quad U_\infty = \{ [1:u_\infty] \mid u_\infty\in \HH\}
	\]
	on the sphere $S^4=\HP^1$. Note that the change of coordinate map is $u_\infty=u_0^{-1}$.Then, the Milnor sphere is homeomorphic to the quotient
	\[
		\Sigma^7_{i,j} = (U_0 \times S^3)\cup_h (U_\infty\times S^3)
	\]
	where $h$ is the map identifying $(u_0,v_0)\in U_0\times S^3$ with
	\[
		h(u_0,v_0)= \left(u_0^{-1}, \frac{u_0^iv_0u_0^j}{\|u_0\|^{i+j}}\right) = \left(u_0^{-1}, \frac{u_0^i (v_0u_0) u_0^{-i}}{\|u_0\|}\right) = (u_\infty, v_\infty),
	\]
	where the second equality follows from $i+j=1$. The remainder of the proof does not change if $i+j=-1$ absent some sign changes, so we assume without loss of generality that $i+j=1$.

	Note that by the clutching construction, $h$ acts as a change of coordinate function between the two ``hemispheres'' $U_0\times S^3$ and $U_\infty\times S^3$.
	Now, consider the real function $f : U_\infty\times S^3\to \R$ defined by
	\[
		f(u_\infty, v_\infty) = \frac{\Re(v_\infty)}{\sqrt{1+\|u_\infty\|^2}}.
	\]
	Under a change of coordinates, this function is
	\[
		f(u_\infty, v_\infty) = \frac{\Re(u_0^i(v_0u_0)u_0^{-i})/\|u_0\|}{\sqrt{1+\|u_0^{-1}\|^2}} = \frac{\Re(v_0u_0)}{\sqrt{\|u_0\|^2+1}}=f(u_0,v_0),
	\]
	where we use the identity $\Re(yxy^{-1})=\Re(x)$ in $\HH$. This shows that $f$ is well-defined on both charts and can thus be extended to a smooth function $f : \Sigma^\infty_{i,j} \to \R$.

	To compute the derivatives of the function $f$, let us use real coordinates. We can write $u_0=x_0+x_1\bm{i}+x_2\bm{j}+x_3\bm{k}$, $v_0 = y_0+y_1\bm{i}+y_2\bm{j}+y_3\bm{k}$, and similarly write $u_\infty$ and $v_\infty$ but with $x'$, $y'$, we have
	\[
		f(u_0, v_0) = \frac{x_0y_0 - x_1y_1 - x_2y_2 - x_3y_3}{\sqrt{1+x_0^2+x_1^2+x_2^2+x_3^2}},\quad
		f(u_\infty, v_\infty) = \frac{x_0'}{\sqrt{1+x_0'^2+x_1'^2+x_2'^2+x_3'^2}}
	\]
	It follows after elementary differentiation that the only critical points are $(u_0, v_0)= (0,\pm 1)$, and that $f$ is non-degenerate at these points. Therefore, $f$ is a Morse function with exactly two critical points so by Reeb's theorem (\ref{thm:reeb}), $\Sigma_{i,j}^7$ is homeomorphic to a sphere.
\end{proof}

\begin{proposition}
There is a diffeomorphism $\overline{\Sigma}_{i,j}=\Sigma_{-i,-j}$. In particular, we can always assume without loss of generality that $i+j=1$.
\end{proposition}

We now can compute the invariants of \cref{sec:invariants-for-homotopy-4k-1-spheres} for Milnor's exotic spheres -- this will tell us if we were successful in our task. First, let $\overline{\pi}_{i,j} : W_{i,j}^8 \to S^4$ be the associated $D^4$ bundle which bounds $\pi_{i,j}$. This coboundary $W_{i,j}^8$ of $\Sigma^7_{i,j}$ is essential for computing invariants.

\begin{proposition}
	For any $(i,j)\in \Z\oplus \Z$ with $i+j=\pm 1$, we have
	\[
		\sigma(W_{i,j}) = i+j\quad\textrm{and}\quad p_1^2[W_{i,j}, \Sigma_{i,j}]=4(i-j)^2.
	\]
\end{proposition}
\begin{proof}
	We will begin by computing the signature. Firstly, note that there is a deformation retraction of $W_{i,j}$ onto $S^4$ by contraction of the disk fibers. Consequently, the middle dimensional homology $H_2(W_{i,j})$ is generated by the image of the zero section of $\overline{\pi}_{i,j}$. The intersection form of $W_{i,j}$ must be multiplication by the Euler number $(i+j)$, but since $i+j=\pm 1$, the signature is equal to $\mathrm{sgn}(i+j)=i+j$.

	For the Pontryagin numbers, we note that just as with the Euler number, the map $p_1: \pi_3(\SO_4) \to \H^4(W_{i,j})$ is a group homomorphism and hence linear in $(i,j)$. Under the reflection $(i,j)\mapsto (-i,-j)$, the map is inverted so it follows that $p_1=\pm C(i-j)\alpha$ for some constant factor $C$. To obtain this factor, note that when $(i,j)=(1,0)$ then $W_{i,j}$ is $\CP^2\setminus \Int(D^4)$. Since the top-dimensional Pontryagin class of an even-dimensional bundle is the square of the Euler class, it follows that $C=\pm 2$ and so we are done.
\end{proof}

Using the Milnor invariant (\cref{def:milnor-invariant-7}), we get:
\begin{corollary}
	$\milinv(\Sigma^7_{i,j})=3+4(i-j)^2\mod 7$.
\end{corollary}

Since there are $4$ quadratic residues modulo $7$, this gives us the bound $|\Theta^7|\geq 4$. Similarly, the Eells-Kupier invariant gives:

\begin{corollary}
	$\ekinv(\Sigma^7_{i,j})=1-(i-j)^2\mod 224$.
\end{corollary}

By the Chinese remainder theorem, there are $28$ quadratic residues modulo $224$, so we get the bound $|\Theta^7|\geq 28$. As it so happens, $\Theta^7$ is exactly of order $28$ so $\lambda_{\textrm{e,k}}$ is a perfect invariant for $\Theta^7$ as mentioned in \cref{sec:lower-bounds}.

\subsection{The Gromoll-Meyer Sphere}\label{sec:gromoll-meyer}

The preceding construction of Milnor spheres takes on a more geometrically in some cases, resulting in not only an exotic smooth structure, but also a canonical Riemannian metric. Such a construction was first carried out by Detlef Gromoll and Wolfgang Meyer in \cite{gromollmeyer1974curvature}, and was the first example of an exotic sphere admitting a metric of non-negative sectional curvature.

Generally, many spaces in geometry and topology can be identified as \defn{homogeneous spaces}[homogeneous space], i.e. the quotient $G/H$ of a Lie group $G$ by a closed subgroup $H$. Such spaces ``look the same everywhere'', and so many results in differential geometry simplify greatly. For example, the scalar curvature of a homogeneous manifold is constant.

A classic examples of homogeneous spaces are the spheres, which admit diffeomorphisms
\[
	S^{n-1}\cong \SO_n/\SO_{n-1},\quad S^{2n-1}\cong \SU_n/\SU_{n-1},\quad\textrm{and}\quad S^{4n-1}\cong \Sp_n/\Sp_{n-1}.
\]
There are other exceptional identifications of spheres as homogeneous manifolds, however these are the only infinite families. For this construction, we are interested in the diffeomorphism $S^7\cong \Sp_2/\Sp_1$. As has been the theme through this section, having found a symmetric way to construct a 7-dimensional sphere, we will proceed to ``twist'' the construction in a way that does not affect the homeomorphism type of the resulting smooth manifold.

Recall that $\Sp_n$ is the group of symplectic $n\times n$ quaternion matrices, i.e. matrices whose conjugate transpose is its inverse. At the lowest dimension, we identify $\Sp_1\cong \SU_2\cong S^3$ as the unit sphere of quaternions. Consider the action of $\Sp_1\times \Sp_1$ on $\Sp_2$ given by
\begin{equation}\label{eq:gromoll-meyer-action}
	(q_1, q_2)\cdot Q \lkxmapsto \begin{pmatrix} q_1 & 0\\ 0 & q_1\end{pmatrix}Q\begin{pmatrix}\overline{q_2} & 0\\ 0 & 1\end{pmatrix}.
\end{equation}

\begin{proposition}
	The quotient of $\Sp_2$ by the action of $\Sp_1\times \Sp_1$ is diffeomorphic to $S^4$.
\end{proposition}
\begin{proof}
	Consider the map
	\[
		\lkxfunc{f}{\Sp_2}{\HH\times \R}{Q}{(2\overline{q_{12}}q_{22}, \|q_{12}\|^2 - \|q_{22}\|^2)}
	\]
	where $q_{ij}$ is the $(i,j)$th row-column entry of $Q$. This action is clearly invariant under the action in \cref{eq:gromoll-meyer-action}, and so descends to a map $\Sp_2/\Sp_1\times \Sp_1 \to \HH\times \R$. Since the total norm of the image in $\HH\times \R\cong \R^5$ has norm $1$, we actually have a map $\Sp_2/\Sp_1\times \Sp_1\to S^4$. This is straightforwardly checked to be a difeomorphism.
\end{proof}

The inclusion of the diagonal $\Delta \subset \Sp_1\times \Sp_1$ induces a map $\Sp_2/\Delta \to \Sp_2/\Sp_1\times \Sp_1$, which is a fiber bundle with fibers $\Sp_1\times \Sp_1/ \Delta$. Since $\Sp_1\times \Sp_1/\Delta\cong S^3$, the bundle
\[
	\Sp_1\times\Sp_1/\Delta \lkxto \Sp_2/\Delta \lkxto \Sp_2/\Sp_1\times \Sp_1
\]
is a bundle over $S^4$ with fibers $S^3$, just like the Milnor spheres in the previous section.

\begin{definition}
	The \defn{Gromoll-Meyer sphere $\gmsp$} is the quotient $\Sp_2/\Delta$.
\end{definition}

\begin{proposition}
	There is a diffeomorphism $\gmsp \cong \Sigma^7_{2,-1}$.
\end{proposition}

This is an example of a general construction known as a biquotient. Whenever $H_1,H_2\subset G$ are subsets of a group, their \defn{biquotient} is the quotient of $G$ by simultaneous left multiplication by $H_1$ and right multiplication by $H_2$. The resulting quotient space is denoted $H_1\backslash G/H_2$. When $H$ is a subgroup of $G\times G$, the notation $G/\!/H$ is sometimes used.

\begin{theorem}
	The Gromoll-Meyer sphere is the only exotic sphere which is diffeomorphic to a biquotient of Lie groups.
\end{theorem}
\begin{proof}
	See Corollary C of \cite{KZ2004biquotients}.
\end{proof}

\subsection{Milnor Spheres in Dimensions Other Than 7}

Generalizing Milnor spheres to 15 dimensions takes a similar approach by twisting the octonionic Hopf bundle. In this case, we get $\pi_7(\SO_8)\cong \Z\oplus \Z$ and for a bundle corresponding to $(i,j)$, we have $p_2=\pm 6(i-j)$. See \cite{shimada1957differentiable} for more details on this construction. 

\begin{remark*}
	Note that in dimension $3$, we have $\pi_1(\SO_2)\cong \Z$ so the only way in which a sphere can be formed as an $S^1$-bundle over $S^2$ is by the Hopf bundle. The other manifolds which are obtainable in this way are the lens spaces, as quotients $S^3/(\Z/p\Z)$ for $p\in \pi_1(\SO_2)$.
\end{remark*}

It turns out that the dimensions $3$, $7$, and $15$ are the only dimensions in which spheres can be constructed as fiber bundles over a sphere. This is related to a beautiful classical problem of homotopy theory known as the Hopf invariant problem, solved in \cite{adams1960}.

\begin{theorem}\label{thm:hopf-homotopy-spheres}
	The dimensions $3, 7,$ and $15$ are the only dimensions in which a topological sphere can be constructed as a fiber bundle over a sphere.
\end{theorem}

\smallrule
