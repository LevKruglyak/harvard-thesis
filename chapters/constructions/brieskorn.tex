\pagebreak
\newcommand{\V}{\mathcal{V}}
\renewcommand{\L}{\mathcal{L}}

\section{Brieskorn Manifolds}\label{sec:brieskorn}

There is an wonderful geometric link between exotic spheres and higher-dimensional knots. While less direct that the plumbing construction, all homotopy spheres in $\bP^{4k}$ can be constructed via knots as well. Most of the content here can be found in yet another classic exposition of Milnor, \cite{milnor1968hypersurfaces}. For a knot theoretic perspective, see Chapter 19 of \cite{kauffman1987knots}. A more general exposition on high-dimensional knot theory is \cite{ranicki1998knot}.

\begin{definition}
  An \defn{$n$-dimensional knot} is a connected codimension $2$ submanifold of $S^{n+2}$.
\end{definition}

\begin{remark}
	Classical knots are $1$-dimensional, and are the images of spheres. Note that any circle embedded in $\R^3$ can be embedded in $S^3$ by one-point compactification.
\end{remark}

\begin{figure}[ht]
	\centering
	\import{diagrams}{knots.pdf_tex}
	\caption{Examples of classical knots.}
\end{figure}

A nice way to view knots is as spherical levels sets of an isolated complex singularity. This approach generalizes easily to higher dimensions. Let $F\in \C[z_0,z_1\ldots, z_n]$ be a non-constant polynomial in $(n+1)$-complex variables.
\begin{definition}
	The \defn{variety} of $F$ is the complex hypersurface given by the zero locus
	\[
		\V(F) = F^{-1}(0)=\left\{ z \in \C^{n+1} \mid F(z)=0\right\} \subset \C^{n+1}.
	\]
\end{definition}

\begin{definition}
	A point $w\in \V(F)$ is a (complex) \defn{singularity}[complex singularity] if $\nabla F(w)$ vanishes. A singularity is \defn{isolated}[isolated singularity] if there is a neighborhood surrounding $w$ which contains no other singularities.
\end{definition}

\begin{definition}
	Let $w\in \V(F)$ be an isolated singularity. The \defn{link} of $F$ at $w$ is the set 
	\[
		\L(F, w) = \V(F) \cap S^{2n+1}_\varepsilon(w) = \left\{ z\in \C^{n+1}  F(z)=0\textrm{ and } |z-w|<\varepsilon\right\}
	\]
	where $\varepsilon > 0$ is some sufficiently small real number so that $\L(F,w)$ is the smooth $(2n-1)$-manifold intersecting the sphere $S^{2n+1}_\varepsilon(w)$ transversally.
\end{definition}

When the isolated singularity is clear, we write $\L(F)$.

\begin{figure}[ht]
  \import{diagrams}{knot-cone.pdf_tex}
  \caption{A link of an isolated singularity.}\label{fig:cone-over-knot}
\end{figure}

\begin{remark}
	In knot theory, a link is a disjoint union of knots. It is not generally the case that $\L(F)$ is connected, so it is referred to as a link.
\end{remark}

The simplest examples of complex polynomials with isolated singularities are known as Brieskorn polynomials.

\begin{definition}
	Let $(a_0,a_1,\ldots, a_n)$ be an $(n+1)$-tuple of integers greater than or equal to $2$. The \defn{Brieskorn polynomial} of the tuple $(a_0,a_1,\ldots, a_n)$ is given by
	\[
		F(z_0,z_1,\ldots, z_n) = z_0^{a_0} + z_1^{a_1} +\cdots + z_n^{a_n}.
	\]
	Correspondingly, we refer to $\V(F)$ as the \defn{Brieskorn variety} of the tuple and to the link $\L(F,0)$ at the origin as the \defn{Brieskorn manifold}. We denote the Brieskorn manifolds by
	\[
		\Sigma(a_0,a_1,\ldots, a_n) =\L(z_0^{a_0}+z_1^{a_1}+\cdots+z_n^{a_n}, 0).
	\]
\end{definition}

\begin{proposition}
	If $p,q\geq 2$ are integers, then $\Sigma(p,q)\subset S^3$ is the torus link of type $(p,q)$. If $\gcd(p,q)=1$ then there is one connected component and so $\Sigma(p,q)$ is a torus knot.
\end{proposition}
\begin{figure}[ht]
	\centering
  \import{diagrams}{torus-knot.pdf_tex}
  \caption{Examples of torus knots.}\label{fig:torus-knots}
\end{figure}

In $3$ dimensions, many classic spaces can be realized as Brieskorn manifolds.
See Chapter 19 of \cite{kauffman1987knots} for some explicit geometric constructions.
\begin{proposition}
	$\Sigma(2,2,2)\cong \RP^3$.
\end{proposition}

\begin{proposition}
	$\Sigma(2,3,5)\cong \mathscr{D}$, where $\mathscr{D}$ is the Poincar\'e dodecahedral space.
\end{proposition}

Note that neither of the spaces $\Sigma(2,2,2)$ or $\Sigma(2,3,5)$ are homotopy spheres. However, $\Sigma(p,q)$ with $\gcd(p,q)=1$ is diffeomorphic to a circle.

\begin{theorem}\label{thm:fibration}
	If $F$ is a complex polynomial in $(n+1)$-variables with an isolated singularity at the origin, then there is a smooth fiber bundle map
	\[
		\lkxfunc{\phi}{S^{2n+1}_\varepsilon \setminus \L(F)}{S^1}{z}{\arg F(z).}
	\]
\end{theorem}

For a given angle $e^{i\theta}\in S^1$, denote the fiber of the bundle $\phi$ as $F_\theta = \phi^{-1}(e^{i\theta})$.

\begin{proposition}
	Each fiber $F_\theta$ is a smooth parallelizable $2n$-manifold.
	Note that the boundary of each fiber $F_\theta$ is $\mathcal{L}(F)$. 
\end{proposition}

\begin{definition}
	A knot/link $\mathcal{L}$ which arises from a fibration $S^\ell \setminus \mathcal{L} \to S^1$ with each fiber bounding $\mathcal{L}$ is known as a \defn{fibered knot/link}.
\end{definition}

For more information on fibered knots, see \cite{kauffmanneumann1977}.


\begin{proposition}
	$\mathcal{L}(F+z_{n+1}^k)$ is a $k$-fold cover of $S^{2n+1}$, branched along $\mathcal{L}(F)$.
\end{proposition}

\begin{figure}[ht]
	\centering
	\import{diagrams}{branched-cover.pdf_tex}
	\caption{A $3$-fold cover of a plane, branched along a line.}
\end{figure}

Let's fix a polynomial $F$ in $(n+1)$ complex variables

\begin{proposition}
	If $n\neq 2$, then $\L$ is homeomorphic to the sphere $S^{2n-1}$ if and only if $\L$ has the homology of a sphere. In fact, $\L$ is a topological sphere if and only if the reduced homology $\widetilde{H}_{n-1}(\L)$ is trivial.
\end{proposition}

Let's now choose an orientation for $F_\theta$.

\begin{proposition}
	The manifold $\L$ is a homology sphere if and only if the intersection form
	\[
		\lkxfunc{Q_{F_\theta}}{\H_n(F_\theta)\times \H_n(F_\theta)}{\Z}
	\]
	has determinant $\pm 1$, i.e. if $Q_{F_\theta}$ is unimodular.
\end{proposition}

\begin{theorem}[Hirzebruch-Mayer] 
	Brieskorn manifolds are parallelizable.
\end{theorem}

\begin{theorem}[Brieskorn]
	Every homotopy sphere in $\bP^{2m}$ with $m\geq 4$ is the link of some isolated singularity.
\end{theorem}

% For a 
