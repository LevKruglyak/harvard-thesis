\pagebreak
\section{Twisted Spheres}\label{sec:twisted-spheres}
The final construction of homotopy spheres we discuss is by means of ``twisting'' together two disks along a diffeomorphism of their boundary.
	%
	% The definition of connected sum for smooth manifolds is slightly stronger than the definition for topological manifolds. In the topological category, we could simply cut out open disks from both manifolds and identify their boundaries, i.e. we set
	% \begin{equation}\label{eq:connected-sum-in-topological-category}
	% 	M_1\# M_2 = (M_1\setminus \Int(\iota_1(D^n)))\cup_g (M_2\setminus \Int(\iota_2(D^n)))
	% \end{equation}
	% where $g : \partial \iota_1(D^n) \to \partial \iota_2(D^n)$ is any orientation-reversing homeomorphism. This definition turns out to be well-defined in the topological category, although proving this takes a considerable amount of work. \todo{cite}
	%
	% However, the connected sum in the topological category will not give a unique connected sum in the smooth category, in fact far from it. Interestingly enough, the failure for \cref{eq:connected-sum-in-topological-category} to give a unique smooth manifold is related to exotic spheres in the following way. 

\begin{definition}
	For any $f : S^{n-1}\to S^{n-1}$ is an orientation-preserving diffeomorphism, the \defn{twisted sphere} associated to $f$ is the space
	\[
		T(f) = D^n\cup_f D^n
	\]
	which identifies the boundaries of $D^n$ by the diffeomorphism $f$.
\end{definition}

Note that topologically, this is a connected sum of two spheres $S^n$. Consequently, $T(f)$ is always homeomorphic to a sphere. However, the smooth structure depends on the diffeomorphism $f$.

\begin{figure}[ht]
	\centering
	\import{diagrams}{twisted-sphere.pdf_tex}
	\caption{A twisted sphere.}
\end{figure}

We can interpret the twisted sphere construction as a map $T: \Diff^+(S^{n-1})\to \Theta^n$ sending an orientation-preserving diffeomorphism $f : S^{n-1} \to S^{n-1}$ to the twisted sphere $T(f)$. For any (smooth) path $\omega : I \to \Diff^+(S^{n-1})$, we can build an $h$-cobordism 
	\[ (D^n\times I)\cup_\omega (D^n\times I) : T(\omega_0) \sohbord T(\omega_1)\]
	where we interpret the path $\omega$ as a smooth homotopy $\omega : I\times S^{n-1}\to\S^{n-1}$ between diffeomorphisms $\omega_0, \omega_1 : S^{n-1} \to S^{n-1}$. For $n\geq 5$, the $h$-cobordism theorem implies that $T(\omega_0)$ is diffeomorphic to $T(\omega_1)$ so the map $T$ only depends on the path component of $f\in \Diff^+(S^{n-1})$.

	This also implies that if $f$ can be extended to a diffeomorphism on the interior of the disk, the resulting twisted sphere must be diffeomorphic to a sphere. Thus, by a similar argument we can reduce to path-components. Altogether, we have an exact sequence (of sets)
	\begin{equation}\label{eq:twisted-sphere-exact-sequence-proto}
		\pi_0 [\Diff^+(D^n)] \lkxto \pi_0 [\Diff^+(S^{n-1})] \lkxto[T] \Theta^n.
	\end{equation}
	Finally, if we take any homotopy sphere $M\in \Theta^n$, cutting out the interiors of any embedded open disks $D_1, D_2\subset M$ gives an $h$-cobordism $M\setminus(D_1\cup D_2) : \partial D_1 \sohbord \partial D_2$. If $n\geq 6$, the $h$-cobordism theorem (\ref{thm:h-cobordism}) implies that $M \setminus (D_1\cup D_2)$ is diffeomorphic to a cylinder $\partial D_2\times [0,1]$. It follows that $M$ is a twisted sphere corresponding to the diffeomorphism $\partial D_1 \cong \partial D_2$ coming from the $h$-cobordism.
	When $n\geq 6$, the exact sequence \cref{eq:twisted-sphere-exact-sequence-proto} therefore extends to 
	\[
		\pi_0 [\Diff^+(D^n)] \lkxto \pi_0 [\Diff^+(S^{n-1})] \lkxto[T] \Theta^n \lkxto 0.
	\]

	Also dimensions $\geq 6$, we have the following result:
	\begin{theorem}[Cerf]
		When $n\geq 6$, we have $\pi_0[\Diff^+(D^n)]\cong 0$.
	\end{theorem}
	\begin{proof}
		This follows as a consequence of the general pseudo-isotopy theorem, see \cite{cerf1970pseudoisotopy}.
	\end{proof}

	Consequently, we have an identification:

	\begin{corollary}\label{cor:twisted-sphere-bijection}
		For $n\geq 6$, there is a bijection $\Theta^n \cong \pi_0[\Diff^+(S^{n-1})]$.
	\end{corollary}

	In my subjective opinion, this is the most canonical way to observe the phenomenon of exotic smooth structures on the spheres, although this construction is by far the hardest to work with. Rather than think of a set of abstract smooth manifolds and diffeomorphisms between them, the twisted sphere construction allows us to interpret the set of smooth structures on a sphere as the set of path-components of diffeomorphisms on a specific sphere. 

	The bijection in \cref{cor:twisted-sphere-bijection} is also a reason as to why some theoretical physicists care about exotic spheres. Theories of physics are generally gauge invariant -- equations should be true independently of the coordinate system in they were calculated.
	For instance, the Ricci curvature scalar $\mathcal{R}$ in general relativity is invariant under any diffeomorphism of spacetime coordinates. If a theory exhibits inconsistencies under such gauge transformations, this could be a problem for its physical viability.

	Note that any proper diffeomorphism $\R^n \to \R^n$ can be extended to a diffeomorphism of the compactification $S^{n} \to S^n$. In higher-dimensional models physics, the space of (proper) diffeomorphisms thus begin to exhibit evidence of exotic smooth structures via the twisted sphere construction. For instance, the space of proper diffeomorphisms of $10$-dimensional spacetime has 992 components since $\Theta^{11}\cong \Z/992$. The study of exotic spheres is therefore intrinsically tied to questions of \defn{global gravitational anomalies} -- inconsistencies arising from diffeomorphisms which are not in the connected component of the identity. For more information, wee Witten's paper \cite{witten1985anomalies} on global anomalies and the book \cite{baadhio1996quantum}.
	\smallrule
