\pagebreak
\section{Characteristic Classes}\label{sec:characteristic-classes}
The study of formal characteristic classes began with the work of Hassler Whitney and Eduard Stiefel in the mid 1930s. Since then, the fundamental idea has remained unchanged -- a vector bundle on a manifold determines certain ``characteristic'' classes in the cohomology of the base manifold. A motivating example is the Euler class, which we discussed in \cref{sec:euler-class}.


By the mid 1940s, these ideas were extended by Lev Pontryagin and Shing-Shen Chern to better capture the geometric data of oriented real and complex vector bundles respectively. In the following decades, characteristic classes quickly joined the toolboxes of mathematicians from a wide range of disciplines, finding connections to prior notions in these fields.
Applications ranged from algebraic topology, differential topology of exotic spheres, complex geometry, index theory, and may others.

We will present a few equivalent formulations of characteristic classes in this thesis, each useful in its own context. 
As with the Euler class, there are a breadth of equivalent formulations of characteristic classes, as
\begin{enumerate}[(a)]
	\item as natural transformations satisfying certain axioms,
	\item as generators of the cohomology ring of a classifying space,
	\item by the Chern-Weil homomorphism as images of invariant polynomials,
	\item as obstructions to problems in homotopy theory,
\end{enumerate}

For a standard introduction which avoids some of the opinionated choices made in this presentation, see \cite{milnorstasheff1974}.
At this stage, we assume a basic knowledge of vector bundles, structure groups, and classifying spaces. For a brief introduction to these topics, see \cref{chap:vector-bundles}.

\subsection{A Universal Perspective on Characteristic Classes}\label{sec:axiomatic-characteristic-classes}

Throughout this section, a cohomology theory will refer either to singular cohomology with some PID coefficient ring such as $\Z$, $\Q$, $\Z/2$, or $\Z[1/2]$, or to de Rham cohomology, with coefficient ring $\R$. The Poincar\'e dual homology theories are then either singular homology with coefficients or compactly supported de Rham cohomology. We use $R$ to denote the coefficient ring.

\begin{definition}\label{defn:characteristic-class}
	A \defn{characteristic class} $c$, valued in a cohomology theory $h$, is a natural transformation of contravariant functors
	\[
		\lkxfunc{c}{\Vect_G}{h^\bullet,}
	\]
	given a structure group $G$. We assume here that all vector bundles are real, since letting $G=\U_n$ recovers complex structure. Denote the set of characteristic classes by $\Class_G^R = [\Vect_G , h^\bullet]$.
\end{definition}

Here, $h^\bullet : \Top \to \Rng$ is the functor sending a space to its  cohomology ring, and $\Vect_G : \Top \to \Set$ is a functor sending a space to the set of isomorphism classes of vector bundles over the space with structure group $G$. We assume that the natural transformation $c$ forgets the ring structure of cohomology when mapping from the set of isomorphism classes of vector bundles.

\begin{example}
	The Euler class is a characteristic class in $\Class^\Z_{\SO_{2m}}$.
\end{example}

For any vector bundle $\mathcal{E} : E \to B$, a cohomology class assigns some cohomology class $c(\mathcal{E})\in h^\bullet(B)$ which is ``characteristic'' of the bundle $\mathcal{E}$.
Part of this ``characteristicness'' is that the assignment must be done in a natural way. Given bundles $\mathcal{E}_1$ and $\mathcal{E}_2$ over bases $B_1$ and $B_2$, whenever a map $f : B_1 \to B_2$ is covered by a bundle map $\mathcal{E}_1 \to \mathcal{E}_2$, we have $f^* c(\mathcal{E}_2) = c(\mathcal{E}_1)$. In particular, isomorphic vector bundles over the same base space $B$ are sent to the same cohomology class. 

\begin{convention*}
	Since every smooth manifold $M$ comes with a canonical vector bundle -- the tangent bundle -- it is common to use the notation $c(M)$ to refer to $c(\TT M)$.
\end{convention*}

\begin{remark}
	In practice, many sources define characteristic classes first as a set of homogeneous classes $c_i\in h^i(\mathcal{E})$ and then refer to the \defn{total characteristic class} $c(\mathcal{E})=\sum_i c_i(\mathcal{E})$. In this thesis, we take a somewhat unorthodox convention and assume that all characteristic classes are inhomogeneous unless otherwise stated.
\end{remark}

For any two characteristic classes $c_1$ and $c_2$, there is a natural notion of their sum $c_1+c_2$, product $c_1\smile c_2$, and scalar product $r\cdot c_1$ for any $r\in R$, simply by applying these operations in $h^\bullet(-)$. This gives the set $\Class_G^R$ of characteristic classes a ring structure.

On a base $B$, the cohomology ring $h^\bullet(B)$ is a graded ring
\[
	h^\bullet(M) = \bigoplus_{k\geq 0} h^k(B),
\]
with the cup product turning $h^\bullet(B)$ into a graded-commutative ring.
With the grading on cohomology, we can give the set of characteristic classes $\Class_G^h$ a grading
\[
	\Class_G^h[k] = [\Vect_G, h^k]
\]
where $h^k: \Top \to \Grp$ denotes the functor sending a space to its $k$-th cohomology group.

\begin{definition}
	A characteristic class $c$ is said to be \defn{homogeneous of degree $k$}[homogeneous characteristic class] if it lies in the graded component $[\Vect_G, h^k]$. We denote this degree by $|c|$.
\end{definition}

By naturality, characteristic classes must preserve cohomological degree since pull-backs preserve degree. We can thus write any characteristic class $c$ as an infinite sum of homogeneous characteristic classes $c=c_0+c_1+c_2+\cdots$, i.e. where each $c_k$ is a natural transformation from $\Vect_G$ to $h^k$.
In other words, the ring of characteristic classes $\Class_G^h$ admits the structure of the completion of a graded-commutative ring:
\[
	\Class_G^R \cong \prod_{k \geq 0} [\Vect_G, h^k].
\]
\begin{remark}
	Recall from the conventions that the completion of a graded ring $A=\bigoplus_{i\in I} A_k$ is the direct product $\widehat{A}=\prod_{i\in I} A_i$.
\end{remark}

The ring structure on $\Class_G^R$ has a simple interpretation in terms of classifying spaces. By the work of \cref{sec:classifying-spaces}, every vector bundle $\mathcal{E} : E \to B$ with structure group $G$ is the pullback of a universal bundle over the classifying space $\BB G$ by a map $f_{\mathcal{E}} : B \to \BB G$.
Every homogeneous characteristic class $c$ of degree $k$ thus determines a universal cohomology class $u\in h^k(\BB G)$ by applying $c$ to the universal bundle over $\BB G$. By naturality, $c(\mathcal{E})$ would be the pullback $f^*u$ of the universal class by the classifying map $f$.

Conversely, any cohomology class in $h^k(\BB G)$ gives a homogeneous characteristic class which sends
\[
	\lkxfunc{c}{\Vect_G(B)}{h^k(B)}{\mathcal{E}}{f_{\mathcal{E}}^*u.}
\]
Accounting for infinite sequences, we get the following theorem:

\begin{theorem}\label{thm:universal-characteristic-classes}
	There is a ring isomorphism $\Class^R_G\cong \widehat{h^\bullet}(\BB G)$
	where $\widehat{h^\bullet}$ denotes the completion of the cohomology ring.
\end{theorem}

This is the universal perspective on characteristic classes -- instead of defining natural transformations between vector bundles and cohomology theories, we can work in the cohomology ring of a single classifying space.

\subsection{General Axioms for Characteristic Classes}

Next, let us look at some common axioms for characteristic classes and investigate their implications. This simplifies many of the following constructions of the various species of characteristic class.

\begin{definition}
	A characteristic class $c$ is said to be \defn{multiplicative}[multiplicative characteristic class] if \begin{equation}\label{eq:whitney-product}
		c(\mathcal{E}_1\oplus \mathcal{E}_2)=c(\mathcal{E}_1)\smile c(\mathcal{E}_2)
	\end{equation}
	whenever the bundles $\mathcal{E}_1$ and $\mathcal{E}_2$ have the same base space.
\end{definition}

Equations of the form \cref{eq:whitney-product} are known as a \defn{Whitney product formula}. The homogeneous version of this equation which commonly appears is
\[
	c_n(\mathcal{E}_1\oplus \mathcal{E}_2) = \sum_{p+q=n}c_p(\mathcal{E}_1)\smile c_q(\mathcal{E}_2).
\]

\begin{remark}
	A multiplicative cohomology class is also multiplicative with respect to manifolds. Given manifolds $M_1$ and $M_2$ and a multiplicative characteristic class $c$, we have
	\[
		\begin{aligned}
			c(M_1\times M_2) & = c(\T(M_1\times M_2))      \\
			                 & = c(\T M_1\oplus \T M_2)    \\
			                 & = c(\T M_1)\smile c(\T M_2) \\
			                 & = c(M_1)\smile c(M_2).
		\end{aligned}
	\]
\end{remark}

\begin{remark}
	When the base space $B$ has finite type (as in the case of manifolds), the \defn{degree}[degree of a cohomology class] of a characteristic class $c(\mathcal{E})$ evaluated at a vector bundle $\mathcal{E} : E \to B$ is the maximal degree of $c(\mathcal{E})\in h^\bullet(B)$, and denoted $|c(\mathcal{E})|$.
\end{remark}

\begin{definition}
	A characteristic class $c$ is said to be \defn{rank-normalized} if $c_0=1$ and for any bundle $\mathcal{E}$ we have $|c(\mathcal{E})|\leq \rank(\mathcal{E})$. Note that the degree $|c(\mathcal{E})|$ is the maximal degree of a homogeneous component of $c(\mathcal{E})$.
\end{definition}

The first example of a characteristic class of an bundle is useful, but not very interesting.

\begin{corollary}
	For any rank-normalized characteristic class $c$ and rank $k$ trivial bundle $\underline{\R}^k$, we have $c(\underline{\R}^k)=1$.
\end{corollary}
\begin{proof}
	A trivial bundle $\underline{\R}^k$ is the pullback of the constant bundle $\R^k\to *$ over a point. This bundle has characteristic class $1$ by the rank normalization condition, and so by naturality pulls back to $1$.
\end{proof}

\begin{corollary}\label{cor:mobius-characteristic-2-torsion}
	If $c$ is a rank-normalized multiplicative characteristic class and $\gamma_1^1$ is the canonical M\"obius bundle over $\RP^1\cong S^1$, then $c(\gamma_1^1)$ is $2$-torsion.
\end{corollary}
\begin{proof}
	Since $\gamma_1^1\oplus \gamma^1_1$ is a trivial bundle (see \cref{fig:trivial-mobius-bundle-sum}), it follows that
	\[
		(1+c_1(\gamma_1^1))(1+c_1(\gamma_1^1)) = 1+2c_1(\gamma_1^1)+c_1^2(\gamma_1^1)=1.
	\]
	However, $H^2(S^1)\cong 0$, so we have $2c_1(\gamma_1^1)=0$.
\end{proof}

\begin{figure}[ht]
	\centering
	\import{diagrams}{placeholder-small.pdf_tex}
	\caption{A Whitney sum of orthogonal M\"obius bundles.}\label{fig:trivial-mobius-bundle-sum}
\end{figure}

\begin{corollary}
	For any multiplicative and rank-normalized characteristic class, we have $c(\mathcal{E}\oplus \underline{\R}^k)=c(\mathcal{E})\smile c(\underline{\R}^k)= c(\mathcal{E})$.
\end{corollary}

\begin{definition}
	A characteristic class $c$ is said to be \defn{stable}[stable characteristic class] if $c(\mathcal{E}\oplus \underline{\R}^k) = c(\mathcal{E})$ for any $k$. A stable characteristic class only depends on the stable isomorphism type of the vector bundle. Every multiplicative and rank-normalized characteristic class is stable.
\end{definition}

\begin{example}
	The Euler class is unstable, since it vanishes for odd rank real vector bundles.
\end{example}

\begin{remark}
	Note that every characteristic class with stable structure group, e.g. $G$ is either $\O, \SO,$ or $\U$, is stable. To compute the ring of stable characteristic classes, we thus would need to compute the cohomology of $\BO$, $\BSO$, or $\BU$.

	For any $n>0$, we have inclusions $G_n \to G_{n+1} \to \cdots \to G$ which correspond to the addition of trivial bundles. These give maps $\BB G_n \to \BB G$, which gives a map on cohomology
	\[
		h^\bullet(\BB G) \lkxto h^\bullet(\BB G_n).
	\]
	Every stable characteristic class in $h^\bullet(\BB G_n)$ must be in the image of this map.
\end{remark}

\subsection{Whitney Duality}

Next, we discuss inversion in the ring of characteristic classes.
In a general completion of a graded ring $\widehat{A}=\prod_{k\geq 0}A_k$, the multiplication of two elements
\[
	(x_0+x_1t +x_2t^2+\cdots)\cdot (y_0+y_1t+y_2t^2+\cdots) = (z_0 + z_1t + z_2t^2+\cdots)
\]
can be expanded in homogeneous components as $z_k=\sum_{p+q=k} x_py_q$, i.e.
\begin{equation}\label{eq:formal-product}
	\begin{aligned}
		z_0 & = x_0y_0,                 \\
		z_1 & = x_0y_1 + x_1y_0,        \\
		z_2 & = x_0y_2 + x_1y_1+x_2y_0, \\
		    & \;\;\vdots
	\end{aligned}
\end{equation}
To invert the series $(x_0+ x_1+\ldots)\in \widehat{A}$, we would like to find some $(y_0+y_1+\cdots)$ such that the resulting product $(z_0+z_1+\cdots)=1$. Setting $z_0=1$, $z_k=1$ for $k>0$ and solving \cref{eq:formal-product} for $y_k$, we get the recursive formula
\begin{equation}\label{eq:formal-inversion}
	y_k = \begin{cases}x_0^{-1}                                        & k=0,   \\
             -x_0^{-1}(x_1y_{k-1}+x_2y_{k-2}+\cdots +x_ky_0) & k > 0.
	\end{cases}
\end{equation}

Let us assume that the algebra is graded-commutative. In this case, we can expand \cref{eq:formal-inversion} to get the expansions
\[
	\begin{aligned}
		y_0 & = x_0^{-1},                        \\
		y_1 & = -x_1x_0^{-2}                     \\
		y_2 & = (x_1^2-x_2)x_0^{-3}              \\
		y_3 & = -(x_1^3 - 2x_1x_2 + x_3)x_0^{-4} \\
		    & \;\;\vdots
	\end{aligned}
\]
More generally, the formula for $y_k$ is given by the sum
\begin{equation}\label{eq:general-formal-inversion}
	y_k = (x_0)^{-(k+1)}\sum_{i_1+2i_2+\cdots+ni_n=n} \frac{(i_1+\cdots+i_n)!}{i_1!\cdots i_n!}(-x_1)^{i_1}\cdots (-x_n)^{i_n}.
\end{equation}
A crucial corollary of \cref{eq:formal-inversion} is that a series $(x_0+x_1+\cdots)$ can be inverted if and only if $x_0\in A_0^\times$ is a unit.

\begin{proposition}\label{prop:formal-inverse}
	If $\widehat{A}$ is a completion of a graded ring, then $\widehat{A}^\times = \widehat{A} \cap A_0^\times$.
\end{proposition}

\begin{remark}
	This is generally not true in the non-completed case. Take for instance in the ring of formal power series $R\fps{z}$, the completion of the polynomial ring $R[z]$. There, we have the identity
	\[
		(1+z)\cdot (1-z+z^2-z^3+\cdots) = 1,
	\]
	which is a rearranged form of the classic equation for the infinite series of a geometric series. Even though $(1-z)$ has monic leading coefficient, it has no inverse in $R[z]$.
\end{remark}

\cref{prop:formal-inverse} has a useful consequence in the context of characteristic classes. Suppose $c$ is a characteristic class with $c_0\in R^\times$. It follows that there is a characteristic class $c^{-1}$ which is the multiplicative inverse of $c$, so that $c\smile c^{-1}=1$ is the trivial characteristic class. If the characteristic class $c$ is multiplicative, then the Whitney product formula gives
\[
	c(\mathcal{E}_1\oplus\mathcal{E}_2)	 = c(\mathcal{E}_1)\smile c(\mathcal{E}_2)
	\quad\implies\quad
	c(\mathcal{E}_1) = c(\mathcal{E}_1\oplus \mathcal{E}_2) \smile c^{-1}(\mathcal{E}_2).
\]
If $\mathcal{E}_1=\TT M$ is the tangent bundle of a manifold $M^k$ and $\mathcal{E}_2=\TT\R^{n}/M$ is the normal bundle of an embedding $M^k\to \R^{n}$, then $\mathcal{E}_1\oplus \mathcal{E}_2$ is trivial since it is the restriction of the trivial tangent bundle of $\R^n$. In the case that $c$ is rank-normalized, we get:
\begin{theorem}[Whitney Duality]\label{thm:whitney-duality}
	If $M^k$ is a submanifold of $\R^n$, and $c$ is a rank-normalized multiplicative characteristic class, we have
	\[
		c(\TT M) = c^{-1}(\TT \R^n/M).
	\]
\end{theorem}

\begin{corollary}
	If $c$ is a rank-normalized multiplicative characteristic class, then $c(S^n)=1$.
\end{corollary}
\begin{proof}
	This follows since $S^n$ embeds into $\R^{n+1}$ with a trivial normal line bundle.
\end{proof}

\begin{remark}
	As it turns out, \emph{any} homotopy sphere $M$ has $c(M)=1$ for a rank-normalized multiplicative characteristic class because the tangent bundle of a homotopy sphere is stably isomorphic to the trivial bundle. We will prove this in \cref{thm:homotopy-sphere-stably-parallelizable} using some hard theorems in homotopy theory.
\end{remark}

We close this section with a nice result which will be useful later.
\begin{proposition}\label{prop:multiplicative-inverse-is-multiplicative}
	If $c$ is a multiplicative characteristic class with $c_0$ a unit, then $c^{-1}$ is as well.
\end{proposition}
\begin{proof}
	\begin{equation}
		\begin{aligned}
			c(\mathcal{E}_1\oplus \mathcal{E}_2)\smile c^{-1}(\mathcal{E}_1\oplus \mathcal{E}_2)    & = 1                                                  \\
			c(\mathcal{E}_1)\smile c(\mathcal{E}_2)\smile c^{-1}(\mathcal{E}_1\oplus \mathcal{E}_2) & = 1                                                  \\
			c(\mathcal{E}_2)\smile
			c^{-1}(\mathcal{E}_1\oplus \mathcal{E}_2)                                               & = c^{-1}(\mathcal{E}_1)                              \\
			c^{-1}(\mathcal{E}_1\oplus \mathcal{E}_2)                                               & = c^{-1}(\mathcal{E}_1)\smile c^{-1}(\mathcal{E}_2). \\
		\end{aligned}
	\end{equation}
	This completes the proof.
\end{proof}

\subsection{Characteristic Numbers}

If we want to compare vector bundles on different spaces, we need a common context in which to compare their characteristic classes since the cohomology rings of the underlying spaces might not be canonically isomorphism. When the base space is a compact $R$-oriented manifold, the Poincar\'e duality isomorphism $h^{n-k}(M) \cong h_k(M)$ allows us to ``integrate'' homogeneous top-dimensional cohomology classes $\alpha\in h^{n}(M)$ along a fundamental class $[M]\in h_n(M)$ to get an element $\alpha[M]\in h_0(M)\cong R$ in the coefficient ring of a corresponding homology theory. The coefficient ring $R$ is this context in which we can compare characteristic classes over different bases.

\begin{definition}\label{def:characteristic-number}
	Given any characteristic class $c$ and closed $n$-dimensional manifold $M$ the \defn{characteristic number} of $M$ is $c[M] = c_n(\TT M)[M]$.
\end{definition}

Note that we must take the degree $n$ homogeneous component of $c$ to apply Poincar\'e duality. Generally, if $\Man_R$ denotes the set of closed $R$-oriented manifolds, any characteristic class $c$ can be interpreted as a map
\begin{equation}\label{eq:characteristic-number}
	\lkxfunc{c}{\Man_R}{R}{M}{c[M].}
\end{equation}
Under some extensions of \cref{def:characteristic-number} to allow for disconnected manifolds, \cref{eq:characteristic-number} is additive with respect to disjoint unions.
\begin{proposition}
	If $c$ is a multiplicative characteristic class, then $c[M_1\times M_2]=c[M_1]\cdot c[M_2]$.
\end{proposition}
\begin{proof}
	By the multiplicative property and the K\"unneth formula, we have.
	\[
		\begin{aligned}
			c[M_1\times M_2]
			 & = c(\TT (M_1\times M_2))[M_1\times M_2]                      \\
			 & = c(\TT M_1\oplus \TT M_2)[M_1\times M_2]                    \\
			 & = (c(\TT M_1)\smile c(\TT M_2))[M_1]\times [M_2]             \\
			 & = c(\TT M_1)[M_1]\cdot c(\TT M_2)[M_2] = c[M_1]\cdot c[M_2].
		\end{aligned}
	\]
	This completes the proof.
\end{proof}

\begin{remark}
	While $\Man_R$ does not have a ring structure due to lack of an identity element, with the inclusion of a cobordism relation on $\Man_R$ it is possible to interpret multiplicative characteristic classes as ring homomorphisms from a cobordism ring to $R$.
\end{remark}

The set of characteristic numbers of a manifold form a topological fingerprint of the manifold. As we will see in \cref{sec:invariants-for-homotopy-4k-1-spheres}, the subtle interplay of their number theoretic properties is one of the main ways to study smooth structure on a manifold.

\subsection{The Chern Class}\label{sec:chern-class}

We will now begin construction of the most fundamental of characteristic classes, the Chern class for complex vector bundles. From the universal perspective on characteristic classes, it suffices to compute the cohomology ring $\H^\bullet(\BU_m)$ of the classifying space. 

Recall that $\BU_m$ is defined as the Grassmannian of $m$-planes in a countably infinite-dimensional Hilbert space $\C^\infty$, i.e. $\BU_m=\Gr_m(\C^\infty)$. Similarly, we can take the Stiefel space $\EU_m=\Fr_m(\C^\infty)$ of orthonormal frames of $\C^\infty$. The projection map $\EU_m \to \BU_m$ sending a frame to its spanning plane then gives the universal principal $\U_m$-bundle
\[
		\U_m \lkxto \EU_m \lkxto \BU_m.
\]
For instance, in the case $m=1$, we have $\BU_1=\CP^\infty$, $\EU_1=S^\infty$, and $\U_1=S^1$ so we have
\[
		S^1\lkxto \EU_1 \lkxto \BU_1.
\]
Since $S^\infty$ is contractible, by the Gysin sequence (\cref{thm:gysin-sequence}) we get isomorphisms
\[
	\H^{k-2}(\BU_1)\lkxto[e\,\smile] \H^k(\BU_1).
\]
In particular, we have a ring isomorphism $\H^\bullet(\BU_1)\cong \Z[\cl_1]$ where $\cl_1$ denotes the Euler class of the universal bundle $\EU_1 \to \BU_1$. 

To compute the cohomology of $\BU_m$ in higher dimensions, we can use an inductive argument. The inclusion map $\U_{m-1} \to \U_m$ induces a projection map $\BU_{m-1} \to \BU_{m}$, with fibers $\U_{m}/\U_{m-1}\cong S^{2m-1}$ so we have an orientable sphere bundle
\[
	S^{2m-1} \lkxto \BU_{m-1} \lkxto \BU_{m}.
\]
For the inductive step, we use a slightly more general lemma.

\begin{lemma}
	If $S^{2m-1} \to E \to B$ is an orientable sphere bundle and $\H^\bullet(E)$ is a polynomial ring $\Z[x_1,\ldots,x_\ell]$ of even-dimensional generators, then there is a ring isomorphism $\H^\bullet(B)\cong \H^\bullet(E)\otimes \Z[e]$ where $e$ is the Euler class of the bundle and $p^* : \H^\bullet(E) \to \H^\bullet(B)$ acts as the identity on $\H^\bullet(E)$.
\end{lemma}
\begin{proof}
	The proof involves a few applications of the Gysin sequence. Since the cohomology of $\H^\bullet(E)$ is concentrated in even dimensions, we can conclude that the cohomology of $\H^\bullet(B)$ is also concentrated in even dimensions. The Gysin sequence thus gives us an exact sequence
\[
	0 \lkxto \H^{k-2m}(B) \lkxto[e\,\smile]\H^k(B) \lkxto[p^*] \H^k(E) \lkxto 0
% 	\begin{tikzcd}
% 	  & \cdots\rar{p^*} \snakenode{X} & \H^{k-1}(\BU_{m-1})\snakearrow{X} \\
% 		{\H^{k-2m}(\BU_m)}\rar{e\,\smile} & {\H^k(\BU_m)}\rar{p^*}\snakenode{Y} & {\H^k(\BU_{m-1})}\snakearrow{Y} \\
% 		{\H^{k-2m+1}(\BU_m)}\rar{e\,\smile} & \cdots
% \end{tikzcd}
\]
for even $k$. From this, we can conclude that $\H^\bullet(B)\cong \H^\bullet(E)\otimes \Z[e]$.
For a full proof, see Proposition 4D.11 of \cite{hatcher2002topology}.
\end{proof}

This leads us to conclude:
\begin{proposition}
	$\H^\bullet(\BU_m)\cong \Z[\cl_1,\ldots, \cl_m]$, where $|\cl_i|=2i$.
\end{proposition}

This is one definition of the \defn{Chern classes}, cohomology classes for complex vector bundles with coefficients $\Z$ in singular homology. The properties of these characteristic classes are directly tied to the group theoretic properties of $\U_m$.
For instance, since the bundle $\BU_{m-1} \to \BU_{m}$ induces an injective map $\H^\bullet(\BU_{m-1}) \to \H^\bullet(\BU_m)$, we can conclude that all Chern classes are stable:
\[
	\H^\bullet(\BU) = \Z[\cl_1,\cl_2,\ldots].
\]
Similarly, multiplicativity of the Chern class $\cl = 1+\cl_1+\cl_2+\cdots$ follows from the group homomorphism $\U_{m_1}\times \U_{m_2} \to \U_{m_1+m_2}$, which induces an injective map 
\[
	\lkxfunc{}{\H^\bullet(\BU_{m_1+m_2})}{\H^\bullet(\BU_{m_1})\otimes \H^\bullet(\BU_{m_2})}{\cl}{\cl \otimes \cl.}
\]
These properties and a normalization condition are enough to fully characterize the Chern class.
\begin{proposition}
	The \defn{Chern class $\cl$}[Chern class] is the unique characteristic class for complex vector bundles ($G=\U$) in singular cohomology with $\Z$ coefficients satisfying:
	\begin{enumerate}[(a)]
		\item $\cl$ is rank-normalized and multiplicative,\footnote{Technically, we only need that $\cl_0=1$, the rank conditions follows from this and axiom (b).}
		\item $\cl(\gamma)=1-\alpha$,
	\end{enumerate}
	where $\gamma$ is the tautological line bundle over $\CP^1$ and $\alpha\in \H^2(\CP^m)$ is the Poincar\'e dual class to $\CP^{m-1}\subset \CP^m$.
\end{proposition}

Note that $\cl_1[\gamma]=-1$ because the Euler number of the tautological bundle is $-!$. See Chapter 14 of \cite{milnorstasheff1974} for more details.


\subsection{Chern Roots}

Another important perspective on the Chern class arises from the geometry of $\U_m$. 

Recall that every closed (connected) abelian Lie group is diffeomorphic to a torus $T \cong \U_1\times \cdots\times \U_1$. A \defn{maximal torus} $T$ of a Lie group $G$ is some properly included torus which is not contained within any other torus. Given a maximal torus in a closed Lie group, the normalizer $N(T)$ is the subgroup of $G$ stabilizing the maximal torus under the action of conjugation, i.e. 
\[
	N(T) = \{ g \in G \mid gTg^{-1} = T\}.
\]
\begin{definition}
	The \defn{Weyl group} of $G$ is the quotient $W(G)=N(T)/T$.
\end{definition}
It turns out that all choices of maximal tori are conjugate, so the exact choice of maximal torus does not affect the structure of the Weyl group.
In the case of the unitary group $\U_m$, a choice basis on $\C^m$ gives a canonical choice of maximal torus, namely the diagonal matrices
\[
	T = \left\{\diag(z_1,\ldots, z_n) \mid |z_i|=1\right\}.
\]
The normalizer of this subgroup is exactly the set of matrices $P_\sigma T$ where $P_\sigma$ is a permutation matrix associated to a permutation $\sigma\in \Sym_m$. From this observation, we get:
\begin{proposition}
	The Weyl group of $\U_m$ is the symmetric group $\Sym_m$.
\end{proposition}

Returning to characteristic classes, the inclusion of the torus $T$ into $\U_m$ induces a map $\BB T \to \BU_m$. 
Since $\BB T = \BU_1^{\times n}$, we get an injective pullback map
\[
		\H^\bullet(\BB U_m) \lkxto \H^\bullet(\BB U_1)\otimes \cdots \otimes \H^\bullet(\BB U_1).
\]
Note that the action of the Weyl group on $\BB T$ permutes the product $\BU_1\times \cdots\times \BU_m$, and hence permutes the factors of the ring $\H^\bullet(\BB U_1)\otimes \cdots\otimes \H^\bullet(\BB U_1)$.

\begin{theorem}[Splitting Principle]
	There is a ring isomorphism \[\H^\bullet(\BU_m)\cong \left(\H^\bullet(\BU_1)\otimes \cdots\otimes \H^\bullet(\BU_1)\right)^{\Sym_m}.\]
\end{theorem}

For some more information of this formulation of the splitting principle, see \cite{toda1987}. This is one of many equivalent formulations of the splitting principle. Essentially, the idea is that any complex bundle pulls back to a sum of line bundles, with Chern class pulling back to a symmetric expression in the first Chern classes of the line bundles.
This more standard statement is found in Theorem~19.3.9 of \cite{dieck2008algebraic} or on page 66 of \cite{hatcher2003ktheory}.

We thus have two ways to look at the cohomology ring $\H^\bullet(\BU_m)$ -- either as a polynomial ring $\Z[\cl_1,\ldots, \cl_m]$, or as symmetric polynomials $\Z[\gamma_1,\ldots, \gamma_m]^{\Sym_m}$. The generators of this latter ring are known as \defn{universal Chern roots}.

\begin{definition}\label{def:chern-roots}
	The \defn{Chern roots} of a complex vector bundle $\mathcal{E}$ over $X$ with classifying map $f : X \to \BU_n$ are given by $\gamma_i(\mathcal{E}) = f^*\gamma_i$. Note that
	\[ \cl(\mathcal{E}) = \prod_{1\leq i \leq n} (1+\gamma_i(\mathcal{E})). \]
\end{definition}

For a simple application of the splitting principle, we get:
\begin{corollary}
	The Chern class of complex projective space is $\cl(\CP^n)=(1-\alpha)^{n+1}$.
\end{corollary}
% \begin{example}
% 	The first few such expressions are given by
% 	\[
% 		\begin{aligned}
% 			\cl(\CP^1) & = 1                              \\
% 			\cl(\CP^2) & = 1+2\alpha+\alpha^2             \\
% 			\cl(\CP^3) & = 1+3\alpha + 3\alpha^2+\alpha^3 \\
% 			           & \;\;\vdots
% 		\end{aligned}
% 	\]
% \end{example}

% \begin{definition}
% 	The \defn{Chern class $\cl$}[Chern class] is the unique characteristic class for complex vector bundles ($G=\U$) in singular cohomology with $\Z$ coefficients satisfying:
% 	\begin{enumerate}[(a)]
% 		\item $\cl$ is rank-normalized and multiplicative,\footnote{Technically, we only need that $\cl_0=1$, the rank conditions follows from this and axiom (b).}
% 		\item $\cl(\OO_{\CP^1}(1))=1+\alpha$.
% 	\end{enumerate}
% 	We denote by $\cl_i$ the degree $2i$ homogeneous component of $c$ in $\H^{2i}(-;\Z)$.
% \end{definition}
%
% Of course, this is not a constructive definition, and it is not at all clear that a characteristic class satisfying these definitions even exists.
% However, being a multiplicative rank-normalized characteristic class with coefficients in a field of characteristic zero, the general results of the previous section hold for the Chern class. The more interesting results follow from axiom (b). By a beautiful theorem known as the splitting principle, it turns out that knowing the Chern classes of line bundles is enough to compute the Chern classes of any complex vector bundle.
%
% \begin{theorem}[Splitting Principle]
% 	Let $\F$ be $\R$ or $\C$ and suppose $\mathcal{E}^k$ is an $\F$-vector bundle over a space $X$. There exists a space $Y$ with map $f : Y \to X$ such that
% 	\begin{enumerate}[(a)]
% 		\item $f^* : h^\bullet(Y) \to h^\bullet(X)$ is injective,
% 		\item there is a bundle isomorphism $f^*\mathcal{E} \cong \mathcal{L}_1\oplus\cdots \oplus \mathcal{L}_k$ for line bundles $\mathcal{L}_i$ over $Y$.
% 	\end{enumerate}
% \end{theorem}
% \begin{proof}
% 	The general idea is as follows. Recall that the frame bundle $\Fr(\mathcal{E})$ is a principal $\GL_n\F$ bundle consisting of all local trivializations of $\mathcal{E}$. Letting $B\subset \GL_n \F$ be the Borel subgroup of upper triangular matrices. By letting $\GL_n$ act on $B\subset \GL_n\F$, we get an associated bundle
% 	\[
% 		\Fl(\mathcal{E}) = \Fr(\mathcal{E})\times_{\GL_n\F}{\GL_n\F/B}
% 	\]
% 	known as the \defn{flag bundle} of $E$. At each point $p\in X$ of the base, the frame bundle has fiber
% 	\[
% 		\Fr_p(\mathcal{E}) = \left\{ b : \F^k \to E_p \mid b \textrm{ is an isomorphism}\right\},
% 	\]
% 	or in other words, the set of ordered bases for $E_p$. Modulo the action of upper triangular matrices, we see that the flag bundle has fibers
% 	\[
% 		\Fl_p(\mathcal{E}) = \left\{ L_1,\ldots, L_k \textrm{ lines in }E_p \mid \span\{L_1,\ldots, L_k\} = \E_p \right\}.
% 	\]
% 	Note that there is a principal $B$-bundle $\Fr(E)\to \Fl(E)$.
%
% 	\begin{remark}
% 		In the complex case, we could reduce the bundle to have structure group $\U_n$, which would force the frame bundle to consist of orthonormal bases. Then to get the flag bundle, we could consider the bundle associated to the action of $\U_n$ on the quotient $\U_n/T$ by its maximal torus of diagonal matrices. This results in a diffeomorphic flag bundle.
% 	\end{remark}
%
% 	We then let $Y=\Fl(E)$ be the total space of the flag bundle, with the injectivity of $f^*$ following from the Leray-Hirsch theorem since it can be shown that $h^\bullet(\GL_n\F/B)$ has no torsion. In fact, the Leray-Hirsch theorem shows the stronger result that we have an isomorphism
% 	\[
% 		h^\bullet(X)\otimes h^\bullet(\GL_n \F/B) \cong h^\bullet(Y)
% 	\]
% 	with the inclusion $f^*$ simply being the precomposition of this isomorphism with $x\mapsto x\otimes 1$.
%
% 	\todo{finish this}
%
% \end{proof}

\subsection{Creating Multiplicative Characteristic Classes}

We now present a powerful method for constructing multiplicative characteristic classes, formalized in by Hirzebruch \cite{hirzebruch1966methods}.
The crux of this method is the fundamental theorem of symmetric polynomials, a simple but incredibly useful result in algebra.

\begin{theorem}[Fundamental Theorem of Symmetric Polynomials]
	For any ring $R$, let $R[x_1,\ldots, x_n]$ be the polynomial ring in $n$-variables and let $R[x_1,\ldots, x_n]^{\Sym_n}$ be the subring of symmetric polynomials, i.e. those invariant under a reordering of the variables. There is a ring isomorphism
	\begin{equation}\label{eq:symmetric-polynomial-isomorphism}
		R[x_1,\ldots, x_n]^{\Sym_n} \cong R[\sigma_1^{(n)},\ldots, \sigma_n^{(n)}].
	\end{equation}
	The variables $\sigma_i$ are the \defn{elementary symmetric polynomials}, defined by the relation
	\begin{equation}\label{eq:symmetric-polynomials}
		1+\sigma_1^{(n)}t+\sigma_2^{(n)}t^2+\cdots+\sigma_n^{(n)}t^n = (1+x_1t)(1+x_2t)\cdots (1+x_nt)
	\end{equation}
	where we grade $|\sigma_i^{(n)}|=|x_i|=i$.
\end{theorem}

By \cref{eq:symmetric-polynomials}, we can expand these elementary symmetric polynomials in terms of $x_i$:
\[
	\begin{aligned}
		\sigma_1^{(n)} & = x_1+x_2+\cdots + x_n                          \\
		\sigma_2^{(n)} & = x_1x_2 + x_2x_3 + x_1x_3 +\cdots + x_{n-1}x_n \\
		               & \;\;\vdots                                      \\
		\sigma_n^{(n)} & = x_1x_2\cdots x_n
	\end{aligned}
\]
More generally, we have the combinatorial formula
\[
	\sigma_i^{(n)} = \sum_{1\leq p_1<\cdots <p_i\leq n} x_{p_1}\cdots x_{p_i}.
\]
Passing to the completed ring, we have a ring isomorphism
\[
	R\fps{x_1,x_2\ldots}^{\Sym_\infty} \cong R\fps{\sigma_1,\sigma_2,\cdots}\quad\textrm{where}\quad \sigma_i = \sum_{0<p_1<\cdots < p_i<\infty}x_{p_1}\cdots x_{p_i}
\]
so that $\sigma_i^{(n)} = \sigma_i \mod (x_{n+1},x_{n+2},\ldots)$. 

\begin{remark}
In the context of Chern classes, the splitting principle for stable complex vector bundles gives us a decomposition
\[
	\cl = \prod_{i\geq 1}(1+\gamma_i)
\]
where $\cl\in \H^\bullet(\BU) \cong \Z[c_1,c_2,\ldots]\cong \Z[\gamma_1,\gamma_2,\ldots]^{\Sym_\infty}$ and $\gamma_i$ are the universal stable Chern roots. Chern classes can then be interpreted as symmetric polynomials in the Chern roots. This is a good picture to keep in mind.
\end{remark}

To generate new multiplicative characteristic classes out of Chern classes, we consider endomorphisms $K$ of the unit group $\Z\fps{c_1,c_2,\ldots}^\times$. Since $K(ab)=K(a)K(b)$, it follows that $K(\cl)$ would be a stable multiplicative characteristic class in $\Class^\Z_{\U}$. The most basic examples of such endomorphism are given below:

\begin{example}
Consider the endomorphism $K(a)=a^{-1}$.
If we let $a=1+a_1t+a_2t^2+\cdots$, we can express the inverse $K(a)=a^{-1}$ directly in terms of $a_i$ by \cref{eq:general-formal-inversion}. 
\begin{equation}\label{eq:chern-inverse-non-symmetric}
	\begin{aligned}
		K(a)=a^{-1} & = (1+a_1t+a_2t^2+\cdots)^{-1}                              \\
		            & = 1-(a_1t+a_2t^2+\cdots) + (a_1t+a_2t^2+\cdots)^2 - \cdots \\
		            & = \sum_{n\geq 0} K_n(a_1,\ldots, a_n)t^n
	\end{aligned}
\end{equation}
where $K=\{K_n\}_{n\geq 0}$ is the family of polynomials
\begin{equation}\label{eq:inverse-K-series}
	K_n(a_1,\ldots, a_n) = \sum_{i_1+2i_2+\cdots+ ni_n=n}\frac{(i_1+\cdots+i_n)!}{i_1!\cdots i_n!}(-a_1)^{i_1}\cdots(-a_n)^{i_n}.
\end{equation}
\end{example}

\begin{example}
	More generally we have the endomorphism $K(a)=a^n$, although its expansion into polynomials has a far more complicated form.
\end{example}

Going in the reverse direction, any sequence of polynomials $K=\{K_n\}$ in which $K_n(\sigma_1,\ldots, \sigma_n)$ is homogeneous of degree $n$ induces a map
\begin{equation}\label{eq:multiplicative-sequence-endomorphism}
	K : \sum_{n\geq 0} \sigma_n t^n \lkxmapsto \sum_{n\geq 0} K_n(\sigma_1,\ldots, \sigma_n) t^n,
\end{equation}
which, by an abuse of notation, we also denote by $K$. 

\begin{definition}
	A \defn{multiplicative sequence} $K=\{K_n\}$ of polynomials in which $K_n(\sigma_1,\ldots, \sigma_n)$ is homogeneous of degree $n$ and $K$ induces an endomorphism of $R\fps{\sigma_1,\sigma_2,\ldots}^\times$.
\end{definition}


When working with multiplicative sequences, the set of polynomials is often far too cumbersome to deal with in practice. For any multiplicative sequence $K$,
expanding out the symmetric polynomials gives
\[
	\sigma = 1+\sigma_1+\sigma_2+\cdots = \prod_{n\geq 1}(1+x_n)
	\quad\implies\quad
	K(\sigma) = \prod_{n\geq 1} K(1+x_n).
\]
We therefore define a formal power series
\[
		Q_K(t) = K(1+t)=\sum_{n\geq 0}K_n(1,0,\ldots, 0) t^n
\]
so that $K(\sigma)$ can be expressed as
\[
		K(\sigma) = \prod_{n\geq 1}Q_K(x_n)
\]

For some simple examples, we have:

\begin{definition}
	The polynomial $Q_K(t)=K(1+t)$ is called the \defn{characteristic series} of the multiplicative sequence $K$.
\end{definition}

\begin{example}
	The Chern class is associated to the characteristic series
	\[Q_\cl(t)=1+t.\]
\end{example}

\begin{example}
	The  inverse Chern class is associated to the characteristic series
	\[Q_{\cl^{-1}}(t)=\frac{1}{1+t}=1-t+t^2-t^3+\cdots.\]
\end{example}

Multiplicative sequences span out a decently large set of endomorphisms of $R\fps{\sigma_1,\sigma_2,\ldots}^\times$, and it turns out that such endomorphisms are completely characterized by their characteristic series.

\begin{theorem}
	There is a bijective correspondence
	\[
		\left\{\parbox{10.5em}{multiplicative sequences}\right\} \quad\lkxleftrightto\quad R\fps{t}^\times
	\]
\end{theorem}
\begin{proof}
	We have seen how to go from a multiplicative sequence to a characteristic series already. In the converse direction, suppose $Q(t)\in R\fps{t}^\times$ is a formal power series. Since the product is symmetric, we expand it in terms of symmetric polynomials graded by $t$
	\[
		 \prod_{n\geq 1} Q(x_n t) = K(\sigma)= \sum_{n\geq 1}K_n(\sigma_1,\ldots, \sigma_n) t^n.
	\]
	Since $\sigma_i$ are independent, each $K_n$ is uniquely determined in this way.
\end{proof}

\subsection{Pontryagin Classes}

The first application of the construction presented in the previous section is the construction of characteristic classes for real vector bundles. Note that:
\begin{proposition}\label{prop:chern-conjugation}
	For any complex vector bundle $\mathcal{E}$, we have
	\[
		\cl_k(\overline{\mathcal{E}}) = (-1)^k\cl(\mathcal{E})
	\]
	where $\overline{\mathcal{E}}$ denotes complex conjugation.
\end{proposition}
\begin{proof}
	See Lemma 14.9 of \cite{milnorstasheff1974}.
\end{proof}

Now given a real vector bundle $\mathcal{E}$, we can complexify it to get a complex vector bundle $\mathcal{E}_\C = \mathcal{E}\otimes \C$. 
Note that complexification of bundles can be interpreted by structure group reduction $\U_m \to \SO_{2m}$, which induces an injective map $\H^\bullet(\BSO_{2m})\to \H^\bullet(\U_m)$. 
The complexified tangent bundle is canonically isomorphic to its conjugate bundle, so by \cref{prop:chern-conjugation} we have
\[
	\cl_{2k+1}(\overline{\mathcal{E}_\C}) = -c_{2k+1}(\mathcal{E}_\C) \quad\implies\quad 2\cdot c_{2k+1}(\mathcal{E}_\C) = 0.
\]
If we mod out by these $2$-torsion elements (for instance taking $R=\Z[\frac{1}{2}])$, we get
\[
	\cl(\mathcal{E}_\C) = 1+\cl_2(\mathcal{E}_\C) + \cl_4(\mathcal{E}_\C)+\cdots
\]

\begin{definition}
\end{definition}

\begin{proposition}
	We have the relation
	\[
		p_n =\cl_n^2 - 2\sum_{1\leq k\leq n}(-1)^k\cl_{n-k}\smile \cl_{n+k}.
	\]
\end{proposition}

In the language of multiplicative sequences, we might say that the Pontryagin class is the characteristic class associated to the characteristic series $Q(t)=1+t^2$.

\begin{proposition}\label{prop:pontryagin-class-complex-projective-space}
	\[
		p(\CP^n) = (1+\alpha^2)^{2n+1}
	\]
\end{proposition}

\begin{example}
	The Pontryagin classes of the first few dimensions of complex projective space are given by:
	\[
		\begin{aligned}
			p(\CP^1) & = 1,                                    \\
			p(\CP^2) & = 1+3\alpha^2,                          \\
			p(\CP^3) & = 1+4\alpha^2,                          \\
			p(\CP^3) & = 1+5\alpha^2 + 10\alpha^4,             \\
			p(\CP^4) & = 1+6\alpha^2 + 15\alpha^4,             \\
			p(\CP^4) & = 1+7\alpha^2 + 21\alpha^4+ 35\alpha^6, \\
		\end{aligned}
	\]
	Note that $p_k[\CP^{2k}]$.
\end{example}

\begin{theorem}\label{thm:cohomology-o}
	The cohomology rings of classifying spaces is given by
	\begin{enumerate}[(a)]
		\item $\H^\bullet(\BU_n;\Z)\cong \Z[c_1,\ldots, c_n]$\hfill where $|c_i|=2i$,
		\item $\H^\bullet(\BSO_{2n+1};\Z[\frac12])\cong \Z[\frac12][p_1,\ldots, p_n]$\hfill where $|p_i|=4i$,
		\item $\H^\bullet(\BSO_{2n+2};\Z[\frac12])\cong \Z[\frac12][p_1,\ldots, p_n, e]$\hfill where $|p_i|=4i$, $|e|=2n$.
	\end{enumerate}
	Consequently, the cohomology rings of stable classifying spaces are 
	\begin{enumerate}[(a)]
		\item $\H^\bullet(\BU;\Z)\cong \Z[c_1,c_2,\ldots]$\hfill where $|c_i|=2i$,
		\item $\H^\bullet(\BSO;\Z[\frac12])\cong \Z[\frac12][p_1,p_2,\ldots]$\hfill where $|p_i|=4i$.
	\end{enumerate}
\end{theorem}

\begin{proposition}\label{prop:pontryagin-}
	If $M$ is a null-cobordant manifold and $c\in \H^\bullet(\BSO, \Z[\frac12])$, then $c[M]=0$.
\end{proposition}

% \begin{definition}
% 	The \defn{Stiefel-Whitney class $w$}[Stiefel-Whitney class] is the unique characteristic class for unoriented vector bundles ($G=
% 		\O$) in singular cohomology with $\Z/2$ coefficients satisfying:
% 	\begin{enumerate}[(a)]
% 		\item $w$ is rank-normalized and multiplicative,
% 		\item $w(\gamma^1_1)=1+\alpha$.
% 	\end{enumerate}
% 	We denote by $w_i$ the degree $i$ homogeneous component of $w$ in $\H^i(-;\Z/2)$.
% \end{definition}
%
% \begin{remark}
% 	Note that axiom (c) does not violate \cref{cor:mobius-characteristic-2-torsion}, since we are working in $\Z/2$ and so every cohomology class has $2$-torsion.
% \end{remark}
%
% \subsection{Pontryagin Classes}

%
% \begin{corollary}
% 	A degree $k$ Pontryagin number $K(p)$ is a well-defined group homomorphism
% 	\[
% 		\lkxfunc{}{\Omega^\SO_k}{\Z}{[M]}{K(p)[M]}
% 	\]
% \end{corollary}
%
% \begin{corollary}
% 	If $K(p)$ is a multiplicative Pontryagin number, then
% \end{corollary}
%
% \subsection{Chern-Weil Theory}\label{sec:chern-weil-theory}
%
% While the axiomatic and universal definitions of characteristic classes are simple and abstract, it is often useful to
%
% \subsection{Wu Classes}\label{sec:wu-classes}
%
% \begin{theorem}
% 	$w=\Sq(v)$.
% \end{theorem}
%
% A refinement
% For complex vector bundles, i.e.
%
% By \cref{thm:universal-characteristic-classes}, it suffices to compute the cohomology of classifying spaces $\BO_n, \BSO_n, \BU_n$ and their stable counterparts.
%
% \begin{theorem}\label{thm:cohomology-o}
% 	The cohomology rings of classifying spaces is given by
% 	\begin{enumerate}[(a)]
% 		\item $\H^\bullet(\BO_n;\Z/2)\cong \Z/2[w_1,\ldots, w_n]$\hfill where $|w_i|=i$,
% 		\item $\H^\bullet(\BU_n;\Z)\cong \Z[c_1,\ldots, c_n]$\hfill where $|c_i|=2i$,
% 		\item $\H^\bullet(\BSO_{2n+1};\Z[\frac12])\cong \Z[\frac12][p_1,\ldots, p_n]$\hfill where $|p_i|=4i$,
% 		\item $\H^\bullet(\BSO_{2n+2};\Z[\frac12])\cong \Z[\frac12][p_1,\ldots, p_n, e]$\hfill where $|p_i|=4i$, $|e|=2n$.
% 	\end{enumerate}
% 	Consequently, the cohomology rings of stable classifying spaces are 
% 	\begin{enumerate}[(a)]
% 		\item $\H^\bullet(\BO;\Z/2)\cong \Z/2[w_1,w_2,\ldots]$\hfill where $|w_i|=i$,
% 		\item $\H^\bullet(\BU;\Z)\cong \Z[c_1,c_2,\ldots]$\hfill where $|c_i|=2i$,
% 		\item $\H^\bullet(\BSO;\Z[\frac12])\cong \Z[\frac12][c_1,c_2,\ldots]$\hfill where $|p_i|=4i$.
% 	\end{enumerate}
% \end{theorem}
%
% \subsection{The Euler Class}\label{sec:euler-class}
%
% \begin{definition}\label{def:euler-class}
% 	The \defn{Euler class}
% \end{definition}
%
% \begin{proposition}
% 	On a closed $n$-manifold manifold $M$, $e[\T M]=\sum_k (-1)^k \rank \H^k(M)$.
% \end{proposition}
%
% \begin{corollary}
% 	The Euler number of a sphere is $e[\T S^n] = (1+(-1)^n)[S^n]$.
% \end{corollary}
% \begin{definition}
% 	The \defn{Euler class} $e$ is a multiplicative characteristic class for oriented real vector bundles in $\Z$ singular cohomology satisfying the following axioms:
% 	\begin{enumerate}[(a)]
% 		\item If a bundle $\mathcal{E}$ has a non-zero section, then $e(\mathcal{E})=0$. 
% 		\item If $-\mathcal{E}$ has opposite orientation to $\mathcal{E}$, then $e(-\mathcal{E})=-e(\mathcal{E})$.
% 	\end{enumerate}
% \end{definition}

\subsection{The Hirzebruch Signature Theorem}

We now use the theory of characteristic classes to come up with an expression for the signature of an oriented manifold. Recall from \cref{sec:cobordism} that the signature can be interpreted as a ring homomorphism
\[
		\Omega_\bullet^\SO \otimes \Q \lkxto \Z.
\]
However, any (stable) multiplicative characteristic class for oriented vector bundles also induces a ring homomorphism. For instance, the Pontryagin class evaluates to
\[
		p[\CP^{2k}] = \binom{2n+1}{2k}
\]
on the generators of $\Omega_\bullet^\SO\otimes \Q$. If we can find some multiplicative characteristic class which evaluates to $1$ on $\CP^{2k}$, we would be able to express the signature in terms of characteristic classes. In general, a ring homomorphism from a cobordism ring is known as a \defn{genus}.

\begin{definition}
	The \defn{$L$-genus} of a closed oriented manifold $M$ is the characteristic number $L[M]$ where $L$ is the characteristic class coming from the characteristic series
	\[
		Q_{L}(t) = \frac{t}{\tanh(t)}=1+\frac{t^2}{3} - \frac{t^4}{45} + \frac{2t^6}{645}-\frac{t^8}{4725}+\cdots.
	\]
\end{definition}

Since this sequence only contains even-degree terms, we can express the associated multiplicative sequence of polynomials in terms of Pontryagin classes.

\begin{example}\label{example:L-genus}
	The first few polynomials in the $L$-genus are given by:
\[
	\begin{aligned}
		L_1 &= \frac{1}{3}p_1,\\
		L_2 &= \frac{1}{45}(-4p_2 + 7p_1^2),\\
		L_3 &= \frac{1}{945}(62p_3-13p_2p_1+2p_1^3),\\
		L_4 &= \frac{1}{14175}(831p_4-71p_3p_1-19p_2^2 + 22p_2p_1^2 - 3p_1^4)
	\end{aligned}
\]
	For information on how these polynomials were computed, see \cref{chap:wolfram}.
\end{example}

\begin{theorem}[Hirzebruch]\label{thm:hirzebruch-signature-theorem}
	Let $M$ be a closed $4k$-manifold. Then we have
	\[
		\sigma(M) = L_k(p_1, \cdots, p_k)[M],
	\]
	where $L_k$ is the $L$-genus.
\end{theorem}
\begin{proof}
	It suffices to check this identity for the generators $\CP^{2k}$. The total Pontryagin class of $\CP^{2k}$ is $p(\CP^{2k})=(1+\alpha^2)^{2k+1}$.
	\[
		\begin{aligned}
			L(p)(\CP^{2k})
			= L((1+\alpha^2)^{2k+1})
			= L(1+\alpha^2)^{2k+1}
			= (\alpha/\tanh \alpha)^{2k+1}
			\quad\in\H^\bullet(\CP^{2k}).
		\end{aligned}
	\]
	Here, we consider $\alpha/\tanh \alpha$ as a formal power series in $\alpha$ truncated by the relation $\alpha^{2k+1}=0$ in the cohomology ring $\H^\bullet(\CP^{2k})$. The characteristic number $L_k(p_1,\ldots,p_k)[\CP^{2k}]$ is then the coefficient of $\alpha^{2k}$ in the expansion of $(\alpha/\tanh \alpha)^{2k+1}$.
	By elementary complex analysis, this coefficient can be extracted by taking a contour integral around a small $\epsilon$-circle about the origin in $\C$:
	\[
		\begin{aligned}
			L_k(p_1,\cdots, p_k)[\CP^{2k}]
			 & = \frac{1}{2\pi i}\oint_{S^1_\epsilon} \frac{dz}{z^{2k+1}} \left(\frac{z}{\tanh z}\right)^{2k+1}
			 &                                                                                                    \\[0.5em]
			 & = \frac{1}{2\pi i}\oint_{S^1_\epsilon} \frac{dz}{\tanh^{2k+1} z}\quad
			 & u  =\tanh(z),\quad
			du =(1-u^2)dz
			\\[0.5em]
			 & = \frac{1}{2\pi i}\oint_{S^1_\epsilon} \frac{1}{u^{2k+1}}\cdot\frac{du}{1-u^2}
			 &                                                                                                    \\[0.5em]
			 & = \frac{1}{2\pi i}\oint_{S^1_\epsilon} \frac{1+u^2+u^4+\cdots}{u^{2k+1}}\,du                       \\[0.5em]
			 & =1.                                                                                              &
		\end{aligned}
	\]
	Since $\sigma(\CP^{2k})=1$, this completes the proof.
\end{proof}

\begin{remark}
	Instead of checking the coefficients from the power series, we might use something like the Lagrange-B\"urmann formula to derive $t/\tanh t$ directly from the series $1/(1-t^4)=1+t^4+t^8+\cdots$ ``generating'' the signature as a map from cobordism. 
\end{remark}

The Hirzebruch signature theorem is a shining example of a truly remarkable theorem in mathematics -- it gives us easy computational means to uncover highly non-trivial relationships between complicated objects. For us, the most useful consequence of the Hirzebruch signature theorem is that it gives us subtle integrability and divisibility theorems. For instance, the expression for the leading coefficient of $L_3$ in \cref{example:L-genus} immediately implies:
\begin{corollary}
	If $M$ is a closed $12$-manifold with $\H^4(M)\cong 0$, then $\sigma(M)$ is divisible by $62$.
\end{corollary}
The observation that for such manifolds $M$, the quantity $\sigma(M)/62$ is an integer, let alone equal to $945p_3[M]$, is far from obvious, and yet it pops out immediately from the signature theorem. These subtle relationships are immensely useful in defining invariants capable of detecting exotic spheres as we will see in the following \cref{sec:invariants-for-homotopy-4k-1-spheres}.

While we are on the topic of the $L$-genus, let us compute its leading coefficient. Often times, the manifolds which we will apply the signature theorem to will be connected enough that the lower order Pontryagin classes vanish -- leaving just the leading term.
Luckily for us, the coefficient of this leading term admits a simple description in terms of the \defn{Bernoulli numbers} $B_{2k}$, a sequence of rational numbers which appear ubiquitously throughout topology, homotopy theory, number theory, and many other disciplines. There are many conventions in the literature, but for our purposes we can define them as the terms appearing in the series expansion:
\[
	\frac{t}{1-e^{-t}} =  1+\frac{1}{2}t + \frac{1}{12}t^2-\frac{1}{720}t^4+\frac{1}{30240}z^6-\frac{1}{1209600}t^8+\cdots=\sum_{n\geq 0}\frac{B_n}{n!}t^n.
\]
\begin{remark}
	The characteristic series $t/(1-e^{-t})$ generates the Todd genus, a fundamental invariant in complex geometry. See \todo{cite} for more information.
\end{remark}
As we are working in the real case, we need only consider the even degree terms. The first few Bernoulli numbers are given by:
\begin{equation}\label{eq:bernoulli_numbers}
	B_0 = 1,\quad B_2 = \frac{1}{6},\quad B_4 = \frac{1}{30},\quad B_6=\frac{1}{42},\quad B_{8}=\frac{1}{30},\quad B_{10} = \frac{5}{66},\quad\cdots
\end{equation}
Bernoulli numbers are ubiquitous in topology and homotopy theory, \todo{cite}

\begin{equation}\label{eq:tanh_series}
	\begin{aligned}
		\tanh t &= t - \frac{1}{3}t^3 + \frac{2}{15}t^5 - \frac{17}{315}t^7+\cdots = \sum_{k\geq 1} (-1)^k\frac{2^{2k}(2^{2k}-1)B_{2k}}{(2k)!}\, t^{2k-1}\\
		\frac{t}{\tanh(t)} &= 1+\frac{1}{3}t^2-\frac{1}{45}t^4+\frac{2}{945}t^6 - \frac{1}{4725}t^8 +\cdots = \sum_{0\leq n}\frac{2^{2n}B_{2n}}{(2n)!}.
	\end{aligned}
\end{equation}

From this, it follows that:

\begin{proposition}\label{prop:leading_coefficient_L_genus}
	The leading coefficient of the $L$-genus is $s_k=2^{2k}(2^{2k-1}-1)B_{2k}/(2k)!$
\end{proposition}
