In this chapter, we'll 

\begin{theorem}\label{thm:homotopy_spheres_are_stably_parallelizable}
  If $\Sigma$ is a homotopy $n$-sphere, then $\T \Sigma \oplus \underline{\R}$ is trivial. 
\end{theorem}
\begin{proof}
	The only obstruction to the triviality of $\T\Sigma\oplus\underline{\R}$ is a well-defined cohomology class:
	\[
		\mathfrak{o}_n(\Sigma) \in \H^n(\Sigma; \pi_{n-1}(\SO_{n+1})) = \pi_{n-1}(\SO_{n+1})
	\]
	The coefficient group may be identified with the stable group $\pi_{n-1}(\SO)$, but these stable groups have been computed by Bott in \cite{bott1957}, for $n\geq 2$ we have:
	\begin{center}
		\begin{tabular}{c|cccccccc}
			\textrm{residue class of $n\mod 8$} & 0 & 1 & 2 & 3 & 4 & 5 & 6 & 7\\
			\hline
			$\pi_{n-1}(\SO)$ & $\Z$ & $\Z_2$ & $\Z_2$ & 0 & $\Z$ & 0 & 0 & 0.
		\end{tabular}
	\end{center}
	If $\pi_{n-1}(\SO)$ is zero, we are done. 

	If $\pi_{n-1}(\SO) = \Z$, then $n=4k$. According to \cite{kervairemilnor1960} and \cite{kervaire1959}, some non-zero multiple of the obstruction class $\mathfrak{o}_n(\Sigma)$ can be identified with the Pontryagin class $p_k(\T \Sigma\oplus \underline{\R}) = p_k(\Sigma)$. \todo{(why?)} But the Hirzebruch signature theorem implies \todo{(why?)} that $p_k(\Sigma)$ is a multiple of the signature $\sigma(\Sigma)$ which is zero since $\H^{2k}(\Sigma)=0$. Thus every homotopy $4k$-sphere is \textsc{s}-parallelizable. 

	Finally, suppose $\pi_{n-1}(\SO)= \Z_2$. It follows from an argument of Rohlin \todo{(what?)} that $J_{n-1}(\mathfrak{o}_n(\Sigma))=0$ where $J_{n-1}$ denotes the Hopf-Whitehead homomorphism
	\[
		\lkxfunc{J_{n-1}}{\pi_{n-1}(\SO_k)}{\pi_{n+k-1}(S^k)}
	\]
	in the stable range $k >n$. But $J_{n-1}$ is injective for $n\equiv 1, 2\mod 8$. This is proven by Adams. \todo{(find)} This means that $\mathfrak{o}_n(\Sigma)=0$.
\end{proof}

In fact, a much stronger result holds true.
\begin{theorem}\label{thm:homotopy_spheres_tangent_bundle}
  If $\Sigma$ is a homotopy $n$-sphere with $f : S^n \to \Sigma$ the homotopy equivalence, then the induced bundle map is an isomorphism $f^*\T\Sigma \approx \T S^n$.
\end{theorem}
