\chapter{Conclusion}

\begin{epigraph}{10em}{}
\end{epigraph}

\section{Computational Stuf}\label{sec:computational stuff}

\section{The Poincar\'e Hypothesis in Low Dimensions}

The reduction of the problem of finding diffeomorphism classes of homotopy spheres to the problem of finding $h$-cobordism classes of spheres was a massive simplification, and allowed us to almost fully classify the set of smooth structures on a sphere.
The lack of an $h$-cobordism theorem in dimensions $<5$ means that practically none of the techniques developed in this thesis work for these low dimensions and so they must be analyzed manually. The uniqueness of smooth structure on a circle is a standard exercise in introductory topology. In dimension $2$, we can give the sphere a complex structure making it a Riemann surface, and by the uniformization theorem, the complex structure must be conformally equivalent to the Riemann sphere. This is the unique complex structure and so it has a unique smooth structure.

For the case of dimension $3$, the uniqueness of a smooth structure is a orders of magnitude harder to prove. In the 1950s, Moise proved the equivalence of topological, PL, and smooth structures for compact $3$-manifolds. These results are outlined in the book \cite{moise1977geometric}. For decades, the uniqueness of smooth structures on the 3-dimensional sphere was thus relegated to a proof of the topological Poincar\'e hypothesis in 3-dimensions -- the original conjecture proposed of Poincar\'e in 1904. The topological Poincar\'e hypothesis had already been proved in dimensions $\geq 5$ by Smale, Stallings, and Zeeman in the early 1960s, and in dimension $4$ by Freedman's 1982 classification \cite{freedman1982manifold} of simply-connected topological $4$-manifolds using many of the techniques we'll discuss in this thesis. Yet, the stubborn third dimension remained.
This proof finally came in 2003 as a consequence of Perelman's proof of Thurston's geometrization conjecture, a general classification for 3-dimensional geometries.
The battle with $3$-manifolds was a tough one, and it took hundreds of pages of hard analysis and Riemannian geometry by Hamilton and Perelman.
The basic idea by Hamilton to prove the Poincar\'e conjecture involved giving a
closed $3$-manifold a Riemannian metric $g_{\mu\nu}$, and then evolving this metric in time $\lambda$ by a differential equation involving the Ricci curvature tensor
\begin{equation}
	\frac{\partial g_{\mu\nu}}{\partial \lambda} = -2\mathcal{R}_{\mu\nu}.
	% \hspace{-3em}\tag{Ricci Flow Equation}
\end{equation}
This is known as the Ricci flow equation, and it forces the metric to change in such a way as to make distances decrease in directions of positive curvature.
Ricci flow can be used to ``smooth out'' the curvature of a nice enough $3$-manifold until it has constant curvature. In particular, simply connected manifolds turn into spheres under this regime, proving the Poincar\'e hypothesis.
Unfortunately, singularities can form when solving the Ricci flow equations, and it takes a careful application of the ideas of surgery to get around them. Perelman proved that this ``Ricci flow with surgery'' was always possible, turning Hamilton's ambitious program into rigorous mathematical machinery. For a comprehensive introduction to Perelman and Hamilton's proof of the geometrization conjecture, see \cite{morgantian2007ricci}.

The last remaining case of the generalized Poincar\'e conjecture is the classification of smooth (and equivalently PL) structures on spheres in dimension $4$. It remains hopelessly unsolved as of the conclusion of this thesis in March 2025. There are some candidates

\todo{talk a bit more about this, exotic $\R^4$s and why people suspect there might be exotic spheres in dimension 4}

\section{Surgery Theory in General}\label{sec:surgery-theory-in-general}

\begin{theorem}[$s$-cobordism]
  test
\end{theorem}

\[
	L_{n+1}(\Z[\pi_1(X)]) \lkxto \mathcal{S}^\DIFF(X) \lkxto {[X, G/\DIFF]} \lkxto L_n(\Z[\pi_1(X)])
\]

\section{Obstructions to Smoothing}\label{sec:smoothing-obstructions}

\begin{theorem}[Freedman, 1982]\label{thm:freedman-actual}
\end{theorem}

\section{Twisted Spheres}\label{sec:twisted-spheres}
Lastly, we will briefly mention the construction of homotopy spheres by means of ``twisting'' together two 

\todo{rephrase}

	The definition of connected sum for smooth manifolds is slightly stronger than the definition for topological manifolds. In the topological category, we could simply cut out open disks from both manifolds and identify their boundaries, i.e. we set
	\begin{equation}\label{eq:connected-sum-in-topological-category}
		M_1\# M_2 = (M_1\setminus \Int(\iota_1(D^n)))\cup_g (M_2\setminus \Int(\iota_2(D^n)))
	\end{equation}
	where $g : \partial \iota_1(D^n) \to \partial \iota_2(D^n)$ is any orientation-reversing homeomorphism. This definition turns out to be well-defined in the topological category, although proving this takes a considerable amount of work. \todo{cite}

	However, the connected sum in the topological category will not give a unique connected sum in the smooth category, in fact far from it. Interestingly enough, the failure for \cref{eq:connected-sum-in-topological-category} to give a unique smooth manifold is related to exotic spheres in the following way. Whenever we have an orientation-preserving diffeomorphism $f: S^{n-1}\to S^{n-1}$, identifying $\partial D^n = S^{n-1}$ allows us to glue together disks to get $T(f)=D^n\cup_f D^n$. This is a smooth manifold which is the (topological) connected sum of two spheres so must be homoemorphic to a sphere. The manifolds $T(f)$ are called \defn{twisted spheres} since they are built by ``twisting together'' two disks by $f$. 

	We can interpret the twisted sphere construction as a map $T: \Diff^+(S^{n-1})\to \Theta^n$ sending an orientation-preserving diffeomorphism $f : S^{n-1} \to S^{n-1}$ to the twisted sphere $T(f)$. For any (smooth) path $\omega : I \to \Diff^+(S^{n-1})$, we can build an $h$-cobordism 
	\[ (D^n\times I)\cup_\omega (D^n\times I) : T(\omega_0) \sohbord T(\omega_1)\]
	where we interpret the path $\omega$ as a smooth homotopy $\omega : I\times S^{n-1}\to\S^{n-1}$ between diffeomorphisms $\omega_0, \omega_1 : S^{n-1} \to S^{n-1}$. For $n\geq 5$, \cref{thm:h-cobordism-diffeomorphism} implies that $T(\omega_0)$ is diffeomorphic to $T(\omega_1)$ so the map $T$ only depends on the path component of $f\in \Diff^+(S^{n-1})$.
	Next, note we can extend $f$ to a diffeomorphism on the interior of the disk, the resulting twisted sphere must be diffeomorphic to a sphere. By a similar argument, we can reduce to path-components. Altogether, we have an exact sequence (of sets)
	\begin{equation}\label{eq:twisted-sphere-exact-sequence-proto}
		\pi_0 [\Diff^+(D^n)] \lkxto \pi_0 [\Diff^+(S^{n-1})] \lkxto[T] \Theta^n.
	\end{equation}
	Finally, if we take any homotopy sphere $\Sigma\in \Theta^n$, cutting out the interiors of any embedded open disks $D_1, D_2\subset \Sigma$ gives an $h$-cobordism $\Sigma\setminus(D_1\cup D_2) : \partial D_1 \sohbord \partial D_2$. If $n\geq 6$, the $h$-cobordism theorem (\ref{thm:h-cobordism}) implies that $\Sigma \setminus (D_1\cup D_2)$ is diffeomorphic to a cylinder $\partial D_2\times [0,1]$. It follows that $\Sigma$ is a twisted sphere corresponding to the diffeomorphism $\partial D_1 \cong \partial D_2$ coming from the $h$-cobordism.
	When $n\geq 6$, the exact sequence \cref{eq:twisted-sphere-exact-sequence-proto} therefore extends to 
	\[
		\pi_0 [\Diff^+(D^n)] \lkxto \pi_0 [\Diff^+(S^{n-1})] \lkxto[T] \Theta^n \lkxto 0.
	\]
	In 1970, Cerf proved the pseudoisotopy theorem \cite{cerf1970pseudoisotopy}, one of the consequences of which implies that $\pi_0[\Diff^+(D^n)]=0$ for $n\geq 6$. From this it follows that:

	\begin{proposition}
		For $n\geq 6$, there is a bijection $\Theta^n \cong \pi_0[\Diff^+(S^{n-1})]$.
	\end{proposition}

	In my subjective opinion, this is the most canonical way to observe the phenomenon of exotic smooth structures on the spheres. Rather than work with a set of abstract smooth manifolds and diffeomorphisms between them, it is possible to interpret the set of smooth structures on a sphere as the set of path-components of diffeomorphisms on a specific sphere. This bijection is also a reason as to why some theoretical physicists care about exotic spheres. \todo{change of coordinates, 10-dimensional change of coordinates (with compact support) has 992 components.} 
	For instance, Witten's 1985 paper \cite{witten1985anomalies} on global anomalies in string theoretic models of gravity contains extensive discussions on exotic spheres, and the use of geometric invariants in detecting them.

\begin{theorem}
  For $n\geq 6$, there is a bijection $\Theta^n \cong \pi_0[\Diff^+(S^{n-1})]$.
\end{theorem}

\subsection{Global Gravitational Anomalies}

\cite{witten1985anomalies}
