\chapter{Conclusion}\label{chap:conclusion}

\begin{epigraph}{15em}{Frank Herbert}
	Beginnings are such delicate times.
\end{epigraph}

We have only barely begun to scratch the surface of the topic of exotic spheres. For one thing, we have spent the vast majority of the thesis working in $\bP^{4k}$, only briefly touching on the case of $\bP^{4k+2}$. Aside from these cases, there is an entire world of exotic spheres which evade the detection techniques discussed in this thesis. Exotic spheres in $\Theta^n\setminus \bP^{n+1}$ are known as \defn{very exotic spheres}, and are far harder to visualize than the spheres we have constructed since they are detectable primarily by homotopy theoretic means. 

In general, there is an

\begin{theorem}\label{thm:framed-surgery-highly-connected}
	Every homotopy sphere in $\bP^{n+1}$ of dimension $n\geq 4$ has a framed highly-connected boundary.
\end{theorem}
\begin{proof}
	This is Theorem 5.5 and 6.6 of \cite{milnorkervaire1963groups}. 
\end{proof}

Since highly-connected manifolds with boundary in odd dimensions are contractible, we have the immediate corollary:

\begin{corollary}\label{cor:odd-dimensional-bP-trivial}
	The groups $\bP^{2k+1}$ are trivial.
\end{corollary}

The remaining non-trivial cases $\bP^{4k}$ and $\bP^{4k+2}$ take some more work. We again black-box some useful theorems from the surgery theory of simply connected manifolds.

\begin{theorem}
	A compact, framed, highly-connected manifold $W$ bounding a homotopy sphere can be framed surgered into a contractible manifold if and only if $\sigma(W)=0$.
\end{theorem}
\begin{proof}
	See Section III of \cite{browder1972surgery}.
\end{proof}

Consequently, since the signature of any framed $4k$-manifold is divisible by $8$, the $E_8$ plumbing from \cref{sec:plumbing} gives a surjective group homomorphism 
\begin{equation}\label{eq:plumbing-map}
		\Z \lkxsurj \bP^{4k}
\end{equation}
which sends an integer $n$ to the connected sum $\partial P^{4k}(E_8)\+\cdots \+ \partial P^{4k}(E_8)$, where we repeat the connected sum $|n|$ times and reverse orientation if $n$ is negative. In particular, we can conclude that $\bP^{4k}$ is a cyclic group. 

%
% By some careful arguments in obstruction theory and homotopic calculations, we can compute the kernel of  \cref{eq:plumbing-map} to arrive at the Milnor-Kervaire theorem:
%
% \begin{theorem}[Milnor-Kervaire]
% 	For every $k>1$, $bP^{4k}$ is a cyclic group of order
% 	\[
% 		|\bP^{4k}| =2^{2k-2}(2^{2k-1}-1)\epsilon(k)\denom(B_{2k}/4k).
% 	\]
% 	Recall here that $\epsilon(k)$ is $2$ when $k$ is odd and $1$ when $k$ is even.
% \end{theorem}
% \begin{proof}
% 	See Lemma 3.5 of \cite{milnor1958manifolds} or Corollary 3.20 of \cite{levine1985lectures} for a summary.
% \end{proof}

% \section{Classification of Exotic Spheres}\label{sec:classification}
%
% We now briefly discuss the classification of exotic spheres using the Pontryagin-Thom map, the $J$-homomorphism, and stable homotopy theory. For more details, we defer to the wonderful expositions of \cite{milnorkervaire1963groups} and \cite{levine1985lectures}.
%
% \subsection{Framed Surgery}
%
% The first major result of the classification of homotopy spheres is a complete determination of the group $\bP^{n+1}$. This is done by means of surgery on a coboundary -- given a homotopy sphere $M\in \bP^{n+1}$ with coboundary $W$, we simplify the topology of $W$ as much as possible while keeping the boundary $M$ fixed. This technique is an example of the operation of \defn{surgery}, also known as \defn{spherical modification}.
%
% The basic observation underlying the operation is the fact that for manifolds $X$ and $Y$, we have:
% \[
% 	\partial (X\times Y) = (\partial X \times Y)\cup (X\times \partial Y).
% \]
% Consequently, for integers $p,q$ the space $S^q\times S^{q-1}$ can be viewed either as the boundary of the $(p+1)$-handle $D^{p+1}\times S^{q-1}$ or as the boundary of the $q$-handle $S^p\times D^q$, since
% \[
% 	\partial (S^p\times D^q) = S^p\times S^{q-1} = \partial (D^{p+1}\times S^{q-1}).
% \]
% Since the two products have the same boundary, whenever we find an embedding of $D^{p}\times S^{q}$ in a manifold, we can cut it out and replace it with $D^{p+1}\times S^{q-1}$.
%
% \begin{example}\label{exam:surgery-on-a-compact-surface}
% 	A simple yet non-trivial example of surgery is on a surface $X_g$ of genus $g$.
% 	When $q=1$, note that $S^q\times D^{n-q} = S^1\times D^1$ is a cylinder, and $D^{q+1}\times S^{n-q-1}=D^2\times S^0$ is a disjoint union of disks.
% 	If we embed a cylinder $S^1\times D^1$ into a ``handle'' of $X_g$, the resulting surgery cuts off this handle and reduces the genus by one. By a sequence of $g$ surgeries along embedded cylinders, we can turn $X_g$ into a sphere $X_0 = S^2$.
% 	\begin{figure}[ht]
% 		\centering
% 		\import{diagrams}{surgery-on-two-holed-torus.pdf_tex}
% 		\caption{Turning a genus 2 surface into a genus 1 surface via surgery.}\label{fig:surgery-on-two-holed-torus}
% 	\end{figure}
% 	Note that corners need to be smoothed, see \cref{sec:smoothing-corners}.
% \end{example}
%
% If the manifolds involved are framed, there is corresponding notion of a \defn{framed surgery}, see Section 2 of \cite{levine1985lectures} for details. 
%
% Using the techniques developed in \cite{milnor1961procedure} for killing homotopy groups of a manifold using surgery operations, Milnor and Kervaire proved:
%
%
%
%
%
% However, it is generally not exotic. The problem of determining when $\partial P^{4k+2}(A_2)$ is exotic is quite hard, and is known as the Kervaire invariant problem. Some sparse results were known to Milnor and Kervaire by the time they wrote \cite{milnorkervaire1963groups}, at the time it was known that the Kervaire sphere was not exotic when $k=1$ and $k=3$. 
%
% \todo{The first major breakthrough was by Browder in \cite{browder1969kervaireinvariant}, who showed that 
%
% and was only recently completed in 2024.
%
% \begin{theorem}
% 	\[
% 		\bP^{2k}\cong \begin{cases}\Z/2 & k\neq 2^\ell-1\\
% 			0 & k=3,7,15,31,162
% 		\end{cases}
% 	\]
% \end{theorem}
%
% \begin{theorem}
% 	The only $k$ for which $\partial P^{4k+2}(A_2)$ is exotic are $k=6,14,30,62,$ and $126$.
% \end{theorem}
%
% See Chapter 4 of \cite{levine1985lectures}.}
%
% \smallrule
%
% \begin{figure}[ht]
% 	\renewcommand{\arraystretch}{1.2}
% 	\centering
% 	\begin{tabular}{r|c|c|c|c|c|c|c|c|c|c|c|c|c|c|c}
% 		\textrm{$n$}               & 1 & 2 & 3 & 4 & 5 & 6 & 7  & 8 & 9 & 10 & 11  & 12 & 13 & 14 & 15    \\
% 		\hline
% 		\textrm{$|\bP^{n+1}|$} & 1 & 1 & 1 & 1 & 1 & 1 & 28 & 1 & 2 & 1  & 992 & 1  & 1  & 1  & 8128\\
% 		\hline
% 		\textrm{$|\Theta^n|$} & 1 & 1 & 1 & 1 & 1 & 1 & 28 & 2 & 8 & 6  & 992 & 1  & 3  & 2  & 16256 \\
% 	\end{tabular}
% \end{figure}
%
% \section{Further Topics}\label{sec:further-topics}
%
% \subsection{The Poincar\'e Hypothesis in Low Dimensions}\label{sec:low-dimensions}
%
% The reduction of the problem of finding diffeomorphism classes of homotopy spheres to the problem of finding $h$-cobordism classes of spheres was a massive simplification, and allowed us to almost fully classify the set of smooth structures on a sphere.
% The lack of an smooth $h$-cobordism theorem in dimensions $<5$ means that practically none of the techniques developed in this thesis work for these low dimensions and so they must be analyzed manually. 
%
% In dimension $1$, the uniqueness of smooth structure on a circle is a standard exercise in introductory topology. In dimension $2$, we can give the sphere a complex structure making it a Riemann surface, and by the uniformization theorem, the complex structure must be conformally equivalent to the Riemann sphere. This is the unique complex structure and hence the unique smooth structure.
%
% For the case of dimension $3$, the uniqueness of a smooth structure is a orders of magnitude harder to prove. In the 1950s, Moise proved the equivalence of topological, PL, and smooth structures for compact $3$-manifolds. These results are outlined in the book \cite{moise1977geometric}. For decades, the uniqueness of smooth structures on the 3-dimensional sphere was thus relegated to a proof of the topological Poincar\'e hypothesis in 3-dimensions -- the original conjecture proposed of Poincar\'e in 1904. The topological Poincar\'e hypothesis had already been proved in dimensions $\geq 5$ by Smale, Stallings, and Zeeman in the early 1960s, and in dimension $4$ by Freedman's 1982 classification \cite{freedman1982manifold} of simply-connected topological $4$-manifolds using many of the techniques we'll discuss in this thesis. Yet, the stubborn third dimension remained.
% This proof finally came in 2003 as a consequence of Perelman's proof of Thurston's geometrization conjecture, a general classification for 3-dimensional geometries.
% The battle with $3$-manifolds was a tough one, and it took hundreds of pages of hard analysis and Riemannian geometry by Hamilton and Perelman.
% The basic idea by Hamilton to prove the Poincar\'e conjecture involved giving a
% closed $3$-manifold a Riemannian metric $g_{\mu\nu}$, and then evolving this metric in time $\lambda$ by a differential equation involving the Ricci curvature tensor
% \begin{equation}
% 	\frac{\partial g_{\mu\nu}}{\partial \lambda} = -2\mathcal{R}_{\mu\nu}.
% 	% \hspace{-3em}\tag{Ricci Flow Equation}
% \end{equation}
% This is known as the Ricci flow equation, and it forces the metric to change in such a way as to make distances decrease in directions of positive curvature.
% Ricci flow can be used to ``smooth out'' the curvature of a nice enough $3$-manifold until it has constant curvature. In particular, simply connected manifolds turn into spheres under this regime, proving the Poincar\'e hypothesis.
% Unfortunately, singularities can form when solving the Ricci flow equations, and it takes a careful application of the ideas of surgery to get around them. Perelman proved that this ``Ricci flow with surgery'' was always possible, turning Hamilton's ambitious program into rigorous mathematical machinery. For a comprehensive introduction to Perelman and Hamilton's proof of the geometrization conjecture, see \cite{morgantian2007ricci}.
%
% The last remaining case of the generalized Poincar\'e conjecture is the classification of smooth (and equivalently PL) structures on spheres in dimension $4$. It remains hopelessly unsolved as of the conclusion of this thesis in March 2025. However, there are some reasons to suspect the existence of $4$-dimensional exotic sphere.
%
% \begin{theorem}
% \end{theorem}
%
% which the smooth $h$-cobordism theorem completely fails in dimension $4$.
%
% There are some candidates
%
% \todo{talk a bit more about this, exotic $\R^4$s and why people suspect there might be exotic spheres in dimension 4}
%
% \subsection{Exotic Spheres and Curvature}
%
% \begin{theorem}[Quarter-Pinched Sphere Theorem]
% 	If $M$ is a simply connected, geodesically complete Riemannian manifold with sectional curvature in the range $(1/4,1]$, then $M$ is homeomorphic to 
% \end{theorem}
%
% \begin{definition}
% \end{definition}
%
% \subsection{Obstructions to Smoothing}\label{sec:smoothing-obstructions}
%
% \begin{theorem}[Freedman, 1982]\label{thm:freedman-actual}
% \end{theorem}
%
