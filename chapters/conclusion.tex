\chapter{Conclusion}\label{chap:conclusion}

\begin{epigraph}{10em}{Frank Herbert}
	Beginnings are such delicate times.
\end{epigraph}

We have only barely began to scratch the surface of the topic of exotic spheres.

\section{Classification of Exotic Spheres}\label{sec:classification}

\section{The Poincar\'e Hypothesis in Low Dimensions}\label{sec:low-dimensions}

The reduction of the problem of finding diffeomorphism classes of homotopy spheres to the problem of finding $h$-cobordism classes of spheres was a massive simplification, and allowed us to almost fully classify the set of smooth structures on a sphere.
The lack of an $h$-cobordism theorem in dimensions $<5$ means that practically none of the techniques developed in this thesis work for these low dimensions and so they must be analyzed manually. The uniqueness of smooth structure on a circle is a standard exercise in introductory topology. In dimension $2$, we can give the sphere a complex structure making it a Riemann surface, and by the uniformization theorem, the complex structure must be conformally equivalent to the Riemann sphere. This is the unique complex structure and so it has a unique smooth structure.

For the case of dimension $3$, the uniqueness of a smooth structure is a orders of magnitude harder to prove. In the 1950s, Moise proved the equivalence of topological, PL, and smooth structures for compact $3$-manifolds. These results are outlined in the book \cite{moise1977geometric}. For decades, the uniqueness of smooth structures on the 3-dimensional sphere was thus relegated to a proof of the topological Poincar\'e hypothesis in 3-dimensions -- the original conjecture proposed of Poincar\'e in 1904. The topological Poincar\'e hypothesis had already been proved in dimensions $\geq 5$ by Smale, Stallings, and Zeeman in the early 1960s, and in dimension $4$ by Freedman's 1982 classification \cite{freedman1982manifold} of simply-connected topological $4$-manifolds using many of the techniques we'll discuss in this thesis. Yet, the stubborn third dimension remained.
This proof finally came in 2003 as a consequence of Perelman's proof of Thurston's geometrization conjecture, a general classification for 3-dimensional geometries.
The battle with $3$-manifolds was a tough one, and it took hundreds of pages of hard analysis and Riemannian geometry by Hamilton and Perelman.
The basic idea by Hamilton to prove the Poincar\'e conjecture involved giving a
closed $3$-manifold a Riemannian metric $g_{\mu\nu}$, and then evolving this metric in time $\lambda$ by a differential equation involving the Ricci curvature tensor
\begin{equation}
	\frac{\partial g_{\mu\nu}}{\partial \lambda} = -2\mathcal{R}_{\mu\nu}.
	% \hspace{-3em}\tag{Ricci Flow Equation}
\end{equation}
This is known as the Ricci flow equation, and it forces the metric to change in such a way as to make distances decrease in directions of positive curvature.
Ricci flow can be used to ``smooth out'' the curvature of a nice enough $3$-manifold until it has constant curvature. In particular, simply connected manifolds turn into spheres under this regime, proving the Poincar\'e hypothesis.
Unfortunately, singularities can form when solving the Ricci flow equations, and it takes a careful application of the ideas of surgery to get around them. Perelman proved that this ``Ricci flow with surgery'' was always possible, turning Hamilton's ambitious program into rigorous mathematical machinery. For a comprehensive introduction to Perelman and Hamilton's proof of the geometrization conjecture, see \cite{morgantian2007ricci}.

The last remaining case of the generalized Poincar\'e conjecture is the classification of smooth (and equivalently PL) structures on spheres in dimension $4$. It remains hopelessly unsolved as of the conclusion of this thesis in March 2025. There are good reasons as to why this case remains intractible. For one thing, even-dimensional exotic spheres (at least in dimensions $\geq 6$) can really only be detectable by homotopy theory since $\bP^{2k+1}\cong 0$. However, the usual methods of surgery theory do not apply since the $h$-cobordism theorem completely fails in dimension $4$. The remaining tools left in the as

which the smooth $h$-cobordism theorem completely fails in dimension $4$.

There are some candidates

\todo{talk a bit more about this, exotic $\R^4$s and why people suspect there might be exotic spheres in dimension 4}

\section{Obstructions to Smoothing}\label{sec:smoothing-obstructions}

\begin{theorem}[Freedman, 1982]\label{thm:freedman-actual}
\end{theorem}


\section{Further Reading}

\cite{witten1985anomalies}
