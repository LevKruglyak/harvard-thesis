\chapter{Fundamentals}\label{chap:fundamentals}

\begin{epigraph}{16em}{Frank Herbert}
	Beginnings are such delicate times.
\end{epigraph}

To begin our journey, we will provide some mathematical essentials which will be used ubiquitously throughout the thesis. These prerequisites can loosely be broken up into three sections. First, in \cref{sec:geometric-topology} we will discuss operations on smooth manifolds in geometric topology -- how to glue manifolds together, cut off handles, attach handles, and so on. We also are concerned with how such operations affect the homology of the manifold. By carefully choosing submanifolds representing homology classes, we will see how to alter the homology of a manifold in a controlled way by means of these operations. These operations form the basic ideas of surgery theory, a topic we will explore in greater depth in \cref{chap:h-cobordism} and \cref{chap:classification}. \todo{mention intersection form}

Altering topology by surgery naturally ties into our next topic of cobordism in \cref{sec:cobordism}, a natural notion of equivalence for manifolds. While cutting off or attaching handles changes the diffeomorphism type of a manifold, such operations do not change its cobordism type. This loose notion of equivalence allows us to fully classify manifolds up to cobordism -- something which is generally impossible for homeomorphism or difeomorphism.\footnote{In high dimensions, classifying manifolds is an \defn{undecidable problem}, meaning there is no finite algorithm that will correctly match a manifold to a list of representatives.}
The determination of manifolds up to cobordism is also one of the entrypoints of homotopy theory into differential topology via the Thom-Pontryagin construction, a massively successful crossover which reduces hard geometric problems to (also admittedly hard) problems of computing homotopy groups. With each flavor of cobordism, a new homotopy problem emerges. Ultimately, it will turn out that most of the computational difficulty in computing the set of smooth structures on a sphere is as a result of these problems in homotopy theory.

\todo{what other sections we need}

\pagebreak
\section{Basics of Geometric Topology}\label{sec:geometric-topology}

Whenever a manifold $N^k$ is embedded a larger ambient manifold $M^n$, there is a short exact sequence of tangent bundles
\begin{equation}
	0 \lkxto \T N\lkxto \T M|_N \lkxto \T M/N \lkxto 0
\end{equation}
where $\T M/N=\T M|_N / \T N$ denotes the \defn{normal bundle} of $N$ in $M$. A \defn{tubular neighborhood} of $N$ is a neighborhood $V\supset N$ in $M$ which is the diffeomorphic image of the set
\begin{equation}
	D(\T M/N) = \{\xi \in \T M/N \mid \|\xi\| <1 \}\subset \T M/N,
\end{equation}
for some given inner product structure on $\T M/N$. Note that $D(\T M/N)$ is the total space of the $(n-k)$-dimensional disk bundle over $N$ arising from the inner product. The restriction of the diffeomorphism to the image of the zero section in $D(\T M/N)$ should be the identity map to $N$.

A tubular neighborhood is then a way of ``thickening'' the submanifold $N$.
The diffeomorphism mapping $D(\T M/N) \to V$ can interpreted as a chart of a tube surrounding $N$ in $M$. Namely, the for each point $p\in N$, local trivializations of the bundle $D(\T M/N)$ are maps $\R^k \times D^{n-k}\to U\subset V$


\begin{theorem}[Tubular Neighborhood Theorem]
	Every properly embedded submanifold has a tubular neighborhood.
\end{theorem}

\begin{theorem}[Collar Neighborhood Theorem]
	Hello world
\end{theorem}

\todo{check boundary}

\subsection{Joining Two Manifolds Along a Submanifolds}

In the case that we have two manifolds $M_1^n$ and $M_2^n$ with interior embeddings $\iota_1 : N \to M_1$ and $\iota_2 : N\to M_2$ of some closed manifold $N^k$, we can join $M_1$ and $M_2$ along $N$ to get a single manifold which we might denote $M_1\cup_N M_2$.
To construct this manifold, let's start by picking tubular neighborhoods $V_1$ and $V_2$ for $\iota_1(N)\subset M_1$ and $\iota_2(N)\subset M_2$ respectively. We then set the joined manifold to be the quotient
\begin{equation}\label{eq:join-definition}
	M_1\cup_N M_2 = \frac{(M_1\setminus \iota_1(N))\sqcup (M_2\setminus \iota_2(N))}{(t\xi_1, \iota_1(p)) \sim ((1-t)\xi_2, \iota_2(p))}
\end{equation}
for all $p\in N$, $t\in(0,1)$, and unit normal vectors $\xi_1\in \T_p M/\iota_1(N)$, $\xi_2\in \T_p M/\iota_2(N)$ with $\|\xi_1\|=\|\xi_2\|=1$. The resulting smooth manifold is called a \defn{join along a submanifold}\footnote{Some authors refer to this operation as ``\defn{pasting}'' or as the ``\defn{generalized connected sum}'', see for instance Section VI.4 in \cite{kosinski1993differential}.} and is one of the fundamental operations of geometric topology.

\begin{remark}
	The tubular neighborhood ``sewing'' procedure in \cref{eq:join-definition} is extremely important in the category of smooth manifolds. If we were in the category of topological manifolds (assuming there were a notion of tubular neighorhood), we could define the join in a simpler way by removing the tubular neighborhoods from both manifolds and identifying the boundaries of the resulting manifolds by some homeomorphism. At each point of the shared boundary, the join would be locally Euclidean since it has a neighborhood which the glueing of two half-disks to get a full disk.

	However, the lack of a globally consistent ``collar'' by which these manifolds were glued leads to many possible smooth structures on the resulting manifold -- even if we require the boundary glueing map to be an orientation-preserving diffeomorphism. We will actually exploit this lack of unique smooth structure in \cref{sec:twisted-spheres} by constructing exotic spheres as the identification of two disks by an orientation-preserving diffeomorphism of their boundaries.
\end{remark}

\subsection{Connected Sum}

The simplest way to join two manifolds $M_1$ and $M_2$ along a submanifold is by embedding a point in both. The resulting join is an operation known as a \defn{connected sum} of $M_1$ and $M_2$. If the manifolds are connected,

and denoted by $M_1\# M_2$.

\begin{theorem}[Disk Theorem]
	test
\end{theorem}
\begin{proof}
	See Theorem B in \cite{palais1960diffeomorphism}.
\end{proof}

\begin{proposition}\label{prop:connected-sum-operation}
	The connected sum
\end{proposition}

\subsection{Spherical Modifications}

\begin{figure}[ht]
	\import{diagrams}{surgery-on-two-holed-torus.pdf_tex}
	\caption{Removing a }
\end{figure}

\subsection{Joining Manifolds Along Submanifolds of the Boundary}

\subsection{Representing Homology Classes by Embedded Submanifolds}\label{sec:representing-homology-classes}

Let $f : N\subset X$ be a smooth map from a compact oriented $p$-dimensional manifold $N$ to $X$. The data of an orientation on a closed manifold gives a fundamental class $[N]\in \H_p(N)$ which can be pushed forward along the map $f : N \hookrightarrow X$ to give a homology class $f_* [N]\in \H_p(X)$. This is the homology class associated to a smooth map $N\to X$.
This correspondence behaves nicely with respect to perturbations, \todo{finish}

Suppose $H : N\times [0,1] \to X$ is a homotopy with $H(x,0)=f(x)$ which perturbs the map $f$. For any $\epsilon>0$, the map $f_\epsilon : N \to X$ given by $f_\epsilon(x)=H(x,\epsilon)$ is homotopic to $f$ and hence induces the same map on homology $\H_p(N)\to \H_p(X)$. The homology class associated to a map $f : N \to X$ thus solely depends on the homotopy type of $f$ so there is a map
\begin{equation}
	\lkxfunc{}{[N,X]}{\H_p(X).}
\end{equation}
Letting $N=S^p$, we can see that this is a generalization of the Hurewicz homomorphism which links homotopy groups to homology groups via a map $\pi_p(X) \to \H_p(X)$.
As with the Hurewicz homomorphism, this correspondence is generally not surjective or injective. If a space is $k$-connected for $k\geq 1$ the Hurewicz homomorphism is an isomorphism $\pi_{k+1}(X) \to \H_{k+1}(X)$. Consequently, every homology cycle in $\H_{k+1}(X)$ can at least be represented by a smooth map of a sphere $S^{k+1}$ into $X$. This smooth map might have ``double-points'', i.e. when multiple points of the sphere map to the same point in the image and prevent the smooth map from being an embedding.

\begin{example}
	In the punctured plane $X=\R^2\setminus \{0\}$ with homology $\H_1(X)\cong \Z$, the only homology cycles which can be represented by embedded submanifolds are $0,\pm 1$, by a circle not containing the origin and circles of both orientations surrounding the origin respectively. A smooth map representing a cycle of higher degree would necessarily have a double-point as in \cref{fig:double-point}.
\end{example}

\begin{figure}[ht]
	\centering
	\import{diagrams}{double-point.pdf_tex}
	\caption{A double-point in a smooth map representing $\pm 2\in \H_1(\R^2\setminus\{0\})$.}\label{fig:double-point}
\end{figure}

That being said, many of the manifolds considered \todo{basis by embedded submanifolds}.
many specially constructed manifolds we will consider in this chapter will at least have a basis by embedded submanifolds, and every homology class will admit a representation by a smooth map. For a classical account of some issues which can arise when representing homology classes by smooth maps, see Chapter II of Ren\'e Thom's seminal paper \cite{thom1954}.

\begin{remark}
	In some cases, homology classes can \emph{always} be represented by embedded submanifolds. For instance, if $M$ is a $4$-manifold, there are isomorphisms
	\begin{equation}
		\H^2(M; \Z) \cong [M, K(\Z,2)] \cong [M, \CP^\infty] \cong [M,\CP^2],
	\end{equation}
	where the first is the representability of singular cohomology by the Eilenberg-Maclane spectrum, the second identifies $\CP^\infty$ as a $K(\Z,2)$ space, and the third uses the cellular approximation theorem to slide maps onto the $4$-skeleton. Any cohomology cycle $\omega\in \H^2(M)$ can be represented by a smooth function $f : M \to \CP^2$. If we choose this function to be transverse to $\CP^1\subset \CP^2$, then $f^{-1}(\CP^1)$ is an embedded $2$-dimensional submanifold of $M$ which corresponds to a Poincar\'e dual class to $\omega$. When $X$ is a compact manifold, Poincar\'e duality tells us that all $2$-dimensional homology cycles arise from $2$-dimensional cohomology cycles and can thus be represented by embedded submanifolds. This is one example of the attractiveness of $4$-manifolds as geometric objects of study.
\end{remark}

\todo{give general conditions for representability}

\pagebreak
\section{The Intersection Form}\label{sec:intersection-form}

One of the fundamental invariants for even-dimensional manifolds is the intersection form, a bilinear form on a lattice which captures the geometric data of submanifold intersections. The lattice in question is the free component of the middle-dimensional singular homology, and the pairing of two homology cycles by the form counts their number of ``intersections''. When the homology cycles of complementary dimension (as in the case of middle-dimensional homology cycles) are represented by smooth immersions, we can perturb them to make them transverse without changing the homology class. If the manifolds are compact, the preimage of their intersection is some finte set of points with orientation -- adding up the orientations of the points gives the orientation number.

This is the geometric interpretation of intersection, and we will explore it in more depth when we construct manifolds with given intersection theory in \cref{sec:plumbing}. For now, we will stick to understanding the algebraic properties of intersections as the adage ``think in terms of intersections, prove in terms of homology'' advises. To start, let us suppose that $M$ is a compact oriented $n$-manifold.
The Poincar\'e-Lefschetz duality gives an isomorphism
\begin{equation}
	\lkxfunc{}{\H^{n-p}(M,\partial M)}{\H_p(M)}{\omega}{\omega\frown [M,\partial M]}
\end{equation}
assuming we have an orientation class $[M,\partial M]\in \H_n(M, \partial M)$ corresponding to the orientation of $M$. Under this duality, the intersection of homology classes is defined as the dual operation to the operation of cup product on cohomology classes. This operation is denoted $\alpha\cdot \beta$ for homology cycles $\alpha\in \H_p(M)$ and $\beta\in \H_q(M)$, and is the top map in \cref{eq:homology-intersection}
\begin{equation}\label{eq:homology-intersection}
	\begin{tikzcd}
		{\H_p(M)\otimes \H_q(M)} & {\H_{n-p-q}(M)} \\
		{\H^{n-p}(M, \partial M)\otimes \H^{n-q}(M,\partial M)} & {\H^{2n-p-q}(M,\partial M)}
		\arrow["\tnsv", from=1-1, to=1-2]
		\arrow[tail reversed, from=1-1, to=2-1]
		\arrow[tail reversed, from=1-2, to=2-2]
		\arrow["\smile", from=2-1, to=2-2]
	\end{tikzcd}
\end{equation}
where the vertical maps are the Poincar\'e-Lefschetz isomorphism. Again, the intuition here should be that $\alpha\cdot \beta$ is the homology class representing the intersection of $\alpha$ and $\beta$ when they are arranged in general ``transverse'' position. Done over homology classes of complementary dimension, the resulting homology intersection class is 0-dimensional and hence pairs with an integer multiple $\ell\cdot [M, \partial M]\in \H_n(M,\partial M)$ of the top-dimensional orientation class. The integer multiple $\ell\in \Z$ is the \defn{intersection number}[intersection number of homology classes] of the cycles $\alpha$ and $\beta$. Removing torsion elements and working in middle dimensional homology so that $\alpha$ and $\beta$ live in the same group, we get an integral bilinear form.

\begin{definition}
	Let $M^{2m}$ be a compact oriented even-dimensional manifold, possibly with boundary. The \defn{intersection form} on middle dimensional homology is the bilinear form
	\begin{equation}
		\lkxfunc{Q_M}{\H_m(M)_{\mathrm{free}}\otimes \H_m(M)_{\mathrm{free}}}{\Z}{\alpha\otimes \beta}{\alpha\tnsv \beta}
	\end{equation}
	where we identify $\H_0(M)\cong \Z$ and $\H_m(M)_{\mathrm{free}}$ denotes the free component of $\H_m(M)$ -- i.e. the quotient by the subgroup of torsion elements.
\end{definition}

\begin{remark}
	If $m$ is even then $Q_X$ is a symmetric bilinear form and if $m$ is odd then $Q_X$ is a skew-symmetric bilinear form. This follows from the graded commutativity of the cup product, and hence the intersection pairing. For brevity, we say that $Q_X$ is \defn{$m$-symmetric} in such cases.
\end{remark}

It's also often helpful to work with the dual intersection pairing on cohomology -- since it is defined explicitly in terms of the cup product, it can be immediately deduced from the multiplicative structure of the cohomology ring.
\begin{definition}
	The intersection form on cohomology is the bilinear form
	\begin{equation}
		\lkxfunc{Q^M}{\H^m(M,\partial M)_{\mathrm{free}}\otimes \H^m(M,\partial M)_{\mathrm{free}}}{\Z}{\alpha\otimes \beta}{\alpha\smile \beta}
	\end{equation}
	where we identify $\H^n(M,\partial M)\cong \Z$.
\end{definition}

\todo{dual lattice perspective using UCT}

\begin{remark}
	For manifolds which do not come with an orientation, the intersection form can be extended in homology/cohomology with $\Z/2$ coefficients. We call this form the \defn{unoriented intersection form}, and denote it by $\widetilde{Q}_M$ or $\widetilde{Q}^M$ depending on if we are working with homology or cohomology. In the context of embedded submanifolds, this form captures the number of transverse intersection points modulo 2, otherwise known as the unoriented intersection number.
\end{remark}

\subsection{Basic Examples of the Intersection Form}

We will see many examples of manifolds and their intersection forms throughout this thesis, so for now let's just consider the most basic examples -- complex projective spaces and tori.

\begin{proposition}\label{prop:intersection-form-complex-projective-plane}
	The intersection form for any complex projective plane $\CP^{2m}$ of even complex dimension is given by $Q=(1)$, and the intersection form for complex projective plane $\CP^{2m+1}$ of odd complex dimension is trivial.
\end{proposition}
\begin{proof}
	Recall that the cohomology ring of complex projective space is given by
	\begin{equation}
		\H^\bullet(\CP^{n}) \approx \C[\alpha]/(\alpha^{n+1})\quad\textrm{with}\quad |\alpha|=2.
	\end{equation}
	A proof of this can be found in any standard algebraic topology book, for instance Theorem~3.19 in \cite{hatcher2002topology}. We can assume without loss of generality that $\alpha^n\in \H^{2n}(\CP^n)$ is the fundamental class corresponding to the canonical orientation on $\CP^n$.
	Note that since the generating element has degree $2$, the middle-dimensional homology $\H^{2m+1}(\CP^{2m+1})$ is trivial and hence so is the intersection form of $\CP^{2m+1}$.

	When the complex dimension is even, the middle-dimension homology $\H^{2m}(\CP^{2m})$ is generated by $\alpha^m$. Since $\alpha^m\smile \alpha^m=\alpha^{2m}$ is a unit multiple of the fundamental class, we have $Q(\alpha^m, \alpha^m)=1$, completing the proof.
\end{proof}

It is illuminating to interpret this result geometrically. Let's begin with the complex vector space $\C^{2m+1}$ equipped with a basis $\{e_0, e_1,\ldots, e_{2m}\}$. Consider the linear subspaces
\begin{equation}
	W = \langle e_0, e_1,\ldots, e_m\rangle \quad\textrm{and}\quad U = \langle e_0, e_{m+1},\ldots, e_{2m}\rangle
\end{equation}
in $\C^{2m+1}$. These complex hyperplanes intersect at a complex line $\langle e_0 \rangle = W\cap U$.

Now, we can pass to the projectivization $\P(\C^{2m+1})=\CP^{2m}$ and realize $W$ and $U$ as embedded submanifolds $\P(W)\approx \CP^m\subset \CP^{2m}$ and $\P(U)\approx \CP^m\subset \CP^{2m}$. Since $W$ and $U$ intersect at a line, their projectivizations $\P(W)$ and $\P(U)$ intersect at a point in $\CP^{2m}$. Furthermore, the intersection is transverse, and descending the orientation on $\C^{2m+1}$ to the embedded submanifolds gives an intersection number of $1$. Both of these embedded submanifolds represent the homology class $a^{2m}\in \H_{2m}(\CP^{2m})$ which is the Poincar\'e dual of the cohomology class $\alpha^{2m}(\CP^{2m})$. We again arrive at $Q(a^{2m}, a^{2m})=1$, although this time through homology intersections.

\begin{figure}[ht]
	\centering
	\caption{\todo{geometric picture of this in complex projective space}}
\end{figure}

\begin{proposition}\label{prop:intersection-form-torus}
	The intersection form for a torus $T^{2m}=S^m\times S^m$ is given by matrices
	\begin{equation}\label{eq:hyperbolic-form-torus}
		Q = \begin{pmatrix}0 & 1 \\ 1 & 0\end{pmatrix}
		\textrm{ when }m\textrm{ is even, and }
		Q = \begin{pmatrix}0 & 1 \\ -1 & 0\end{pmatrix}
		\textrm{ when }m\textrm{ is odd.}
	\end{equation}
\end{proposition}
\begin{proof}
	Let us again begin with a cohomology computation. The cohomology of a sphere is
	\begin{equation}
		\H^\bullet(S^n) = \C[\alpha]/(\alpha^2)\quad\textrm{with}\quad |\alpha| = n,
	\end{equation}
	and by the K\"unneth formula (see Theorem~3.15 in \cite{hatcher2002topology}), we have
	\begin{equation}
		\begin{aligned}
			\H^\bullet(T^{2m})=\H^\bullet(S^m\times S^m)\cong \H^\bullet(S^m)\otimes \H^\bullet(S^m)
			 & \cong \Z[\alpha]/(\alpha^2)\otimes \Z[\beta]/(\beta^2) \\
			 & \cong \Z[\alpha,\beta]/(\alpha^2,\beta^2).
		\end{aligned}
	\end{equation}
	Assuming $\alpha$ and $\beta$ are fundamental classes for the spheres, the fundamental class of the torus is $\alpha\smile \beta$. From this multiplicative structure and fundamental class, we clearly have
	\begin{equation}
		Q(\alpha, \alpha)=0, \quad Q(\beta,\beta)=0, \quad Q(\alpha,\beta)=1,\quad Q(\beta,\alpha)=(-1)^m Q(\alpha,\beta)=(-1)^m.
	\end{equation}
	These give exactly the matrices described in \cref{eq:hyperbolic-form-torus}.
\end{proof}

The geometric proof of this claim is analogous. The Poincar\'e duals of $\alpha$ and $\beta$, denoted $a$ and $b$ in $\H_{m}(T^{2m})$, are represented
by the embedded submanifolds $S^m\times \{p\}$ and $\{p\}\times S^m$ for some basepoint $p\in S^m$. Shifting an individual embedded sphere to a disjoint embedding by a path taking $p\mapsto p'$ disjoint shows that the self-intersection numbers of $a$ and $b$ are zero. These are the zeroes along the diagonal of the matrices in \cref{eq:hyperbolic-form-torus}. However, the embedded spheres representing $a$ and $b$ intersect transversally at the point $(p,p)\in T^{2m}$. We choose orientations on $S^m\times \{p\}$ and $\{p\}\times S^m$ so that $a\cdot b=1$, then by graded-commutativity we get $b\cdot a=(-1)^m$.

\begin{figure}[ht]
	\centering
	\caption{\todo{geometric picture intersections on a torus}}
\end{figure}

\begin{remark}
	The symmetric form in \cref{eq:hyperbolic-form-torus} is known as the \defn{hyperbolic form}, denoted by
	\[
		H=\begin{pmatrix} 0 & 1\\ 1 & 0 \end{pmatrix}.
	\]
	The hyperbolic form is a fundamental building block for symmetric bilinear forms over the integers and $\Z/2$.
\end{remark}

\subsection{Properties of the Intersection Form}

\begin{proposition}\label{prop:connected-sum-intersection-form}
	For compact manifolds $M_1^{2m}$ and $M_2^{2m}$, we have
	$Q_{M_1\+M_2} \cong Q_{M_1}\oplus Q_{M_2}.$
\end{proposition}
\begin{proof}
\end{proof}

An immediate corollary of \cref{prop:connected-sum-operation} and \cref{prop:connected-sum-intersection-form} is
that the intersection form is a homomorphism of commutative monoids, i.e. sets with a commutative associative binary operation and identity elements. On one side, we have the monoid $\mathcal{M}^{2m}$ of oriented compact $2m$-manifolds connected sum, and on the other side we have $\mathcal{Q}(\Z)$ of bilinear forms valued in $\Z$ under the operation of direct sum. Similarly, the unoriented intersection form maps the monoid of unoriented compact $2m$-manifolds $\widetilde{\mathcal{M}}^{2m}$ to $\mathcal{Q}(\Z)$.
\begin{equation}\label{eq:monoid-homomorphism-intersection-form}
	\lkxfunc{}{\mathcal{M}^{2m}}{\mathcal{Q}(\Z),}{X}{Q_X,}
	\quad\textrm{and}\quad
	\lkxfunc{}{\widetilde{\mathcal{M}}^{2m}}{\mathcal{Q}(\Z/2),}{X}{\widetilde{Q}_X.}
\end{equation}
The monoidal structure of the intersection form is quite useful throughout geometric topology, especially in classification problem.

\subsection{Classification of Manifolds by Intersection Form}
An illustrative case in low dimensions is the classification of compact (unoriented) surfaces up to homeomorphism. Recall that every compact surface is homeomorphic to exactly one of the following surfaces
\[
	S^2,\quad T^2\#\cdots\# T^2,\quad\textrm{or}\quad \RP^2\#\cdots\# \RP^2,
\]
i.e. it is either a sphere, a torus with some number of holes, or an unorientable surface formed by gluing together M\"obius strips. For instance, a Klein bottle is the connected sum $\RP^2\#\RP^2$.
A standard cominatorial proof of this classification by polygonal presentations can be found in Chapter 6 of \cite{lee2011topological}.

By similar arguments to \cref{prop:intersection-form-complex-projective-plane} and \cref{prop:intersection-form-torus}, the unoriented intersection forms of these generating surfaces are given by
\[
	\widetilde{Q}_{S^2}=(0),\quad \widetilde{Q}_{T^2}=\begin{pmatrix}0 & 1 \\ 1 & 0\end{pmatrix},\quad \textrm{and}\quad\widetilde{Q}_{\RP^2} = (1).
\]
For instance, the intersection form of $T^2\+ \RP^2$ is given by
\[
	\widetilde{Q}_{T^2\+ \RP^2} = H\oplus (1)=
	\begin{pmatrix}
		1 & 0 & 0 \\
		0 & 0 & 1 \\
		0 & 1 & 0
	\end{pmatrix}.
\]
This matrix represents a bilinear form, and so the transformation $Q\mapsto P^\intercal Q P$ does not affect the form. In this case, $Q^\intercal =Q$ and $Q^2=I\mod 2$, the transformation $Q\mapsto Q^\intercal Q Q$ gives the form
\[
	\widetilde{Q}_{T^2\+ \RP^2}
	\lkxto \begin{pmatrix}
		1 & 0 & 0 \\
		0 & 1 & 0 \\
		0 & 0 & 1
	\end{pmatrix} =\oplus^3(1)= \widetilde{Q}_{\RP^2\+\RP^2\+\RP^2}.
\]
As it turns out, the underlying surfaces $T^2\+\RP^2$ and $\RP^2\+\RP^2\+\RP^2$ are homoemorphic. This has an easy geometric interpretation. The operation $T^2\+$ can be thought of as adding a handle, and $\RP^2\+\RP^2\+$ being connected sum with a Klein bottle can be thought of as adding a handle in a twisted manner, i.e. one spout on one side of the surface and the other spout on the other side.

However, if we add a handle to a projective plane $\RP^2$ (a M\"obius band with boundary collapsed), we can move one spout around the twist of $\RP^2$ to get a twisted handle (as depicted in \cref{fig:twisted-handle-to-handle}). Thus, the surfaces $T^2\+\RP^2$ and $\RP^2\+\RP^2\+\RP^2$ are homeomorphic, a geometric fact which was detected in part by the algebraic identity of forms $H\oplus (1)=\oplus^3(1)$ in $\mathcal{Q}(\Z/2)$.

\begin{figure}
	\centering
	\todo{figure}
	\medskip
	\caption{Turning $T^2\+\RP^2$ into $\RP^2\+\RP^2\+\RP^2$.}\label{fig:twisted-handle-to-handle}
\end{figure}

\begin{proposition}
	There is a presentation
	\[\mathcal{Q}_{\mathrm{skew}}(\Z/2) = \langle H, (1) \mid H\oplus (1) = \oplus^3 (1)\rangle\]
	where $\mathcal{Q}_{\mathrm{skew}}(\Z/2)$ is the monoid of skew-symmetric bilinear forms over $\Z/2$.
\end{proposition}
\begin{proof}
	See Chapter III of \cite{milnorhuse1973forms} for a generalized statement and proof.
\end{proof}

Just like $H\oplus (1)= \oplus^3 (1)$ is the defining relation for skew-symmetric bilinear forms over $\Z/2$, so too is $T^2\+ \RP^2 = \RP^2\+\RP^2\+\RP^2$ the defining relation for closed surfaces. This leads to an elegant restatement of the classification theorem for closed surfaces.

\begin{theorem}[Classification of Compact Surfaces]
	Let $\mathcal{S}^2\subset \widetilde{\mathcal{M}}^2$ be the monoid of \textit{closed} unoriented surfaces under connected sum. The unoriented intersection form is an isomorphism of monoids
	\[
		\lkxfunc{\widetilde{Q}}{\mathcal{S}^2}{\mathcal{Q}_{\mathrm{skew}}(\Z/2).}
	\]
\end{theorem}

The classification of compact surfaces by the intersection form is a model result of algebraic topology -- a complete algebraic classification of a class of manifolds. Better yet, simple algebraic manipulations correspond to non-trivial topological equivalences. This is part of why intersection forms are so useful -- algebraic intuition scales far better with dimension than does geometric intuition and so bilinear forms are a much more comfortable setting in which to study higher-dimensional topology. For instance, the classification theorem of Michael Freedman in his 1982 paper \cite{freedman1982manifold} is formulated entirely in terms of the intersection form and an additional $\Z/2$-valued invariant detecting the existence of a smooth structure.

\begin{theorem}[Freedman, 1982] Let $\mathcal{S}^4\subset \mathcal{M}^4$ be the monoid of simply-connected closed topological $4$-manifolds. The intersection form
	\[
		\lkxfunc{Q}{\mathcal{S}^4}{\mathcal{Q}_{\mathrm{sym}}(\Z)}
	\]
	is at most two-to-one, i.e. a symmetric intersection form corresponds to at most two topological $4$-manifolds, one which admits a smooth structure and one which does not.
\end{theorem}
An accessible exposition to this remarkable theorem can be found in \cite{behrens2021discembedding}. We will explore this theorem in a bit more depth in \cref{sec:smoothing-obstructions}.

\subsection{Intersection Form Invariants}\label{sec:intersection-form-invarians}

While the complete algebraic data of a


\medskip
\todo{geometric interpretation of lattices,
	\href{https://math.stackexchange.com/questions/2944104/obtaining-two-holed-torus-as-a-quotient-of-bbb-c}{maybe this article?}}

\pagebreak
\section{Cobordism}\label{sec:cobordism}

The basic principle of cobordism is to declare two manifolds equivalent if there is a manifold a dimension higher which connects the two manifolds. As an equivalence relation, cobordism is far looser than the notions of homoemorphism or diffeomorphism and so allows for a full classification of manifolds. Many notions in geometry and topology -- for instance characteristic classes -- depend solely on the cobordism type of a manifold, so understanding the structure of cobordism is immensely helpful.

\begin{remark}
	Note that the implied compactness assumption throughout the thesis is important here, otherwise any manifold $M$ is trivially the boundary of $M\times [0,\infty)$.
\end{remark}

\begin{definition}
	An \defn{unoriented cobordism} between closed $n$-manifolds $M_1$ and $M_2$ is an $(n+1)$-manifold $W$ with $\partial W = M_1\sqcup M_2$. We use the notation $W : M_1\bord M_2$ to refer to the cobordism.
\end{definition}

\begin{definition}
	An \defn{oriented cobordism} between closed oriented $n$-manifolds $M_1$ and $M_2$ is an oriented $(n+1)$-manifold $W$ with $\partial W = M_1\sqcup (-M_2)$. We use the notation $W : M_1\sobord M_2$ to refer to the cobordism.
\end{definition}

\begin{remark}
	\todo{General structure on a cobordism}
\end{remark}

Note that the oriented cobordism group can be thought of as a $\Z$-module, with multiplication action on a closed manifold $M$ given by
\[
	n \cdot M = \begin{cases} M\sqcup \cdots \sqcup M & n > 0,\\ (-M)\sqcup \cdots \sqcup (-M) & n < 0,\\ \emptyset & n=0,\end{cases}
\]
for all $n\in \Z$, where $\sqcup$ is repeated $|n|$ times. Since there is no notion of negation in the unoriented case, the unoriented cobordism group is a $\Z/2$-module.

\begin{figure}[ht]
	\centering
	\import{diagrams}{pair-of-pants.pdf_tex}
	\caption{A cobordism $W$ between $S^1$ and $(S^1\sqcup S^1)$.}\label{fig:pair-of-pants}
\end{figure}

For a simple example of a cobordism between a circle and a disjoint union of circles, see \cref{fig:pair-of-pants}. Note that this cobordism could be made much simpler by removing the handle. Simplifying cobordisms in this way is one of the major applications of surgery theory.

\begin{definition}
	The \defn{$k$-th oriented cobordism group}[oriented cobordism group] $\Omega^\SO_k$ is the abelian group of oriented cobordism classes of closed $k$-manifolds\footnote{We do not require manifolds to be connected in this definition.}
	under disjoint union. The identity component is the empty set $\varnothing$, and negation is given by reversing orientation. Similarly, the \defn{$k$-th unoriented cobordism group}[unoriented cobordism group] $\Omega_k$ is the abelian group of cobordism classes of closed $k$-manifolds under disjoint union.
\end{definition}

\begin{example}
	There is an isomorphism $\Omega_0\cong \Z/2$. An unoriented 0-dimensional manifold is just a set of points. Any pair of points is cobordant to the empty set by a path connecting them. Since adding pairs of points doesn't change the cobordism type, the number of points modulo 2 determines the cobordism class entirely.
\end{example}

\begin{example}
	There is an isomorphism $\Omega_0^\SO \cong \Z$. An oriented 0-dimensional manifold is still a set of points, however the orientation now equips each point with a ``charge'', we might label them as $+$ or $-$. Note that points of opposite ``charges'' cancel out by a path between them oriented from $-$ to $+$. Given some set of points of various charges, we can always eliminate pairs of opposite charges and are left with a homogeneous set of charge. Adding up all of the pluses or minuses, we get an integer. This integer determines the cobordism class, and is invariant to adding or removing pairs of opposing charge.
\end{example}

\begin{example}
	Both the oriented and unoriented cobordism groups are trivial in dimension 1, since every circle is the boundary of a disk.
\end{example}

In higher dimensions, the classification becomes much more interesting. 

\begin{figure}[ht]
	\renewcommand{\arraystretch}{1.2}
	\centering
	\begin{tabular}{r||c|c||c|c}
		$k$ & $\Omega_k$          & generators                                          & $\Omega_k^\SO$ & generators                 \\
		\hline
		$0$ & $\Z/2$              & a point                                             & $\Z$           & a point                    \\
		$1$ & $0$                 &                                                     & $0$            &                            \\
		$2$ & $\Z/2$              & $\RP^2$                                             & $0$            &                            \\
		$3$ & $0$                 &                                                     & $0$            &                            \\
		$4$ & $\Z/2\oplus \Z/2$   & $\RP^4$, $\RP^2\times \RP^2$                        & $\Z$           & $\CP^2$                    \\
		$5$ & $\Z/2$              & $\SU_3/\SO_3$                                       & $\Z/2$         & $\SU_3/\SO_3$              \\
		$6$ & $(\Z/2)^{\oplus 3}$ & $\RP^6$, $\RP^2\times \RP^4$, $(\RP^2)^{\times 3}$, & $0$            &                            \\
		$7$ & $\Z/2$              & $(\SU_3/\SO_3) \times \RP^2$                        & $0$            &                            \\
		$8$ & $(\Z/2)^{\oplus 4}$ & $\RP^8, \RP^6\times \RP^2, \cdots$                  & $\Z\oplus \Z$  & $\CP^4, \CP^2\times \CP^2$ \\
	\end{tabular}
	\medskip
	\caption{Structure of unoriented and oriented cobordism groups.}\label{fig:cobordism-structure-table}
\end{figure}

The structure of \cref{fig:cobordism-structure-table} makes a lot more sense in the context of 

\begin{proposition}
	The product of manifolds is a well-defined operation with respect to cobordism.
\end{proposition}
\begin{proof}
	\todo{proof}
\end{proof}

\begin{definition}
	The \defn{oriented cobordism ring} $\Omega^\SO_\bullet$ is the set of oriented cobordism classes of closed manifolds under the operations of disjoint union and product.
\end{definition}

The oriented cobordism ring has a grading by
\[
	\Omega_\bullet^\SO = \bigoplus_{k\geq 0} \Omega^\SO_k.
\]

\begin{theorem}[Thom-Pontryagin]\label{thm:oriented-cobordism-structure}
	There is a ring isomorphism
	\[
		\Omega_\bullet^\SO \otimes \Q \lkxto \Q[x_4, x_8, x_{12}, \ldots]
	\]
	where $x_{4k}$ are cobordism classes representing $\CP^{2k}$.
\end{theorem}

\subsection{The Thom-Pontryagin Construction}

\begin{theorem}[Pontryagin-Thom Isomorphism]
	Given a closed manifold $M$ of dimension $n>k$, there is a bijective correspondence
	\begin{equation}
		\Omega^\fr_k(M) \lkxto[] [M, S^{n-k}].
	\end{equation}
\end{theorem}

\begin{corollary}
	Since every closed framed manifold can be framed embedded in a sphere of sufficiently large dimension, there is an isomorphism
	\begin{equation}
		\Omega^\fr_k \lkxto \pi_k S
	\end{equation}
	where $\pi_k S = \lim_{n\to\infty} \pi_n S^{n-k}$ is the $k$-stable homotopy group of the spheres.
\end{corollary}

\begin{theorem}
	There exists a space $\MSO$ such that there is a
\end{theorem}

\pagebreak
\section{Characteristic Classes}
The study of characteristic classes began with the work of Hassler Whitney and Eduard Stiefel in the mid 1930s. Since then, the fundamental idea has remained unchanged -- a vector bundle on a manifold determines certain ``characteristic'' classes in the homology or cohomology of the base manifold.
By the mid 1940s, these ideas were extended by Lev Pontryagin and Shing-Shen Chern to better capture the geometric data of oriented real and complex vector bundles respectively. In the following decades, characteristic classes quickly joined the toolboxes of mathematicians from a wide range of disciplines, finding connections to prior notions in these fields. 
Applications ranged from algebraic topology, differential topology of exotic spheres, complex geometry, index theory, and may others. 

We will present a few equivalent formulations of characteristic classes in this thesis, each useful in its own context. These formulations are, 
\begin{itemize}
	\item as natural transformations satisfying certain axioms in \cref{sec:axiomatic-characteristic-classes},
	\item as generators of the cohomology ring of a classifying space in \cref{sec:universal-characteristic-classes},
	\item by the Chern-Weil homomorphism as images of invariant polynomials in \cref{sec:chern-weil-theory},
	\item as obstructions to problems in homotopy theory in \cref{sec:obstruction-theory}.
\end{itemize}
At this stage, we assume a basic knowledge of vector bundles, structure groups, and classifying spaces. For a brief introduction to these topics, see \cref{chap:vector-bundles}.

\subsection{Axiomatic Perspective on Characteristic Classes}\label{sec:axiomatic-characteristic-classes}

\begin{definition}\label{defn:characteristic-class}
A \defn{characteristic class} $c$, valued in a cohmology theory $h$, is a natural transformation of contravariant functors
\[
	\lkxfunc{c}{\Vect_G}{h^\bullet,}
\]
given a structure group $G$. 
\end{definition}

Here, $h^\bullet : \Top \to \Rng$ sends a space to its  cohomology ring, and $\Vect_G : \Top \to \Set$ sends a space to the set of isomorphism classes of vector bundles over the space with structure group $G$.

Given a manifold $M$, a characteristic class assigns a vector bundle $\xi$ over $M$ to a cohomology class $c(\xi)\in h^\bullet(M)$.
This assignment is done in a natural way, i.e. given bundles $\xi_1$ and $\xi_2$ over manifolds $M_1$ and $M_2$, whenever a map $f : M_1 \to M_2$ is covered by a bundle map $\xi_1 \to \xi_2$, we have $f^* c(\xi_2) = c(\xi_1)$. 

\begin{convention*}
	Since every smooth manifold $M$ comes with a canonical vector bundle -- the tangent bundle -- it's common to use the notation $c(M)$ to refer to $c(\T M)$.
\end{convention*}

\begin{remark} 
	The cohomology ring $h^\bullet(M)$ has a $\Z$-grading
\[
		h^\bullet(M) = \bigoplus_{k\in \Z} h^k(M).
\]
Often times, characteristic classes are defined as a sequence of homogeneous classes $c_i\in h^{i}(\xi)$. It then makes sense to consider the \defn{total characteristic class} $c(\xi)=\sum_i c_i(\xi)\in h^\bullet(M)$. Many formulas are simpler when working with total characteristic classes rather than their homogeneous components. In this thesis, all arbitrary characteristic classes are assumed to be inhomogeneous in $h^\bullet(M)$.
\end{remark}

When bundles over the same base space are isomorphic, the naturality of a characteristic class implies that the corresponding characteristic classes are the same. This is useful for classifying vector bundles over a space but doesn't help when we need to compare vector bundles over distinct spaces.
When a cohomology theory admits a Poincar\'e duality isomorphism $h^{n-k}(M) \cong h_k(M)$ for some class of closed $n$-dimensional manifolds, we can ``integrate'' homogeneous top-dimensional cohomology classes $\alpha\in h^{n}(M)$ along a fundamental class $[M]\in h_n(M)$ to get an element $\alpha[M]\in h_0(M)\cong h_0(*)$ in the coefficient ring of a corresponding homology theory. The coefficient ring $h_0(*)$ is a common context in which to compare characteristic classes.

\begin{definition}\label{defn:characteristic-numbers}
	Given a homogeneous characteristic class $c$ of degree $n$ and a closed $n$-dimensional manifold $M$, the \defn{characteristic number} of a vector bundle $\xi$ over $M$ is the $c(\xi)[M] \in h_0(*)$.
\end{definition}

\begin{convention*}
	When referring to a characteristic number of a tangent bundle of a closed manifold, we use the notation $c[M]$.
\end{convention*}

\begin{remark}\label{rmk:characteristic-number-monomial-polynomial}
	Whenever we have some family $\{c_i\}$ of homogeneous characteristic classes and an $n$-dimensional manifold $M$, it is common to associate characteristic numbers of the tangent bundle to partitions $k_0|c_0|+k_1|c_1|+\cdots+k_\ell|c_\ell| = n$. We then get a characteristic class $c_0^{k_0}\cdots c_\ell^{k_\ell}$, homogeneous of degree $n$. For each closed $n$-dimensional manifold $M$, the monomial $c_0^{k_0}\cdots c_\ell^{k_\ell}$ thus has an associated characteristic number $c_0^{k_0}\cdots c_\ell^{k_\ell}[M]$. 

	More generally, given a polynomial in $\ell$ variables $K\in R[x_1,\ldots, x_\ell]$ with $K(z^{|c_0|}, \ldots, z^{|c_\ell|})$ homogeneous of degree $n$, the class $K(c_0,\ldots, c_\ell)$ is a homogeneous characteristic class of dimension $n$ and so has
	an associated characteristic number $K(c_0, \ldots, c_\ell)[M] \in R$.
\end{remark}

\subsection{Stiefel-Whitney Classes}

Stiefel-Whitney classes are some of the simplest characteristic classes, defined for vector bundles with structure group $\GL_n\R$ and taking values in singular cohomology with $\Z/2$ coefficients. Characterisic numbers of Stiefel-Whitney classes thus take values in $\Z/2$ -- these are called \defn{Stiefel-Whitney numbers} and are an important topological invariant when computed from the tangent bundle of a manifold.

\begin{definition}
	The \defn{(total) Stiefel-Whitney class}[total Stiefel-Whitney class] $w$ is the unique characteristic class for unoriented vector bundles in $\Z/2$ singular cohomology satisfying the following axioms:
	\begin{enumerate}[(a)]
		\item The degree $0$ component of $w$ is always $1\in H^0(-; \Z/2)$.
		\item The total degree of $w$ is never greater than the dimension of the base space.
		\item For bundles $\xi_1$ and $\xi_2$ over a common base, we have $w(\xi_1\oplus \xi_2)=w(\xi_1)\smile w(\xi_2)$. 
		\item If $\gamma$ is the canonical line bundle over the circle $\RP^1$, we have $w(\gamma)=1+\alpha$ in the ring $\H^\bullet(\RP^1; \Z/2)\cong \Z/2[\alpha]/(\alpha^2)$.
	\end{enumerate}
	Here, we set $w_i$ to be the homogeneous component of $w$ in $\H^i(-;\Z/2)$.
\end{definition}

\begin{remark}
	Axiom (c) is known as the \defn{Whitney product formula}. The classical form of the Whitney product formula in homogeneous components is given by
	\[
		w_i(\xi_1\oplus \xi_2) = \sum_{p+q=i} w_p(\xi_1)\smile w_q(\xi_2).
	\]
\end{remark}

Of course, this definition is not constructive, and it is not at all clear that a characteristic class satisfying these definitions even exists. That being said, it is instructive to explore the immediate consequences of these axioms before considering explicit constructions.

\begin{corollary}
	For any trivial bundle $\underline{\R}^k$, we have $w(\underline{\R}^k)=1$.
\end{corollary}
\begin{proof}
	A trivial bundle $\underline{\R}^k$ is the pullback of the constant bundle $\R^k\to *$ over a point. This bundle has trivial Stiefel-Whitney class by (a) and (b) since the point is zero-dimensional.
\end{proof}

\begin{corollary}
	By the Whitney product formula, $c(\xi\oplus \underline{\R}^k)=c(\xi)\smile c(\underline{\R}^k)= c(\xi)$.
\end{corollary}

In other words, adding on trivial bundles to a vector bundle does not affect the Stiefel-Whitney class. This equivalence relation on vector bundles is known as stable isomorphism, namely two vector bundles $\xi_1$ and $\xi_2$ are said to be \defn{stably isomorphic}[stable isomorphism of vector bundles] if there is an isomorphism of vector bundles $\xi_1\oplus \underline{\R}^{k_1} \cong \xi_2\oplus \underline{\R}^{k_2}$. \todo{stable structure group}. 

\begin{definition}
	A characteristic class $c$ is said to be \defn{stable}[stable characteristic class] if $c(\xi\oplus \underline{\R}^k) = c(\xi)$ for any $k$.
\end{definition}

The sensitivity of Stiefel-Whitney classes to stable isomorphism types has a nice geometric corollary. If a manifold $M$ can be embedded \todo{finish}
This notion is known as \defn{stable parallelizability},\todo{finish} 
we will prove in \cref{thm:homotopy-sphere-stably-parallelizable} that any manifold homeomorphic sphere in dimension $n\geq 5$ has a tangent bundle which is stably isomorphic to the trivial bundle, and hence all characteristic classes of exotic spheres are trivial. This triviality is one reason for the subtleties in detecting exotic spheres, since they cannot be distinguished by stable characteristic classes. 

\subsection{Wu Classes}

A refinement 

\subsection{Chern Classes}

\subsection{Pontryagin Classes}\label{sec:pontryagin-classes}
\begin{theorem}[Whitney Duality]

\end{theorem}


Note that since 

\subsection{A Universal Perspective on Characteristic Classes}\label{sec:universal-characteristic-classes} 

\begin{definition}
	The \defn{infinite-dimensional Grassmannian}, denoted $\Gr_k$ or $\Gr_k(\R^\infty)$ is defined as the direct limit
	\[
		\Gr_k(\R^\infty) = \varinjlim_n \Gr_k(\R^n)
	\]
\end{definition}

\begin{theorem}
	\[
		\H^\bullet(\Gr_k; \Z/2) \cong \Z/2[w_1(\gamma), \ldots, w_n(\gamma)]
	\]
\end{theorem}

\subsection{Chern-Weil Theory}\label{sec:chern-weil-theory}

While the axiomatic and universal definitions of characteristic classes are simple and abstract, it is often useful to
