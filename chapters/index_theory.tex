
\section{The Hirzebruch Signature Theorem}

\begin{theorem}[Hirzebruch Signature Theorem]\label{thm:hirzebruch_signature}
  \todo{todo}
\end{theorem}

\begin{proof}
\end{proof}

\subsection*{Harmonic Forms}

Given a function $\psi : U \to \R$ on an open region of $\R^n$, the Laplace equation\index{Laplace equation} is the second-order partial differential equation
\begin{equation}\label{eq:laplace}
    \Delta \psi = 0
    \quad\implies\quad
    \frac{\partial^2 \psi}{\partial x_1^2}+\cdots+\frac{\partial^2 \psi}{\partial x_n^2}=0.
\end{equation}
Functions satisfying this equation are known as \defn{harmonic functions}[harmonic function] and are ubiquitous across mathematics and physics. The Laplacian operator $\Delta$

\section{The Atiyah-Singer Index Theorem}

\begin{theorem}[Atiyah-Singer Index Theorem]\label{thm:atiyah-singer_index}
  Let $X$ be a closed oriented $n$-manifold and let $(E,D)$ be an elliptic complex on $X$. Then we have
  \[
    \ind(E, D) = (-1)^{n(n+1)/2}\int_X \mathrm{ch}(E,D)\smile \Td(\T X_\C).
  \]
\end{theorem}
