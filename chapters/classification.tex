\chapter{Classification}\label{chap:classification}

\begin{epigraph}{9em}{David Hilbert}
	Wir w\"ussen wissen.\\
	Wir werden wissen.
\end{epigraph}

In this penultimate chapter, we will prove general theorems about the group $\Theta^n$ of smooth structures on the sphere, in some respects, a full classification depending on the solutions of problems in homotopy theory. As we have discussed in \cref{sec:groups-of-homotopy-spheres}, there are generally two classes of exotic spheres -- those that bound parallelizable manifolds, i.e. $\bP^{n+1}\subset \Theta^n$, and those that do not, i.e. $\Theta^n/\bP^{n+1}$.
Exotic spheres represented by the latter set are known as \defn{very exotic spheres}, and are far harder to construct or visualize since they can only really be detected by homotopy theoretic means.
We have seen how to construct exotic spheres in $\bP^{n+1}$ in \cref{chap:constructions}, and have even computed divisibility lower bounds on the size of $\bP^{4k}$ using geometric invariants in \cref{chap:invariants}. We will now
\begin{itemize}
	\item prove that $\bP^{4k}$ is a cyclic group and derive a formula for its order (\cref{thm:kervaire-milnor}), generalizing the results of \cref{chap:invariants},
	\item prove that the groups $\bP^{4k+2}$ are either trivial or $\Z/2$ depending on the Kervaire invariant problem (\cref{thm:kervaire-invariant-problem}),
	\item prove that the groups $\bP^{2k+1}$ are trivial (\cref{cor:odd-dimensional-bP-trivial}),
	\item prove that $\Theta^n$ are finite groups,
	\item relate the quotient $\Theta^n/\bP^{n+1}$ of very exotic spheres to the $J$-homomorphism and the stable homotopy groups of spheres.
\end{itemize}
Altogether, these results

Many of these results fit into an 

\pagebreak
\section{Framed Surgery Theory}

In \cref{sec:plumbing} and \cref{sec:brieskorn}, we constructed highly-connected manifolds

\begin{theorem}\label{thm:framed-surgery-highly-connected}
  Every compact framed manifold of dimension $n\geq 4$ with boundary $\partial M$ a homology sphere can be made highly-connected by a finite sequence of framed surgeries.
\end{theorem}

This is a bulky theorem, and so we will tackle it in parts. First, let us understand the effect of 

\begin{proof}
\end{proof}

\begin{corollary}\label{cor:odd-dimensional-bP-trivial}
	The groups $|\bP^{2k+1}|$ are trivial.
\end{corollary}

\subsection{The Surgery Invariant}\label{sec:surgery-invariant}

\begin{theorem}[Milnor-Kervaire]\label{thm:kervaire-milnor}
	For every $k>1$, $bP^{4k}$ is a cyclic group of order
	\[
	  |\bP^{4k}| =2^{2k-2}(2^{2k-1}-1)\epsilon(k)\denom(4k/B_{2k}).
	\]
\end{theorem}

\begin{theorem}[Kervaire Invariant Problem]\label{thm:kervaire-invariant-problem}
\end{theorem}

\begin{theorem}\label{thm:homotopy-sphere-stably-parallelizable}
	Every homotopy sphere is stably parallelizable.
\end{theorem}
\begin{proof}
	Let $\Sigma$ be a homotopy $n$-sphere.

	The only obstruction to the triviality of $T^sM$ is a well-defined cohomology class:
	\[
		\mathfrak{o}_n(\Sigma) \in \H^n(\Sigma; \pi_{n-1}(\SO_{n+1})) = \pi_{n-1}(\SO_{n+1})
	\]
	The coefficient group may be identified with the stable group $\pi_{n-1}(\SO)$, but these stable groups have been computed by Bott in \cite{bott1959stable}, for $n\geq 2$ we have:
	\begin{center}
		\begin{tabular}{c|cccccccc}
			\textrm{$n\mod 8$} & 0 & 1 & 2 & 3 & 4 & 5 & 6 & 7\\
			\hline
			$\pi_{n-1}(\SO)$ & $\Z$ & $\Z/2$ & $\Z/2$ & 0 & $\Z$ & 0 & 0 & 0.
		\end{tabular}
	\end{center}
	If $\pi_{n-1}(\SO)$ is zero, we are done. 

	If $\pi_{n-1}(\SO) = \Z$, then $n=4k$. According to \cite{kervairemilnor1960} and \cite{kervaire1959}, some non-zero multiple of the obstruction class $\mathfrak{o}_n(\Sigma)$ can be identified with the Pontryagin class $p_k(T^s M) = p_k(TM)$. \todo{(why?)} But the Hirzebruch signature theorem implies \todo{(why?)} that $p_k(\Sigma)$ is a multiple of the signature $\sigma(\Sigma)$ which is zero since $\H^{2k}(\Sigma)=0$. Thus every homotopy $4k$-sphere is \textsc{s}-parallelizable. 

	Finally, suppose $\pi_{n-1}(\SO)= \Z_2$. It follows from an argument of Rohlin \todo{(what?)} that $J_{n-1}(\mathfrak{o}_n(\Sigma))=0$ where $J_{n-1}$ denotes the Hopf-Whitehead homomorphism
	\[
		\lkxfunc{J_{n-1}}{\pi_{n-1}(\SO_k)}{\pi_{n+k-1}(S^k)}
	\]
	in the stable range $k >n$. But $J_{n-1}$ is injective for $n\equiv 1, 2\mod 8$. This is proven by Adams. \todo{(find)} This means that $\mathfrak{o}_n(\Sigma)=0$.
\end{proof}

\begin{theorem}
	The quotient group $\Theta^n / \bP^{n+1}$ is finite.
\end{theorem}
\begin{proof}
	Let $M$ be an \textsc{s}-parallelizable $n$-manifold. 

	Imbed it as $i : M \to S^{n+k}$ for some $k>n+1$ so that it's normal bundle is trivial.

	For each normal $k$-frame $\varphi$, we get an element of $\pi_{n+k}(S^k) = \pi^s(S^n)$ by the Pontryagin-Thom construction. Let's call the set of these elements (as $\varphi$ is allowed to vary) $p(\Sigma)$. \todo{(elaborate)}

	\todo{add lemmas}

	\begin{lemma}
		There is a homomorphism:
		\[
			\lkxfunc{p'}{\Theta_n}{\pi^s(S^n)/p(S^n)}
		\]
	\end{lemma}

	Furthermore, the kernel of $p'$ contains $h$-cobordism classes of homotopy $n$-spheres which bound parallelizable manifolds \todo{(provide lemma)}, which is exactly $bP_{n+1}$. By the first isomorphism theorem, it follows that $\Theta_n/bP_{n+1}$ is isomorphic to a subgroup of $\pi^s(S^n)$ which is finite.
\end{proof}

\section{Very Exotic Spheres}

\subsection{Obstruction Theory}\label{sec:obstruction-theory}

% \section{Kervaire Invariant}
