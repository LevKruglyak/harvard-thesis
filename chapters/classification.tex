\chapter{Classification}\label{chap:classification}

\begin{epigraph}{8.6em}{David Hilbert}
	Wir m\"ussen wissen.\\
	Wir werden wissen.
\end{epigraph}

In this penultimate chapter, we will prove general theorems about the group $\Theta^n$ of smooth structures on the sphere, in some respects, a full classification depending on the solutions of problems in homotopy theory. As we have discussed in \cref{sec:groups-of-homotopy-spheres}, there are generally two classes of exotic spheres -- those that bound parallelizable manifolds, i.e. $\bP^{n+1}\subset \Theta^n$, and those that do not, i.e. $\Theta^n/\bP^{n+1}$.
Exotic spheres represented by the latter set are known as \defn{very exotic spheres}, and are far harder to construct or visualize since they can only really be detected by homotopy theoretic means.
We have seen how to construct exotic spheres in $\bP^{n+1}$ in \cref{chap:constructions}, and have even computed divisibility lower bounds on the size of $\bP^{4k}$ using geometric invariants in \cref{chap:invariants}. We will now
\begin{itemize}
	\item prove that $\bP^{4k}$ is a cyclic group and derive a formula for its order (\cref{thm:kervaire-milnor}), making exact the lower bounds of \cref{chap:invariants},
	\item prove that the groups $\bP^{4k+2}$ are either trivial or $\Z/2$ depending on the Kervaire invariant problem (\cref{thm:kervaire-invariant-problem}),
	\item prove that the groups $\bP^{2k+1}$ are trivial (\cref{cor:odd-dimensional-bP-trivial}),
	\item prove that $\Theta^n/\bP^{n+1}$ are finite groups (\cref{thm:finite-very-exotic-spheres}) and consequently $\Theta^n$ are finite,
	\item relate the quotient $\Theta^n/\bP^{n+1}$ of very exotic spheres to the $J$-homomorphism and the stable homotopy groups of spheres.
\end{itemize}
Altogether, these results

Many of these results fit into an

\pagebreak
\section{Homotopy Theory and Geometry}

One of the most elegant correspondences in mathematics is that of homotopy theory with geometry and topology. Due to the wide array of computational tools available in homotopy theory, any time a geometric problem can be converted to a statement in homotopy theory, it is automatically reduced to computation. Admittedly, these computations are often extremely difficult, but they are doable, which is an improvement from the often hopeless landscape of high-dimensional topology done by hand.

A familiar example of this the problem of classifying principal bundles over a space. Given the data of the geometric structure of bundle, i.e. the structure group, we can construct a universal bundle over a classifying space for that group. Every principal bundle is then a pullback of this universal bundle, unique modulo homotopy of the function along which to pull back.
In this section we apply a similar principle to manifolds up a notion of cobordism, representing a manifolds by homotopy classes of maps to a classifying space.

\subsection{The Pontryagin-Thom Construction}

Let us begin with the classical theory for framed cobordism, originally due Pontryagin in 1950. We defer the reader to Pontryagin's short book on the topc \cite{pontryagin1959homotopy} for the technical details of various constructions in this section.

The basic idea is simple and geometric. Suppose we have closed ambient manifold $M^n$ with submanifold $N^k$ possessing a tubular neighborhood $T_N\supset N$. Recall that a framing on $N$ is a diffeomorphism $\varphi : T_N \to N\times \R^{n-k}$. If we project onto the second factor, we get a map $\varphi'=\pi_2\circ \varphi : T_N \to \R^{n-k}$. We can then include $\R^{n-k}$ into its one-point compactification $\R^{n-k}\cup \{\infty\}=S^{n-k}$ by stereographic projection, and extend the map $\varphi'$ to $S^{n-k}$ by sending $N\setminus T_N$ to $\infty$. We can associate this map $f : M \to S^{n-k}$ to the framed embedding $N\subset M$. For a geometric picture of this construction, see \cref{fig:pontryagin-thom-forward}.
\[\begin{tikzcd}
		N & {T_N} & M \\
		{\{0\}} & {\R^{n-k}} & {S^{n-k}}
		\arrow[hook, from=1-1, to=1-2]
		\arrow[from=1-1, to=2-1]
		\arrow[hook, from=1-2, to=1-3]
		\arrow["{\varphi'}", from=1-2, to=2-2]
		\arrow["f", from=1-3, to=2-3]
		\arrow[hook, from=2-1, to=2-2]
		\arrow[hook, from=2-2, to=2-3]
	\end{tikzcd}\]

This construction is not too surprising, $N$ is sent to the point $0\in S^{n-k}$, and $M\setminus T_N$ is sent to the $\infty \in S^{n-k}$ (although any distinct pair of points would suffice). The tubular neighborhood fills in the gaps, encoding the ``twistedness'' of the framing of $N\subset M$ in the homotopy type of the map $f$.

\begin{figure}[ht]
	\import{diagrams}{placeholder-small.pdf_tex}
	\caption{Turning a framed submanifold of $M$ into a map $M\to S^{n-k}$.}\label{fig:pontryagin-thom-forward}
\end{figure}

\begin{proposition}
	The map $f$ is smooth, and $0\in S^{n-k}$ is a regular value.
\end{proposition}

In the reverse direction, suppose we had a smooth map $f : M \to S^{n-k}$ with a regular value at some point $p\in S^{n-k}$. By the preimage theorem, the preimage $N=f^{-1}(p)$ is a $k$-dimensional submanifold of $M$. Furthermore, since the tangent bundle of $S^{n-k}$ restricted to $p$ is trivial, we can pull it back to get a trivial normal bundle on $N\subset M$. We thus have a correspondence
\[
	\left\{\parbox{9.5em}{framed $k$-dimensional submanifolds of $M$}\right\}
	\quad\lkxleftrightto\quad
	\left\{
		\parbox{10em}{maps $f : M \to S^{n-k}$ with $0$ a regular value}
	\right\}.
\]
There are natural equivalences on both sides, although it is not at all obvious that the correspondence will be well-defined with respect to either.
On the right side of the correspondence, we introduce the relation of homotopy. On the left side, we introduce a notion of framed cobordism restricted to lie within $M$. More precisely:

\begin{definition}
	If $N_1$ and $N_2$ are embedded submanifolds of $M$, a \defn{framed cobordism in $M$}[framed cobordism] $W : N_1 \frbord_{M} N_2$ of $N_1$ and $N_2$ is a framed submanifold\,\footnote{Recall the assumption held throughout the thesis that submanifolds preserve boundaries.} $W\subset M\times [0,1]$ such that $\partial W$ is $N_1\sqcup N_2$
\end{definition}

\begin{example}
	A good intuitive example of framed cobordism is in the simplest case of a circle $M=S^1$.

\begin{figure}[ht]
	\import{diagrams}{placeholder-small.pdf_tex}
	\caption{Framed cobordisms in $S^1$.}
\end{figure}
\end{example}


\begin{definition}
	The group of framed cobordism classes of framed $k$-dimensional submanifolds of $M$ is denoted $\Omega^\fr_k(M)$.
\end{definition}

A good example and source of intuition for this construction comes from framed knots. In knot theory, a \defn{framed knot} is an embedding $\iota$ of $S^1$ into $S^3$ along with a section $s$ of its normal bundle $\T S^3/S^1$. A framed knot should be thought of as an embedded ``ribbon'' in three dimensional space, the underlying knot determining the knottedness of the ribbon and the framing section determining the twistedness of the ribbon. For some examples of framed knots, see \cref{fig:framed-knot-examples}.

\begin{figure}[ht]
	\centering
	\import{diagrams}{placeholder.pdf_tex}
	\caption{A framed unknot and trefoil knot.}\label{fig:framed-knot-examples}
\end{figure}

\begin{remark}
	With a Riemannian structure on $S^3$, we can assume without loss of generality that $s$ has unit norm, and get an orthonormal section $s^\perp$ to $s$. This recovers our notion of a framed submanifold, since $s\oplus s^\perp$ defines trivializations for a tubular neighborhood of the (unframed) knot $\iota(S^1)$ in $S^3$.
\end{remark}

Under the construction described above, a framed knot corresponds to a map $f : S^3\to S^{2}$.
\todo{self-linking}

\begin{theorem}[Pontryagin Isomorphism]\label{thm:thom-pontryagin-isomorphism}
	Given a closed manifold $M^n$ of dimension $n>k$, there is a bijective correspondence
	\begin{equation}
		\lkxfunc{p^\fr_k(M)}{\Omega^\fr_k(M)}{{}[M, S^{n-k}].}
	\end{equation}
\end{theorem}

\subsection{Stable Homotopy Theory}

\begin{definition}
	The \defn{stable homotopy groups} of a pointed space $X$ are the colimits
	\[
		\pi_n^{s}(X) = \varinjlim_k \pi_{n+k}(\Sigma^k X).
	\]
\end{definition}

\begin{definition}
	The susepsnion
\end{definition}

\begin{definition}
	The \defn{sphere spectrum} is the suspension spectrum of the point.
\end{definition}

\begin{theorem}[Stable Pontryagin Isomorphism]
	There is a group isomorphism
	\[
		\lkxfunc{p^\fr_k}{\Omega_k^\fr}{\pi_k(\mathcal{S})}
	\]
\end{theorem}

\subsection{Obstruction Theory}

\pagebreak
\section{Framed Surgery Theory}

In \cref{sec:plumbing} and \cref{sec:brieskorn}, we constructed highly-connected manifolds

\begin{theorem}\label{thm:framed-surgery-highly-connected}
	Every compact framed manifold of dimension $n\geq 4$ with boundary $\partial M$ a homology sphere can be made highly-connected by a finite sequence of framed surgeries.
\end{theorem}

This is a bulky theorem, and so we will tackle it in parts. First, let us understand the effect of

\begin{proof}
\end{proof}

\begin{corollary}\label{cor:odd-dimensional-bP-trivial}
	The groups $|\bP^{2k+1}|$ are trivial.
\end{corollary}

\subsection{The Surgery Invariant}\label{sec:surgery-invariant}

\begin{theorem}[Milnor-Kervaire]\label{thm:kervaire-milnor}
	For every $k>1$, $bP^{4k}$ is a cyclic group of order
	\[
		|\bP^{4k}| =2^{2k-2}(2^{2k-1}-1)\epsilon(k)\denom(4k/B_{2k}).
	\]
\end{theorem}

\begin{theorem}[Kervaire Invariant Problem]\label{thm:kervaire-invariant-problem}
\end{theorem}

\begin{theorem}\label{thm:homotopy-sphere-stably-parallelizable}
	Every homotopy sphere is stably parallelizable.
\end{theorem}
\begin{proof}
	Let $\Sigma$ be a homotopy $n$-sphere.

	The only obstruction to the triviality of $T^sM$ is a well-defined cohomology class:
	\[
		\mathfrak{o}_n(\Sigma) \in \H^n(\Sigma; \pi_{n-1}(\SO_{n+1})) = \pi_{n-1}(\SO_{n+1})
	\]
	The coefficient group may be identified with the stable group $\pi_{n-1}(\SO)$, but these stable groups have been computed by Bott in \cite{bott1959stable}, for $n\geq 2$ we have:
	\begin{center}
		\begin{tabular}{c|cccccccc}
			\textrm{$n\mod 8$} & 0    & 1      & 2      & 3 & 4    & 5 & 6 & 7  \\
			\hline
			$\pi_{n-1}(\SO)$   & $\Z$ & $\Z/2$ & $\Z/2$ & 0 & $\Z$ & 0 & 0 & 0.
		\end{tabular}
	\end{center}
	If $\pi_{n-1}(\SO)$ is zero, we are done.

	If $\pi_{n-1}(\SO) = \Z$, then $n=4k$. According to cite{kervairemilnor1960} and cite{kervaire1959}, some non-zero multiple of the obstruction class $\mathfrak{o}_n(\Sigma)$ can be identified with the Pontryagin class $p_k(T^s M) = p_k(TM)$. \todo{(why?)} But the Hirzebruch signature theorem implies \todo{(why?)} that $p_k(\Sigma)$ is a multiple of the signature $\sigma(\Sigma)$ which is zero since $\H^{2k}(\Sigma)=0$. Thus every homotopy $4k$-sphere is \textsc{s}-parallelizable.

	Finally, suppose $\pi_{n-1}(\SO)= \Z_2$. It follows from an argument of Rohlin \todo{(what?)} that $J_{n-1}(\mathfrak{o}_n(\Sigma))=0$ where $J_{n-1}$ denotes the Hopf-Whitehead homomorphism
	\[
		\lkxfunc{J_{n-1}}{\pi_{n-1}(\SO_k)}{\pi_{n+k-1}(S^k)}
	\]
	in the stable range $k >n$. But $J_{n-1}$ is injective for $n\equiv 1, 2\mod 8$. This is proven by Adams. \todo{(find)} This means that $\mathfrak{o}_n(\Sigma)=0$.
\end{proof}

\section{Very Exotic Spheres}

\begin{theorem}\label{thm:finite-very-exotic-spheres}
	The quotient group $\Theta^n / \bP^{n+1}$ is finite.
\end{theorem}
\begin{proof}
	Let $M$ be an \textsc{s}-parallelizable $n$-manifold.

	Imbed it as $i : M \to S^{n+k}$ for some $k>n+1$ so that it's normal bundle is trivial.

	For each normal $k$-frame $\varphi$, we get an element of $\pi_{n+k}(S^k) = \pi^s(S^n)$ by the Pontryagin-Thom construction. Let's call the set of these elements (as $\varphi$ is allowed to vary) $p(\Sigma)$. \todo{(elaborate)}

	\todo{add lemmas}

	\begin{lemma}
		There is a homomorphism:
		\[
			\lkxfunc{p'}{\Theta_n}{\pi^s(S^n)/p(S^n)}
		\]
	\end{lemma}

	Furthermore, the kernel of $p'$ contains $h$-cobordism classes of homotopy $n$-spheres which bound parallelizable manifolds \todo{(provide lemma)}, which is exactly $bP_{n+1}$. By the first isomorphism theorem, it follows that $\Theta_n/bP_{n+1}$ is isomorphic to a subgroup of $\pi^s(S^n)$ which is finite.
\end{proof}

\subsection{Obstruction Theory}\label{sec:obstruction-theory}

% \section{Kervaire Invariant}
