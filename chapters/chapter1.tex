\providecommand{\bo}{\Sigma}
\providecommand{\co}{B}

\begin{flushleft}
	\textsl{A mathematician is a blind man in }\\
	\textsl{a dark room looking for a black cat}\\
	\textsl{which isn’t there.}\\
	\rule[0pt]{15em}{0.5pt}\\
	\textsl{-- Unknown}
	\vspace{2em}
\end{flushleft}

Before we begin attempts to construct an exotic sphere, we should answer the following question, without which we will get nowhere:
\begin{center}
	\textsl{How do we know if a given homotopy-sphere is exotic?}
\end{center}

This is a tricky problem -- evidenced in part by the fact that it's still open in dimension 4. We need an invariant associated to a differentiable manifold $\Sigma$ which is sensitive enough to discern smooth structure even for manifolds with the same homeomorphism type.
These requirements immediately exclude solely topological invariants such as the Euler characteristic, homotopy groups, the cohomology ring, etc. Since we're working with manifolds that have smooth structure, the natural thing to consider would be invariants built from the tangent bundle $\T \Sigma \to \Sigma$. For instance, we could build invariants out of characteristic classes (see \cref{chap:characteristic_classes} for a comprehensive review).

Even this faces some major challenges. For one thing, while the tangent bundle of a homotopy sphere isn't generally trivial (ordinary spheres, for instance, are only parallelizable in dimensions $1$, $3$, and $7$), the bundle is \emph{stably} trivial.
\begin{theorem*}
	If $\Sigma$ is a homotopy sphere, then $\T \Sigma \oplus \underline{\R}^k$ is trivial for some $k$.
\end{theorem*}
\begin{proof}
	See \cref{thm:homotopy_spheres_are_stably_parallelizable}.
\end{proof}
This isn't an obvious result by any means, but it does save us some wasted time. In particular, characteristic classes such as the Stiefel-Whitney and Pontryagin classes are only sensitive to the stable isomorphism type of a bundle by the Whitney product formula -- for instance
\[
	w(E\oplus\underline{\R}^k) = w(E)\smile w(\R^k) = w(E)
\]
in the Stiefel-Whitney case. Consequently, for any homotopy-sphere, the Pontryagin and Stiefel-Whitney numbers are all zero.

In fact, a much stronger, lesser known result holds true.
\begin{theorem*}
	If $\Sigma$ is a homotopy $n$-sphere with $f : S^n \to \Sigma$ the homotopy equivalence, then the induced bundle map is an isomorphism $f^*\T\Sigma \approx \T S^n$.
\end{theorem*}
\begin{proof}
	See \cref{thm:homotopy_spheres_tangent_bundle}.
\end{proof}

The triviality of characteristic classes is not a cause for mourning but rather a blessing in disguise. Recall that if all Stiefel-Whitney classes of a manifold vanish, then by \cref{thm:vanishing_stiefel_whitney} the manifold is null-cobordant -- it is the boundary of manifold $B$ one dimension higher.\footnote{In specific dimensions, there are much easier ways to prove this. For instance, in Milnor's first paper \cite{milnor1956manifolds} on the topic, the null-cobordism of homotopy-spheres is implied by the triviality of the $7$-dimensional cobordism group $\Omega^\SO_7$.} We'll refer to such a manifold $B$ as a \defn{coboundary} of the original manifold $\Sigma$. Although this extra data of a coboundary is far from unique, perhaps we could define some invariant of $B$ which only depends on its boundary $\partial B=\Sigma$. This approach will turn out to be incredibly fruitful in detecting exotic spheres, and contains some beautiful geometry.

\begin{remark}
	This idea of passing to a coboundary is an example of constructing a \defn{secondary invariant}. When primary invariants, in this case characteristic forms, turn out to be zero, we lift to a case where they are not zero, and use the descent data to measure ``how'' the original invariants vanished. This is a central idea in Chern-Simons theory \cite{chernsimons1974geometricinvariants}, \todo{and we'll explore it later in this chapter.}
\end{remark}

% In particular, we can use Poincar\'e duality
% \begin{theorem}[Poincar\'e Duality]
% 	Suppose $X$ is an $n$-dimensional closed manifold. Given a fundamental class $[X]\in \H^{n}(X)$, there is a duality isomorphism
% 	\[
% 		\lkxfunc{}{\H^k(X)}{\H_{n-k}(X)}{\omega}{\omega\frown [X]}
% 	\]
% 	given by cap product with the fundamental class.
% \end{theorem}

\section{Secondary Invariants}

In order to define invariants of the coboundary, we need to generalize our

So far, the invariants we've discussed have been primary invariants, defined entirely out of the intrinsic geometric and topological data of a manifold. Before we can define secondary invariants, we first need to
This requires understanding smooth invariants of manifolds with boundary, and this will be the focus of the section.
Throughout let's assume $\Sigma$ is a closed $n$-manifold with oriented coboundary $B$.
First of all, we'll note that many \emph{topologically} defined invariants generalize naturally to the case of manifolds with boundary. For instance, the Euler characteristic can be defined as a topological invariant at least for any finite CW complex.
Generalizing the intersection form and correspondingly the signature requires a slight generalization of the Poincar\'e duality theorem:

\begin{theorem}[Poincar\'e-Lefschetz Duality]
	Suppose $X$ is an $n$-dimensional manifold with boundary $\partial X$. Given a fundamental class $[X, \partial X]\in \H^{n}(X, \partial X)$, there is a duality isomorphism
	\[
		\lkxfunc{}{\H^k(X, \partial X)}{\H_{n-k}(X)}{\omega}{\omega\frown [X,\partial X]}
	\]
	given by cap product with the fundamental class.
\end{theorem}
\begin{proof}
	See Theorem~18.6.1 in \cite{dieck2008algebraic}.
\end{proof}
As in the case of ordinary Poincar\'e duality, the isomorphism $\H_0(X)\approx \R$ allows us to interpret the cap product as integration when $\omega$ is top dimensional -- there is an isomorphism $\H^m(X)\to \R$ which sends $\omega$ to $\int_X \omega$.

This allows us to define the signature and intersection form of a $4k$-dimensional manifold with boundary.

\begin{definition}
	If $X$ is an $4k$-manifold with boundary $\partial X$, the (relative) \defn{intersection form} is the symmetric bilinear form given by
	\[
		\lkxfunc{I_{X}}{\H^{2k}(X, \partial X)\otimes \H^{2k}(X, \partial X)}
		{\R}{\alpha\otimes \beta}{\int_X \alpha\smile \beta.}
	\]
	The (relative) \defn{signature} $\sigma(X, \partial X)$ of $X$ is the signature of this bilinear form.
\end{definition}

\begin{convention*}
	Most of the literature simply uses the notation $\sigma(X)$ even when $X$ has a non-empty boundary, and we will adopt this convention outside of this chapter. However, in this chapter, we would like to keep the distinction meaningful for better clarity.
\end{convention*}

While there are no topological constraints on a manifold in order to generalize the signature or Euler characteristic, constraints do appear when generalizing characteristic forms to the relative setting.
Characteristic forms are not a priori relative cohomology classes, so pulling them back to obtain \emph{relative} characteristic forms in order to integrate requires additional assumptions about the topology of the boundary $\Sigma$.
For any integer $\ell$, the pair $(B, \partial B) = (B, \Sigma)$ gives us a long exact sequence of cohomology groups
\begin{equation}\label{eq:relative_characteristic_classes_exact_sequence}
	\H^{\ell-1}(\Sigma) \lkxto \H^{\ell}(B, \Sigma) \lkxto[j] \H^{\ell}(B) \lkxto \H^{\ell}(\Sigma)
\end{equation}
where $j : \H^{\ell}(B, \Sigma) \to \H^{\ell}(B)$ is the induced map of the inclusion $(B,\emptyset) \to (B, \Sigma)$. This is an isomorphism if the groups on either side of \cref{eq:relative_characteristic_classes_exact_sequence} are trivial. In this case, we can pullback:

\begin{definition}\label{defn:relative_characteristic_form}
	Suppose that $\H^{\ell}(\Sigma)$ and $\H^{\ell-1}(\Sigma)$ are trivial. For a characteristic form $c_\ell(B) \in \H^{\ell}(B)$, the \defn{relative characteristic form} is the pullback
	\[
		c_\ell(B, \Sigma) = j^{-1} c_\ell(B) \quad\in \H^{\ell}(B, \Sigma).
	\]
	The \defn{relative characteristic number} is the integral $c_\ell[B,\Sigma]=\int_B c_\ell(B,\Sigma)$.
\end{definition}

For instance, we could define relative Pontryagin numbers in this way:

\begin{definition}\label{defn:relative_pontryagin_number}
	Given a polynomial $K\in \Q[x_1,\ldots, x_k]$ satisfying the conditions of \cref{defn:pontryagin_number}, suppose that $\H^{4i}(\Sigma)$ and $\H^{4i-1}(\Sigma)$ are trivial for all $i$ for which $K$ has a $x_i$ term.
	In this case, we define the \defn{relative Pontryagin number} to be the integral
	\[
		\begin{aligned}
			K(p_1, \ldots, p_k)[B,\Sigma]
			 & = \int_B K(p_1, \ldots, p_k)(B,\Sigma)       \\
			 & = \int_B K(j^{-1}p_1, \ldots, j^{-1}p_k)(B).
		\end{aligned}
	\]
\end{definition}

\begin{remark}
	Note that we pullback \emph{before} applying the polynomial $K$. This is because pulling back a top-dimensional form on $B$ is not generally possible since $\H^{n}(\Sigma)$ is non-trivial and generated by the fundamental class.
\end{remark}

\begin{remark}
	Note that if $\Sigma=\emptyset$, relative characteristic forms and numbers correspond exactly to the non-relative versions since $j$ becomes the identity map.
\end{remark}

We've defined some useful relative invariants -- we now have the relative signature and relative Pontryagin numbers, although the latter comes with some topological restrictions. Our original goal was to use relative invariants of the coboundary $B$ to get an invariant for the boundary $\Sigma$. Thus, our next question should be:
\begin{center}
	\textsl{How do relative invariants change with the coboundary?}
\end{center}

There is a elegant trick we can use to help us answer this. If $B_1$ and $B_2$ are coboundaries for $\Sigma$, we can form a closed $(n+1)$-manifold $C$ by glueing $B_1$ and $B_2$ along their boundary $\Sigma$. There is a unique smooth structure on $C$ which agrees with the smooth structures of $B_1$ and $B_2$, and we can give $C$ the orientation which agrees with the orientation of $B_1$ and therefore with the reverse orientation of $B_2$.

By the Mayer-Vietoris sequence, we have an exact sequence
\[
	\H^{\ell-1}(\Sigma)\lkxto \H^{\ell}(C) \lkxto[\mu] \H^{\ell}(B_1)\oplus \H^{\ell}(B_2) \lkxto \H^{2k}(\Sigma)
\]
for any $\ell$, where $\mu$ is the map which restricts a form $\omega\in \H^\ell(C)$ to the form $\omega|_{B_1}\oplus \omega|_{B_2}\in \H^{\ell}(B_1)\oplus \H^{\ell}(B_2)$.
The relative version of the exact sequence is of the form
\[
	0 \lkxto \H^{\ell}(C; \Sigma) \lkxto[\rho] \H^{\ell}(B_1;\Sigma)\oplus \H^{\ell}(B_2;\Sigma) \lkxto 0,
\]
so we have an isomorphism $\rho$.
These maps are related neatly by the inclusion isomorphisms in \cref{eq:relative_characteristic_classes_exact_sequence}, and we can use these to form the commutative square:
\begin{equation}\label{eq:closing_coboundaries_square}
	\begin{tikzcd}
		{\H^{\ell}(C,\Sigma)} & {\H^{\ell}(B_1,\Sigma)\oplus\H^{\ell}(B_2,\Sigma)} \\
		{\H^{\ell}(C)} & {\H^{\ell}(B_1)\oplus\H^{\ell}(B_2)}
		\arrow["j_1\oplus j_2"', from=1-2, to=2-2]
		\arrow["\rho"', from=1-1, to=1-2]
		\arrow["j"', from=1-1, to=2-1]
		\arrow["\mu"', from=2-1, to=2-2]
		\arrow["h", from=1-2, to=2-1, dashed]
	\end{tikzcd}
\end{equation}
In the case that $\H^{\ell-1}(\Sigma)$ and $\H^\ell(\Sigma)$ are trivial, every map in this diagram is an isomorphism. Otherwise, we can only assume that the top map $\rho$ is an isomorphism.
Of particular interest to us is the diagonal map $h = j\circ \rho^{-1}$, which ``glues'' together relative forms on $B_1$ and $B_2$ to a form on $C$.

This glueing map satisfies the naturality properties:

\begin{proposition}\label{prop:variation_naturality_poincare}
	If $\alpha\in \H^{n+1}(B_1, \Sigma)$ and $\beta\in \H^{n+1}(B_2,\Sigma)$, then we have
	\[
		\int_C h(\alpha\oplus \beta) = \int_{B_1}\alpha - \int_{B_2}\beta.
	\]
\end{proposition}
\begin{proof}
	\todo{do this proof}
\end{proof}

\begin{proposition}\label{prop:variation_naturality_cup}
	If
	$\alpha_i\in \H^{\ell_i}(B_1,\Sigma)$ and $\beta_i \in \H^{\ell_i}(B_2,\Sigma)$ for $i=1,2$, then we have
	\[
		h(\alpha_1\oplus\beta_1) \smile h(\alpha_2\oplus \beta_2) = h(\alpha_1\smile \alpha_2 \oplus \beta_1\smile \beta_2).
	\]
\end{proposition}
\begin{proof}
	\todo{do this proof}
\end{proof}

As an immediate corollary of these two properties, we now have:
\begin{corollary}
	Suppose $\alpha_i\in \H^{\ell_i}(B_1, \Sigma)$ and $\beta_i\in \H^{\ell_i}(B_2,\Sigma)$ for $0\leq i < k$, and $K\in \Q[x_1,\ldots,x_k]$ a polynomial with $K(x^{\ell_1}, \ldots, x^{\ell_k})$ homogenous of degree $n+1$. Then, we have
	\[
		\int_C K(h(\alpha_1\oplus \beta_1),\ldots, h(\alpha_k\oplus \beta_k))
		=
		\int_{B_1} K(\alpha_1,\ldots, \alpha_k) - \int_{B_2} K(\beta_1,\ldots, \beta_k).
	\]
\end{corollary}

In particular, this gives us a formula for the change in relative Pontryagin numbers under a change of coboundary:
\begin{proposition}\label{prop:relative_pontryagin_number_variation}
	For $K\in \Q[x_1,\ldots, x_k]$ and $\Sigma$ as in \cref{defn:relative_pontryagin_number}, we have
	\begin{equation}\label{eq:relative_pontryagin_number_variation}
		K(p_1,\ldots,p_k)[B_1,\Sigma] - K(p_1,\ldots, p_k)[B_2,\Sigma] = K(p_1,\ldots,p_k)[C].
	\end{equation}
\end{proposition}

Another corollary relates to the signature. When $n=4k-1$, it makes sense to talk about the intersection forms of $B_1,B_2,$ and $C$. In this case, for forms $\alpha_1,\alpha_2\in \H^{2k}(B_1, \Sigma)$ and $\beta_1,\beta_2\in \H^{2k}(B_2,\Sigma)$ we can set $\alpha=h(\alpha_1\oplus\alpha_2)$ and $\beta=h(\beta_1\oplus \beta_2)$ in $\H^{2k}(C)$ and get
\[
	\int_{C} \alpha\smile \beta = \int_{B_1} \alpha_1\smile \alpha_2 - \int_{B_2}\beta_1\smile \beta_2.
\]
If $h$ is an isomorphism in dimension $2k$, for instance if $\H^{2k}(\Sigma)$ and $\H^{2k-1}(\Sigma)$ are trivial, then every element of $\H^{2k}(C)$ admits such a decomposition. This means that under the identification of $\H^{2k}(B_1,\Sigma)\oplus \H^{2k}(B_2,\Sigma)$ with $\H^{2k}(C)$ by $h$, we have
\[
	I_C = \begin{pmatrix}I_{B_1} & 0 \\ 0 & -I_{B_2}\end{pmatrix}.
\]
In other words, the intersection form of $C$ is the difference of the intersection form of $B_1$ and $B_2$. For the signature, this has the following implication, similar to \cref{prop:relative_pontryagin_number_variation}:
\begin{proposition}\label{prop:signature_variation}
	If $\H^{2k}(\Sigma)$ and $\H^{2k-1}(\Sigma)$ are trivial, then the signature satisfies the relation
	\begin{equation}\label{eq:signature_variation}
		\sigma(B_1, \Sigma) - \sigma(B_2, \Sigma) = \sigma(C).
	\end{equation}
\end{proposition}

Overall, from perspective of secondary invariants:
\begin{center}
	\textsl{The change of a secondary invariant with coboundary is expressible in terms}\\
	\textsl{of the invariant applied to a closed manifold.}
\end{center}

% We now have all of the tools needed to construct some basic invariants of homotopy spheres.

\section{The Hirzebruch Signature Theorem}

Thus far, we've defined the signature of a manifold entirely in topological terms.
However there is a surprising and beautiful way to express the signature an integral of a geometric quantity involving Pontryagin classes over a given manifold.
The connections between global topological invariants and locally defined geometric quantities run deep, and this is the start of a beautifully rich area of mathematics known as index theory.

For our purposes of defining secondary invariants for homotopy spheres, it will be important to understand the signature of a closed manifold. After all, as we saw in the previous section, varying the coboundary changes the signature by the signature of a closed manifold.

\begin{proposition}
	For closed manifolds $X_1$ and $X_2$, we have:
	\begin{enumerate}
		\item $\sigma(X_1\+ X_2) = \sigma(X_1)+\sigma(X_2)$.
		\item $\sigma(X_1\times X_2) = \sigma(X_1)\cdot\sigma(X_2)$.
	\end{enumerate}
\end{proposition}
\begin{proof}
	\todo{this proof}
\end{proof}

See \cref{chap:cobordism} for a review of cobordism.

\begin{proposition}
	If there is an oriented cobordism $X_1 \sobord X_2$, then $\sigma(X_1)=\sigma(X_2)$.
\end{proposition}
\begin{proof}
	\todo{this proof}
\end{proof}

These two propositions lead to the observation that $\sigma$ is a ring homomorphism
\[
	\lkxfunc{\sigma}{\Omega^\SO_\bullet}{\Z}
\]
from the ring of oriented cobordism classes of closed manifolds to the integers. Recall that the operations on $\Omega^\SO_\bullet$ are exactly the connected sum and product operations of manifolds. In particular, this homomorphism factors through $\Omega^\SO_\bullet\otimes \Q$ so it really can be considered as a homomorphism
\[
	\lkxfunc{\sigma}{\Omega^\SO_\bullet\otimes \Q}{\Z.}
\]
We already know what $\Omega^\SO_\bullet\otimes \Q$ is -- it is a polynomial ring generated by the cobordism classes of the complex projective spaces $[\CP^{2n}]$. For a proof of this decomposition, see \cref{thm:thom-pontryagin_oriented_cobordism}.

Recall that the cohomology rings of complex projective spaces is given by
\[
	\H^\bullet(\CP^{m}) = \R[\alpha]/(\alpha^{m+1})\quad\textrm{where}\quad |\alpha|=2.
\]
In particular, this means that $\dim \H^{2k}(\CP^{2k})=1$ so the intersection form is the $1\times 1$ identity matrix and the signature is $\sigma(\CP^{2k})=1$. Given a cobordism decomposition of any manifold $M$ into connected sums and products of complex projective spaces -- say $M\sobord F(\CP^2,\CP^4,\ldots)$ for a polynomial $F$ on $\Omega^\SO_\bullet\otimes \Q$ -- we can compute its signature $\sigma(M)=F(1,1,\ldots)$ by substituting $1$'s into the polynomial.

Cobordism decompositions of a manifold into complex projective spaces are tricky to compute in general, so we would like an easier way to compute the signature.
Recall that if $K(p_1,\ldots, p_k)[X]$ is a Pontryagin number, it is additive with respect to connected sum, i.e.
\[
	K(p_1,\ldots,p_k)[X_1\+ X_2] = K(p_1,\ldots,p_k)[X_1]+K(p_1,\ldots,p_k)[X_2]
\]
for closed manifolds $X_1$ and $X_2$. Furthermore, the Pontryagin classes are cobordism invariant (see \cref{prop:characteristic_numbers_cobordism_invariant}). This means that a Pontryagin number is a \emph{group} homomorphism
\[
	\lkxfunc{K(p_1,\ldots,p_k)}{\Omega^\SO_{4k}\otimes \Q}{\Z.}
\]
To replicate the signature we need to assemble such Pontryagin number group homomorphisms into a \emph{ring} homomorphism $\Omega^\SO_\bullet\otimes \Q\to \Z$, which in particular is multiplicative across degrees. At this point the algebraic formalism of multiplicative sequences and characteristic series is quite useful.

\subsection*{Multiplicative Sequences and Characteristic Series}

Let $\Lambda$ be a commutative ring and suppose $A=A_0\oplus A_1\oplus\cdots$ is a $\Z_{\geq 0}$-graded $\Lambda$-algebra. We can form the ring of formal power series $A\fps{t} = \left\{ a =a_0+ a_1 t^1+ a_2 t^2+\cdots \mid t_i\in A_i\right\}$.

\begin{definition}
	Given a sequence of polynomials $K_n(x_1, \ldots, x_n)$ homogeneous of degree $n$, define
	\[
		\lkxfunc{K}{A\fps{t}}{A\fps{t}}{x}{1+K_1(x_1)\cdot t^1+K_2(x_1,x_2)\cdot t^2+\cdots}
	\]
	The sequence $\{K_n\}$ is a \defn{multiplicative sequence} if $K(xy)=K(x)K(y)$ for all series $x,y\in A\fps{t}$.
\end{definition}

\begin{proposition}
	There is a bijective correspondence
	\[
		\left\{\right\}
		\quad\iff\quad
		\left\{\right\}
	\]
\end{proposition}

\subsection*{The $L$-Genus}

\begin{definition}
	The \defn{$\bm{L}$-genus} $\{L_k\}$ is the multiplicative sequence of polynomials corresponding to the series $Q(z) = \sqrt{z}/\tanh\sqrt{z}$.
\end{definition}

\begin{theorem}[Hirzebruch Singature Theorem]\label{thm:hirzebruch_signature}
	Let $X$ be a closed (oriented) $4k$-manifold. Then we have
	\[
		\sigma(X) = \int_X L_k(p_1, \cdots, p_k)(X),
	\]
	where $L_k$ is the $L$-genus.
\end{theorem}
\begin{proof}
	It suffices to check this identity for the generators $\CP^{2k}$. The total Pontryagin class of $\CP^{2k}$ is $p(\CP^{2k})=(1+\alpha^2)^{2k+1}$. Using multiplicativity of $L$ and $L(1+z)=\sqrt{z}/\tanh\sqrt{z}$, we have
	\[
		\begin{aligned}
			L(p)(\CP^{2k})
			= L\left((1+\alpha^2)^{2k+1}\right)
			= L(1+\alpha^2)^{2k+1}
			= \left(\alpha/\tanh \alpha\right)^{2k+1}
			\quad\in\H^\bullet(\CP^{2k}).
		\end{aligned}
	\]
	Here, we consider $\alpha/\tanh \alpha$ as a formal power series in $\alpha$ truncated by the relation $\alpha^{2k+1}=0$ in the cohomology ring $\H^\bullet(\CP^{2k})$. The characteristic number $L_k(p_1,\ldots,p_k)[\CP^{2k}]$ is then the coefficient of $\alpha^{2k}$ in the expansion of $(\alpha/\tanh \alpha)^{2k+1}$.
	By elementary complex analysis, this coefficient can be extracted by taking a contour integral around a small $\varepsilon$-circle about the origin in $\C$:
	\[
		\begin{aligned}
			L_k(p_1,\cdots, p_k)[\CP^{2k}]
			 & = \frac{1}{2\pi i}\oint_{S^1_\varepsilon} \frac{dz}{z^{2k+1}} \left(\frac{z}{\tanh z}\right)^{2k+1}
			 &                                                                                                       \\[0.5em]
			 & = \frac{1}{2\pi i}\oint_{S^1_\varepsilon} \frac{dz}{\tanh^{2k+1} z}\quad
			 & u  =\tanh(z),\quad
			du =(1-u^2)dz
			\\[0.5em]
			 & = \frac{1}{2\pi i}\oint_{S^1_\varepsilon} \frac{1}{u^{2k+1}}\cdot\frac{du}{1-u^2}
			 &                                                                                                       \\[0.5em]
			 & = \frac{1}{2\pi i}\oint_{S^1_\varepsilon} \frac{1+u^2+u^4+\cdots}{u^{2k+1}}\,du                       \\[0.5em]
			 & =1.                                                                                                 &
		\end{aligned}
	\]
	Since $\sigma(\CP^{2k})=1$, this completes the proof.
\end{proof}

The Hirzebruch signature theorem is a shining example of a truly remarkable theorems in mathematics -- it gives us easy computational means to uncover highly non-trivial relationships between complicated objects. For us, the most useful consequence of the Hirzebruch signature theorem is that it gives us subtle integrability and divisibility theorems. For instance, with relatively little effort we can calculate a few $L$-genus polynomials to get:
\begin{equation}\label{eq:L-genus}
	\begin{aligned}
		 & L_1 = \frac{p_1}{3},\quad
		 & L_2 = \frac{7p_2 - p_1^2}{45},\quad
		 & L_3 = \frac{62p_3 - 13p_1p_2 + 2p_1^3}{945},\quad\cdots
	\end{aligned}
\end{equation}
Having done this, the expression for the leading coefficient of $L_3$ in \cref{eq:L-genus} immediately implies:
\begin{corollary}
	If $X$ is a closed $12$-manifold with trivial $\H^4(X)$, then $\sigma(X)$ is divisible by $62$.
\end{corollary}
The observation that for such manifolds $X$, the quantity $\sigma(X)/62$ is an integer, in fact equal to $945p_3[X]$, is far from obvious, and yet it pops out immediately from the signature theorem. These subtle relationships are immensely useful in defining invariants capable of detecting exotic spheres as we will see in the following \cref{sec:invariants_from_integrability}.

While we're on the topic of the $L$-genus, let's compute its leading coefficient. Often times, the manifolds which we'll apply the signature theorem to will be connected enough that the lower order Pontryagin classes vanish -- leaving just the leading term.
As it turns out, the coefficient of this leading term admits a simple description in terms of the \defn{Bernoulli numbers} $B_{2k}$, a sequence of rational numbers which appear surprisingly ubiquitously throughout topology. For our purposes, we can define them as the terms appearing in the series expansion of $\tanh z$:
\begin{equation}\label{eq:tanh_series}
	\tanh z = z - \frac{z^3}{3} + \frac{2z^5}{15} - \frac{17z^7}{315}+\cdots = \sum_{k\geq 1} (-1)^k\frac{2^{2k}(2^{2k}-1)B_{2k}}{(2k)!}\, z^{2k-1}.
\end{equation}
With this definition, the first few Bernoulli numbers are given:
\begin{equation}\label{eq:bernoulli_numbers}
	B_0 = 1,\quad B_2 = \frac{1}{6},\quad B_4 = \frac{1}{30},\quad B_6=\frac{1}{42},\quad B_{8}=\frac{1}{30},\quad B_{10} = \frac{5}{66},\quad\cdots
\end{equation}

\todo{complexity of the sequence indicates how complicated the geometry can get}

\begin{proposition}\label{prop:leading_coefficient_L_genus}
	The leading coefficient of the $L$-genus is $s_k=2^{2k}(2^{2k-1}-1)B_{2k}/(2k)!$
\end{proposition}
\begin{proof}
	\todo{this proof}
\end{proof}

\todo{define genera}

\section{Invariants From Integrality}\label{sec:invariants_from_integrability}

With the signature theorem, we now have all the tools in place to construct our sought after secondary invariants which will be capable of detecting exotic spheres. Since the following notions are subtle, we'll start in a low dimensional case.

\subsection*{Milnor's Invariant for $7$-Manifolds}

Let's see what types of invariants can be constructed out of relative Pontryagin classes and the relative signature. Suppose $\Sigma$ is a $7$-dimensional homotopy-sphere with an $8$-dimensional coboundary $B$. Based on the cohomology, we know
\[
	\begin{aligned}
		\H^3(\Sigma)=0,  & \quad \H^4(\Sigma)=0 \\
		\H^7(\Sigma)=\R, & \quad \H^8(\Sigma)=0 \\
		\H^3(\Sigma)=0,  & \quad \H^4(\Sigma)=0
	\end{aligned}
	\quad\implies\quad
	\begin{aligned}
		 & p_1^2\textrm{ does have a relative generalization}         \\
		 & p_2\textrm{ does not have a relative generalization}       \\
		 & \sigma\textrm{ satisfies \cref{prop:signature_variation}.}
	\end{aligned}
\]
Thus, the two invariants of interest to us are
\[
	p_1^2[B, \partial B]
	\quad\textrm{and}\quad
	\sigma(B, \partial B).
\]
Meanwhile, the second Pontryagin number $p_2[B]$ does not have a relative generalization.
Now, for a \emph{closed} $8$-manifold $X$, rearranging the signature theorem gives us the expression:
\begin{equation}\label{eq:7-manifold_rearrangement}
	\sigma(X) = L_2(p_1, p_2)[X] = \frac{7p_2[X] - p_1^2[X]}{45}
	\quad\implies\quad
	p_2[X] = \frac{45\sigma(X) + p_1^2[X]}{7}.
\end{equation}
This suggests that maybe there \emph{is} some analogue of the second Pontryagin class for $B$. For manifolds with boundary, we could define the number
\[
	\widetilde{p_2}[B, \partial B] = \frac{45\sigma(B, \partial B) + p_1^2[B, \partial B]}{7}.
\]
This is a \emph{rational} number,
which reduces to the second Pontryagin number $p_2[B]$, an integer, when $\partial B=\emptyset$. How does the quantity change under a change in coboundary, say if $B_1$ and $B_2$ were coboundaries? Letting $X$ be the $8$-manifold obtained by glueing them together, we see that
\[
	\begin{aligned}
		\widetilde{p_2}[B_1,\partial B] - \widetilde{p_2}[B_2,\partial B]
		 & = \frac{45\sigma(B_1,\partial B) + p_1^2[B_1,\partial B]}{7} - \frac{45\sigma(B_2, \partial B) + p_1^2[B_2,\partial B]}{7} \\
		 & =\frac{45\sigma(X) + p_1^2[X]}{7} = p_2[X].
	\end{aligned}
\]
But this last term is just an ordinary Pontryagin number, and hence an integer. While $\widetilde{p_2}$ is a priori a rational number for a given coboundary, it varies by an integer -- namely by the Pontryagin number $p_2[X]$ of a closed manifold.
Taking the fractional part of $\widetilde{p_2}$ thus gives us an invariant of $\Sigma$ which is \emph{independent of the coboundary}! This is exactly the secondary invariant we've been after.

\begin{definition}\label{defn:milnor_invariant_7}
	Let $\Sigma$ be a closed $7$-manifold bounding an $8$-manifold $B$ with $\H^3(\Sigma)$ and $\H^4(\Sigma)$ trivial.
	The \defn{Milnor invariant}\footnote{This differs from Milnor's original definition in \cite{milnor1956manifolds} by a factor of $2$, and here we write it in fraction form -- Milnor's original definition was $\lambda= 2p_1^2-\sigma\mod 7$.} of $\Sigma$ is given by
	\[
		\boxed{\lambda_{\mathrm{milnor}}(\Sigma^7) = \frac{1}{7}\left(3\sigma(B,\partial B)+p_1^2[B,\partial B]\right)\mod 1}
	\]
	where $B$ is any coboundary of $\Sigma$.
\end{definition}

It's clear that this is a diffeomorphism invariant -- the hard part is understanding why it's well defined. The key point is that if $\Sigma_1$ and $\Sigma_2$ are diffeomorphic manifolds and $B$ is a coboundary of $\Sigma_1$, we can form a diffeomorphic coboundary $B'$ by gluing $\Sigma_2$ to $B$ along the diffeomorphism $\Sigma_1\approx\Sigma_2$. The resulting coboundary $B'$ is then diffeomorphic to $B$ and so the Milnor invariants are equal.

Since the ordinary sphere $S^7$ is the boundary of a ball $B^8$ which is contractible, the Milnor invariant of the ordinary sphere $\lambda_{\textrm{milnor}}(S^7)=0$ is zero.
If we can find a homotopy $7$-sphere $\Sigma$ with $\lambda_{\textrm{milnor}}(\Sigma)\neq 0$, we will have found an exotic sphere.

\begin{remark} The ``resolution'' of this invariant is $7$ -- since it can take on only $7$ values, this invariant can detect at most $7$ types of homotopy-spheres in this dimension. There are refinements of this invariant which can detect all $28$ classes of homotopy-spheres in $7$-dimensions -- we will see such a refinement in \cref{sec:eells-kupier_invariant}.\end{remark}

\subsection*{The First Exotic Sphere}

At this point we'll take a brief detour and go over the original construction by Milnor of an exotic sphere since this was the invariant used to prove its exoticity.

\todo{write up, feel free to skip}

\subsection*{Milnor's Invariant for $(4k-1)$-Manifolds}

Returning back to the invariant, let's try generalizing the invariant.
The basic ideas outlined for $7$-manifolds should work for $(4k-1)$-manifolds in general, so let's see what we can do. If we require that $\H^{2k}(\Sigma)$ and $\H^{4i}(\Sigma)$ trivial for all $i<k$, then Poincar\'e duality ensures that $\H^{2k-1}(\Sigma)$ and $\H^{4i}(\Sigma)$ are trivial as well. For full generality in defining the invariant, we should also assume that $\Sigma$ is null-cobordant (which all homotopy-spheres are) so that a coboundary exists.
Just as in the $7$-dimensional case, all but the top-dimensional Pontryagin classes can be generalized to a coboundary $B$.

If $X$ is a closed $4k$-manifoldm then using the signature theorem, we can do a rearrangement similar to \cref{eq:7-manifold_rearrangement} in order to get the expression
\begin{equation}\label{eq:4k-1-manifold_rearrangement}
	\begin{aligned}
		\sigma(X) = L_k(p_1, \ldots, p_k)[X] =
		 & L_k(p_1,\ldots,p_{k-1},0)[X] + s_k \cdot p_k[X] \\
		 & \quad\implies\quad
		p_k[X] = \frac{\sigma(X) - L_k(p_1,\ldots, p_{k-1}, 0)[X]}{s_k},
	\end{aligned}
\end{equation}
where recall $s_k$ is the leading coefficient of the $L$-genus (see \cref{prop:leading_coefficient_L_genus}).
This expression lets us define a rational number
\[
	\widetilde{p_k}[B, \partial B] = \frac{\sigma(B, \partial B) - L_k(p_1,\ldots, p_{k-1}, 0)[B,\partial B]}{s_k}
\]
which acts as a generalization of the $k$-th Pontryagin class to a manifold with boundary. If we have two coboundaries $B_1$ and $B_2$, applying \cref{prop:relative_pontryagin_number_variation} and \cref{prop:signature_variation} gives us
\[
	\widetilde{p_k}[B_1, \Sigma] - \widetilde{p_k}[B_2, \Sigma] = p_k[X],
\]
where $X$ is the glueing of $B_1$ with $B_2$. Since the $k$-th Pontryagin number of a closed manifold is an integer, we take the fractional part of $\widetilde{p_k}[B,\Sigma]$ and arrive at a generalization of the Milnor invariant for a $(4k-1)$-manifold:
\begin{definition}\label{defn:milnor_invariant}
	Let $\Sigma$ be a closed $(4k-1)$-manifold bounding a $4k$-manifold $B$ with $\H^{4i}(\Sigma)$ and $\H^{2k}(\Sigma)$ trivial for all $i<n$. The \defn{Milnor invariant} of $\Sigma$ is given by
	\[
		\boxed{
			\lambda_{\mathrm{milnor}}(\Sigma^{4k-1}) = \frac{1}{s_k}\Big(\sigma(B, \partial B) - L_k(p_1, \ldots, p_{k-1},0)[B,\partial B]\Big)\mod 1
		}
	\]
	where $s_k = L_k(0,\ldots,0,1)$ is the leading coefficient of the $L$-genus.
\end{definition}

\begin{example}
	Using \cref{eq:L-genus}, we can compute the first few Milnor invariants
	\[
		\begin{aligned}
			\lambda_{\mathrm{milnor}}(\Sigma^7)
			 & = \frac{4}{7}\cdot \sigma(B, \partial B) - \frac{1}{7}\cdot p_1^2[B, \partial B],
			 \\[0.5em]
			\lambda_{\mathrm{milnor}}(\Sigma^{11})
			 & = \frac{15}{62}\cdot \sigma(B,\partial B) - \frac{1}{62}\left(2p_1^3 - 13p_1p_2\right)[B,\partial B],
			 \\[0.5em]
			 \lambda_{\mathrm{milnor}}(\Sigma^{15})
			 &= \frac{101}{127}\cdot \sigma(B,\partial B) - \frac{1}{381}\left(3p_1^4 + 22p_1^2p_2 - 19p_2^2 - 71p_1p_3\right)[B,\partial B].
		\end{aligned}
	\]
\end{example}

If we wish to generalize these invariants further, it's helpful to take a bird's eye view of what happened here. We started with the observation that the signature and Pontryagin numbers of a homotopy-sphere were trivial, so we lifted to a coboundary. Once on this coboundary, we constructed a characateristic number which takes on a restricted set of values for closed manifolds, the closed manifolds representing a ``change of coboundary''. Moding out by the image of such changes, we get an invariant purely of the boundary. To summarize, we have a loose procedure which turns:
\[
	\left\{\parbox{12.5em}{An integrality theorem for a characteristic number of a closed $(n+1)$-manifold}\right\}
	\quad\implies \quad
	\left\{\parbox{12em}{A diffeomorphism invariant for a class of $n$-manifolds}\right\}
\]
In this case, the integrality of the Pontryagin number $p_k$ of a closed $4k$-manifold led us to Milnor's invariant for $(4k-1)$-manifolds with certain vanishing cohomology groups.
If we use a different integrality theorem, would this procedure give us a usefully different invariant? The answer turns out to be yes, but we need a fancier integrality theorem.

\todo{skip notice: At this point, we have a pretty good invariant which can be used to see some basic examples of exotic spheres. As such, the impatient reader should feel free to skip ahead to }

\subsection{The Eells-Kupier Invariant}\label{sec:eells-kupier_invariant}

\begin{proposition}
	If $\Sigma_1$ and $\Sigma_2$ are homotopy spheres, then we have \[\lambda_{\mathrm{milnor}}(\Sigma_1\+\Sigma_2) = \lambda_{\mathrm{milnor}}(\Sigma_1) + \lambda_{\mathrm{milnor}}(\Sigma_2).\]
\end{proposition}


\subsection{The Witten Genus and Exotic $\mathbf{23}$-Spheres}
\begin{definition*}
	Let $\Sigma$ be a closed $23$-manifold with $\H^i(\Sigma)=0$ for $i<23$ with $i\equiv 0,3\mod 4$. Then if $\Sigma$ has an $11$-connected string coboundary $B$, define:
	\[
		\mu(\Sigma) = \frac{153945\sigma(B) + 2591p_2^3[B, \partial B]}{521432801280}\mod 1
	\]
	where $\sigma(B)$ is the signature of $B$.
\end{definition*}

