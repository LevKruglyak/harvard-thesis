\chapter{Differential Geometry}

\begin{flushleft}
	\textsl{I admire the elegance of your method of computation;}\\
	\textsl{it must be nice to ride through these fields upon the}\\
  \textsl{horse of true mathematics while the like of us have to}\\
  \textsl{make our way laboriously on foot.}\\
	\rule[0pt]{24em}{0.5pt}\\
	-- \textsc{Albert Einstein} to \textsc{Tullio Levi-Civita}\\
	\vspace{2em}
\end{flushleft}

If geometry is the study of 
\begin{definition}
  A (linear) \defn{symmetry type} is a pair $(G_n,\rho)$ consisting 
  of an $n$-dimensional Lie group $G_n$ and a smooth representation $\rho : G_n \to \GL_n$.
\end{definition}

\begin{figure}[ht]
  \centering
  \begin{tabular}{|c|c|}
    \hline
    \textbf{Geometry} & \textbf{Lie group} \\
    \hline
    Riemannian & $\O_n$ \\
    Oriented Riemannian & $\SO_n$ \\
    Lorentzian & $\O_{1,n-1}$ \\
    Spin & $\Spin_n$\\
    \hline
  \end{tabular}\vspace{0.5em}
  \caption{Symmetry groups}
\end{figure}

% \begin{definition}
%   A \defn{stable symmetry type} consists of a pairs $(G, \rho)$ where $G$ is a Lie group
% \end{definition}

\begin{definition}[Associated Fiber Bundles]
  Let $F$ be a smooth manifold with a left $G$-action. Then to any principal $G$-bundle $P \to X$ we get an associated fiber bundle $P\times_G F \to X$ with total space
  \[
    P\times_G F = \frac{P\times F}{(p,f) \sim (pg, g^{-1}f)}.
  \]
\end{definition}

Let $G, H$ be Lie groups and $\rho : G \to H$ a homomorphism. This induces a functor:
\[
    \lkxfunc{\rho}{\Bun_G(X)}{\Bun_H(X)}{(P \to X)}{(P\times_G H \to X)}
\]

  Assume $\mathcal{G} = (G_n, \rho)$ is a symmetry type.

\begin{definition}
  Let $X$ be a smooth $n$-manifold.
  A \defn{$\mathcal{G}$-structure} on a manifold $M$ is an isomorphism
  \[
    \lkxfunc{\theta}{\B X}{\B X\times_{G_n} \GL_n}
  \]
\end{definition}

\section{Vector Bundles}

Smooth manifolds posses many \todo{write the introduction}

\begin{theorem}[Serre-Swan]\label{eq:serre_swan}
Let $M$ be a connected smooth manifold.
  The module of global sections of a vector bundle gives an equivalence
  \[
    \lkxfunc{\Gamma}{\Vect(M)}{\Proj\left(\Omega^0(M)\right)}
  \]
  between the category of finite rank vector bundles over $M$ and the category of finitely generated projective $\Omega^0(M)$-modules.
\end{theorem}
\begin{proof}
  \todo{citation}
\end{proof}

\begin{remark} 
  This theorem is rather remarkable \todo{discussion about vector bundles over a ringed space in algebraic geometry}
\end{remark}

In fact, an even stronger statement is true. Both $\Vect(M)$ and $\Proj\left(\Omega^0(M)\right)$ are \defn{monoidal categories}[monoidal category] -- i.e. they have a bifunctor $\otimes : \Cscr\times \Cscr \to \Cscr$ that is associative and unital up to natural isomorphism. In the category of vector bundles, this bifunctor is the tensor product, and in the category of $\Omega^0(M)$-modules, this bifunctor is the tensor product over $\Omega^0(M)$. 

\begin{theorem} $\Gamma$ is an additive, monoidal functor, 
  \[
    \Gamma(E\oplus F) = \Gamma(E)\oplus\Gamma(F),\quad
    \Gamma(E\otimes F) = \Gamma(E)\otimes_{\Omega^0(M)}\Gamma(F),\qtq{and}
    \Gamma(E^\d) = \Gamma(E)^\d
  \]
\end{theorem}


Throughout differential geometry, it's useful to twist the bundle of differential forms over a manifold $M$ by some vector bundle $E\to M$. Such twisted forms are known as vector-valued differential forms.

\begin{definition}
  The bundle of \defn{$E$-valued differential $q$-forms}[vector-valued differential forms] on $M$ is $E\otimes \alt^q \T^*M$. An $E$-valued differential $q$-form is a section of this bundle, the vector space of which we denote by
  \[
    \Omega^q(M; E) = \Gamma\left(E\otimes \alt^q \T^\d M\right).
  \]
\end{definition}

If we let $E=\underline{\R}$ be the trivial line bundle, then $\Omega^q(M; \underline{\R})=\Omega^q(M)$ so the notion of vector-valued differential forms is a generalization of the ordinary notion of a differential form. As a convention, if $E=\underline{V}$ is a trivial bundle we omit the $\underline{\phantom{V}}$ and write $\Omega^q(M; V)=\Omega^q(M; \underline{V})$ .

More generally, there is a natural isomorphism
\[
  \lkxfunc{}{\Omega^p(M)\otimes_{\Omega^0(M)} \Gamma(E)}{\Omega^p(M; E).}
\]
When $E=\underline{V}$ is a trivial vector bundle, this isomorphism gives \[\Omega^p(M;V)\lkxto \Omega^p(M)\otimes_{\Omega^0(M)}\underline{V}\lkxto \Omega^p(M)\otimes V.\]
In such cases, it's common to write forms $\theta\in \Omega^p(M; E)$ as $\theta =  \theta^i e_i$ where $\{e_i\}$ is some basis for $V$ and $\theta^i\in \Omega^p(M)$. Thus, vector-valued forms can actually be thought of as vectors of ordinary forms.

Given two vector bundles $E_1,E_2 \to M$, and bundle map $f : E_1\to E_2$

An essential operation for differential forms is the wedge product, which acts as a graded-commutative product

% Now suppose $E_1, E_2, E_3 \to M$ are vector bundles, and $B : E_1\otimes E_2 \to E_3$ is a bilinear form. $B$ then extends to a bilinear form
% \[
%   \lkxfunc{B}{\Omega^p(M; E_1)\otimes \Omega^q(M; E_2)}{\Omega^n(M; E_3)}{(\alpha\cdot \xi)\otimes (\beta\cdot \eta)}{(\alpha\wedge\beta)\cdot B(\xi, \eta)}
% \]
% \todo{check this.}
% For instance, suppose $E_1=E_2=E_3=\underline{M_n\R}$ is the constant vector bundle of $n\times n$ matrices, and let $B(A,B) = AB$. Then


\section{Fiber Bundles}

Recall that a fiber bundle simply consists of a map $\pi : P \to X$ -- being a fiber bundle is a condition, not data. Taking the kernel of the differential $\pi_*$ of this fiber bundle map gives us a sub-bundle of $\T P$:
\begin{definition}
  The \defn{vertical tangent bundle} is the bundle $\T P /X=\ker \pi_*$.
\end{definition}
Pulling back $\T X$ to $P$, we have a short exact sequence of vector bundles over $P$:
\begin{equation}\label{eq:ses_fiber_bundle}
    0 \lkxto \T P/X \lkxto \T P \lkxto[\pi_*] \pi^* \T X\lkxto 0.
\end{equation}
The bundle $\T P/X$ consists of \defn{vertical tangent vectors}[vertical tangent vector] in $\T P$, and this notion comes canonically from the data of the fiber bundle. Objects which ``vanish'' or are complementary to vertical tangent vectors are referred to as horizontal. While this notion of being horizontal is canonical, actual instances of horizontal objects are usually not canonical, and introduce additional data to a bundle.

\begin{definition}\label{defn:horizontal_differential_form}
  A differential $q$-form $\omega\in \Omega^q(P)$ is said to be a \defn{horizontal differential form} if it vanishes on vertical tangent vectors $\xi\in \T P / X$. Equivalently, $\ker \omega \subset \T P/X$. We denote the space of horizontal differential $q$-forms by $\Omega^q_=(P)$.
\end{definition}

There is another interpretation of horizontal differential forms. Recall that a differential $q$-form is a section of the $q$-th alternating power $\alt^q \T^\d P$ of the cotangent bundle. The surjective differential $\pi_* : \T P \to \pi^* \T X$ gives an injective pullback map $\pi^* \T^\d X \to \T^\d P$ and thus an inclusion of bundles
\[
  \lkxfunc{}{\alt^q \pi^* \T^\d X}{\alt^q \T^\d P}
\]
The image of this inclusion is exactly the bundle of horizontal $q$-forms, so there is an isomorphism
\[
    \lkxfunc{}{\Gamma(\alt^q \pi^* \T^\d X)}{\Omega^q_=(P).}
\]
Similarly, there is an isomorphism $\Gamma(\alt^q \pi^* \T^\d X)\cong \Gamma(\alt^q \T^\d X) = \Omega^q(X)$. (Note that the left hand side consists of sections over $P$ and the right hand side consists of sections over $X$) The composition of these two isomorphisms $\Omega^q(X) \cong \Gamma(\alt^q\pi^*\T^\d X) \cong \Omega^q_=(P)$ is exactly the pullback $\pi^*$:

\begin{proposition}[Fiber Bundle Descent Conditions]\label{prop:fiber_descent_conditions}
  The pullback is an isomorphism
  \[
    \lkxfunc{\pi^*}{\Omega^q(X)}{\Omega^q_=(P).}
  \]
\end{proposition}

In other words, a differential form defined on $P$ \emph{descends} to a form on $X$ -- i.e. is the pullback of a form on $X$ -- if and only if it is horizontal. As we introduce more structure on the bundle $\pi$, these descent conditions will get more complex.

What happens if we consider vector-valued differential forms? \todo{what }
In our case of a fiber bundle $P\to X$, suppose $E\to P$ is some vector bundle. \cref{defn:horizontal_differential_form} of a horizontal differential form still makes sense here, and we use the notation $\Omega^q_=(P; E)$ to refer to the space of horizontal $E$-valued differential forms. As before, we have an inclusion of bundles
\[
    E\otimes \alt^q \pi^* \T^\d X \lkxto E\otimes \alt^q \T^\d P
\]
and consequently an isomorphism of vector spaces
\[
    \Gamma(E\otimes \alt^q \pi^* \T^\d X) \lkxto \Omega^q_=(P; E).
\]
Unlike before, there is no canonical isomorphism $\Gamma(E\otimes \alt^q \pi^* \T^\d X)\cong \Omega^q(X; ?)$ -- for this to even make sense we would need some way form an ``associated bundle'' to $E$ over $X$. Without additional constraints or data on $E$ there is no analogue of \cref{prop:fiber_descent_conditions} for vector-valued differential forms in the general case.

\subsection{Horizontal Distributions}

So far, we've only worked with the data of the fiber bundle, which in particular gives rise to a canonical notion of a vertical tangent vector in $\T P$. The notion of a \defn{horizontal tangent vector} on the other hand is not canonical, and requires the extra data of a splitting of \cref{eq:ses_fiber_bundle}.
\begin{definition}
  A \defn{horizontal distribution}\footnote{This is also known as an \defn{Ehresmann connection}.}
  is a splitting $\sigma : \pi^*\T X \to \T P$ of the sequence \cref{eq:ses_fiber_bundle}. Equivalently, a horizontal distribution is a distribution $H\subset \T P$ complementary to $\T P/X$.
\end{definition}
Let $\mathcal{A}_\pi$ be the set of all horizontal distributions. 
\begin{remark}\label{rmk:hor_distr_affine}
  By \cref{prop:splittings_affine}, the set $\mathcal{A}_\pi$ is an affine space over 
\[
  \Hom(\pi^* \T X, \T P /X) = \Gamma(\T P/X \otimes \pi^* \T^\d X) = \Omega^1_=(P; \T P /X).
\]
This global parallelism on the space of horizontal distributions \todo{blah blah blah}
\end{remark}

\todo{parallel transport}

\subsection{Curvature}

One of the first things we can ask about a distribution is whether or not it is integrable -- if the total space $M$ can be foliated by submanifolds whose tangent subbundles are the distribution.

\begin{theorem}[Frobenius Theorem]\label{thm:frobenius}
  Let $E\subset \T M$ be a distribution. Then $E$ is integrable if and only if $\mathfrak{X}(E)\subset \mathfrak{X}(M)$ is a Lie subalgebra -- i.e. the commutator or bracket of two vector fields in $E$ lies in $E$.
\end{theorem}

The obstruction to integrability is the failure of $[\xi, \eta]$ to lie in $E$ for $\xi,\eta\in \mathfrak{X}(E)$. As with many constructions in differential geometry, we can package this data into a single tensorial object.

\begin{definition}
  The \defn{Frobenius tensor} of a distribution $E\subset \T M$ is given by
  \[
    \lkxfunc{\phi_E}{E\otimes E}{\T M / E}{\xi\otimes \eta}{[\xi, \eta]\mod E}
  \]
\end{definition}

Since it is skew-symmetric, the Frobenius tensor may be regarded as a section
\[
    \phi_E \in \Gamma(\T M / E\otimes \alt^2 E^\d).
\]
In the case of a fiber bundle, any horizontal distribution $H\in \mathcal{A}_\pi$ gives us a Frobenius tensor 
\[
  \begin{aligned}
    \phi_H \in \Gamma(\T P / H\otimes \alt^2 H^\d) 
    &= \Gamma(\T P / X \otimes \alt^2 H^\d)\\ 
    &= \Gamma(\T P / X \otimes \alt^2 (\pi^* \T^\d X))\\
    &= \Omega^2_=(P; \T P/X).
  \end{aligned}
\]
Up to a sign, this tensor can be thought of as the curvature of the fiber bundle with horizontal distribution. We flip the sign for compatibility with later formulas for the curvature.
\begin{definition}
 The \defn{curvature} of a horizontal distribution is $-\phi_H \in \Omega^2_=(P; \T P/X)$. We usually denote the curvature form by $\Omega$.
\end{definition}

\section{Principal Bundles}

Throughout, let $G$ be a Lie group.

\begin{definition}
  A \defn{principal $G$-bundle} $\pi : P \to X$ is a fiber bundle with a right $G$-torsor structure on each of the fibers.
\end{definition}

\pagebreak
\section{Differential Forms on Principal Bundles}

The kernel of $\pi^*$ consists of \defn{vertical vector fields}.

If we let $E=\underline{\R}$ be the trivial real line bundle, there is a canonical isomorphism $\Omega^q(X; \underline{\R}) = \Omega^q(X)$. This notion of a vector-valued differential form is thus a strict generalization of the standard notion of a differential form.

\todo{explain the adjoint bundle associated to the adjoint representation.}
For a principal $G$-bundle $\pi : P \to X$ and a vector bundle $E\to X$, any linear representation $\rho : G \to \Aut(E)$ gives us an associated vector bundle $E\times_G P \to X$.

\begin{definition}
  A differential form $\omega\in \Omega^q(P; E)$ is said to be a \defn{horizontal form} if $\ker\pi_* \subset \ker \omega$. We denote the subspace of horizontal forms as $\Omega^q_=(P; E) \subset \Omega^q(P; E)$.
\end{definition}

Now suppose $\rho : G \to \Aut(E)$ is a linear representation. 
\begin{definition}
  A differential form $\omega\in \Omega^q(P; E)$ is said to be \defn{$G$-equivariant} (under $\rho$) if
  \[
    R_g^*\omega = \rho(g)^{-1}\omega.
  \]
  We denote the subspace of $G$-equivariant forms as $\Omega^q(P; E)^G_\rho$ or simply $\Omega^q(P; E)^G$ when the representation is clear.
\end{definition}

\begin{proposition}[Descent Conditions]\label{prop:principal_descent_conditions}
  Pullback of differential forms gives an isomorphism
  \[
    \lkxfunc{\pi_*}{\Omega^q(X; E\times_{G} P)}{\Omega^q_=(P; E)^G_\rho \subset \Omega^q(P; E).}
  \]
\end{proposition}

This proposition provides necessary and sufficient conditions for a differential form defined on $P$ to descend to a form on $X$ -- namely it must be horizontal and $G$-equivariant.

Note that when $E$ is the trivial vector bundle with trivial representation $\rho$, this simplifies to:
\begin{corollary}
  The pullback map gives an isomorphism $\pi_* : \Omega^q(X) \to \Omega^q_=(P)\subset \Omega^q(P)$.
\end{corollary}

\section{Connections}


For us, the prototypical principal $G$-bundle is the frame bundle of a manifold.
\begin{definition}
  Let $X$ be a smooth $n$-manifold. The \defn{frame bundle} of $X$ is the fiber product
  \[
    \B X = \underbrace{\T X\times_X \cdots \times_X \T X}_{n\textrm{ times}}. 
  \]
\end{definition}
At a point $x\in X$, this frame bundle has fibers
\[
    \B_x X = \End(\R^n, \T_x X)
\]
consisting of bases for the tangent space $\T_x X$.

Let $\pi : P \to X$ be a principal $G$-bundle.

\begin{definition}
  A \defn{connection} on $\pi$ is a $G$-invariant horizontal distribution $H\subset \T P$.
\end{definition}

\begin{proposition}\label{prop:connection_definition}
  There is a bijective correspondence:
  \[
    \left\{\parbox{11em} {
        1-forms $\Theta\in \Omega^1(P; \mathfrak{g})^G_\Ad$ which  
        satisfy $\iota_x^*\Theta = \theta_\MC^{P_x}$
    }\right\}
    \lkxto[\ker]
    \left\{\parbox{10em}{
        $G$-invariant horizontal distributions $H\subset \T P$
      }
    \right\}
  \]
\end{proposition}


\begin{proof}
\end{proof}

\begin{definition}
  A \defn{connection} on $\pi$ is a member of either side of the correspondence described in \cref{prop:connection_definition}. To disambiguate, we refer to a $1$-form $\Theta$ as a \defn{connection form} and to a distribution $H$ simply as a \defn{horizontal distribution}.
\end{definition}

% \begin{definition}
%   A \defn{connection} on $\pi$ is a $1$-form $\Theta\in \Omega^1(P; \mathfrak{g})$ satisfying
%   \[
%     \iota_x^*\Theta = \theta_\MC(P_x)
%     \quad
%     \textrm{and}
%     \quad 
%     R_g^* \Theta = \Ad_{g^{-1}}\Theta.
%   \]
% \end{definition}

\section{Linear Connections}
Let $X$ be a smooth manifold and let $\varpi : \B X \to X$ be its principal $\GL_n$-bundle of frames. There is a canonical identity representation $\GL_n \to \Aut(\R^n)$. With this representation, the associated bundle $\R^n\times_{\GL_n} \B X$ can be identified with $\T X$.

\begin{proposition}
  There are bijective correspondences:
\[
  \Omega^1_=(\B X; \R^n)^{\GL_n}_\id
  \lkxto[\textrm{descent}]
  \Omega^1(X; \T X)
  \lkxto
  \Bun_X(\T X, \T X).
\]
\end{proposition}

\begin{proof}
  The first map is a bijection by \cref{prop:principal_descent_conditions}. For the second map, note that $\Omega^1(X; \T X)$ by definition is the space of sections of the bundle $\T X\otimes \T^\d X$ which is canonically isomorphic to the bundle $\Hom(\T X, \T X)$. Sections of this bundle are exactly bundle maps $\T X \to \T X$.
\end{proof}

\begin{definition}
  The \defn{soldering form} $\theta\in \Omega^1(\B X; \R^n)$ is the form associated to the identity map $\T X\to \T X$ under these correspondences.
\end{definition}

We can interpret the action of the soldering form as follows: Suppose $b\in \B_x X$ is a frame at a point $x\in X$. This consists of an isomorphism $b : \R^n \to \T_x X$.
Now for any tangent vector $\xi\in \T_b \B X$ to the frame bundle, there is a pullback vector $\pi_* \xi\in \T_x X$ in the base. 
The soldering form then satisfies
\[
    \pi_* \xi = b\circ \theta(\xi),
\]
so in other words $\theta(\xi)$ expresses the projection of a tangent vector to a frame in the coordinates of the frame. This justifies the name ``soldering form'', since in some sense $\theta$ solders the frame bundle to the base.

Armed with this canonical form, let's now turn our attention to connections in this context.

\begin{definition}
  A \defn{linear connection} on $X$ is a connection on $\varpi : \B X \to X$.
\end{definition}

\subsection{Local Coordinates}

Let $e_1,\ldots, e_n$ be the standard basis for $\R^n$, and $e^1,\ldots,e^n$ the corresponding basis for the dual space $(\R^n)^\d$. Using these bases, let $E_i^j = e_i\otimes e^j \in \R^n\otimes(\R^n)^\d \cong M_n(\R) \cong \gl_n$ be a basis for the space of matrices. Finally, let $\widehat{E}_i^j\in \mathfrak{X}(\mathscr{B}X)$ be the corresponding vertical vector fields.

For any basis $b\in \B_x X$ and tangent vector $\xi\in T_b \B X$ we have
\[
  \pi_*\xi =  b\circ \theta(\xi) = \theta^i(\xi) b(e_i)\quad\textrm{where}\quad \theta = \theta^i e_i.
\]
This justifies the naming of $\theta$ as the ``soldering form'' -- it solders the frame bundle to the base.

% \[\iota_{\widehat{E}_i^j} \theta = 0\]

\section{Chern-Weil Theory}

\section{Characteristic Classes}\label{sec:characteristic classes}

A fundamental piece of data given to us by a smooth structure on $M$ is the tangent bundle $\T M \to M$, and by extension the frame bundle $\B M \to M$.
These bundles allow us to do some differential geometry on $M$. In general, given a vector bundle $E\to M$, picking any linear connection $\Theta\in \Omega^1(\B E; \gl_n)$ with curvature form $F_\Theta \in \Omega^2(\B E; \gl_n)$, gives the Chern-Weil homomorphism
\[
  \lkxfunc{}{(\Sym^\bullet \gl_n^*)^{\GL_n}}{\H^\bullet(M; \R)}
  {h}{h(F_\Theta\wedge\cdots\wedge F_\Theta)}
\]
which sends a $\GL_n$-invariant polynomial $h$ on $\gl_n$ to its evaluation on the curvature form $F_\Theta$, considered as a de Rham cohomology class. A basic result of Chern-Weil theory proves that this homomorphism is independent (up to cohomology of course) of the original choice of connection. The image of the Chern-Weil homomorphism forms the ring of characteristic forms -- an essential ingredient to the study of smooth manifolds. 

There are several bases for this ring of characteristic forms. For instance, the \defn{Pontryagin classes} are given by
\[
  p_k(E) = \left[f_{2k}\left(\frac{i}{2\pi}F_\Theta\right)\right]\quad\in\H^{4k}(M; \R)
\]
where $f_{2k}(\lambda)$ are the coefficients of even degree in the characteristic polynomial $f(\lambda) = \det(\lambda I+F_\Theta)$. Here, the normalization factor of $i/2\pi$ ensures that Pontryagin classes integrate to integers. It is often useful to consider the \defn{total Pontryagin class} as associated to the characteristic polynomial
\[
  p(E) = \sum_k p_k(E) = \det\left(I + \frac{i}{2\pi} F_\Theta\right) \quad\in \H^\bullet(M; \R).
\]
\begin{proposition}[Whitney Product Formula] If $E_1$ and $E_2$ are vector bundles over $M$, then we have $p(E_1\oplus E_2) = p(E_1)\cdot p(E_2)$.
\end{proposition}

\begin{theorem}[Chern-Simons]\label{thm:chern-simons}
\end{theorem}

\section{Hirzebruch Signature Theorem}

\begin{theorem}{Hirzebruch}\label{thm:hirzebruch_signature}
\end{theorem}

\section{Multiplicative Sequences}\label{sec:multiplicative_sequences}
