\chapter*{Prelude}
\addcontentsline{toc}{chapter}{Prelude} 

% Epigraph
\begin{flushleft}
	\slshape{A mathematician is a blind man in }\\
	\slshape{a dark room looking for a black cat}\\
	\slshape{which isn’t there.}\\
	\rule[0pt]{15em}{0.5pt}\\
	\slshape{-- Unknown}
\end{flushleft}

\vspace{2em}

% In the summer of 2024, I went with some friends on a mountaineering trip up Banner peak, a picturesque mountain in the eastern Sierra Nevada range of California. 
% As we descended the glacier towards our base camp, ice axes in hand, I 
So how \emph{do} we study objects which we not only can't see, can't even hope to visualize in their entirety? As with any hopeless endeavor, 

There are two levels to understanding.
\begin{enumerate}
	\item First, we
\end{enumerate}

In 5th grade science class,
