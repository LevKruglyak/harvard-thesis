\chapter{Geometric Invariants}\label{chap:invariants}

\begin{epigraph}{15em}{Unknown}
	A mathematician is a blind man in \\
	a dark room looking for a black cat \\
	which isn’t there.
\end{epigraph}

In this chapter, we explore a rich theory of invariants for homotopy spheres arising from a class of theorems known as index theorems. 

\pagebreak
\section{Characteristic Classes}
The study of characteristic classes began with the work of Hassler Whitney and Eduard Stiefel in the mid 1930s. Since then, the fundamental idea has remained unchanged -- a vector bundle on a manifold determines certain ``characteristic'' classes in the homology or cohomology of the base manifold.
By the mid 1940s, these ideas were extended by Lev Pontryagin and Shing-Shen Chern to better capture the geometric data of oriented real and complex vector bundles respectively. In the following decades, characteristic classes quickly joined the toolboxes of mathematicians from a wide range of disciplines, finding connections to prior notions in these fields.
Applications ranged from algebraic topology, differential topology of exotic spheres, complex geometry, index theory, and may others.

We will present a few equivalent formulations of characteristic classes in this thesis, each useful in its own context. These formulations are,
\begin{enumerate}[(a)]
	\item as natural transformations satisfying certain axioms,
	\item as generators of the cohomology ring of a classifying space,
	\item by the Chern-Weil homomorphism as images of invariant polynomials,
	\item as obstructions to problems in homotopy theory.
\end{enumerate}
We begin with a somewhat unconventional blend of the (a) and (b) perspective, closing with a brief discussion of (c) in \cref{sec:chern-weil-theory}. The obstruction theory perspective is immensely useful for classifying homotopy spheres, however it is not immediately relevant to the topic of geometric invariants so we will discuss it in \cref{sec:obstruction-theory}. For a standard introduction which avoids some of the opinionated choices made in this presentation, see \cite{milnorstasheff1974}.
At this stage, we assume a basic knowledge of vector bundles, structure groups, and classifying spaces. For a brief introduction to these topics, see \cref{chap:vector-bundles}.

\subsection{A Universal Perspective on Characteristic Classes}\label{sec:axiomatic-characteristic-classes}

Throughout this section, a cohomology theory will refer either to singular cohomology with some PID coefficient ring such as $\Z$, $\Q$, $\Z/2$, or $\Z[1/2]$, or to de Rham cohomology, with coefficient ring $\R$. The Poincar\'e dual homology theories are then either singular homology with coefficients or compactly supported de Rham cohomology. We use $R$ to denote the coefficient ring.

\begin{definition}\label{defn:characteristic-class}
	A \defn{characteristic class} $c$, valued in a cohomology theory $h$, is a natural transformation of contravariant functors
	\[
		\lkxfunc{c}{\Vect_G}{h^\bullet,}
	\]
	given a structure group $G$. We assume here that all vector bundles are real, since letting $G=\U_n$ recovers complex structure. Denote the set of characteristic classes by $\Class_G^R = [\Vect_G , h^\bullet]$.
\end{definition}

Here, $h^\bullet : \Top \to \Rng$ is the functor sending a space to its  cohomology ring, and $\Vect_G : \Top \to \Set$ is a functor sending a space to the set of isomorphism classes of vector bundles over the space with structure group $G$. We assume that the natural transformation $c$ forgets the ring structure of cohomology when mapping from the set of isomorphism classes of vector bundles.

For any vector bundle $\mathcal{E} : E \to B$, a cohomology class assigns some cohomology class $c(\mathcal{E})\in h^\bullet(B)$ which is ``characteristic'' of the bundle $\mathcal{E}$.
Part of this ``characteristicness'' is that the assignment must be done in a natural way. Given bundles $\mathcal{E}_1$ and $\mathcal{E}_2$ over bases $B_1$ and $B_2$, whenever a map $f : B_1 \to B_2$ is covered by a bundle map $\mathcal{E}_1 \to \mathcal{E}_2$, we have $f^* c(\mathcal{E}_2) = c(\mathcal{E}_1)$. In particular, isomorphic vector bundles over the same base space $B$ are sent to the same cohomology class. It is not hard to see why such a notion would be useful.

\begin{convention*}
	Since every smooth manifold $M$ comes with a canonical vector bundle -- the tangent bundle -- it is common to use the notation $c(M)$ to refer to $c(\TT M)$.
\end{convention*}

\begin{remark}
	In practice, many sources define characteristic classes first as a set of homogeneous classes $c_i\in h^i(\mathcal{E})$ and then refer to the \defn{total characteristic class} $c(\mathcal{E})=\sum_i c_i(\mathcal{E})$. In this thesis, we take a somewhat unorthodox convention and assume that all characteristic classes are inhomogeneous unless otherwise stated.
\end{remark}

For any two characteristic classes $c_1$ and $c_2$, there is a natural notion of their sum $c_1+c_2$, product $c_1\smile c_2$, and scalar product $r\cdot c_1$ for any $r\in R$, simply by applying these operations in $h^\bullet(-)$. This gives the set $\Class_G^R$ of characteristic classes a ring structure.

On a base $B$, the cohomology ring $h^\bullet(B)$ is a graded ring
\[
	h^\bullet(M) = \bigoplus_{k\geq 0} h^k(B),
\]
with the cup product turning $h^\bullet(B)$ into a graded-commutative ring.
With the grading on cohomology, we can give the set of characteristic classes $\Class_G^h$ a grading
\[
	\Class_G^h[k] = [\Vect_G, h^k]
\]
where $h^k: \Top \to \Grp$ denotes the functor sending a space to its $k$-th cohomology group.

\begin{definition}
	A characteristic class $c$ is said to be \defn{homogeneous of degree $k$}[homogeneous characteristic class] if it lies in the graded component $[\Vect_G, h^k]$. We denote this degree by $|c|$.
\end{definition}

By naturality, characteristic classes must preserve cohomological degree since pull-backs preserve degree. We can thus write any characteristic class $c$ as an infinite sum of homogeneous characteristic classes $c=c_0+c_1+c_2+\cdots$, i.e. where each $c_k$ is a natural transformation from $\Vect_G$ to $h^k$.
In other words, the ring of characteristic classes $\Class_G^h$ admits the structure of the completion of a graded-commutative ring:
\[
	\Class_G^R \cong \prod_{k \geq 0} [\Vect_G, h^k].
\]
\begin{remark}
	Recall from the conventions that the completion of a graded ring $A=\bigoplus_{i\in I} A_k$ is the direct product $\widehat{A}=\prod_{i\in I} A_i$.
\end{remark}

The ring structure on $\Class_G^R$ has a simple interpretation in terms of classifying spaces. By the work of \cref{sec:classifying-spaces}, every vector bundle $\mathcal{E} : E \to B$ with structure group $G$ is the pullback of a universal bundle over the classifying space $\BB G$ by a map $f_{\mathcal{E}} : B \to \BB G$.
Every homogeneous characteristic class $c$ of degree $k$ thus determines a universal cohomology class $u\in h^k(\BB G)$ by applying $c$ to the universal bundle over $\BB G$. By naturality, $c(\mathcal{E})$ would be the pullback $f^*u$ of the universal class by the classifying map $f$.

Conversely, any cohomology class in $h^k(\BB G)$ gives a homogeneous characteristic class which sends
\[
	\lkxfunc{c}{\Vect_G(B)}{h^k(B)}{\mathcal{E}}{f_{\mathcal{E}}^*u.}
\]
Accounting for infinite sequences, we get the following theorem:

\begin{theorem}\label{thm:universal-characteristic-classes}
	There is a ring isomorphism $\Class^R_G\cong \widehat{h^\bullet}(\BB G)$
	where $\widehat{h^\bullet}$ denotes the completion of the cohomology ring.
\end{theorem}
\begin{proof}
	\todo{yoneda lemma.}
\end{proof}

This is the universal perspective on characteristic classes -- instead of defining natural transformations between vector bundles and cohomology theories, we can work in the cohomology ring of a single classifying space.

\subsection{General Axioms for Characteristic Classes}

Next, let us look at some common axioms for characteristic classes and investigate their implications. This simplifies many of the following constructions of the various species of characteristic class.

\begin{definition}
	A characteristic class $c$ is said to be \defn{multiplicative}[multiplicative characteristic class] if \begin{equation}\label{eq:whitney-product}
		c(\mathcal{E}_1\oplus \mathcal{E}_2)=c(\mathcal{E}_1)\smile c(\mathcal{E}_2)
	\end{equation}
	whenever the bundles $\mathcal{E}_1$ and $\mathcal{E}_2$ have the same base space.
\end{definition}

Equations of the form \cref{eq:whitney-product} are known as a \defn{Whitney product formula}. The homogeneous version of this equation which commonly appears is
\[
	c_n(\mathcal{E}_1\oplus \mathcal{E}_2) = \sum_{p+q=n}c_p(\mathcal{E}_1)\smile c_q(\mathcal{E}_2).
\]

\begin{remark}
	A multiplicative cohomology class is also multiplicative with respect to manifolds. Given manifolds $M_1$ and $M_2$ and a multiplicative characteristic class $c$, we have
	\[
		\begin{aligned}
			c(M_1\times M_2) & = c(\T(M_1\times M_2))      \\
			                 & = c(\T M_1\oplus \T M_2)    \\
			                 & = c(\T M_1)\smile c(\T M_2) \\
			                 & = c(M_1)\smile c(M_2).
		\end{aligned}
	\]
\end{remark}

\begin{remark}
	When the base space $B$ has finite type (as in the case of manifolds), the \defn{degree}[degree of a cohomology class] of a characteristic class $c(\mathcal{E})$ evaluated at a vector bundle $\mathcal{E} : E \to B$ is the maximal degree of $c(\mathcal{E})\in h^\bullet(B)$, and denoted $|c(\mathcal{E})|$.
\end{remark}

\begin{definition}
	A characteristic class $c$ is said to be \defn{rank-normalized} if $c_0=1$ and for any bundle $\mathcal{E}$ we have $|c(\mathcal{E})|\leq \rank(\mathcal{E})$. Note that the degree $|c(\mathcal{E})|$ is the maximal degree of a homogeneous component of $c(\mathcal{E})$.
\end{definition}

The first example of a characteristic class of an bundle is useful, but not very interesting.

\begin{corollary}
	For any rank-normalized characteristic class $c$ and rank $k$ trivial bundle $\underline{\R}^k$, we have $c(\underline{\R}^k)=1$.
\end{corollary}
\begin{proof}
	A trivial bundle $\underline{\R}^k$ is the pullback of the constant bundle $\R^k\to *$ over a point. This bundle has characteristic class $1$ by the rank normalization condition, and so by naturality pulls back to $1$.
\end{proof}

\begin{corollary}\label{cor:mobius-characteristic-2-torsion}
	If $c$ is a rank-normalized multiplicative characteristic class and $\gamma_1^1$ is the canonical M\"obius bundle over $\RP^1\cong S^1$, then $c(\gamma_1^1)$ is $2$-torsion.
\end{corollary}
\begin{proof}
	Since $\gamma_1^1\oplus \gamma^1_1$ is a trivial bundle (see \cref{fig:trivial-mobius-bundle-sum}), it follows that
	\[
		(1+c_1(\gamma_1^1))(1+c_1(\gamma_1^1)) = 1+2c_1(\gamma_1^1)+c_1^2(\gamma_1^1)=1.
	\]
	However, $H^2(S^1)\cong 0$, so we have $2c_1(\gamma_1^1)=0$.
\end{proof}

\begin{figure}[ht]
	\centering
	\import{diagrams}{placeholder-small.pdf_tex}
	\caption{A Whitney sum of orthogonal M\"obius bundles.}\label{fig:trivial-mobius-bundle-sum}
\end{figure}

\begin{corollary}
	For any multiplicative and rank-normalized characteristic class, we have $c(\mathcal{E}\oplus \underline{\R}^k)=c(\mathcal{E})\smile c(\underline{\R}^k)= c(\mathcal{E})$.
\end{corollary}

\begin{definition}
	A characteristic class $c$ is said to be \defn{stable}[stable characteristic class] if $c(\mathcal{E}\oplus \underline{\R}^k) = c(\mathcal{E})$ for any $k$. A stable characteristic class only depends on the stable isomorphism type of the vector bundle. Every multiplicative and rank-normalized characteristic class is stable.
\end{definition}

\begin{remark}
	Note that every characteristic class with stable structure group, e.g. $G$ is either $\O, \SO,$ or $\U$, is stable. To compute the ring of stable characteristic classes, we thus would need to compute the cohomology of $\BO$, $\BSO$, or $\BU$.

	For any $n>0$, we have inclusions $G_n \to G_{n+1} \to \cdots \to G$ which correspond to the addition of trivial bundles. These give maps $\BB G_n \to \BB G$, which gives a map on cohomology
	\[
		h^\bullet(\BB G) \lkxto h^\bullet(\BB G_n).
	\]
	Every stable characteristic class in $h^\bullet(\BB G_n)$ must be in the image of this map.
\end{remark}

Next, we discuss inversion in the ring of characteristic classes.
In a general completion of a graded ring $\widehat{A}=\prod_{k\geq 0}A_k$, the multiplication of two elements
\[
	(x_0+x_1+x_2+\cdots)\cdot (y_0+y_1+y_2+\cdots) = (z_0 + z_1 + z_2+\cdots)
\]
can be expanded in homogeneous components as $z_k=\sum_{p+q=k} x_py_q$, i.e.
\begin{equation}\label{eq:formal-product}
	\begin{aligned}
		z_0 & = x_0y_0,                 \\
		z_1 & = x_0y_1 + x_1y_0,        \\
		z_2 & = x_0y_2 + x_1y_1+x_2y_0, \\
		    & \;\;\vdots
	\end{aligned}
\end{equation}
To invert the series $(x_0+ x_1+\ldots)\in \widehat{A}$, we would like to find some $(y_0+y_1+\cdots)$ such that the resulting product $(z_0+z_1+\cdots)=1$. Setting $z_0=1$, $z_k=1$ for $k>0$ and solving \cref{eq:formal-product} for $y_k$, we get the recursive formula
\begin{equation}\label{eq:formal-inversion}
	y_k = \begin{cases}x_0^{-1}                                        & k=0,   \\
             -x_0^{-1}(x_1y_{k-1}+x_2y_{k-2}+\cdots +x_ky_0) & k > 0.
	\end{cases}
\end{equation}
A crucial corollary of \cref{eq:formal-inversion} is that a series $(x_0+x_1+\cdots)$ can be inverted if and only if $x_0\in A_0^\times$ is a unit. This is generally not true in the non-completed case. Take for instance in the ring of formal power series $R\fps{t}$, the completion of the polynomial ring $R[t]$. There, we have the identity
\[
	(1-t)\cdot (1+t+t^2+t^3+\cdots) = 1,
\]
which is a rearranged form of the classic equation for the infinite series of a geometric series. Even though $(1-t)$ has monic leading coefficient, it has no inverse in $R[t]$.

\begin{proposition}\label{prop:formal-inverse}
	If $\widehat{A}$ is a completion of a graded ring, then $\widehat{A}^\times = \widehat{A} \cap A_0^\times$.
\end{proposition}

This observation has a useful consequence in the context of characteristic classes. Suppose $c$ is a characteristic class with $c_0\in R^\times$. By \cref{prop:formal-inverse}, there is a characteristic class $c^{-1}$ which is the multiplicative inverse of $c$, so that $c\smile c^{-1}=1$ is the trivial characteristic class. If the characteristic class $c$ is multiplicative, then the Whitney product formula gives
\[
	c(\mathcal{E}_1\oplus\mathcal{E}_2)	 = c(\mathcal{E}_1)\smile c(\mathcal{E}_2)
	\quad\implies\quad
	c(\mathcal{E}_1) = c(\mathcal{E}_1\oplus \mathcal{E}_2) \smile c^{-1}(\mathcal{E}_2).
\]
If $\mathcal{E}_1=\TT M$ is the tangent bundle of a manifold $M^k$ and $\mathcal{E}_2=\TT\R^{n}/M$ is the normal bundle of an embedding $M^k\to \R^{n}$, then $\mathcal{E}_1\oplus \mathcal{E}_2$ is trivial since it is the restriction of the trivial tangent bundle of $\R^n$. In the case that $c$ is rank-normalized, we get:
\begin{theorem}[Whitney Duality]\label{thm:whitney-duality}
	If $M^k$ is a submanifold of $\R^n$, and $c$ is a rank-normalized multiplicative characteristic class, we have
	\[
		c(\TT M) = c^{-1}(\TT \R^n/M).
	\]
\end{theorem}

\begin{corollary}
	If $c$ is a rank-normalized multiplicative characteristic class, then $c(S^n)=1$.
\end{corollary}
\begin{proof}
	This follows since $S^n$ embeds into $\R^{n+1}$ with a trivial normal line bundle.
\end{proof}

\begin{remark}
	As it turns out, \emph{any} homotopy sphere $M$ has $c(M)=1$ for a rank-normalized multiplicative characteristic class because the tangent bundle of a homotopy sphere is stably isomorphic to the trivial bundle. We will prove this in \cref{thm:homotopy-sphere-stably-parallelizable} using some hard theorems in homotopy theory.
\end{remark}

\subsection{Characteristic Numbers}

If we want to compare vector bundles on different spaces, we need a common context in which to compare their characteristic classes since the cohomology rings of the underlying spaces might not be canonically isomorphism. When the base space is a compact $R$-oriented manifold, the Poincar\'e duality isomorphism $h^{n-k}(M) \cong h_k(M)$ allows us to ``integrate'' homogeneous top-dimensional cohomology classes $\alpha\in h^{n}(M)$ along a fundamental class $[M]\in h_n(M)$ to get an element $\alpha[M]\in h_0(M)\cong R$ in the coefficient ring of a corresponding homology theory. The coefficient ring $R$ is this context in which we can compare characteristic classes over different bases.

\begin{definition}\label{def:characteristic-number}
	Given any characteristic class $c$ and closed $n$-dimensional manifold $M$ the \defn{characteristic number} of $M$ is $c[M] = c_n(\TT M)[M]$.
\end{definition}

Note that we must take the degree $n$ homogeneous component of $c$ to apply Poincar\'e duality. Generally, if $\Man_R$ denotes the set of closed $R$-oriented manifolds, any characteristic class $c$ can be interpreted as a map
\begin{equation}\label{eq:characteristic-number}
	\lkxfunc{c}{\Man_R}{R}{M}{c[M].}
\end{equation}
Under some extensions of \cref{def:characteristic-number} to allow for disconnected manifolds, \cref{eq:characteristic-number} is additive with respect to disjoint unions.
\begin{proposition}
	If $c$ is a multiplicative characteristic class, then $c[M_1\times M_2]=c[M_1]\cdot c[M_2]$.
\end{proposition}
\begin{proof}
	By the multiplicative property and the K\"unneth formula, we have.
	\[
		\begin{aligned}
			c[M_1\times M_2]
			 & = c(\TT (M_1\times M_2))[M_1\times M_2]                      \\
			 & = c(\TT M_1\oplus \TT M_2)[M_1\times M_2]                    \\
			 & = (c(\TT M_1)\smile c(\TT M_2))[M_1]\times [M_2]             \\
			 & = c(\TT M_1)[M_1]\cdot c(\TT M_2)[M_2] = c[M_1]\cdot c[M_2].
		\end{aligned}
	\]
	This completes the proof.
\end{proof}

\begin{remark}
	While $\Man_R$ does not have a ring structure due to lack of an identity element, with the inclusion of a cobordism relation on $\Man_R$ it is possible to interpret multiplicative characteristic classes as ring homomorphisms from a cobordism ring to $R$.
\end{remark}

The set of characteristic numbers of a manifold form a topological fingerprint of the manifold. As we will see in \cref{chap:invariants}, the subtle interplay of their number theoretic properties is one of the main ways to study smooth structure on a manifold.

\subsection{Construction of Characteristic Classes}

We will now begin construction of most fundamental characteristic classes, the Chern class for complex vector bundles. Recall that for complex projective spaces we have the cohomology ring
\[
	\H^\bullet(\CP^n;\Z)\cong \Z[\alpha]/(\alpha^{n+1}
\]
where $\alpha$ is the degree $2$ class corresponding to the Poincar\'e dual $\CP^{n-1}\subset \CP^n$.
In the infinite dimensional limit, we get the purely polynomial rings
\[
		\H^\bullet(\CP^\infty;\Z)\cong \Z[\alpha].
\]
However since $\BU_1=\CP^\infty$, by \cref{thm:universal-characteristic-classes} we have
\begin{equation}\label{eq:cczu1-a}
	\Class_{\U_1}^{\Z}\cong \Z\fps{\alpha}
\end{equation}
The generator $\alpha$ thus corresponds to a homogeneous characteristic classes of degree $2$. We can use this as the inductive base or a general definition.

\begin{definition}
	The \defn{Chern class $\cl$}[Chern class] is the unique characteristic class for complex vector bundles ($G=\U$) in singular cohomology with $\Z$ coefficients satisfying:
	\begin{enumerate}[(a)]
		\item $\cl$ is rank-normalized and multiplicative,\footnote{Technically, we only need that $\cl_0=1$, the rank conditions follows from this and axiom (b).}
		\item $\cl(\OO_{\CP^1}(1))=1+\alpha$.
	\end{enumerate}
	We denote by $\cl_i$ the degree $2i$ homogeneous component of $c$ in $\H^{2i}(-;\Z)$.
\end{definition}

Of course, this is not a constructive definition, and it is not at all clear that a characteristic class satisfying these definitions even exists.
However, being a multiplicative rank-normalized characteristic class with coefficients in a field of characteristic zero, the general results of the previous section hold for the Chern class. The more interesting results follow from axiom (b). By a beautiful theorem known as the splitting principle, it turns out that knowing the Chern classes of line bundles is enough to compute the Chern classes of any complex vector bundle.

\begin{theorem}[Splitting Principle]
	Let $\F$ be $\R$ or $\C$ and suppose $\mathcal{E}^k$ is an $\F$-vector bundle over a space $X$. There exists a space $Y$ with map $f : Y \to X$ such that 
	\begin{enumerate}[(a)]
		\item $f^* : h^\bullet(Y) \to h^\bullet(X)$ is injective, 
		\item there is a bundle isomorphism $f^*\mathcal{E} \cong \mathcal{L}_1\oplus\cdots \oplus \mathcal{L}_k$ for line bundles $\mathcal{L}_i$ over $Y$.
	\end{enumerate}
\end{theorem}
\begin{proof}
	The general idea is as follows. Recall that the frame bundle $\Fr(\mathcal{E})$ is a principal $\GL_n\F$ bundle consisting of all local trivializations of $\mathcal{E}$. Letting $B\subset \GL_n \F$ be the Borel subgroup of upper triangular matrices. By letting $\GL_n$ act on $B\subset \GL_n\F$, we get an associated bundle
	\[
		\Fl(\mathcal{E}) = \Fr(\mathcal{E})\times_{\GL_n\F}{\GL_n\F/B}
	\]
	known as the \defn{flag bundle} of $E$. At each point $p\in X$ of the base, the frame bundle has fiber
	\[
		\Fr_p(\mathcal{E}) = \left\{ b : \F^k \to E_p \mid b \textrm{ is an isomorphism}\right\},
	\]
	or in other words, the set of ordered bases for $E_p$. Modulo the action of upper triangular matrices, we see that the flag bundle has fibers
	\[
		\Fl_p(\mathcal{E}) = \left\{ L_1,\ldots, L_k \textrm{ lines in }E_p \mid \span\{L_1,\ldots, L_k\} = \E_p \right\}.
	\]
	Note that there is a principal $B$-bundle $\Fr(E)\to \Fl(E)$.

	\begin{remark}
		In the complex case, we could reduce the bundle to have structure group $\U_n$, which would force the frame bundle to consist of orthonormal bases. Then to get the flag bundle, we could consider the bundle associated to the action of $\U_n$ on the quotient $\U_n/T$ by its maximal torus of diagonal matrices. This results in a diffeomorphic flag bundle.
	\end{remark}

	We then let $Y=\Fl(E)$ be the total space of the flag bundle, with the injectivity of $f^*$ following from the Leray-Hirsch theorem since it can be shown that $h^\bullet(\GL_n\F/B)$ has no torsion. In fact, the Leray-Hirsch theorem shows the stronger result that we have an isomorphism
	\[
			h^\bullet(X)\otimes h^\bullet(\GL_n \F/B) \cong h^\bullet(Y)
	\]
	with the inclusion $f^*$ simply being the precomposition of this isomorphism with $x\mapsto x\otimes 1$.

	\todo{finish this}

	See Theorem~19.3.9 of \cite{dieck2008algebraic} or page 66 of \cite{hatcher2003ktheory} for a more detailed proof.
\end{proof}

\begin{corollary}
	The Chern class of complex projective space is $\cl(\CP^n)=(1+\alpha)^{n+1}$.
\end{corollary}
\begin{example}
	For instance, we have
	\[
		\begin{aligned}
			\cl(\CP^1) &= 1\\
			\cl(\CP^2) &= 1+2\alpha+\alpha^2\\
			\cl(\CP^3) &= 1+3\alpha + 3\alpha^2+\alpha^3\\
								 &\;\;\vdots
		\end{aligned}
	\]
\end{example}

More generally, if we apply the splitting principle to the universal bundle $\EU_n$ over $\BU_n$, we can show that the flag manifold $\Fl(\EU_n)\cong (\BU_1)^{\times n}$, with the map $(\BU_1)^{\times n} \to \BU_n$ corresponding to the inclusion of the central torus $\U_1^{\times n}\subset \U_n$. However, the cohomology of $(\BU_1)^{\times n}$ is given by
\[
	\H^\bullet((\BU_1)^{\times n}) \cong \H^\bullet(\BU_1)\otimes\cdots\otimes \H^\bullet(\BU_1) \cong \Z[\gamma_1,\ldots,\gamma_n]
\]
with the injective pullback map $\H^\bullet(\BU_n) \to \H^\bullet(\BU_1^{\times n})$ having image $\Z[\gamma_1,\ldots, \gamma_n]^{S_n}$ the symmetric polynomials.

\todo{Talk about $U_n$, its maximal torus, and normalizer group isomorphic to $S_n$. There is some cool geometry happening here under the hood}

\begin{definition}\label{def:chern-roots}
	The \defn{Chern roots} of a complex vector bundle $\mathcal{E}$ over $X$ with classifying map $f : X \to \BU_n$ are given by $\gamma_i(\mathcal{E}) = f^*\gamma_i$. Note that
	\[ \cl(\mathcal{E}) = \prod_{1\leq i \leq n} (1+\gamma_i(\mathcal{E})). \]
\end{definition}

\subsection{Creating Multiplicative Characteristic Classes}

We now present a powerful method for constructing multiplicative characteristic classes, formalized in by Hirzebruch \cite{hirzebruch1966methods}. The crux of this method is the fundamental theorem of symmetric polynomials, a simple but incredibly useful result in algebra.

\begin{theorem}[Fundamental Theorem of Symmetric Polynomials]
	For any ring $R$, let $R[x_1,\ldots, x_n]$ be the polynomial ring in $n$-variables and let $R[x_1,\ldots, x_n]^{S_n}$ be the subring of symmetric polynomials, i.e. those invariant under a reordering of the variables. There is a ring isomorphism
	\begin{equation}\label{eq:symmetric-polynomial-isomorphism}
		R[x_1,\ldots, x_n]^{S_n} \cong R[\sigma_1,\ldots, \sigma_n].
	\end{equation}
	The variables $\sigma_i$ are the \defn{elementary symmetric polynomials}, defined by the relation
	\begin{equation}\label{eq:symmetric-polynomials}
			1+\sigma_1+\sigma_2+\cdots+\sigma_n = (1+x_1)(1+x_2)\cdots (1+x_n)
	\end{equation}
	where we grade $|\sigma_i|=|x_i|=i$.
\end{theorem}

By \cref{eq:symmetric-polynomials}, we can expand these elementary symmetric polynomials in terms of $x_i$:
\[
	\begin{aligned}
		\sigma_1 &= x_1+x_2+\cdots + x_n\\
		\sigma_2 &= x_1x_2 + x_2x_3 + x_1x_3 +\cdots + x_{n-1}x_n\\
						 &\;\;\vdots\\
		\sigma_n &= x_1x_2\cdots x_n
	\end{aligned}
\]
More generally, we have the combinatorial formula
\[
	\sigma_i = \sum_{1\leq p_1<\cdots <p_i\leq n} x_{p_1}\cdots x_{p_i}.
\]
\begin{remark}
	For full notational clarity, we should denote these symmetric polynomials $\sigma_i^{(n)}$ since they depend on the number of variables $n$ in the polynomial ring $R[x_1,\ldots, x_n]$.
\end{remark}

Now letting $x_i=\gamma_i$ be the Chern roots, the symmetric polynomials $\sigma_i$ are exactly the homogeneous components $\cl_i$ of the Chern class. 
\[
	\cl = \prod_{i\geq 1} (1+\alpha)
\]

\begin{example}
	The \defn{Chern character}
\end{example}

\subsection{Characteristic Classes for Real Bundles}

\begin{definition}
	The \defn{Stiefel-Whitney class $w$}[Stiefel-Whitney class] is the unique characteristic class for unoriented vector bundles ($G=
		\O$) in singular cohomology with $\Z/2$ coefficients satisfying:
	\begin{enumerate}[(a)]
		\item $w$ is rank-normalized and multiplicative,
		\item $w(\gamma^1_1)=1+\alpha$.
	\end{enumerate}
	We denote by $w_i$ the degree $i$ homogeneous component of $w$ in $\H^i(-;\Z/2)$.
\end{definition}

\begin{remark}
	Note that axiom (c) does not violate \cref{cor:mobius-characteristic-2-torsion}, since we are working in $\Z/2$ and so every cohomology class has $2$-torsion.
\end{remark}

\subsection{Pontryagin Classes}

\begin{definition}
	The \defn{Pontryagin class $p$} is the unique characteristic class for oriented real vector bundles ($G=\SO$) in singular cohomology with $\Z[1/2]$ coefficients satisfying:
	\begin{enumerate}[(a)]
		\item $p$ is rank-normalized and multiplicative,
		\item
	\end{enumerate}
\end{definition}

\begin{proposition}\label{prop:pontryagin-class-complex-projective-space}
	\[
		p(\CP^n) = (1+\alpha^2)^{2n+1}
	\]
\end{proposition}

\begin{example}
	The Pontryagin classes of the first few dimensions of complex projective space are given by:
	\[
		\begin{aligned}
			p(\CP^1) & = 1,                                    \\
			p(\CP^2) & = 1+3\alpha^2,                          \\
			p(\CP^3) & = 1+4\alpha^2,                          \\
			p(\CP^3) & = 1+5\alpha^2 + 10\alpha^4,             \\
			p(\CP^4) & = 1+6\alpha^2 + 15\alpha^4,             \\
			p(\CP^4) & = 1+7\alpha^2 + 21\alpha^4+ 35\alpha^6, \\
		\end{aligned}
	\]
	Note that $p_k[\CP^{2k}]$.
\end{example}

\begin{proposition}\label{prop:pontryagin-}
	Let $K(p)$ be a Pontryagin class. If $M^{4k}$ is an oriented manifold which is the boundary of a smooth compact oriented manifold $W^{4k+1}$, then $K(p)[M]=0$.
\end{proposition}

\begin{corollary}
	A degree $k$ Pontryagin number $K(p)$ is a well-defined group homomorphism
	\[
		\lkxfunc{}{\Omega^\SO_k}{\Z}{[M]}{K(p)[M]}
	\]
\end{corollary}

\begin{corollary}
	If $K(p)$ is a multiplicative Pontryagin number, then
\end{corollary}

\subsection{Chern-Weil Theory}\label{sec:chern-weil-theory}

While the axiomatic and universal definitions of characteristic classes are simple and abstract, it is often useful to

\subsection{Wu Classes}\label{sec:wu-classes}

\begin{theorem}
	$w=\Sq(v)$.
\end{theorem}

A refinement
For complex vector bundles, i.e.

By \cref{thm:universal-characteristic-classes}, it suffices to compute the cohomology of classifying spaces $\BO_n, \BSO_n, \BU_n$ and their stable counterparts.

\begin{theorem}\label{thm:cohomology-o}
	The cohomology rings of classifying spaces is given by
	\begin{enumerate}[(a)]
		\item $\H^\bullet(\BO_n;\Z/2)\cong \Z/2[w_1,\ldots, w_n]$\hfill where $|w_i|=i$,
		\item $\H^\bullet(\BU_n;\Z)\cong \Z[c_1,\ldots, c_n]$\hfill where $|c_i|=2i$,
		\item $\H^\bullet(\BSO_{2n+1};\Z[\frac12])\cong \Z[\frac12][p_1,\ldots, p_n]$\hfill where $|p_i|=4i$,
		\item $\H^\bullet(\BSO_{2n+2};\Z[\frac12])\cong \Z[\frac12][p_1,\ldots, p_n, e]$\hfill where $|p_i|=4i$, $|e|=2n$.
	\end{enumerate}
	Consequently, the cohomology rings of stable classifying spaces are 
	\begin{enumerate}[(a)]
		\item $\H^\bullet(\BO;\Z/2)\cong \Z/2[w_1,w_2,\ldots]$\hfill where $|w_i|=i$,
		\item $\H^\bullet(\BU;\Z)\cong \Z[c_1,c_2,\ldots]$\hfill where $|c_i|=2i$,
		\item $\H^\bullet(\BSO;\Z[\frac12])\cong \Z[\frac12][c_1,c_2,\ldots]$\hfill where $|p_i|=4i$.
	\end{enumerate}
\end{theorem}

\subsection{The Euler Class}\label{sec:euler-class}

\begin{definition}\label{def:euler-class}
	The \defn{Euler class}
\end{definition}

\begin{proposition}
	On a closed $n$-manifold manifold $M$, $e[\T M]=\sum_k (-1)^k \rank \H^k(M)$.
\end{proposition}

\begin{corollary}
	The Euler number of a sphere is $e[\T S^n] = (1+(-1)^n)[S^n]$.
\end{corollary}
\begin{definition}
	The \defn{Euler class} $e$ is a multiplicative characteristic class for oriented real vector bundles in $\Z$ singular cohomology satisfying the following axioms:
	\begin{enumerate}[(a)]
		\item If a bundle $\mathcal{E}$ has a non-zero section, then $e(\mathcal{E})=0$. 
		\item If $-\mathcal{E}$ has opposite orientation to $\mathcal{E}$, then $e(-\mathcal{E})=-e(\mathcal{E})$.
	\end{enumerate}
\end{definition}

\subsection{The Hirzebruch Signature Theorem}

However, we know by \cref{prop:intersection-form-complex-projective-plane} that the signature of these generators is $\sigma(\CP^{2n})=1$.

\begin{theorem}[Hirzebruch]\label{thm:hirzebruch-signature-theorem}
	Let $M$ be a closed $4k$-manifold. Then we have
	\[
		\sigma(M) = \int_X L_k(p_1, \cdots, p_k)(M),
	\]
	where $L_k$ is the $L$-genus.
\end{theorem}
\begin{proof}
	It suffices to check this identity for the generators $\CP^{2k}$. The total Pontryagin class of $\CP^{2k}$ is $p(\CP^{2k})=(1+\alpha^2)^{2k+1}$. Using multiplicativity of $L$ and $L(1+z)=\sqrt{z}/\tanh\sqrt{z}$, we have
	\[
		\begin{aligned}
			L(p)(\CP^{2k})
			= L\left((1+\alpha^2)^{2k+1}\right)
			= L(1+\alpha^2)^{2k+1}
			= \left(\alpha/\tanh \alpha\right)^{2k+1}
			\quad\in\H^\bullet(\CP^{2k}).
		\end{aligned}
	\]
	Here, we consider $\alpha/\tanh \alpha$ as a formal power series in $\alpha$ truncated by the relation $\alpha^{2k+1}=0$ in the cohomology ring $\H^\bullet(\CP^{2k})$. The characteristic number $L_k(p_1,\ldots,p_k)[\CP^{2k}]$ is then the coefficient of $\alpha^{2k}$ in the expansion of $(\alpha/\tanh \alpha)^{2k+1}$.
	By elementary complex analysis, this coefficient can be extracted by taking a contour integral around a small $\epsilon$-circle about the origin in $\C$:
	\[
		\begin{aligned}
			L_k(p_1,\cdots, p_k)[\CP^{2k}]
			 & = \frac{1}{2\pi i}\oint_{S^1_\epsilon} \frac{dz}{z^{2k+1}} \left(\frac{z}{\tanh z}\right)^{2k+1}
			 &                                                                                                    \\[0.5em]
			 & = \frac{1}{2\pi i}\oint_{S^1_\epsilon} \frac{dz}{\tanh^{2k+1} z}\quad
			 & u  =\tanh(z),\quad
			du =(1-u^2)dz
			\\[0.5em]
			 & = \frac{1}{2\pi i}\oint_{S^1_\epsilon} \frac{1}{u^{2k+1}}\cdot\frac{du}{1-u^2}
			 &                                                                                                    \\[0.5em]
			 & = \frac{1}{2\pi i}\oint_{S^1_\epsilon} \frac{1+u^2+u^4+\cdots}{u^{2k+1}}\,du                       \\[0.5em]
			 & =1.                                                                                              &
		\end{aligned}
	\]
	Since $\sigma(\CP^{2k})=1$, this completes the proof.
\end{proof}

The Hirzebruch signature theorem is a shining example of a truly remarkable theorem in mathematics -- it gives us easy computational means to uncover highly non-trivial relationships between complicated objects. For us, the most useful consequence of the Hirzebruch signature theorem is that it gives us subtle integrability and divisibility theorems. For instance, with relatively little effort we can calculate a few $L$-genus polynomials to get:
\begin{equation}\label{eq:L-genus}
	\begin{aligned}
		 & L_1 = \frac{p_1}{3},\quad
		 & L_2 = \frac{7p_2 - p_1^2}{45},\quad
		 & L_3 = \frac{62p_3 - 13p_1p_2 + 2p_1^3}{945},\quad\cdots
	\end{aligned}
\end{equation}
Having done this, the expression for the leading coefficient of $L_3$ in \cref{eq:L-genus} immediately implies:
\begin{corollary}
	If $M$ is a closed $12$-manifold with trivial $\H^4(M)$, then $\sigma(M)$ is divisible by $62$.
\end{corollary}
The observation that for such manifolds $M$, the quantity $\sigma(M)/62$ is an integer, let alone equal to $945p_3[M]$, is far from obvious, and yet it pops out immediately from the signature theorem. These subtle relationships are immensely useful in defining invariants capable of detecting exotic spheres as we will see in the following \cref{sec:invariants-for-homotopy-4k-1-spheres}.

While we are on the topic of the $L$-genus, let us compute its leading coefficient. Often times, the manifolds which we will apply the signature theorem to will be connected enough that the lower order Pontryagin classes vanish -- leaving just the leading term.
Luckily for us, the coefficient of this leading term admits a simple description in terms of the \defn{Bernoulli numbers} $B_{2k}$, a sequence of rational numbers which appear ubiquitously throughout topology, homotopy theory, number theory, and many other disciplines. There are many conventions in the literature, but for our purposes we can define them as the terms appearing in the series expansion of $\tanh z$:
\begin{equation}\label{eq:tanh_series}
	\tanh z = z - \frac{z^3}{3} + \frac{2z^5}{15} - \frac{17z^7}{315}+\cdots = \sum_{k\geq 1} (-1)^k\frac{2^{2k}(2^{2k}-1)B_{2k}}{(2k)!}\, z^{2k-1}.
\end{equation}
With this definition, the first few Bernoulli numbers are given by:
\begin{equation}\label{eq:bernoulli_numbers}
	B_0 = 1,\quad B_2 = \frac{1}{6},\quad B_4 = \frac{1}{30},\quad B_6=\frac{1}{42},\quad B_{8}=\frac{1}{30},\quad B_{10} = \frac{5}{66},\quad\cdots
\end{equation}

\todo{complexity of the sequence indicates how complicated the geometry can get}

\begin{proposition}\label{prop:leading_coefficient_L_genus}
	The leading coefficient of the $L$-genus is $s_k=2^{2k}(2^{2k-1}-1)B_{2k}/(2k)!$
\end{proposition}
\begin{proof}
	\todo{this proof}
\end{proof}

\subsection{The $\Ahat$ Genus}

\pagebreak
\section{Invariants for Homotopy \texorpdfstring{${(4k-1)}$}{(4k-1)}-Spheres}\label{sec:invariants-for-homotopy-4k-1-spheres}

Armed with the power of the signature theorem, let's now try to build some invariants for homotopy spheres.
Due to the topological simplicity of homotopy spheres, it's difficult to imagine how we could construct such an invariant which can distinguish smooth structure. One of the great ideas of 20th century topology is to pass to a coboundary in situations like this. Namely, if the usual invariants on a manifold $M$ vanish, find a manifold $W$ a dimension higher which has $M$ as a boundary (this isn't always possible since not all manifolds are null-cobordant).
In the case such a manifold exists, it is referred to as a \defn{coboundary} of $M$. If we are careful, we can use the topology of the coboundary $W$ to construct an invariant which does not depend on the choice of coboundary $W$ -- in some sense extracting the \todo{this}

\begin{remark}
	This idea of passing to a coboundary is an example of constructing a \defn{secondary invariant}. When primary invariants, in this case characteristic forms, turn out to be zero, we lift to a case where they are not zero, and use the descent data to measure ``how'' the original invariants vanished. This is a central idea of Chern-Simons theory \cite{chernsimons1974geometric}, a topic which has found widespread application in constructing quantum field theories.
\end{remark}

\subsection{Invariants of Manifolds with Boundary}\label{sec:relative-invariants}

While there are no topological constraints needed to define invariants such as the signature or Euler characteristic on a manifold, constraints do appear when generalizing characteristic numbers to the manifolds with boundary. We will explore these subtleties in this section in preparation for the construction of exotic sphere invariants in \cref{sec:invariants-for-homotopy-4k-1-spheres}.

Let's say that we have some cohomology and homology theory $h$ with coefficient ring $h_0(*)=R$.
Characteristic classes on an $n$-dimensional manifold $W$ are defined as cohomology classes in $c\in h^k(W)$. When the manifold is closed and the characteristic class is homogeneous of degree $n$, Poincar\'e duality allows us to associate a characteristic number $c[W]\in h_0(W)$ to $c$ given a fundamental class $[W]\in h_n(W)$. If $W$ has non-empty boundary, Poincar\'e-Lefschetz duality $h^k(W)\to h_{n-k}(W,\partial W)$ gives us an element $c[W]\in h_0(W,\partial W)$. However, this group is trivial for connected manifolds with non-empty boundary and so is a useless context in which to define relative characteristic numbers.

Instead, we try to pull back characteristic classes to the relative cohomology group $h^n(W,\partial W)$ so that Poincar\'e-Lefschetz duality gives us characteristic numbers in $h_0(W)\cong \Z$ for a fundamental class $[W,\partial W]\in \H_n(W,\partial W)$.
For any integer $\ell$, the pair $(W, \partial W)$ gives a long exact sequence of cohomology groups
\begin{equation}\label{eq:relative-characteristic-classes-exact-sequence}
	h^{\ell-1}(\partial W) \lkxto h^{\ell}(W, \partial W) \lkxto[j] h^{\ell}(W) \lkxto h^{\ell}(\partial W)
\end{equation}
where $j : h^{\ell}(W, \partial W) \to h^{\ell}(W)$ is the induced map of the inclusion $(W,\emptyset) \to (W, \partial W)$. This is an isomorphism if the groups on either side of \cref{eq:relative-characteristic-classes-exact-sequence} are trivial, which will allow us to pull back.

\begin{definition}\label{defn:relative-characteristic_form}
	Suppose that $h^{\ell}(\partial W)$ and $h^{\ell-1}(\partial W)$ are trivial. For a homogeneous characteristic class $c_\ell(W) \in h^{\ell}(W)$ of degree $\ell$, the \defn{relative characteristic form} is the pullback
	\[
		c_\ell(W, \partial W) = j^{-1} c_\ell(W) \quad\in h^{\ell}(W, \partial W).
	\]
	When the class is top-dimensional, i.e. when $\ell=n$, the \defn{relative characteristic number} is the Poincar\'e-Lefschetz dual number $c_\ell[W,\partial W] \in h_0(W)\approx R$.
\end{definition}

\begin{remark}
	There is an intuitive reason for why the topological invariance of characteristic numbers can break down if the manifold has boundary, and this perspective might be illuminating to physics-minded readers. Let's assume we are working with de Rham cohomology. Characteristic classes are cohomology classes, and so any invariant built from them should be invariant under exact gauge transformations $c\mapsto c+d\Lambda$. For example, a change of Riemannian metric in Chern-Weil theory gives such a transformation with $\Lambda$ the relative Chern-Simons form. The corresponding variation in a characteristic number is then
	\[
		c[M] = \int_M c \quad\lkxto\quad \delta c[M] = \int_M c+d\Lambda-\int_M c = \int_{\partial M} \Lambda.
	\]
	The variation $\delta c[M]$ should vanish if $c[M]$ is to be a topological invariant. If the differential equation $\Lambda=dH$ has a solution on $\partial M$, the variation does vanishes and topological invariance is preserved. This is implied by the condition that $\HdR^{\ell-1}(\partial M)$ is trivial, but such a strong condition is not always needed. If there is some extra structure on $M$ restricting the choice of $\Lambda$ to be exact, it is possible to construct invariants even when $\HdR^{\ell-1}(\partial M)$ is non-trivial. See Section V of \cite{witten1985anomalies} for example of such a construction in the context of physics.
\end{remark}

It is commonly the case that we have a family $\{c_i\}$ of characteristic classes.

\begin{definition}\label{defn:relative-characteristic-number-polynomial}
	Given a polynomial $K\in R[x_0,x_1,\ldots, x_\ell]$ such that $K(z^{|c_0|}, z^{|c_1|}, \ldots, z^{|c_\ell|})$ is homogeneous of degree $n$, the relative characteristic number associated to $K$ is given by
	\[
		K(c_1, \ldots, c_\ell)[W,\partial W] = K(j^{-1}c_1(W), \ldots, j^{-1}c_\ell(W))[W,\partial W]
	\]
	whenever $h^{|c_i|-1}(\partial W)$ and $h^{|c_i|}(\partial W)$ are trivial for all $i$ such that $K$ contains an $x_i$ term.
\end{definition}

\begin{remark}
	Note that we pull back \emph{before} applying the polynomial $K$. This is because pulling back a top-dimensional form on $W$ is not generally possible since $h^{n-1}(W)$ is non-trivial and generated by the fundamental class.
\end{remark}

\begin{remark}
	Note that if $\partial W=\emptyset$, relative characteristic forms and numbers correspond exactly to the non-relative versions since $j$ becomes the identity map.
\end{remark}

Turning back to our goal of defining an invariant for a manifold $M$ by constructing an invariant of a coboundary $W$ of $M$, we next would like to understand how relative invariants such as the relative characteristic numbers change with respect to a change of coboundary.
There is a elegant trick we can use to help us answer this. If $W_1$ and $W_2$ are coboundaries for an $n$-dimensional manifold $M$, we can form a closed $(n+1)$-manifold $C$ by gluing $W_1$ and $W_2$ along their boundary $M$.

\begin{proposition}
	There is a unique smooth structure on $C$ which agrees with the smooth structures of $W_1$ and $W_2$. Furthermore, we can give $C$ an orientation which agrees with the orientation of $W_1$ and with the reverse orientation of $W_2$.
\end{proposition}
\begin{proof}
	This is simply a join operation (see \cref{sec:smooth-manifold-operations}) of $W_1$ and $W_2$ along collar neighborhoods (see \cref{thm:collar-neighborhood}) of the boundary $M$.
\end{proof}

For any $\ell$, the Mayer-Vietoris sequence gives an exact sequence
\[
	h^{\ell-1}(M)\lkxto h^{\ell}(C) \lkxto[\mu] h^{\ell}(W_1)\oplus h^{\ell}(W_2) \lkxto h^{\ell}(M)
\]
for any $\ell$, where $\mu$ is the ``restriction'' map. This is an isomorphism if $h^{\ell-1}(M)$ and $h^\ell(M)$ are trivial.
The relative version of the Mayer-Vietoris sequence is of the form
\[
	0 \lkxto h^{\ell}(C, M) \lkxto[\rho] h^{\ell}(W_1,M)\oplus h^{\ell}(B_2,M) \lkxto 0,
\]
since the boundary terms $\H^\ell(M,M)$ vanish, so we have an isomorphism $\rho$.
By the inclusion isomorphisms in \cref{eq:relative-characteristic-classes-exact-sequence}, and we have a commutative diagram
\begin{equation}\label{eq:closing_coboundaries_square}
	\begin{tikzcd}
		{h^{\ell}(C,M)} & {h^{\ell}(B_1,M)\oplus h^{\ell}(B_2,M)} \\
		{h^{\ell}(C)} & {h^{\ell}(B_1)\oplus h^{\ell}(B_2)}
		\arrow["j_1\oplus j_2"', from=1-2, to=2-2]
		\arrow["\rho"', from=1-1, to=1-2]
		\arrow["j"', from=1-1, to=2-1]
		\arrow["\mu"', from=2-1, to=2-2]
		\arrow["h", from=1-2, to=2-1, dashed]
	\end{tikzcd}
\end{equation}
In the case that $h^{\ell-1}(M)$ and $h^\ell(M)$ are trivial, every map in this diagram is an isomorphism. Otherwise, we can only assume that the top map $\rho$ is an isomorphism.
Of particular interest to us is the diagonal map $h = j\circ \rho^{-1}$, which ``glues'' together relative forms on $B_1$ and $B_2$ to a form on $C$.

\begin{proposition}\label{prop:invariant-variation-naturality}
	This gluing map satisfies the following conditions:
	\begin{enumerate}[(a)]
		\item If $\alpha\in h^{n+1}(W_1, M)$ and $\beta\in h^{n+1}(W_2,M)$, then
		      $h(\alpha\oplus \beta)[C] = \alpha[W_1, M] - \beta[W_2, M]$.
		\item If $\alpha_i\in h^{\ell_i}(W_1,M)$ and $\beta_i \in h^{\ell_i}(W_2,M)$ for $i=1,2$, then
		      \[h(\alpha_1\oplus\beta_1) \smile h(\alpha_2\oplus \beta_2) = h(\alpha_1\smile \alpha_2 \oplus \beta_1\smile \beta_2).\]
	\end{enumerate}
\end{proposition}
\begin{proof}
	\todo{do this proof}
\end{proof}

\begin{corollary}\label{prop:relative-characteristic-number-variation}
	Given a polynomial $K\in R[x_0,x_1,\ldots, x_\ell]$ satisfying the conditions of \cref{defn:relative-characteristic-number-polynomial},
	\[
		K(c_1,\ldots, c_\ell)[C] = K(c_1,\ldots, c_\ell)[W_1, M] - K(c_1,\ldots, c_\ell)[W_2, M].
	\]
\end{corollary}

\begin{proposition}\label{prop:signature-variation}
\end{proposition}

We now have all of the tools needed to construct some basic invariants of homotopy spheres.

\subsection{Milnor's Method for Constructing Invariants}

Let us now use the theory developed in \cref{sec:relative-invariants} to construct an invariant for exotic spheres in 7-dimensions. For our cohomology theory, let's use singular cohomology with rational coefficients and consider the Pontryagin numbers due to their connection with the signature.
Given a 7-dimensional homotopy sphere $M$ with 8-dimensional coboundary $W$, there are three invariants which we might consider -- the signature $\sigma$, and the Pontryagin classes $p_1^2$ and $p_2$.
Based on the cohomology of $M$, we have
\[
	\begin{aligned}
		\H^3(M; \Q)=0,  & \quad \H^4(M; \Q)=0 \\
		\H^7(M; \Q)=\Q, & \quad \H^8(M; \Q)=0 \\
		\H^3(M; \Q)=0,  & \quad \H^4(M; \Q)=0
	\end{aligned}
	\quad\implies\quad
	\begin{aligned}
		 & p_1^2\textrm{ does have a relative generalization.}        \\
		 & p_2\textrm{ does not have a relative generalization.}      \\
		 & \sigma\textrm{ satisfies \cref{prop:signature-variation}.}
	\end{aligned}
\]
Therefore, the two invariants of interest to us are
\[
	p_1^2[W,M]
	\quad\textrm{and}\quad
	\sigma(W).
\]
Now, for a \emph{closed} $8$-dimensional manifold $X$, rearranging the Hirzebruch signature theorem gives us the expression
\begin{equation}\label{eq:7-manifold_rearrangement}
	\sigma(X) = \frac{7p_2[X] - p_1^2[X]}{45}
	\quad\implies\quad
	p_2[X] = \frac{45\sigma(X) + p_1^2[X]}{7}.
\end{equation}
This suggests that there \emph{is} some analogue of the second Pontryagin number for $W$, even though the usual pull back strategy fails. For manifolds with boundary, we might consider the number
\[
	\widetilde{p_2}[W, M] = \frac{45\sigma(W) + p_1^2[W, M]}{7} \quad\in\Q.
\]
We use the notation $\widetilde{p_2}$ to emphasize that this is \emph{not} the relative Pontryagin number arising from a pullback of $p_2(W)$, but rather a formal expression which reduces to the Pontryagin number $p_2[W]$ when $W$ is closed by the Hirzebruch signature theorem.
How does the quantity change under a change in coboundary, say if $W_1$ and $W_2$ were coboundaries? Letting $C$ be the $8$-manifold obtained by gluing them together, we see that
\[
	\begin{aligned}
		\widetilde{p_2}[W_1,M] - \widetilde{p_2}[W_2,M]
		 & = \frac{45\sigma(W_1,M) + p_1^2[W_1,M]}{7} - \frac{45\sigma(W_2, M) + p_1^2[W_2,M]}{7} \\
		 & =\frac{45\sigma(C) + p_1^2[C]}{7} = p_2[C].
	\end{aligned}
\]
But this last term is just an ordinary Pontryagin number, and hence an integer. While $\widetilde{p_2}$ is a priori a rational number for a given coboundary, it changes by an integer -- namely by the Pontryagin number $p_2[C]$ of a closed manifold.
Taking the fractional part of $\widetilde{p_2}$ thus gives us an invariant of $M$ which is \emph{independent of the coboundary}!

\begin{remark}
	Given a homotopy sphere $M$, the quantity $\widetilde{p_2}[W,M]$ is a well-defined element of $\Q/\Z$ for any coboundary $W$. It is common to multiply both sides by $7$ so that $7\cdot \widetilde{p_2}[W,M]$ is a well-defined element of $\Z/7$, although this is purely a convention and has no mathematical difference.
\end{remark}

\begin{definition}\label{def:milnor-invariant-7}
	Let $M$ be a null-cobordant\footnote{This is a redundant condition since all oriented $7$-manifolds are null-cobordant.}
	closed oriented $7$-manifold with $\H^3(M; \Q)$ and $\H^4(M; \Q)$ trivial. The \defn{Milnor invariant}\footnote{This differs from Milnor's original definition in \cite{milnor1956manifolds} by a factor of $2$ (mod 7) -- there he defined $\lambda=2p_1^2-\sigma \mod 7$.} of $M$ is
	\[
		\milinv(M) = 3\sigma(W)+p_1^2[W,M]\mod 7.
	\]
	for $W$ any coboundary of $M$.
\end{definition}

Note that this expression is exactly $7\cdot \widetilde{p_2}[W,M]\mod 7$.

\begin{remark}
	Since it takes values in $\Z/7$, the ``resolution'' of the Milnor invariant is at most $7$, meaning that it can detect at most $7$ distinct smooth structures on a $7$-dimensional sphere. We will see a refinement of this invariant in \cref{sec:eells-kupier-invariant} which is able to detect all $28$ smooth structures.
\end{remark}

The basic ideas outlined for $7$-manifolds should work for $(4k-1)$-manifolds in general, so let us work through this case. If we require that $\H^{2i}(M)$ and $\H^{4i}(M)$ are trivial for all $i<k$, then Poincar\'e duality ensures that $\H^{2i-1}(M)$ and $\H^{4i}(M)$ are trivial as well. For full generality, we should also assume that $M$ is null-cobordant (which all homotopy-spheres are) so that a coboundary exists, although results in \cref{sec:cobordism} imply that all $(4k-1)$-manifolds are oriented null-cobordant.

As in the $7$-dimensional case, all but the top-dimensional Pontryagin classes can be generalized to a coboundary $W$.
If we let $X$ be a closed $4k$-manifold and use the Hirzebruch signature theorem, we can do a rearrangement similar to \cref{eq:7-manifold_rearrangement} in order to get the expression
\begin{equation}\label{eq:4k-1-manifold_rearrangement}
	\sigma(X) = L_k(p_1, \ldots, p_k)[X]\quad\implies\quad
	p_k[X] = \frac{\sigma(X) - L_k(p_1,\ldots, p_{k-1}, 0)[X]}{s_k}.
\end{equation}
Here $s_k=L_k(0,\ldots, 0, 1)$ is the coefficient of $x_k$ in the $L$-genus $L(x_1,\ldots, x_k)$.
As before, we use the rearrangement \cref{eq:4k-1-manifold_rearrangement} to get a rational number
\[
	\widetilde{p_k}[W, M] = \frac{\sigma(W) - L_k(p_1,\ldots, p_{k-1}, 0)[W,M]}{s_k}
\]
which acts as a rational generalization of the $k$-th Pontryagin class. By \cref{prop:relative-characteristic-number-variation} and \cref{prop:signature-variation}, given any coboundaries $W_1$ and $W_2$ we get
\[
	\widetilde{p_k}[W_1, M] - \widetilde{p_k}[W_2, M] = p_k[C],
\]
where $C$ is the gluing of $B_1$ and $B_2$. Since the $k$-th Pontryagin number of a closed manifold is an integer, we take the fractional part of $\widetilde{p_k}[W,M]$ and multiplying by the common denominator, we arrive at an generalization of the Milnor invariant for a $7$-manifold.

\begin{definition}
	Let $M$ be a null-cobordant closed $(4k-1)$-manifold with $\H^{4i}(M)$ and $\H^{2i}(M)$ trivial for all $i<n$. The \defn{Milnor invariant} of $M$ is
	\[
		\milinv(M) = \denom(s_k)\cdot \left(\sigma(W) - L_k(p_1, \ldots, p_{k-1},0)[W,M]\right)\mod \numer(s_k),
	\]
	where $s_k = L_k(0,\ldots,0,1)$ is the coefficient of the last term in the $L$-polynomial and $\numer$ and $\denom$ take the numerator and denominator of a rational number respectively.
\end{definition}

\begin{example}
	The first few Milnor invariants are:
	\[
		\renewcommand{\arraystretch}{1.2}
		\begin{array}{r@{\;}ll}
			\milinv(M^7)
			 & = 4\sigma(W) - p_1^2[W, M]
			 & \mod 7,                                                        \\
			\milinv(M^{11})
			 & = 15\sigma(W) - (2p_1^3-13p_1p_2)[W,M]
			 & \mod 62,                                                       \\
			\milinv(M^{15})
			 & = 303\sigma(W) - (3p_1^4-22p_1^2p_2 + 19p_2^2 + 71p_1p_3)[W,M]
			 & \mod 381.
		\end{array}
	\]
	These invariants were generated with Wolfram, see \cref{chap:wolfram} for details.
\end{example}

If we would like to generalize these invariants further, it is helpful to take a bird's eye view of what happened here. We started with the idea that lifting to a coboundary could shed geometric insights to the boundary. Once on this coboundary, we pick a characateristic number which takes on a restricted set of values for closed manifolds representing a ``change of coboundary''. Moding out by the image of such changes, we get an invariant purely of the boundary. To summarize, we have a loose procedure
\[
	\left\{\parbox{12.5em}{An integrality theorem for a characteristic number of a closed $(n+1)$-manifold}\right\}
	\quad\lkxto\quad
	\left\{\parbox{9.5em}{A smooth invariant for certain $n$-dimensional manifolds}\right\}
\]
In the case of the Milnor invariant, the integrality of the top-dimensional Pontryagin number of a closed $4k$-manifold led us to Milnor's invariant for $(4k-1)$-manifolds with certain vanishing cohomology groups.
If we use a different integrality theorem, this procedure gives a different invariant.

\subsection{The Eells-Kupier Invariant}\label{sec:eells-kupier-invariant}

Consider the characteristic

\begin{theorem}\label{thm:Ahat-integrality}
	If $M$ is an oriented closed $2m$-dimensional manifold with a spin structure, then $\Ahat[M]$ is an integer. If $m\equiv 2\mod 4$, then $\Ahat[M]$ is an even integer.
\end{theorem}

\begin{theorem}[Rochlin]\label{thm:rochlin}
	The signature of an oriented closed $4$-manifold is divisible by $16$.
\end{theorem}
\begin{proof}
	If $M$ is a oriented closed $4$-manifold, we have
	\[
		\sigma(M) = \frac{p_1[M]}{3}\quad\textrm{and}\quad p_1[M] = -24\cdot\Ahat[M].
	\]
	If follows that $\sigma(M)=-8\cdot\Ahat[M]$, and by \cref{thm:Ahat-integrality} we know that $\Ahat[M]$ is even. Therefore, the signature is divisible by $8$.
\end{proof}

\begin{remark}
	Rochlin's theorem implies that there is a topological manifold which admits no smooth structure, see \cref{rmk:E8-manifold}.
\end{remark}

\subsection{Index Theory}

\subsection{A ``Twisted'' Eells-Kupier Invariant}
\begin{definition}
	We have
	\[
		\Ahat(M,\T M_\C) = \frac{1}{158505984000}\left(-25167p_2^3 - 18832p_3^2 + 38828p_2p_4 - 4976p_6\right)[M]
	\]
\end{definition}
See page 98 of \cite{hopkinsmahowald2002bo8}.


\begin{theorem}[Mahowald-Hopkins]
	A closed String manifold $M$ has $24\mid \Ahat(M,T)$.
\end{theorem}

\begin{corollary}
	Let $M$ be a closed $23$-manifold with $\H^{4i}(M)$ and $\H^{4i+3}(M)$ trivial. For any $11$-connected String coboundary $W$, the \defn{twisted Eells-Kupier invariant}
	\[
		\lambda_{\mathrm{m,h}}(M) = 153945\sigma(W) + 2591p_2^3[W,M]\mod 521432801280
	\]
	depends only on the smooth structure of $M$.
\end{corollary}
