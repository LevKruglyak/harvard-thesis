\chapter*{Prologue}

\begin{epigraph}{19.5em}{Rav. Joseph Soloveitchik}
	Existence plays a mischievous game with us,\\
	as though to tease and provoke us. In the \\
	midst of knowledge there yet again arises \\
	the mystery; in the midst of contemplation\\
	the riddle gains new strength.
\end{epigraph}

During my last summer before graduating college, some friends and I set out on a mountaineering trip up Banner Peak, a picturesque mountain in the eastern Sierra Nevada range of California. On our way down, the conversation turned towards the upcoming year, where I mentioned I would be writing a thesis on exotic spheres.
As I began to elaborate, I struggled to find the words to explain what exotic spheres were, what about them made them so ``exotic'', and why anyone should care.
It was apparent that, aside from folklore I had heard from graduate students, I had no intuition about exotic spheres at all so I hastily changed the subject.

Our next trip to the mountains was several months later.
When the topic of theses came up again, I had much more to say -- I talked at length about how we detect can exotic spheres, how we can construct them as generalizations of knots in higher dimensions, how they relate to the geometry of lattices, why a physicist looking to investigate the consistency of their theory might care about exotic spheres, etc. Over course of the semester, I had learned first-hand that exotic spheres were not some artificial construction but were of central importance in differential topology. They appear naturally yet surprisingly in a tremendous breadth of disciplines, forming a far-reaching riddle of curious phenomena.
In this thesis, I hope to share some of the intuition I've gained over the last year, providing a comprehensive introduction to the detection, construction, and classification of exotic spheres. In particular, I hope to emphasize the rich geometric aspects of the theory which are sometimes skipped in more modern accounts. 

\section*{The Shape of Space}

The simplest, oldest, and arguably most intuitive conceptualization of space is that of Euclidean space -- an infinitely extending flat canvas of points in which parallel lines do not intersect. Most of classical geometry occurred in this context, either in $2$ or $3$ dimensions.
Less often, mathematicians worked in higher dimensional versions of Euclidean geometry. These were less intuitive and tricker to draw on paper, but the basic geometric rules still applied.
Throughout the 19th century, mathematicians such as Gauss, Riemann, and Lobachevsky began to uncover models of geometry in which many of the assumptions of Euclidean geometry no longer held.
For instance, using the curved surface of a ball as a canvas for geometry gives fundamentally different results than when using a plane. For starters, parallel ``lines'' on the surface of a ball always intersect (think longitudinal lines on a globe).
More quantitatively, the case of a ball has the sum of angles in a triangle add up to less than $180^\circ$ while in case of a plane the sum of angles in a triangle add up to exactly $180^\circ$. The failure of a triangle's angles to add up to $180^\circ$ or for parallel lines to avoid intersection is an example of \emph{curvature} -- deviation from the flat geometry of Euclidean space.

It's important to note that while geometers had worked with circles, spheres, and other curved surfaces for millenia, they were always embedded in Euclidean space as in the case of a ball in $3$-dimensional space. The novel idea behind non-Euclidean models was that the curved space was \emph{all there was} -- the $2$-dimensional surface of the ball could exist as a geometrical space \emph{independently} of its embedding in $3$-dimensional space.
The curvature of geometric spaces was not the curving of something \emph{into} an ambient space but rather an intrinsic geometric feature of the sapce itself. After all, measuring angles and drawing lines can be done entirely without leaving the surface of the ball. Even without an understanding of $3$-dimensions, a flat creature living on the ball could conclude that they existed in a curved non-Euclidean space.
By dispensing with the Euclidean axiom of flatness, mathematicians opened up a cornucopia of rich geometry to explore, and these explorations have only grown grander in scope over time.

\begin{figure}[ht]
	\centering
	\import{diagrams}{placeholder.pdf_tex}
	\caption{Triangles in Euclidean (left) and non-Euclidean geometries (right).}\label{fig:triangles}
\end{figure}

Near the end of the 19th century, a sparse set of theorems and loose definitions began to consolidate into a field known as topology. Although there weren't many clear definitions in the early days of the field, the central idea was simple. Rather than just removing the axiom of flatness like in the case of non-Euclidean geometries, topology removes most of the axioms, leaving only the \emph{shape} of space, detached from any quantitative information. Unlike a geometric space, a topological space does not come with any notions of distance, angle, perpendicularity, area or volume -- all that is left is a notion of continuity.
In the geometry that most people are familiar with, two shapes are said to be congruent, or geometrically equivalent, if we can transform between them in a way that does not alter distances,angles, and consequently areas.
The only requirement for a topological equivalence is that it is continuous, that is, close together points in the first shape should be close together after the transformation. More informally, ripping the shape is not allowed, but any squishy deformation is totally fine. Such a transformation is known as a \emph{homeomorphism} -- coming from Greek roots for ``similar shape''.
\begin{figure}[ht]
	\centering
	\import{diagrams}{congruence-vs-homeomorphism.pdf_tex}
	\caption{Geometric equivalence (congruence) vs topological equivalence (homeomorphism)}
\end{figure}

In many areas of topology, we require our spaces to be \emph{manifolds}.
A space is said to be a manifold if it can be built out of patches of coordinate space. For instance, our intuitive model of the surface of the earth is flat, specified by the perpendicular directions -- north/south and east/west. On human scales, this coordinate system is indistinguishable from a flat plane. However, if we keep going north we'd end up at the north pole, at which point the notion of ``further north'' no longer makes sense. At the north pole, we'd have to come up with a different coordinate system but that's fine -- as long as a space has a local coordinate system at each point, the space is a manifold. The number of coordinates needed to describe a manifold is known as its \emph{dimension}. The surface of the earth for example can be represented by a two-dimensional manifold, this is a two-dimensional sphere. Similarly, a circle can be thought of as a one-dimensional manifold since locally it looks like a line, a circle can be thought of as a one-dimensional sphere.

While the study of manifolds is a deep and fascinating discipline in itself, many manifolds have some extra geometric structure which makes them more interesting to work with. Many of these structures were originally inspired by physics -- for instance Einstein's manifolds modelling the fabric of space and time came with a way to measure the curvature caused by the presence of a heavy mass or large energy density.
The common feature behind many of these geometric structures was that they required a notion of ``calculus'' on manifolds, i.e. infinitesimals, derivatives, and the like. Mathematicians did what they do best -- remove unnecessary axioms and work primarily in this most general context. For calculus on manifolds, the most general definition was that of a \emph{smooth structure}, the word smooth here meaning ``infinitely differentiable''. In the spirit of topology, while a smooth structure might not allow you to get the precise numerical solutions to differential equations, it does allow you to explore the dimensionality of the solutions -- how many \emph{types} of solutions are there rather than what exactly they are. Instead of only requiring two equivalent manifolds to just have the same shape, we now also require equivalences between them to be infinitely differentiable, excluding infinitely sharp kinks or bends. These smooth equivalences are called \emph{diffeomorphisms}.

With just topology, the dressing room of a manifold is dizzyingly vast.
A circle could be represented by homeomorphism with a square, a squiggly blob, a spiky star, a horrifically jagged fractal snowflake, or literally any other closed loop drawn without lifting your pencil. Any of these outfits ``fit'' the abstract topological circle.
Adding a smooth structure onto a manifold downsizes its walk-in closet of possible representations to a tiny travel suitcase, eliminating any outfits which are not ``smooth''.

\section*{So what is an Exotic Sphere?}

As with any construction in mathematics, we can ask if (a) a smooth structure always exists on a manifold, and (b) if a smooth structure is unique (at least up to diffeomorphism). The reasonable guess would be yes to both of these questions. After all, how topologically twisted can a manifold get for it not to have a smooth structure at all? Remember, ordinary manifolds still locally look like ordinary Euclidean space. Surely any jaggedness can be smoothed out? If you're drawing a closed loop on a sheet of paper and it comes out jagged, you can always redraw it and hold your pen tighter, the resulting shape would still be the same. Similarly, if you're building a clay model and the surface is rough, you can always gently brush over it with water to get a nice smooth surface. You certainly wouldn't expect smoothing it one way to result in a fundamentally different smooth structure than if you smoothed it a different way.

Our intuition is correct in these low-dimensional cases -- in fact every one, two, and three-dimensional manifold admits a unique smooth structure. 
Note that spheres of any dimension have at least one smooth structure -- we can always draw a sphere in Euclidean space and do calculus there. Curiously, when we move up to four dimensions, the question of whether the sphere has a unique smooth structure becomes extremely hard. As of the writing of this thesis, the problem has remained wide open for almost 100 years and is the last remaining case of a problem known as the generalized Poincar\'e conjecture.

\begin{figure}[ht]
	\renewcommand{\arraystretch}{1.2}
	\centering
	\begin{tabular}{r|c|c|c|c|c|c|c|c|c|c|c|c|c|c|c}
		\textrm{dimension:}               & 1 & 2 & 3 & 4 & 5 & 6 & 7  & 8 & 9 & 10 & 11  & 12 & 13 & 14 & 15    \\
		\hline
		\textrm{\# of smooth structures:} & 1 & 1 & 1 & ? & 1 & 1 & 28 & 2 & 8 & 6  & 992 & 1  & 3  & 2  & 16526 \\
	\end{tabular}
	\caption{The number of smooth structures on the spheres.}
\end{figure}

In dimensions 5 and 6 there is always a unique smooth structure, but things begin to break down in dimension 7.
On the seven-dimensional sphere, there are exactly \emph{twenty-eight}\footnote{This includes pairs of smooth structures which come in a ``right-handed'' and ``left-handed'' version, so depending on what you count there may be less.} distinct smooth structures, no more no less.
Any time we find a smooth structure on a sphere which is not diffeomorphic to the ordinary one, we call such a sphere \emph{exotic}.
One of these twenty eight structures corresponds to the ordinary smooth structure, but the rest are all exotic spheres.
Next up in dimension, an eight-dimensional has only two smooth structures, while a nine-dimensional sphere has eight smooth structures. In fact, the only dimensions in which a unique smooth structure is known to exist are $1,2,3,5,6,12,56,$ and $61$. This seems oddly specific.
At first glance, the number of smooth structures on the sphere does not appear to follow an obvious pattern.


\section*{Ramifications of Exoticity}

% \todo{mention how exotic spheres can be thought of as knots, talk about torsion phenomena}
