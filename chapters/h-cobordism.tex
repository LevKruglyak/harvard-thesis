\chapter{The \texorpdfstring{$h$}{h}-Cobordism Theorem}\label{chap:h-cobordism}

\todo{prove h-cobordism, define groups of homotopy spheres}

\section{Cobordism}\label{sec:cobordism}

The basic principle of cobordism is to declare two manifolds equivalent if there is a manifold a dimension higher which connects the two manifolds. As an equivalence relation, cobordism is far looser than the notions of homoemorphism or diffeomorphism and so allows for a full classification of manifolds. Many notions in geometry and topology -- for instance characteristic classes -- depend solely on the cobordism type of a manifold, so understanding the structure of cobordism is immensely helpful.

\begin{remark}
	Note that the implied compactness assumption throughout the thesis is important here, otherwise any manifold $M$ is trivially the boundary of $M\times [0,\infty)$.
\end{remark}

\begin{definition}
	An \defn{unoriented cobordism} between closed $n$-manifolds $M_1$ and $M_2$ is an $(n+1)$-manifold $W$ with $\partial W = M_1\sqcup M_2$. We use the notation $W : M_1\bord M_2$ to refer to the cobordism.
\end{definition}

\begin{definition}
	An \defn{oriented cobordism} between closed oriented $n$-manifolds $M_1$ and $M_2$ is an oriented $(n+1)$-manifold $W$ with $\partial W = M_1\sqcup (-M_2)$. We use the notation $W : M_1\sobord M_2$ to refer to the cobordism.
\end{definition}

\begin{remark}
	\todo{General structure on a cobordism}
\end{remark}

Note that the oriented cobordism group can be thought of as a $\Z$-module, with multiplication action on a closed manifold $M$ given by
\[
	n \cdot M = \begin{cases} M\sqcup \cdots \sqcup M & n > 0,\\ (-M)\sqcup \cdots \sqcup (-M) & n < 0,\\ \emptyset & n=0,\end{cases}
\]
for all $n\in \Z$, where $\sqcup$ is repeated $|n|$ times. Since there is no notion of negation in the unoriented case, the unoriented cobordism group is a $\Z/2$-module.

\begin{figure}[ht]
	\centering
	\import{diagrams}{pair-of-pants.pdf_tex}
	\caption{A cobordism $W$ between $S^1$ and $(S^1\sqcup S^1)$.}\label{fig:pair-of-pants}
\end{figure}

For a simple example of a cobordism between a circle and a disjoint union of circles, see \cref{fig:pair-of-pants}. Note that this cobordism could be made much simpler by removing the handle. Simplifying cobordisms in this way is one of the major applications of surgery theory.

\begin{definition}
	The \defn{$k$-th oriented cobordism group}[oriented cobordism group] $\Omega^\SO_k$ is the abelian group of oriented cobordism classes of closed $k$-manifolds\footnote{We do not require manifolds to be connected in this definition.}
	under disjoint union. The identity component is the empty set $\varnothing$, and negation is given by reversing orientation. Similarly, the \defn{$k$-th unoriented cobordism group}[unoriented cobordism group] $\Omega_k$ is the abelian group of cobordism classes of closed $k$-manifolds under disjoint union.
\end{definition}

\begin{example}
	There is an isomorphism $\Omega_0\cong \Z/2$. An unoriented 0-dimensional manifold is just a set of points. Any pair of points is cobordant to the empty set by a path connecting them. Since adding pairs of points doesn't change the cobordism type, the number of points modulo 2 determines the cobordism class entirely.
\end{example}

\begin{example}
	There is an isomorphism $\Omega_0^\SO \cong \Z$. An oriented 0-dimensional manifold is still a set of points, however the orientation now equips each point with a ``charge'', we might label them as $+$ or $-$. Note that points of opposite ``charges'' cancel out by a path between them oriented from $-$ to $+$. Given some set of points of various charges, we can always eliminate pairs of opposite charges and are left with a homogeneous set of charge. Adding up all of the pluses or minuses, we get an integer. This integer determines the cobordism class, and is invariant to adding or removing pairs of opposing charge.
\end{example}

\begin{example}
	Both the oriented and unoriented cobordism groups are trivial in dimension 1, since every circle is the boundary of a disk.
\end{example}

In higher dimensions, the classification becomes much more interesting.

\begin{figure}[ht]
	\renewcommand{\arraystretch}{1.2}
	\centering
	\begin{tabular}{r||c|c||c|c}
		$k$ & $\Omega_k$          & generators                                          & $\Omega_k^\SO$ & generators                 \\
		\hline
		$0$ & $\Z/2$              & a point                                             & $\Z$           & a point                    \\
		$1$ & $0$                 &                                                     & $0$            &                            \\
		$2$ & $\Z/2$              & $\RP^2$                                             & $0$            &                            \\
		$3$ & $0$                 &                                                     & $0$            &                            \\
		$4$ & $\Z/2\oplus \Z/2$   & $\RP^4$, $\RP^2\times \RP^2$                        & $\Z$           & $\CP^2$                    \\
		$5$ & $\Z/2$              & $\SU_3/\SO_3$                                       & $\Z/2$         & $\SU_3/\SO_3$              \\
		$6$ & $(\Z/2)^{\oplus 3}$ & $\RP^6$, $\RP^2\times \RP^4$, $(\RP^2)^{\times 3}$, & $0$            &                            \\
		$7$ & $\Z/2$              & $(\SU_3/\SO_3) \times \RP^2$                        & $0$            &                            \\
		$8$ & $(\Z/2)^{\oplus 4}$ & $\RP^8, \RP^6\times \RP^2, \cdots$                  & $\Z\oplus \Z$  & $\CP^4, \CP^2\times \CP^2$ \\
	\end{tabular}
	\medskip
	\caption{Structure of unoriented and oriented cobordism groups.}\label{fig:cobordism-structure-table}
\end{figure}

The structure of \cref{fig:cobordism-structure-table} makes a lot more sense in the context of

\begin{proposition}
	The product of manifolds is a well-defined operation with respect to cobordism.
\end{proposition}
\begin{proof}
	\todo{proof}
\end{proof}

\begin{definition}
	The \defn{oriented cobordism ring} $\Omega^\SO_\bullet$ is the set of oriented cobordism classes of closed manifolds under the operations of disjoint union and product.
\end{definition}

The oriented cobordism ring has a grading by
\[
	\Omega_\bullet^\SO = \bigoplus_{k\geq 0} \Omega^\SO_k.
\]

\begin{theorem}[Thom-Pontryagin]\label{thm:oriented-cobordism-structure}
	There is a ring isomorphism
	\[
		\Omega_\bullet^\SO \otimes \Q \lkxto \Q[x_4, x_8, x_{12}, \ldots]
	\]
	where $x_{4k}$ are cobordism classes representing $\CP^{2k}$.
\end{theorem}

\pagebreak
\section{Morse Theory}\label{sec:morse-theory}

The last tool in our arsenal is that of Morse theory, introduced in \cref{sec:morse-theory}.

\todo{Morse theory is a technique connecting the structure of a manifold to the behavior of a (nice) smooth function from the manifold to the real numbers. By investigating the discrete set of critical points of such a function, questions of attaching and removing high-dimensional handles is simplified to questions of points on the one-dimensional real line. For this thesis, the main application of Morse theory will be in the proof of the $h$-cobordism theorem of \cref{chap:h-cobordism}, however we make occasional use of Morse theory in historical remarks and for explicit constructions of exotic spheres scattered throughout.}

While a fully comprehensive introduction to Morse theory is outside of the scope of this thesis, we'll include a basic overview for completeness. A great classical introduction to Morse theory can be found in Milnor's book on the subject \cite{milnor1963morse}.

If $f : M \to \R$ is a smooth function on a manifold $M$, the points $p\in M$ where the differential $df_p : \T_p M \to \T_{f(p)} \R = \R$ vanish are known as \defn{critical points}, and their images in $\R$ are called \defn{critical values}. In terms of local coordinates $\{x^1,\ldots, x^n\}$ at $p$, this means that
\begin{equation}
	\frac{\partial f}{\partial x^1}=\cdots=\frac{\partial f}{\partial x^n} = 0.
\end{equation}
A critical point $p\in M$ is said to be \defn{non-degenerate}[non-degenerate point] if the matrix
\begin{equation}
	\everymath={\displaystyle}
	\renewcommand*{\arraystretch}{2}
	H_f(p) = \begin{pmatrix}
		\frac{\partial^2 f}{\partial x^1\partial x^1} & \cdots &
		\frac{\partial^2 f}{\partial x^1\partial x^n}                   \\
		\vdots                                        & \ddots & \vdots \\
		\frac{\partial^2 f}{\partial x^n\partial x^1} & \cdots &
		\frac{\partial^2 f}{\partial x^n\partial x^n}                   \\
	\end{pmatrix}(p)
\end{equation}
is invertible at $p$. This is called the \defn{Hessian matrix} of $f$ at $p$, and in this formulation depends on our chosen coordinate system.
There is a coordinate independent way to define the Hessian as a symmetric bilinear form on $\T_p M$ which makes the coordinate invariance of the condition of non-degeneracy manifestly apparent.

\begin{definition}
	The \defn{index}[index of a function] of $f$ at a point $p$ is the maximal dimension of a subspace on which $H_f(p)$ is negative definite, i.e. it is the dimension of $\{v\in \R^n \mid v^\intercal H_f(p) v < 0\}.$
\end{definition}

The index of a function at a point essentially describes the ``shape'' of the function out of a list of finitely many possible shapes. Remember, the index of a function on an $n$-dimensional manifold must be an integer between $0$ and $n$. For instance, in the case of a surface there are three possible shapes -- when both coordinates curve up we get a bowl facing up, when one curves up and one curves down we get a saddle, and when both coordinates curve down we get a bowl facing down. These shapes correspond to indices of $0$, $1$, and $2$ respectively.
\begin{figure}[ht]
	\centering
	\import{diagrams}{placeholder.pdf_tex}
	\caption{An upward bowl, saddle, and downward bowl.}
\end{figure}

The fundamental lemma of Morse theory makes rigorous this notion of a manifold having a shape dictated by a real-valued function -- there is always a local coordinate system which puts the function into a standard form depending on the index.

\begin{lemma}[Morse Lemma]\label{lemma:morse}
	Let $p$ be a non-degenerate critical point of $f$. There is a local coordinate system $(y^1,\ldots, y^n)$ at $p$ such that
	\begin{equation}
		f(y^1,\ldots, y^n)=f(0)-\left[(y^1)^2 + \cdots + (y^{\ell})^2\right] + \left[(y^{\ell + 1})^2 + \cdots + (y^n)^2\right].
	\end{equation}
	where $\ell$ is the index of $f$ at $p$.
\end{lemma}
\begin{proof}
	Let's assume without loss of generality that $f(p)=0$. Given any local coordinate system $(x^1,\ldots, x^n)$ at $p$, we can write
	\begin{equation}
		f(x^1,\ldots, x^n) = \sum_{1\leq j \leq n} x^j g_j(x^1,\ldots, x^n)
	\end{equation}
	where $g_j$ are functions satisfying $g_j(0)=(\partial f / \partial x^j)(0)$.
	This can be achieved by setting $g_j(x_1,\ldots, x_n) = \int_0^1 (\partial f/\partial x^j)(tx_1, \ldots, tx_n)\,dt$. Since $p$ is a critical point

	\todo{basic idea}
\end{proof}

Inspired by this lemma, we might call a function $f : M \to \R$ a \defn{Morse function} if all critical points are non-degenerate.

\begin{corollary}
	Non-degenerate critical points are isolated.
\end{corollary}

For a brief demonstration of the power of the Morse lemma, we'll prove Reeb's theorem, a Morse theoretic criterion for a compact manifold to be homeomorphic to a sphere. Throughout the thesis, we'll usually defer to the more powerful $h$-cobordism theorem to prove that a manifold is homoemorphic to a sphere.
However, it is useful to not always rush for the flamethrower when trying to kill a fly -- a simple swatter might do the trick. We'll see a direct application of this lighter theorem in \cref{sec:milnor-spheres}.

\begin{theorem}[Reeb]\label{thm:reeb}
	If $M$ is a compact manifold and $f$ is a Morse function with exactly $2$ critical points, then $M$ is homeomorphic to a sphere.
\end{theorem}
\begin{proof}
	Firstly, by compactness of $M$ we can find a global minimum $f(x_0)$ and global maximum $f(x_1)$ for some distinct points $x_0$ and $x_1$ (otherwise the function would be constant and not a Morse function with $2$ critical points). We can normalize the function $f$ to have $f(x_0)=0$ and $f(x_1)=1$ without loss of generality. It follows that $x_0$ is a non-degenerate critical point of index 0 and $x_1$ is a non-degenerate critical point of index $n$.
	By the Morse lemma (\ref{lemma:morse}), there is a neighborhood $x_0\in U_0$ with local coordinates $\{y^1,\ldots, y^n\}$ such that
	\begin{equation}
		f(y^1,\ldots, y^n) = (y^1)^2 + \cdots + (y^n)^2.
	\end{equation}
	This gives a Riemannian metric $(dy^1)^2+\cdots+(dy^n)^2$ on $U$ which can be extended to all of $M$ by partitions of unity.
	% A Riemannian metric on a manifold $M$ determines an isomorphism $\T^\d M \cong \T M$, and hence an isomorphism $\Omega^1(M)=\Gamma(\T^\d M) \cong \Gamma(\T M)=\X(M)$ between the space of $1$-forms and the space of vector fields. Composing this isomorphism with the exterior derivative $d : \Omega^0(M)\to \Omega^1(M)$ gives the gradient operator $\nabla : \Omega^0(M) \to \X(M)$ sending a function to its vector field. 

	Given a Riemmanian structure, there is a gradient operator $\nabla : \Omega^0 \to \X(M)$ sending functions to vector fields.
	In our case, the vector field $\nabla f$ is non-zero everywhere except for at $x_0$ and $x_1$. We thus have a normalized vector field $\nabla f/\|\nabla f\|^2$
	defined everywhere except for at $x_0$ and $x_1$. Let $\varphi_t : M \to M$ be the global flow corresponding to this vector field, i.e. the unique solution to the differential equation
	\begin{equation}
		\left.\frac{d\varphi_t(p)}{dt}\right|_{t=0} = \frac{\nabla f(p)}{\|\nabla f(p)\|^2}
	\end{equation}
	for all $p\in M\setminus \{x_0,x_1\}$. By the chain rule, it follows that
	\begin{equation}
		\frac{d(f\circ \varphi_t(p))}{dt}=\left\langle \frac{d\varphi_t(p)}{dt}, \nabla f\right\rangle = \left\langle \frac{\nabla f}{\|\nabla f\|^2}, \nabla f\right\rangle=1.
	\end{equation}
	In particular, this implies that $f\circ \varphi_t(p) = f(p)+t$.

	\todo{finish the proof}
\end{proof}

The basic ideas used in the proof of Reeb's theorem can be radically generalized.

\begin{definition}
	For any $a\in \R$, the \defn{level set} of a smooth function $f : M \to \R$ is the set
	\begin{equation}
		M_a = f^{-1}(-\infty, a] = \{ p\in M \mid f(p)\leq a)\}.
	\end{equation}
\end{definition}

\begin{figure}
\end{figure}

\begin{theorem}
	Let $f : M \to \R$ be a smooth function and suppose $f^{-1}[a,b]$ contains no critical points of $f$ for real numbers $a<b$. Then $M_a$ is diffeomorphic to $M_b$. Furthermore $M_a$ is a deformation retract of $M_b$.
\end{theorem}

\begin{theorem}
	Let $f : M \to \R$ be a smooth function and let $p$ be a non-degenerate critical point of index $\ell$. Letting $c=f(p)$, suppose that $f^{-1}[c-\epsilon, c+\epsilon]$ is compact and contains no critical points of $f$ aside from $p$. For sufficiently small $\epsilon$, the level set $M^{c+\epsilon}$ has the homotopy type of $M^{c-\epsilon}$ with an $\ell$-cell attached.
\end{theorem}

\begin{theorem}
	If $f$ is a Morse function with compact level sets (for instance if $M$ is compact), then $M$ has the homotopy type of a CW-complex with a cell in each dimension $\ell$ for each critical point of index $\ell$.
\end{theorem}

\todo{finish}

\begin{theorem}[Morse Inequality]
	Let $M^n$ be a closed manifold with $f : M\to \R$ a Morse function. Let $\beta_i=\rank \H_i(M)$ be the $i$-th Betti number of $M$ and let $c_i$ be the number of critical points of $f$ of index $i$. Then, for every $\ell\in \Z^{\geq 0}$ we have
	\begin{equation}
		\beta_\ell - \beta_{\ell-1} + \beta_{\ell-2} - \cdots +(-1)^\ell \beta_0 \leq c_\ell - c_{\ell-1} + c_{\ell-2} - \cdots + (-1)^\ell c_0.
	\end{equation}
\end{theorem}

\section{Handlebodies}

\begin{theorem}[$h$-cobordism]\label{thm:h-cobordism}
	Within some category of manifolds $\mathscr{C}$, if $M$ and $N$ are closed, simply-connected manifolds of dimensions $\geq 5$ and $W : M \hbord N$ is a simply-connected $h$-cobordism, then $W$ is $\mathscr{C}$-isomorphic to the cylinder $M\times [0,1]$. Furthermore, the isomorphism can be chosen to be the identity on $M\times \{0\}$.
\end{theorem}

\begin{example}
	The simple-connectedness assumption of the $h$-cobordism theorem is required, since the spaces $L(7,1)\times S^4$ and $L(7,2)\times S^4$ are $h$-cobordant but not diffeomorphic.
\end{example}

\begin{theorem}
	In the manifold categories $\TOP$ and $\PL$, the generalized Poincar\'e conjecture is true in dimensions $\geq 5$.
\end{theorem}
\begin{proof}
	\todo{cone construction, fails for $\DIFF$ because you can't take a smooth cone.}
	\todo{mention method of engulfing?}
\end{proof}

\begin{corollary}\label{thm:h-cobordism-diffeomorphism}
	In the smooth oriented manifold category, two simply-connected closed manifolds of dimensions $\geq 5$ are $h$-cobordant if and only if they are diffeomorphic (by an orientation preserving diffeomorphism).
\end{corollary}
\begin{proof}
	If $f : M_1 \to M_2$ is a diffeomorphism between manifolds $M_1$ and $M_2$, they are $h$-cobordant by the manifold $W=M_1\times [0,1]\cup_f M_2$, where we glue $M_2$ onto $M_1\times \{1\}$ in $M_1\times [0,1]$ by $f$.
	Conversely, if $W : M_1\sohbord M_2$ is an $h$-cobordism, by the $h$-cobordism theorem (\ref{thm:h-cobordism}) there must be a diffeomorphism $f : W \to M_2$ must map to $M_1\times \{1\}$, this gives a diffeomorphism $M_2 \to M_1$. If we choose $f$ to reverse orientation on $M_1\to M_1\times \{0\}$, then the restriction $f|_{M_2}$ will preserve orientation.
\end{proof}

\pagebreak
\section{Groups of Homotopy Spheres}\label{sec:groups-of-homotopy-spheres}

With the $h$-cobordism theorem and \cref{thm:h-cobordism-diffeomorphism}, we have arrived at our first major simplification to the problem of classifying smooth structures in dimensions $\geq 5$ -- to classify smooth structures on $S^n$, it is enough to classify the $h$-cobordism classes of smooth manifolds which are homeomorphic to $S^n$, and to find smooth manifolds which are homeomorphic to $S^n$,. it suffices to consider smooth manifolds which have the homotopy type of $S^n$.

\begin{definition}
	A \defn{homotopy $n$-sphere}[homotopy sphere] is a smooth manifold which is homotopy equivalent to the sphere $S^n$. By the generalized Poincar\'e conjecture, all homotopy spheres are homeomorphic to spheres.
\end{definition}

\begin{definition}
	Let $\Theta^n$ be the pointed set of $h$-cobordism classes of homotopy $n$-spheres (the basepoint is the ordinary sphere $S^n$).
\end{definition}

So far, this is a fairly general setup. Instead of spheres, we could use any simply-connected base manifold and consider the set of $h$-cobordism classes of manifolds homotopy equivalent to it. This is known as the \defn{surgery structure set}, and we will expand on this briefly in \cref{sec:surgery-theory-in-general}. However, in the case of spheres, there is a special bit of extra data which does not usually generalize -- a group structure under the connected sum operation defined in \cref{sec:connected-sum}.

\begin{theorem}\label{thm:group-of-homotopy-spheres}
	The connected sum turns $\Theta^n$ into a group with identity element $S^n$.
\end{theorem}
\begin{proof}
	We have already proved that the connected sum is well-defined up to diffeomorphism, associative, and commutative up to orientation preserving diffeomorphism.  
	The remaining proof splits as three lemmas -- these are Lemmas 2.2, 2.3, and 2.4 in \cite{milnorkervaire1963groups}.

	\begin{lemma}
		The connected sum is a well-defined operation on $\Theta^n$.
	\end{lemma}
	\begin{proof}
		We will prove something slightly more general. Let $M_1, M_1'$ and $M_2, M_2'$ be closed, simply-connected, and oriented manifolds, and suppose $W_1 : M_1\sohbord M_1'$ and $W_2 : M_2\sohbord M_2'$ are $h$-cobordisms. The goal is to construct an $h$-cobordism $W : (M_1\+M_2)\sohbord (M_1'\+M_2')$.

		\begin{figure}[ht]
			\centering
			\import{diagrams}{placeholder.pdf_tex}
			\caption{A join of two $h$-cobordisms along embedded arcs.}\label{fig:connected-sum-of-h-cobordisms}
		\end{figure}

		Recall that the connected sum $M_1\+M_2$ was defined as the join of $M_1$ and $M_2$ along points $p_1\in M_1$ and $p_2\in M_2$. Let us assume these were the points along which the connected sum $M_1\+M_2$ was taken, and $p_1'\in M_1'$ and $p_2'\in M_2'$ the points along which the connected sum $M_1'\+M_2'$ was taken. Choose smooth paths $\gamma_i : [0,1] \to W_i$ which send $p_i\mapsto p_i'$, and assume without loss of generality that $\gamma_i$ are embeddings which are transverse to the boundary of $\partial W_i$, making it possible to get tubular neighborhoods
		\[
			\lkxfunc{\iota_i}{\R^n\times [0,1]}{W_i.}
		\]
		Let $W$ be the join of $W_1$ and $W_2$ along the image of the arcs $\gamma_i$, as depicted in \cref{fig:connected-sum-of-h-cobordisms}. This is an oriented smooth manifold with boundary
		\[
			\partial W = (M_1\#M_2)\sqcup -(M_1'\+M_2')
		\]
		so $W$ is a cobordism $W : (M_1\+M_2)\sobord (M_1'\+M_2')$. We just need to prove that $W$ is an $h$-cobordism to complete the proof.

		\begin{lemma}\label{lemma:removing-point-homotopy-equivalence-h-cobordism}
			If $W : M \sohbord M'$ is an $h$-cobordism with $M,M'$ compact oriented and simply connected, $p\in M$ is a point, and $T$ is a tubular neighborhood of an arc from $p$ to $p'\in M'$, then the inclusion $M\setminus\{p\} \to W\setminus T$ is a homotopy equivalence.
		\end{lemma}
		\begin{proof}
			The long exact sequence of the inclusion $j : (M,M\setminus\{p\}) \to (W, W\setminus T)$ gives
			\[
				\H_{i+1}(W\setminus T, M\setminus\{p\})\lkxto
				\H_i(M\setminus\{p\}) \lkxto[j_*] \H_i(W\setminus T) \lkxto \H_i(W\setminus T, M\setminus\{p\})
			\]
			Note that by excision, $\H_i(W\setminus T, M\setminus\{p\})\cong \H_i(W,M)$, and since $W$ is an $h$-cobordism, $\H_i(W,M)\cong 0$. Consequently, $j$ induces isomorphisms on homology and since $M,M'$, and $W$ are simply connected, Whitehead's theorem and the Hurewicz isomorphism imply that $j$ is a homotopy equivalence.
		\end{proof}

		With \cref{lemma:removing-point-homotopy-equivalence-h-cobordism} in mind, we use the Mayer-Vietoris sequence
		\[\begin{tikzcd}
				{\H_i(S^{n-1})} & {\H_i(M_1\setminus\{p_1\})\oplus\H_i(M_2\setminus\{p_2\})} & {\H_{i}(M_1\+M_2)} & {\H_{i-1}(S^{n-1})} \\
				{\H_i(S^n)} & {\H_i(W_1\setminus T_1)\oplus\H_i(W_2\setminus T_2)} & {\H_{i}(W)} & {\H_{i-1}(S^n)}
				\arrow[from=1-1, to=1-2]
				\arrow[from=1-1, to=2-1]
				\arrow["f"', from=1-2, to=1-3]
				\arrow["{(j_1)_*\oplus (j_2)_*}", from=1-2, to=2-2]
				\arrow[from=1-3, to=1-4]
				\arrow["j_*", from=1-3, to=2-3]
				\arrow[from=1-4, to=2-4]
				\arrow[from=2-1, to=2-2]
				\arrow["g"', from=2-2, to=2-3]
				\arrow[from=2-3, to=2-4]
			\end{tikzcd}\]
			where $j_i : M_i\setminus \{p_i\} \to W_i\setminus T_i$ and $j : M_1\+ M_2\to W$ are inclusions. Note that $(j_1)_*\oplus (j_2)_*$ is an isomorphism by \cref{lemma:removing-point-homotopy-equivalence-h-cobordism}. When $i$ is not $1,n-1,$ or $n$, it follows immediately that $f$ and $g$ are isomorphisms and so $j$ is an isomorphism. In the remaining cases, a simple application of the  snake lemma does the trick.
			By simple connectedness, Whitehead's theorem, and the Hurewicz isomorphism, it follows that the inclusion of $M_1\+M_2 \to W$ is a homotopy equivalence, completing the proof.
	\end{proof}

	\begin{remark}
		In the previous lemma, the simple-connectedness assumption was not strictly required -- only used to reduce the problem of homotopy equivalence to a homology problem by Whitehead's theorem. It is possible to directly construct an explicit homotopy equivalence with a more careful argument.
	\end{remark}

	Next, we will prove a simple characterization of when a homotopy sphere is equal to the identity element in $\Theta^n$.

	\begin{lemma}\label{lemma:null-h-cobordant-iff-bounds-contractible}
		A simply-connected manifold $M$ is $h$-cobordant to $S^n$ if and only if $M$ bounds a contractible manifold.
	\end{lemma}
	\begin{proof}
		The forward direction is fairly straightforward. If $W : M\sohbord S^n$ is an $h$-cobordism, we can glue a tubular neighborhood of $D^{n+1}$ in $\R^{n+1}$ to $(-S^n)\subset \partial W$ along a collar neighborhood of $-S^n$ in $\partial W$. Any homotopy equivalence of $W$ to $S^n$ can then be extended to a contraction along $D^{n+1}$.

		In the reverse direction, let us assume that $M$ bounds a contractible manifold $W$. Pick any point in the interior of $M$ and remove the interior of an embedded disk $D^{n+1}\subset W$. This gives a simply-connected cobordism $W'=W\setminus \Int(D^{n+1}) : M\sobord (-S^n)$. By the long exact sequence of the inclusion $j : (D^{n+1}, S^n) \to (W,W')$, we have
		\[
			\H_i(W', S^n) \lkxto H_i(S^n)\lkxto[j_*] \H_i(W')\lkxto \H_{i-1}(W',S^n),
		\]
		and by contractability of $W$ the boundary terms vanish and so $S^n \to W'$ is a homotopy equivalence. Then, by the relative Poincar\'e duality isomorphism $\H_k(W',M)\cong \H^{n+1-k}(W', S^n)$ and the long exact sequence of the pair $(W',M)$, it follows that $M\to W'$ is a homotopy equivalence and so $W'$ is in fact an $h$-cobordism. 
	\end{proof}

	\begin{lemma}
		If $M$ is a homotopy sphere, then $M\+(-M)$ bounds a contractible manifold.
	\end{lemma}
	\begin{proof}
		\begin{figure}[ht]
			\centering
			\import{diagrams}{placeholder.pdf_tex}
			\caption{A contractible boundary of $M\+(-M)$.}
		\end{figure}
	\end{proof}


	\noindent This completes the proof.
\end{proof}

\subsection{Framed Manifolds}

There is a final

\begin{definition}
\end{definition}

\begin{definition}
\end{definition}
