\begin{flushleft}
	\textsl{}\\
	\rule[0pt]{15em}{0.5pt}\\
	\textsl{-- }
	\vspace{2em}
\end{flushleft}

\todo{introduction}

\section{The First Exotic Sphere}\label{sec:first-exotic-sphere}

We mentioned briefly in \cref{sec:plumbing} that the original constructions of $7$-dimensional exotic spheres by Milnor in \cite{milnor1956manifolds} were as total spaces of $S^{3}$ bundles over $S^4$. In this section, we'll explicitly work through this case, directly constructing a smooth manifold and proving that it is homeomorphic but not diffeomorphic to the standard $7$-dimensional sphere.

First, recall that there is an equivalence between $S^3$ bundles, $D^4$ bundles, and rank $4$ vector bundles by taking associated bundles since the linear symmetry groups of every one of $S^3\subset D^4\subset \R^4$ are isomorphic. Such bundles on $S^4$ are classified by the clutching function, which lives as an element in $\pi_3(\SO_4)$. For a review of the clutching construction, see \cref{sec:vector-bundles-over-spheres}. 
Understanding $S^3$ bundles over $S^4$ thus involves understanding the homotopy group $\pi_3(\SO_4)$. Note that there is a canonical identification of $S^3$ with the set of quaternions of unit norm, and of $\SO_4$ with rotations of the quaternionic plane $\HH$. 

\begin{proposition}
	There is an isomorphism $\Z\oplus \Z\cong \pi_3(\SO_4)$ which sends a pair $(i,j)$ to the map sending a unit quaternion $u\in S^3$ to the rotation $(v\mapsto u^i\cdot v\cdot u^j)\in \SO_4$.
\end{proposition}

\begin{proof}
	\color{red}
	\todo{this proof}

The direct approach involves viewing $\SO_4$ as the group of orientation-preserving isometries of $\R^4$, or equivalently as rotations of $S^3$. Let's consider $S^3\cong \SU_2\subset \HH$ as the set of unit norm quaternions. Letting $\SO_3$ act on the $\{i,j,k\}$ plane in $S^3$, note that any rotation $R\in \SO_4$ gives a rotation $R^{-1}(1)R\in \SO_3$ which now fixes $1$ and thus only acts on the $\{i,j,k\}$ plane. 
\begin{changemargins}
\begin{lemma}\label{prop:SO4-decomposition}
	There is a homeomorphism
	\begin{equation}
		\lkxfunc{}{\SO_4}{\SU_2\times \SO_3}{R}{\left(R(1),R^{-1}(1)R\right)}
	\end{equation}
	with inverse given by sending $(\xi, R)\in \SU_2\times \SO_3$ to the rotation $v\mapsto \xi \cdot R(v)$ where $\cdot$ denotes quaternion multiplication in $\SU_2$.
\end{lemma}
\end{changemargins}

% Next, note that 
%
% \begin{proposition}
% \end{proposition}
% \begin{proof}
% \end{proof}

Next, we recall that $\SU_2$ can be viewed as a double cover of $\SO_3$. One way to construct this double cover is by starting with the 
conjugation map $\rho : \HH \to \Aut(\HH)$ which sends $\xi$ to the automorphism $(v\mapsto \xi\cdot v\cdot \xi^{-1})$. If we restrict $\rho$ to $\SU_2\subset \HH$, the map becomes $\rho : \SU_2 \to \SO_3$. This can be shown to be a double cover by an isomorphism of $\SO_3$ with $\RP^3$. The double cover map leads to an identification of $\SU_2$ with the spin group $\Spin_3$.
\begin{proposition}
	There is an exceptional isomorphism $\Spin_3\cong \SU_2$.
\end{proposition}

It follows that $\pi_3(\SU_2)$

A bit more abstractly, we can double cover $\SO_4$ by the spin group $\Spin_4$. 
\begin{proposition}\label{prop:exceptional-isomorphism-spin4}
	There is an exceptional isomorphism $\Spin_4\cong \SU_2\times \SU_2$.
\end{proposition}
\begin{proof}
	To see that there is a hoemorphism at all, note that by \cref{prop:SO4-decomposition} we have a homeomorphism $\SO_4\cong \SU_2\times \SO_3$. Since there is an exceptional isomorphism $\Spin_3\cong \SU_2$, it follows that the double cover of $\SO_4$ would be homeomorphic to $\SU_2\times \SU_2\times \SU_2$ since this is the unique

	For a full proof that there is an isomorphism of Lie groups, see Theorem 8.1 in \cite{lawson1989spin}.
	\todo{cite spin geometry}
\end{proof}
Since covering spaces induce isomorphisms of higher homotopy groups, we have $\pi_3(\SO_4)\cong \pi_3(\Spin_4)$, and by \cref{prop:exceptional-isomorphism-spin4} we have isomorphisms 
\[\pi_3(\Spin_4)\cong \pi_3(\SU_2\times \SU_2) \cong \pi_3(S^3\times S^3)\cong \pi_3(S^3)\oplus \pi_3(S^3) = \Z\oplus \Z\]
since $\SU_2\cong S^3$. The result $\pi_3(\SO_4)\cong \Z\oplus \Z$ follows as with the direct computation, although it's trickier to see in this case that quaternion conjugation gives the representatives of homotopy classes.
\end{proof}

Let's now see which $(i,j)\in \pi_3(\SO_4)$ give us spaces homeomorphic to spheres. Let $\Sigma_{i,j}^7$ denote the total space of the $S^3$ bundle corresponding to $(i,j)$. Note that by giving $S^3$ and $S^4$ their usual smooth structures, $\Sigma_{i,j}^7$ gets a smooth structure as well. These manifolds are known as \defn{Milnor manifolds}.

\begin{proposition}
	When $i+j=1$, there is a homeomorphism $\Sigma_{i,j}^7\cong S^7$.
\end{proposition}

There are two proofs, both of which are enlightening. In the first proof, we make use of Morse theory in the form of Reeb's theorem which requires us to construct a Morse function with exactly two critical points to prove a manifold is homeomorphic to a sphere. For the second proof, we compute the homotopy type and use the topological Poincar\'e conjecture in dimension $7$.

\begin{proof}[First Proof]
	Recall that the clutching construction for $\Sigma_{i,j}^7$ is the quotient space
	\[
		\Sigma_{i,j}^7 = (D^4_+\times S^3)\cup_{h_{i,j}} (D^4_-\times S^3)
	\]
	where $h_{i,j}$ identifies $(u,v)\in\partial D^4_+ \times S^3$ with $(u, u^i\cdot v\cdot u^j)\in \partial D^4_-\times S^3$.
	Consider the function
	\[
		f : 
	\]
\end{proof}

\begin{proof}[Second Proof]
Since we have a fiber bundle
\[
		S^3 \lkxto \Sigma_{i,j}^7 \lkxto S^4,
\]
we can apply the long exact sequence of a fibration to get 
\[
	\cdots \lkxto \pi_{k+1}(S^{4}) \lkxto[\delta] \pi_k(S^{3}) \lkxto \pi_k(\Sigma_{i,j}^7) \lkxto \pi_{k}(S^{4})\lkxto[\delta] \cdots
\]
It's clear that $\Sigma_{i,j}^7$
\end{proof}




\[
	S^{n-1} \lkxto M \lkxto S^n
\]
for a manifold $M$, there is a long exact sequence of homotopy groups
\[
	\cdots \lkxto \pi_{k+1}(S^{n}) \lkxto[\delta] \pi_k(S^{n-1}) \lkxto \pi_k(M) \lkxto \pi_{k}(S^{n})\lkxto[\delta] \cdots
\]
If the connecting map $\delta : \pi_{n}(S^n) \to \pi_{n-1}(S^{n-1})$ is an isomorphism,

\todo{finish}

\begin{definition}
	% Let $\xi
	\todo{Milnor manifold}
\end{definition}

\todo{use Reeb's theorem to explicitly show that they are topological spheres}

\section{Gromoll-Meyer Exotic Sphere}

\begin{theorem}
	The Gromoll-Meyer sphere has non-negative sectional curvature.
\end{theorem}

\section{Dodecahedral Space}

\section{Exotic Spheres as Knots}

\subsection{Complex Singularities}

Let $F\in \C[z_0,z_1\ldots, z_n]$ be a non-constant polynomial in $(n+1)$-complex variables.
\begin{definition}
	The \defn{variety} of $F$ is the complex hypersurface given by the zero locus
	\[
		\V(F) = F^{-1}(0)=\left\{ z \in \C^{n+1}  F(z)=0\right\} \subset \C^{n+1}.
	\]
\end{definition}

\todo{cauchy riemann equations}

\begin{definition}
	The \defn{gradient} of a complex analytic function $F : \C^{n+1} \to \C$ is the $(n+1)$-tuple
	\[
		\nabla_F = \left(\frac{\partial F}{\partial z_0}, \frac{\partial F}{\partial z_1},\ldots, \frac{\partial F}{\partial z_n}\right).
	\]
	\todo{better definition}
\end{definition}

\begin{definition}
	A point $w\in \V(F)$ is a (complex) \defn{singularity}[complex singularity] if $\nabla_F(w)$ vanishes. A singularity is \defn{isolated}[isolated singularity] if there is a neighborhood surrounding $w$ which contains no other singularities.
\end{definition}

\begin{theorem}
	For small $\varepsilon>0$ the intersection of $\V(F)$ with $D_\varepsilon(w)$
\end{theorem}

\begin{proposition}
	Every sufficiently small sphere around an isolated singularity of $F$ intersects $\V(F)$ transversally in a smooth manifold.
\end{proposition}

\begin{definition}
	Let $w\in \V(F)$ be an isolated singularity. The \defn{link} of $F$ at $w$ is the intersection
	\[
		\L(F, w) = \V(F) \cap S^{2n+1}_\varepsilon(w) = \left\{ z\in \C^{n+1}  F(z)=0\textrm{ and } |z-w|<\varepsilon\right\}
	\]
	where $\varepsilon > 0$ is some sufficiently small real number so that $\L(F,w)$ is a smooth manifold intersecting the sphere $S^{2n+1}_\varepsilon(w)$ transversally.
\end{definition}

When the isolated singularity is clear, we write $\L(F)$.

\subsection{Brieskorn Manifolds}
The simplest examples of complex polynomials with isolated singularities are \todo{this}

\begin{definition}
	Let $(a_0,a_1,\ldots, a_n)$ be an $(n+1)$-tuple of integers greater than or equal to $2$. The \defn{Brieskorn polynomial} of the tuple $(a_0,a_1,\ldots, a_n)$ is given by
	\[
		F(z_0,z_1,\ldots, z_n) = z_0^{a_0} + z_1^{a_1} +\cdots + z_n^{a_n}.
	\]
	Correspondingly, we refer to $\V(F)$ as the \defn{Brieskorn variety} of the tuple and to the link at the origin $\L(F,0)$ origin as the \defn{Brieskorn manifold}. We'll use the notation
	\[
		\Sigma(a_0,a_1,\ldots, a_n) =\L(z_0^{a_0}+z_1^{a_1}+\cdots+z_n^{a_n}, 0)
	\]
	to refer to these Brieskorn manifolds.
\end{definition}


\begin{proposition}
	If $p,q\geq 2$, then $\Sigma(p,q)\subset S^3$ is the torus link of type $(p,q)$.
\end{proposition}

\begin{proposition}
	There is a homeomorphism $\Sigma(2,2,2)\cong \RP^3$.
\end{proposition}

\begin{proposition}
	There is a homeomorphism $\Sigma(2,3,5)\cong \mathscr{D}$.
\end{proposition}

\subsection{The Fibration Theorem}

\begin{theorem}\label{thm:fibration}
	If $F$ is a complex polynomial in $(n+1)$-variables with an isolated singularity at the origin, then there is a smooth fiber bundle map
	\[
		\lkxfunc{\phi}{S^{2n+1}_\varepsilon - \L(F)}{S^1}{z}{\arg F(z).}
	\]
\end{theorem}

For a given angle $e^{i\theta}\in S^1$, we'll denote the fiber of the bundle $\phi$ as $F_\theta = \phi^{-1}(e^{i\theta})$.

\begin{proposition}
	Each fiber $F_\theta$ is a smooth parallelizable $2n$-manifold.
\end{proposition}

\subsection{When is the link a topological sphere?}

Let's fix a polynomial $F$ in $(n+1)$ complex variables

\begin{proposition}
	If $n\neq 2$, then $\L$ is homeomorphic to the sphere $S^{2n-1}$ if and only if $\L$ has the homology of a sphere. In fact, $\L$ is a topological sphere if and only if the reduced homology $\widetilde{H}_{n-1}(\L)$ is trivial.
\end{proposition}

Let's now choose an orientation for $F_\theta$.

\begin{proposition}
	The manifold $\L$ is a homology sphere if and only if the intersection form
	\[
		\lkxfunc{Q_{F_\theta}}{\H_n(F_\theta)\times \H_n(F_\theta)}{\Z}
	\]
	has determinant $\pm 1$ -- in other words if $Q_{F_\theta}$ is unimodular.
\end{proposition}

\subsection{Kervaire Invariant}

\begin{theorem}[Brieskorn-Pham]
\end{theorem}

\begin{theorem}[Levine]
	If $n$ is odd, the Kervaire invariant is given by
	\[
		c(F_0) = \begin{cases}
			0 & \textrm{if }\Delta(-1)\equiv \pm 1\mod 8 \\
			1 & \textrm{if }\Delta(-1)\equiv \pm 3\mod 8
		\end{cases}
	\]
\end{theorem}

\begin{theorem}[Hirzebruch-Mayer] Smooth Brieskorn varieties are parallelizable.
\end{theorem}
