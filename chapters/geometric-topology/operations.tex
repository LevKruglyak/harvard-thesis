\pagebreak
\section{Smooth Manifold Operations}\label{sec:smooth-manifold-operations}

Cutting and pasting is the category of smooth manifolds is a subtle craft. 
Unlike in the topological manifold category, combining manifolds resembles ``sewing'' more than ``gluing''. In order to ensure that the resulting manifold will have a unique smooth structure, we must identify the spaces in a small region surrounding the submanifold. The extra space provided by the region allows one manifold to ``smoothly curve into'' the other.

\begin{remark} 
	We will see in \cref{sec:twisted-spheres} that exotic spheres can be formed by identifying two disks along \emph{just} their boundary by an orientation preserving diffeomorphism. This always gives an ordinary sphere in the topological category but not in the smooth category.
\end{remark}

We begin with some technical lemmas from differential topology. Whenever a manifold $N^k$ is embedded into an ambient manifold $M^n$, there is a short exact sequence of vector bundles on $N$
\begin{equation}
	0 \lkxto \TT N\lkxto \TT M|_N \lkxto \TT M/N \lkxto 0.
\end{equation}
Here $\TT M/N=\TT M|_N / \TT N$ denotes the \defn{normal bundle} of $N$ in $M$, which has rank $(n-k)$.

A \defn{tubular neighborhood} of $N$ is a neighborhood $\widetilde{N}\supset N$ in $M$ which is the diffeomorphic image of the total space of the normal bundle $\T M/N$ with the diffeomorphism $\tau : \T M/N \to \widetilde{N}$ satisfying:
\begin{enumerate}[(a)]
	\item $\tau$ is the identity on the image of the zero section $s_0 : N \to \T M/N$,
	\item For any point $p\in \partial N$ on the boundary, $\tau(\T_p M/N)\subset \partial M$.
\end{enumerate}


A tubular neighborhood should be thought of as a way of ``thickening'' the submanifold $N$ to have the same dimension as $M$, as depicted in \cref{fig:tubular-neighborhood}. 

\begin{figure}[ht]
	\centering
	\import{diagrams}{tubular-neighborhood.pdf_tex}
	\caption{Tubular neighborhoods of some submanifolds.}\label{fig:tubular-neighborhood}
\end{figure}

The diffeomorphism $\tau : \T M/N \to \widetilde{N}$ can interpreted as a chart of a ``tube'' surrounding $N$ in $M$. Namely, for each point $p\in N$, local trivializations of the normal bundle $\T M/N$ are maps $\R^k \times \R^{n-k}\to U\subset \widetilde{N}$ which consist of coordinates $(x^1,\ldots,x^k)$ on $N$ and coordinates $(v^{k+1},\ldots, v^{n})$ on an orthogonal plane to $N$.

\begin{remark}
	A common convention is to define a tubular neighborhood as the diffeomorphic image of the associated disk bundle $\D(\T M/N)$. This convention makes explicit the ``tube'' of a  ``tubular neighborhood''. We will refer to such neighborhoods as \defn{closed tubular neighborhoods}[closed tubular neighborhood], since they are the closure of a tubular neighborhood.
\end{remark}

This leads to a basic technical lemma of differential topology.

\begin{theorem}[Tubular Neighborhood Theorem]\label{thm:tubular-neighborhood}
	Every embedded submanifold has a tubular neighborhood.\footnote{This result does not require $M$ to be compact.}
\end{theorem}
\begin{proof}
	See Chapter II of \cite{kosinski1993differential} or Theorem 6.24 of \cite{lee2013smooth} for the case $M=\R^n$.
\end{proof}

\begin{warning}
	In Conventions, we assumed all smooth manifolds to be compact and all submanifolds to be properly embedded and neat. Absent these assumptions, the tubular neighborhood theorem as stated is not true.
\end{warning}

In a similar vein, when we have just a manifold $M^n$, a \defn{collar neighborhood} of its boundary is a neighborhood $\widetilde{\partial M}\supset \partial M$ which is the diffeomorphic image of the ``half'' normal bundle. This can be defined as the subset of the restricted normal bundle $\T M|_{\partial M} / \T \partial M$ of vectors whose local geodesics lie strictly in $M$. Since $\partial M$ has codimension $1$ in $M$, this is a fiber bundle with fibers $[0,\infty)$. Just as for tubular neighborhoods, we require that the diffeomorphism is the identity on the zero-section of this bundle.

\begin{theorem}[Collar Neighborhood Theorem]\label{thm:collar-neighborhood}
	Every manifold with non-empty boundary has a collar neighborhood.
\end{theorem}
\begin{proof}
	See Theorem 9.25 of \cite{lee2013smooth}.
\end{proof}

\begin{proposition}
	If $N\subset M$ is a submanifold, there is a collar neighborhood of $\partial M$ which restricts to a collar neighborhood of $\partial N$ in $N$.
\end{proposition}
\begin{proof}
	See Proposition~2.8.2 in Chapter II of \cite{kosinski1993differential}.
\end{proof}

\begin{figure}[ht]
	\centering
	\import{diagrams}{collar-neighborhood.pdf_tex}
	\caption{A collar neighborhood.}\label{fig:collar-neighborhoods}
\end{figure}

\subsection{Joining Two Manifolds Along a Submanifolds}

We can now address one of the fundamental operations in geometric topology -- the joining of two manifolds by submanifolds.

Let us suppose we have manifolds $M_1^n$ and $M_2^n$ of the same dimension, with embeddings $\iota_1 : N\to M_1$ and $\iota_2 : N\to M_2$ of a manifold $N^k$. Our goal is to construct a joined manifold $M$ which we denote $M_1\cup_N M_2$.
We start by picking tubular neighborhoods $\widetilde{N_i}$ for $\iota_i(N)\subset M_i$ and let $\tau_i: \T M/\iota_i(N) \to \widetilde{N_i}$ be the diffeomorphism.
We define $M$ to be the quotient
\begin{equation}\label{eq:join-definition}
	M_1\cup_N M_2 = \frac{M_1\setminus \iota_1(N)\sqcup M_2\setminus \iota_2(N)}{\tau_1(p, t\xi) \sim \tau_2(p, t^{-1}\xi)}
\end{equation}
for all $p\in N$, $t\in(0,\infty)$, and unit vectors $\xi\in \T_p M/\iota_i(N)$ with respect to some inner product on $\T M/N_i$. The resulting smooth manifold $M=M_1\cup_N M_2$ is called a \defn{join along a submanifold}\footnote{Some authors refer to this operation as ``\defn{pasting}'' or as the ``\defn{generalized connected sum}'', see for instance Section VI.4 in \cite{kosinski1993differential}.} and is one of the fundamental operations of geometric topology.

\begin{figure}[ht]
	\centering
	\import{diagrams}{join-along-submanifold.pdf_tex}
	\caption{A join of two manifolds along a submanifold.}\label{fig:join-along-submanifold}
\end{figure}

\begin{remark}
	As we have remarked, the ``sewing'' procedure in \cref{eq:join-definition} by tubular neighborhoods is vital in the category of smooth manifolds. In the category of topological manifolds\footnote{Assuming, of course, that there is a topological notion of  tubular neighborhoods.}, we could define the join by removing the tubular neighborhoods entirely (rather the just the zero sections) and identifying the resulting boundaries.
	At each point of the shared boundary, the join would be locally Euclidean since it has a neighborhood which the gluing of two half-disks to get a full disk.
\end{remark}

\begin{proposition}\label{prop:join-along-submanifolds-well-defined}
	The join $M$ of two manifolds along a submanifold has a smooth structure that agrees with the smooth structures of $M_1\setminus \iota_1(N)$ and $M_2\setminus \iota_2(N)$ in $M$. 
\end{proposition}
\begin{proof}
	There is a technical but simple proof which involves writing out the transition functions for charts on $M$ arising from charts on $M_1$ and $M_2$ which did not intersect $\iota_1(N)$ or $\iota_2(N)$. The result then follows from the smoothness of $t \mapsto t^{-1}$ on $t\in (0,\infty)$.
\end{proof}

\begin{proposition}\label{prop:join-along-submanifolds-orientation}
	If $M_1$, $M_2$, and $N$ are oriented, with $\iota_1$ orientation-preserving and $\iota_2$ orientation-reversing, then $M$ has an orientation compatible with $M_1\setminus \iota_1(N)$ and $M_2\setminus \iota_2(N)$.
\end{proposition}
\begin{proof}
	This follows since $t\mapsto t^{-1}$ and $\iota_2$ both reverse orientations. We omit the technical details for brevity.
\end{proof}

\begin{theorem}
	Up to orientation-preserving diffeomorphism, the join of manifolds along a submanifold is independent of the choices of tubular neighborhood.
\end{theorem}

\subsection{Connected Sum}\label{sec:connected-sum}

The simplest submanifold by which to join two manifolds is a point.

\begin{definition}
	The \defn{connected sum} $M_1\+M_2$ of two manifolds $M_1$ and $M_2$ is their join along an embedded point.
\end{definition}

Visually, this operation can be thought of as cutting disks out of both manifolds and connecting them with a tube, as depicted in \cref{fig:connected-sum}.

\begin{remark}
	Note that by our assumption of submanifolds preserving boundary, the point along which the manifolds are joined cannot lie on the boundary of either manifold.
\end{remark}

\begin{figure}[ht]
	\centering
	\import{diagrams}{connected-sum.pdf_tex}
	\caption{A connected sum of two surfaces.}\label{fig:connected-sum}
\end{figure}

This operation is (up to orientation-preserving orientation) independent of the choice of embedding, justifying the basepoint-free notation of $M_1\+ M_2$.\footnote{As stated in Conventions, we assume that all manifolds are connected. Otherwise the connected sum would only be well-defined assuming a choice of connected component of $M_1$ and $M_2$.}
Proving the independence of connected sum from the choice of basepoint is non-trivial, and follows from a technical result by Richard Palais.

\begin{theorem}[Disk Theorem]\label{thm:disk}
	If $M^n$ is an oriented manifold and $\iota_1, \iota_2 : D^n \to M$ are orientation-preserving disk embeddings, then there is a diffeomorphism $f : M \to M$ such that $\iota_1 = f\circ \iota_2$.
\end{theorem}
\begin{proof}
	See Theorem B in \cite{palais1960diffeomorphism}.
\end{proof}

\begin{corollary}\label{cor:connected-sum-operation}
		The connected sum is well-defined, associative, and commutative up to orientation-preserving diffeomorphism.
\end{corollary}

\begin{corollary}
	For any manifold $M^n$, there is a diffeomorphism $M\+ S^n\cong M$.
\end{corollary}

\begin{corollary}
	$\partial(M_1\+M_2) = \partial M_1\sqcup \partial M_2$.
\end{corollary}

Let us now briefly discuss the effect of connected sum on homology. If we have two oriented manifolds $M_1$ and $M_2$ of dimension $n>1$, their connected sum $M_1\# M_2$ can be decomposed as a union of open sets diffeomorphic to $M_1\setminus\{p\}$ and $M_2\setminus\{p\}$. We denote these open sets $M_1^\circ$ and $M_2^\circ$ respectively. Their intersection is diffeomorphic to a tubular neighborhood of $S^{n-1}$, so by the Mayer-Vietoris sequence, we have an exact sequence
\[
	\H_k(S^{n-1})\lkxto \H_k(M_1^\circ)\oplus \H_k(M_2^\circ)\lkxto[p_k] \H_k(M_1\+M_2)\lkxto \H_{k-1}(S^{n-1})
\]
In lowest dimension $k=0$, we know that $\H_0(M_1\+ M_2)\cong \Z$ since the connected sum is connected. Next, when $k=1$, the edge term $\H_{k-1}(S^{n-1})$ is non-trivial, but the kernel is trivial by a rank argument in the $k=0$ segment and so $p_1$ is also an isomorphism.
When $k$ is not $n-1$, the boundary terms vanish completely and so $p_k$ is also an isomorphism. In the remaining cases, we get the exact sequence
\begin{equation}\label{eq:connected-sum-cohomology}
% 	\begin{tikzcd}
% 	0 & {\H_n(M_1^\circ)\oplus \H_n(M_2^\circ)} & {\H_n(M_1\+M_2)} & {\H_{n-1}(S^{n-1})} \\
% 	0 & {\H_{n-1}(M_1\+M_2)} & {\H_{n-1}(M_1^\circ)\oplus \H_{n-1}(M_2^\circ)}
% 	\arrow[from=1-1, to=1-2]
% 	\arrow["p_n", from=1-2, to=1-3]
% 	\arrow["q", from=1-3, to=1-4]
% 	\arrow[from=1-4, to=2-3]
% 	\arrow[from=2-2, to=2-1]
% 	\arrow["p_{n-1}"',from=2-3, to=2-2]
% \end{tikzcd}
%
	\begin{tikzcd}
		0\rar & {\H_n(M_1^\circ)\oplus \H_n(M_2^\circ)} \rar{p_n}\snakenode{X} & {\H_n(M_1\+M_2)}\snakearrow{X} \\
		\H_{n-1}(S^{n-1}) \rar & {\H_{n-1}(M_1^\circ)\oplus \H_{n-1}(M_2^\circ)}\rar{p_{n-1}} & {\H_{n-1}(M_1\+M_2)} \\
\end{tikzcd}
\end{equation}
If $M_1$ and $M_2$ are both closed, so is $M_1\+M_2$, and hence $q$ is an isomorphism by a fundamental class argument. This implies that $p_{n-1}$ is an isomorphism, so we conclude:

\begin{proposition}\label{prop:homology-connected-sum-closed}
	If $n>1$ and $M_1$ and $M_2$ are closed oriented manifolds, we have
	\[
		\H_k(M_1\+M_2) \cong \begin{cases}
			\H_k(M_1^\circ)\oplus \H_k(M_2^\circ) & 0 < k < n,\\
			\Z & k=0\textrm{ or }n,\\
			0 & \textrm{otherwise.}
		\end{cases}
	\]
\end{proposition}

\begin{example}
	The compact surface $X_g$ of genus $g$ can be broken down as a $g$-repeated connected sum of the torus $T^2=S^1\times S^1$, i.e. $X_g \cong \+[g] T^2$. It follows that by \cref{prop:homology-connected-sum-closed} that the Betti numbers are $\beta_0=1$, $\beta_1=2g$, and $\beta_2=1$.
\end{example}

The preceding is an example of the trivial case of a join of two manifolds. In general, the situation is not as nice. First of all, the submanifold $N$ might not have a trivial normal sphere bundle so computing the homology of the intersection of $M_1^\circ$ and $M_2^\circ$ becomes more complex. Additionally, in many applications the manifolds involved will not be closed, the homology in dimensions $n-1$ and $n$ behaves differently than in \cref{prop:homology-connected-sum-closed}.

\subsection{Joining Manifolds Along Submanifolds of the Boundary}\label{sec-joins-along-boundary-submanifolds}

There is an operation related to the join of two manifolds along a submanifold which is useful throughout geometric topology. Namely, suppose we had a manifold $M$ and a submanifold $N\subset \partial M$. Note that by our assumption that all submanifolds are neat and properly embedded, this requires $N$ to be closed. If we have two manifolds $M_1$ and $M_2$ with embeddings $\iota_1 : N \to \partial M_1$ and $\iota_2 : N \to \partial M_2$, we would like to define a join of these two manifolds along $N$. The trouble is that we do not have tubular neighborhoods since $\iota_i(N)$ does not intersect $\partial M_i$ transversally. To generalize this notion easily, we use a trick.

\begin{definition}
	The \defn{doubling}[doubling of a manifold] of a smooth manifold $M$ with non-empty boundary is the closed smooth manifold $M\cup_{\partial M} M$ obtained by identifying the collar neighborhoods of $\partial M$.
\end{definition}
\begin{figure}[ht]
	\centering
	\import{diagrams}{doubling-manifold.pdf_tex}
	\caption{A doubling of a manifold.}
\end{figure}
Note that the image of $\partial M$ in the doubling $M\cup_{\partial M} M$ is a submanifold. We can glue the images of the left and right collar neighborhoods to get a tubular neighborhood for $\partial M$ in the doubling $M\cup_{\partial M} M$.
In the setup for the join operation, we can then double the manifolds $M_1$ and $M_2$ so that the embeddings $\iota_i$ embed $N$ as a submanifold of the doubled manifolds, perform the ordinary join $(M_1\cup_{\partial M_1} M_2)\cup_N (M_2\cup_{\partial M_2} M_2)$, and finally separate the doubles. See \cref{fig:boundary-connected-sum} for a visual demonstration in the case of an embedded point. We denote the resulting manifold simply by $M_1\cup_N M_2$, although we the choice of embedding matters. This is the \defn{join of two manifolds along a submanifold of the boundary}.

\begin{remark}
	To simplify the verification of technical details, we require that the tubular neighborhoods chosen for $N$ in the doubled manifolds should be restrictions of the tubular neighborhood of $\partial M_i$. Since the boundary has codimension $1$, its tubular neighborhood is a line bundle and hence the removal of the zero section gives two disconnected components. This observation is what allows us to perform the separation back to a manifold with boundary.
\end{remark}

The main application of the operation of a join along a boundary submanifold is when we would like to take the connected sum of two manifolds with non-empty boundary while keeping their boundary connected. In this case, rather than joining along a point in the interior of the manifold like in the case of connected sum, we join along a point on the boundary.

\begin{definition}
	If two manifolds $M_1$ and $M_2$ have connected boundaries, their \defn{boundary connected sum} is their join along an embedded point of their boundaries. We denote this operation $(M_1,\partial M_1)\+ (M_2,\partial M_2)$.
\end{definition}

\begin{figure}[ht]
	\centering
	\import{diagrams}{boundary-connected-sum.pdf_tex}
	\caption{A boundary connected sum of two manifolds.}\label{fig:boundary-connected-sum}
\end{figure}

In fact, this operation restricts to an ordinary connected sum on the boundary itself.

\begin{proposition} We have $\partial\left[(M_1,\partial M_1)\+ (M_2,\partial M_2)\right] = \partial M_1\+\partial M_2$.
\end{proposition}

We will conclude this section with a brief remark on the resulting homology. If we do a Mayer-Vietoris decomposition of $M_1$ and $M_2$, letting $M_i^\circ=M_i\setminus D^n$\footnote{Here $D^n$ intersects the point on the boundary along which we are joining.} as in the case of the ordinary connected sum, the intersection of $M_1^\circ$ and $M_2^\circ$ will be contractible. Since the edge terms of the Mayer-Vietoris sequence vanish, we have a isomorphisms of reduced homology
\[
		\Hr_k(M_1^\circ)\oplus \Hr_k(M_2^\circ) \cong \Hr_k((M_1,\partial M_1)\+ (M_2,\partial M_2))
\]
for all $k$. Alternatively, we could collapse the intersection to a point and conclude that $(M_1,\partial M_1)\+(M_2,\partial M_2)$ has the homotopy type of $M_1\vee M_2$.

\subsection{Smoothing out Corners}\label{sec:smoothing-corners}

It is commonly the case when working with smooth manifolds with non-empty boundary that an operation results in a canonical structure everywhere except for at a codimension $2$ submanifold lying in the boundary. The simplest example of this occurs when we take the product of two smooth manifolds $M_1$ and $M_2$. The boundary of the product is
\[
		\partial(M_1\times M_2) = (\partial M_1 \times M_2) \cup (M_1\times \partial M_2).
\]
At the points $\partial M_1\times \partial M_2\subset \partial (M_1\times M_2)$, we do not have a smooth structure with boundary. Instead, this can be considered as a \defn{manifold with corners}, i.e. we have charts to $[0,\infty)^k\times \R^{n-k}$. We might refer to this as a codimension $k$ corner. Note that when $k=1$, the space $[0,\infty)\times \R^{n-1}$ is just a half-plane and so a codimension $1$ corner is just a point on the boundary. In the case of the product of two manifolds with non-empty boundary, we have codimension $2$ corners. 

We do not have to worry about this technicality in such cases since ``corners can be smoothed'' by introducing a new coordinate system. For instance, a corner $[0,\infty)^2$ can be transformed to a half-plane $[0,\infty)\times \R$ by the transformation
\[
	(r\cos(\theta), r\sin(\theta) )\lkxmapsto (r\cos(2\theta), r\sin(2\theta))
\]
in polar coordinates $r\in[0,\infty)$, $\theta\in [0,\pi/2]$. Of course, this coordinate change is not smooth at the origin $r=0$, but it does allow us to canonically transform any chart to $[0,\infty)^2$ into a chart to $[0,\infty)\times \R$, thus extending the smooth structure in a canonical way. Similar procedures work in higher-dimensions. For some more details on this technicality, see the appendix of \cite{milnor1958manifolds}.
