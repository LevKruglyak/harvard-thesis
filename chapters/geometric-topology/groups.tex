\pagebreak
\section{Groups of Homotopy Spheres}\label{sec:groups-of-homotopy-spheres}

So far, this is a fairly general setup. Instead of spheres, we could use any simply-connected base manifold and consider the set of $h$-cobordism classes of manifolds homotopy equivalent to it. This is known as the \defn{surgery structure set}, one of the central objects of study in surgery theory. In the case of spheres, there is a special bit of extra data which does not usually generalize -- a group structure under the connected sum operation defined in \cref{sec:connected-sum}. 
As we will see in \cref{chap:invariants} and \cref{chap:conclusion}, this group structure on $\Theta^n$ simplifies the work of classification tremendously.

\begin{theorem}\label{thm:group-of-homotopy-spheres}
	The connected sum turns $\Theta^n$ into a group with identity element $S^n$.
\end{theorem}

\begin{proof}
	We have already proved that the connected sum is well-defined up to diffeomorphism, associative, and commutative up to orientation preserving diffeomorphism.  
	The remaining proof splits as three lemmas -- these are Lemmas 2.2, 2.3, and 2.4 in \cite{milnorkervaire1963groups}.

	\begin{lemma}
		The connected sum is a well-defined operation on $\Theta^n$.
	\end{lemma}
	\begin{proof}
		We will prove something slightly more general. Let $M_1, M_1'$ and $M_2, M_2'$ be closed, simply-connected, and oriented manifolds, and suppose $W_1 : M_1\sohbord M_1'$ and $W_2 : M_2\sohbord M_2'$ are $h$-cobordisms. The goal is to construct an $h$-cobordism $W : (M_1\+M_2)\sohbord (M_1'\+M_2')$.

		\begin{figure}[ht]
			\centering
			\import{diagrams}{join-h-cobordism.pdf_tex}
			\caption{A join of two $h$-cobordisms along embedded arcs.}\label{fig:connected-sum-of-h-cobordisms}
		\end{figure}

		Recall that the connected sum $M_1\+M_2$ was defined as the join of $M_1$ and $M_2$ along points $p_1\in M_1$ and $p_2\in M_2$. Let us assume these were the points along which the connected sum $M_1\+M_2$ was taken, and $p_1'\in M_1'$ and $p_2'\in M_2'$ the points along which the connected sum $M_1'\+M_2'$ was taken. Choose smooth paths $\gamma_i : [0,1] \to W_i$ which send $p_i\mapsto p_i'$, and assume without loss of generality that $\gamma_i$ are embeddings which are transverse to the boundary of $\partial W_i$, making it possible to get tubular neighborhoods
		\[
			\lkxfunc{\iota_i}{\R^n\times [0,1]}{W_i.}
		\]
		Let $W$ be the join of $W_1$ and $W_2$ along the image of the arcs $\gamma_i$, as depicted in \cref{fig:connected-sum-of-h-cobordisms}. This is an oriented smooth manifold with boundary
		\[
			\partial W = (M_1\#M_2)\sqcup -(M_1'\+M_2')
		\]
		so $W$ is a cobordism $W : (M_1\+M_2)\sobord (M_1'\+M_2')$. We just need to prove that $W$ is an $h$-cobordism to complete the proof.

		\begin{lemma}\label{lemma:removing-point-homotopy-equivalence-h-cobordism}
			If $W : M \sohbord M'$ is an $h$-cobordism with $M,M'$ compact oriented and simply connected, $p\in M$ is a point, and $T$ is a tubular neighborhood of an arc from $p$ to $p'\in M'$, then the inclusion $M\setminus\{p\} \to W\setminus T$ is a homotopy equivalence.
		\end{lemma}
		\begin{proof}
			The long exact sequence of the inclusion $j : (M,M\setminus\{p\}) \to (W, W\setminus T)$ gives
			\[
				\H_{i+1}(W\setminus T, M\setminus\{p\})\lkxto
				\H_i(M\setminus\{p\}) \lkxto[j_*] \H_i(W\setminus T) \lkxto \H_i(W\setminus T, M\setminus\{p\})
			\]
			Note that by excision, $\H_i(W\setminus T, M\setminus\{p\})\cong \H_i(W,M)$, and since $W$ is an $h$-cobordism, $\H_i(W,M)\cong 0$. Consequently, $j$ induces isomorphisms on homology and since $M,M'$, and $W$ are simply connected, Whitehead's theorem and the Hurewicz isomorphism imply that $j$ is a homotopy equivalence.
		\end{proof}

		With \cref{lemma:removing-point-homotopy-equivalence-h-cobordism} in mind, we use the Mayer-Vietoris sequence
		\[\begin{tikzcd}
				{\H_i(S^{n-1})} & {\H_i(M_1\setminus\{p_1\})\oplus\H_i(M_2\setminus\{p_2\})} & {\H_{i}(M_1\+M_2)} & {\H_{i-1}(S^{n-1})} \\
				{\H_i(S^n)} & {\H_i(W_1\setminus T_1)\oplus\H_i(W_2\setminus T_2)} & {\H_{i}(W)} & {\H_{i-1}(S^n)}
				\arrow[from=1-1, to=1-2]
				\arrow[from=1-1, to=2-1]
				\arrow["f"', from=1-2, to=1-3]
				\arrow["{(j_1)_*\oplus (j_2)_*}", from=1-2, to=2-2]
				\arrow[from=1-3, to=1-4]
				\arrow["j_*", from=1-3, to=2-3]
				\arrow[from=1-4, to=2-4]
				\arrow[from=2-1, to=2-2]
				\arrow["g"', from=2-2, to=2-3]
				\arrow[from=2-3, to=2-4]
			\end{tikzcd}\]
			where $j_i : M_i\setminus \{p_i\} \to W_i\setminus T_i$ and $j : M_1\+ M_2\to W$ are inclusions. Note that $(j_1)_*\oplus (j_2)_*$ is an isomorphism by \cref{lemma:removing-point-homotopy-equivalence-h-cobordism}. When $i$ is not $1,n-1,$ or $n$, it follows immediately that $f$ and $g$ are isomorphisms and so $j$ is an isomorphism. In the remaining cases, a simple application of the  snake lemma does the trick.
			By simple connectedness, Whitehead's theorem, and the Hurewicz isomorphism, it follows that the inclusion of $M_1\+M_2 \to W$ is a homotopy equivalence, completing the proof.
	\end{proof}

	\begin{remark}
		In the previous lemma, the simple-connectedness assumption was not strictly required -- only used to reduce the problem of homotopy equivalence to a homology problem by Whitehead's theorem. It is possible to directly construct an explicit homotopy equivalence with a more careful argument.
	\end{remark}

	Next, we will prove a simple characterization of when a homotopy sphere is equal to the identity element in $\Theta^n$.

	\begin{lemma}\label{lemma:null-h-cobordant-iff-bounds-contractible}
		A simply-connected manifold $M$ is $h$-cobordant to $S^n$ if and only if $M$ bounds a contractible manifold.
	\end{lemma}
	\begin{proof}
		The forward direction is fairly straightforward. If $W : M\sohbord S^n$ is an $h$-cobordism, we can glue a tubular neighborhood of $D^{n+1}$ in $\R^{n+1}$ to $(-S^n)\subset \partial W$ along a collar neighborhood of $-S^n$ in $\partial W$. Any homotopy equivalence of $W$ to $S^n$ can then be extended to a contraction along $D^{n+1}$.

		In the reverse direction, let us assume that $M$ bounds a contractible manifold $W$. Pick any point in the interior of $M$ and remove the interior of an embedded disk $D^{n+1}\subset W$. This gives a simply-connected cobordism $W'=W\setminus \Int(D^{n+1}) : M\sobord (-S^n)$. By the long exact sequence of the inclusion $j : (D^{n+1}, S^n) \to (W,W')$, we have
		\[
			\H_i(W', S^n) \lkxto H_i(S^n)\lkxto[j_*] \H_i(W')\lkxto \H_{i-1}(W',S^n),
		\]
		and by contractability of $W$ the boundary terms vanish and so $S^n \to W'$ is a homotopy equivalence. Then, by the relative Poincar\'e duality isomorphism $\H_k(W',M)\cong \H^{n+1-k}(W', S^n)$ and the long exact sequence of the pair $(W',M)$, it follows that $M\to W'$ is a homotopy equivalence and so $W'$ is in fact an $h$-cobordism. 
	\end{proof}

	\begin{lemma}
		If $M$ is a homotopy sphere, then $M\+\overline{M}$ bounds a contractible manifold.
	\end{lemma}
	\begin{proof}
		There is a thorough proof of this in \cite{milnorkervaire1963groups} (see Lemma 2.4) so we only sketch a geometric picture here. The connected sum $M\+\overline{M}$ can be thought of as a union of punctured spaces $M^\circ$ and $\overline{M}^\circ$. Turning the punctured homotopy sphere $M^\circ$ inside out through the opening gives the reverse orientation $\overline{M}^\circ$.

		\begin{figure}[ht]
			\centering
			\import{diagrams}{group-inverse-lemma.pdf_tex}
			\caption{Turning a homotopy sphere inside out through a puncture.}\label{fig:group-inverse-lemma}
		\end{figure}

		\noindent
		Taking the union of the intermediate stages of this inversion procedure gives a contractible manifold $W$ which bounds $M\+\overline{M}$, as in \cref{fig:group-inverse-lemma}.
	\end{proof}

	\noindent
	These lemmas complete the proof.
\end{proof}

% \subsection{The Subgroup \texorpdfstring{$\bP^{n+1}$}{bP^{n+1}}}\label{sec:framed-manifolds}

The full group $\Theta^n$ is often too complicated for direct classification. There is a subgroup, denoted $\bP^{n+1}$, of ``simpler'' homotopy spheres which are generally much easier to work with. In fact, all direct constructions of exotic spheres in this thesis are of these simpler homotopy spheres. Homotopy spheres in the complement $\Theta^n\setminus \bP^{n+1}$ of this subgroup are referred to as \defn{very exotic spheres}, and can only be detected and constructed via homotopy theoretic means. The relationship between $\bP^{n+1}$ and $\Theta^n$ will be discussed in \cref{chap:conclusion}.

\begin{definition}
	Let $\bP^{n+1}\subset \Theta^n$ be the set of homotopy spheres which have a parallelizable coboundary.
\end{definition}

Clearly, the identity $S^n$ is in this subset, and the subset is clearly closed under inversion. It is a bit tricker to see why it is closed under the group operation of connected sum. 

\begin{proposition}
	$\bP^{n+1}$ is a subgroup of $\Theta^n$.
\end{proposition}
\begin{proof}
	It suffices to show that $\bP^{n+1}$ is closed under connected sum. Suppose $M_1$ and $M_2$ are homotopy spheres bound by parallelizable manifolds $W_1$ and $W_2$.Note that the connected sum of $M_1$ and $M_2$ has a coboundary $W=(W_1,\partial W_1)\+(W_2,\partial W_2)$.

	\begin{figure}[ht]
		\centering
		\import{diagrams}{join-vector-bundle.pdf_tex}
		\caption{A join of vector bundles.}
	\end{figure}
	
	Now let us choose trivializations $\varphi_i : W_i\times \R^{n+1} \to \T W_i$ for the original coboundaries. If $p$ is the point along which $M_1$ and $M_2$ are glued, we have a fiber embeddings $\R^{n+1}\to \T_p W_i$. Then, we choose a tubular neighborhood of the fiber which agrees with the tangent bundle structure so that the join $\T W_1\cup_{\R^{n+1}}\T W_2$ is a fiber bundle over $W$. We also get a trivialization $\varphi : W\times \R^{n+1} \to \T W$ which restricts to $\varphi_i$.
\end{proof}
