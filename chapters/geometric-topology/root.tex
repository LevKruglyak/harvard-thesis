\chapter{Tools of Geometric Topology}\label{chap:fundamentals}

% \begin{epigraph}{21em}{Pythagoras}
% 	There is geometry in the humming of the strings,\\
% 	and there is music in the spacing of the spheres
% \end{epigraph}
%
%
% \begin{epigraph}{15em}{Frank Herbert}
% 	Beginnings are such delicate times.
% \end{epigraph}

\begin{epigraph}{20em}{Sir Michael Atiyah}
	Algebra is the offer made by the devil to the \\
	mathematician. The devil says: ``I will give \\
	you this powerful machine, it will answer any \\
	question you like. All you need to do is give\\
	me your soul: give up geometry.
\end{epigraph}

\noindent
To begin our journey, we will provide some geometric topology essentials which will be used ubiquitously throughout the thesis.

First, in \cref{sec:smooth-manifold-operations} we will discuss operations on smooth manifolds -- how to glue manifolds together, cut off handles, attach handles, stitch together manifolds at multiple points, and so on. The two operations most of interest to us are the \defn{connected sum}, which glues together two manifolds of the same dimension, and \defn{surgery}, which adds or removes handles. 
Armed with a powerful set of topological cutlery, we begin to investigate how smooth manifold operations change the topology of the manifold. 

In \cref{sec:intersection-form}, we spend a fair bit of time discussing the \defn{intersection form}, a bilinear form which captures a tremendous amount of topological information 


\todo{what other sections we need}

\subimport{/}{operations}
\subimport{/}{intersection}
\subimport{/}{morse}
\subimport{/}{cobordism}
\subimport{/}{groups}
