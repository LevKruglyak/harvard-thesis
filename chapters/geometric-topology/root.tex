\chapter{Tools of Geometric Topology}\label{chap:fundamentals}

% \begin{epigraph}{21em}{Pythagoras}
% 	There is geometry in the humming of the strings,\\
% 	and there is music in the spacing of the spheres
% \end{epigraph}
%
%
% \begin{epigraph}{15em}{Frank Herbert}
% 	Beginnings are such delicate times.
% \end{epigraph}

\begin{epigraph}{20em}{Sir Michael Atiyah}
	Algebra is the offer made by the devil to the \\
	mathematician. The devil says: ``I will give \\
	you this powerful machine, it will answer any \\
	question you like. All you need to do is give\\
	me your soul: give up geometry.
\end{epigraph}

\noindent
To begin our journey, we review some fundamental tools of geometric topology.

We start in \cref{sec:smooth-manifold-operations} with a discussion of operations on manifolds and the technicalities which arise in the smooth case. Aside from standard operations on general topological spaces, we introduce the connected sum and boundary connected sum for manifolds with boundary. Both of these perspectives can be viewed as special cases of a more general operation of a join of two manifolds along a submanifold. Our next topic in \cref{sec:intersection-form} is the intersection form, a fundamental invariant of a manifold which encodes the data of middle-dimensional submanifold intersections. The intersection form relates this geometric data to the algebraic data of cohomology, and forms a wide bridge by which geometric problems can be transferred to algebraic ones. Related to this is the topic of the Euler class in \cref{sec:euler-class}, our first example of a characteristic class. For us, the Euler class provides a number of bridges linking intersection theory with the geometry of vector bundles, obstruction theory, characteristic classes, and others.

Finally, in section \cref{sec:cobordism} we discuss cobordism, a loose equivalence for manifolds which allows for complete classification. Cobordism serves two essential purposes in this thesis. First of all, we note that many topological invariants are invariant under cobordism. A complete determination of manifolds up to cobordism can thus help us relate two topological invariants, since we need only check the representative classes. As for the second application, there is a strengthening of the condition of cobordism known as $h$-cobordism. 
In dimensions $\geq 5$, a non-trivial result known as the $h$-cobordism theorem then turns the problem the existence problem for diffeomorphisms between two manifolds into an existence problem for a manifold with a certain homotopy type.
Such a reduction massively simplifies the problem of finding exotic spheres, and we conclude with these simplifications in \cref{sec:groups-of-homotopy-spheres}.

\pagebreak
\section{Smooth Manifolds Operations}\label{sec:smooth-manifold-operations}

Cutting and pasting is the category of smooth manifolds is a subtle craft. 
Unlike in the topological manifold category, combining manifolds resembles ``sewing'' more than ``gluing''. In order to ensure that the resulting manifold will have a unique smooth structure, we must identify the spaces in a small region surrounding the submanifold. The extra space provided by the region allows one manifold to ``smoothly curve into'' the other.

\begin{remark} 
	We will see in \cref{sec:twisted-spheres} that exotic spheres can be formed by identifying two disks along \emph{just} their boundary by an orientation preserving diffeomorphism. This always gives an ordinary sphere in the topological category but not in the smooth category.
\end{remark}

We begin with some technical lemmas from differential topology. Whenever a manifold $N^k$ is embedded into an ambient manifold $M^n$, there is a short exact sequence of vector bundles on $N$
\begin{equation}
	0 \lkxto \TT N\lkxto \TT M|_N \lkxto \TT M/N \lkxto 0.
\end{equation}
Here $\TT M/N=\TT M|_N / \TT N$ denotes the \defn{normal bundle} of $N$ in $M$, which has rank $(n-k)$.

A \defn{tubular neighborhood} of $N$ is a neighborhood $\widetilde{N}\supset N$ in $M$ which is the diffeomorphic image of the total space of the normal bundle $\T M/N$ with the diffeomorphism $\tau : \T M/N \to \widetilde{N}$ satisfying:
\begin{enumerate}[(a)]
	\item $\tau$ is the identity on the image of the zero section $s_0 : N \to \T M/N$,
	\item For any point $p\in \partial N$ on the boundary, $\tau(\T_p M/N)\subset \partial M$.
\end{enumerate}


A tubular neighborhood should be thought of as a way of ``thickening'' the submanifold $N$ to have the same dimension as $M$, as depicted in \cref{fig:tubular-neighborhood}. 

\begin{figure}[ht]
	\centering
	\import{diagrams}{placeholder.pdf_tex}
	\caption{Tubular neighborhoods of some submanifolds.}\label{fig:tubular-neighborhood}
\end{figure}

The diffeomorphism $\tau : \T M/N \to \widetilde{N}$ can interpreted as a chart of a ``tube'' surrounding $N$ in $M$. Namely, for each point $p\in N$, local trivializations of the normal bundle $\T M/N$ are maps $\R^k \times \R^{n-k}\to U\subset \widetilde{N}$ which consist of coordinates $(x^1,\ldots,x^k)$ on $N$ and coordinates $(v^{k+1},\ldots, v^{n})$ on an orthogonal plane to $N$.

\begin{remark}
	A common convention is to define a tubular neighborhood as the diffeomorphic image of the associated disk bundle $\D(\T M/N)$. This convention makes explicit the ``tube'' of a  ``tubular neighborhood''. We will refer to such neighborhoods as \defn{closed tubular neighborhoods}[closed tubular neighborhood], since they are the closure of a tubular neighborhood.
\end{remark}

This leads to a basic technical lemma of differential topology.

\begin{theorem}[Tubular Neighborhood Theorem]\label{thm:tubular-neighborhood}
	Every embedded submanifold has a tubular neighborhood.\footnote{This result does not require $M$ to be compact.}
\end{theorem}
\begin{proof}
	See Chapter II of \cite{kosinski1993differential} or Theorem 6.24 of \cite{lee2013smooth} for the case $M=\R^n$.
\end{proof}

\begin{warning}
	In Conventions, we assumed all smooth manifolds to be compact and all submanifolds to be properly embedded and neat. Absent these assumptions, the tubular neighborhood theorem as stated is not true.
\end{warning}

In a similar vein, when we have just a manifold $M^n$, a \defn{collar neighborhood} of its boundary is a neighborhood $\widetilde{\partial M}\supset \partial M$ which is the diffeomorphic image of a trivial ``ray bundle'' $\partial M\times [0,\infty)$. Just as for tubular neighborhoods, we require that the diffeomorphism is the identity on the zero-section $\partial M\times\{0\}$.

\begin{theorem}[Collar Neighborhood Theorem]\label{thm:collar-neighborhood}
	Every manifold with non-empty boundary has a collar neighborhood.
\end{theorem}
\begin{proof}
	See Theorem 9.25 of \cite{lee2013smooth}.
\end{proof}

\begin{proposition}
	If $N\subset M$ is a submanifold, there is a collar neighborhood of $\partial M$ which restricts to a collar neighborhood of $\partial N$ in $N$.
\end{proposition}
\begin{proof}
	See Proposition~2.8.2 in Chapter II of \cite{kosinski1993differential}.
\end{proof}

\begin{figure}[ht]
	\centering
	\import{diagrams}{placeholder.pdf_tex}
	\caption{Collar neighborhoods.}\label{fig:collar-neighborhoods}
\end{figure}

\subsection{Joining Two Manifolds Along a Submanifolds}

We can now address one of the fundamental operations in geometric topology -- the joining of two manifolds by submanifolds.

Let us suppose we have manifolds $M_1^n$ and $M_2^n$ of the same dimension, with embeddings $\iota_1 : N\to M_1$ and $\iota_2 : N\to M_2$ of a manifold $N^k$. Our goal is to construct a joined manifold $M$ which we denote $M_1\cup_N M_2$.
We start by picking tubular neighborhoods $\widetilde{N_i}$ for $\iota_i(N)\subset M_i$ and let $\tau_i: N\times \R^{n-k} \to \widetilde{N_i}$ be the diffeomorphism.
We define $M$ to be the quotient
\begin{equation}\label{eq:join-definition}
	M_1\cup_N M_2 = \frac{M_1\setminus \iota_1(N)\sqcup M_2\setminus \iota_2(N)}{\tau_1(p, t\xi) \sim \tau_2(p, t^{-1}\xi)}
\end{equation}
for all $p\in N$, $t\in(0,\infty)$, and unit vectors $\xi\in S^{n-k-1}\subset \R^{n-k}$. The resulting smooth manifold $M=M_1\cup_N M_2$ is called a \defn{join along a submanifold}\footnote{Some authors refer to this operation as ``\defn{pasting}'' or as the ``\defn{generalized connected sum}'', see for instance Section VI.4 in \cite{kosinski1993differential}.} and is one of the fundamental operations of geometric topology.

\begin{figure}[ht]
	\centering
	\import{diagrams}{placeholder.pdf_tex}
	\caption{A join of two manifolds along a submanifold.}\label{fig:join-along-submanifold}
\end{figure}

\begin{remark}
	As we have remarked, the ``sewing'' procedure in \cref{eq:join-definition} by tubular neighborhoods is vital in the category of smooth manifolds. In the category of topological manifolds\footnote{Assuming, of course, that there is a topological notion of  tubular neighborhoods.}, we could define the join by removing the tubular neighborhoods entirely (rather the just the zero sections) and identifying the resulting boundaries.
	At each point of the shared boundary, the join would be locally Euclidean since it has a neighborhood which the gluing of two half-disks to get a full disk.
\end{remark}

\begin{proposition}\label{prop:join-along-submanifolds-well-defined}
	The join $M$ of two manifolds along a submanifold has a smooth structure that agrees with the smooth structures of $M_1\setminus \iota_1(N)$ and $M_2\setminus \iota_2(N)$ in $M$. 
\end{proposition}
\begin{proof}
	There is a technical but simple proof which involves writing out the transition functions for charts on $M$ arising from charts on $M_1$ and $M_2$ which did not intersect $\iota_1(N)$ or $\iota_2(N)$. The result then follows from the smoothness of $t \mapsto t^{-1}$ on $t\in (0,\infty)$.
\end{proof}

\begin{proposition}\label{prop:join-along-submanifolds-orientation}
	If $M_1$, $M_2$, and $N$ are oriented, with $\iota_1$ orientation-preserving and $\iota_2$ orientation-reversing, then $M$ has an orientation compatible with $M_1\setminus \iota_1(N)$ and $M_2\setminus \iota_2(N)$.
\end{proposition}
\begin{proof}
	This follows since $t\mapsto t^{-1}$ and $\iota_2$ both reverse orientations. We omit the technical details for brevity.
\end{proof}

\begin{theorem}
	Up to orientation-preserving diffeomorphism, the join of manifolds along a submanifold is independent of the choices of tubular neighborhood.
\end{theorem}
\begin{proof}
	\todo{cite}
\end{proof}

\subsection{Connected Sum}\label{sec:connected-sum}

The simplest submanifold by which to join two manifolds is a point.

\begin{definition}
	The \defn{connected sum} $M_1\+M_2$ of two manifolds $M_1$ and $M_2$ is their join along an embedded point.
\end{definition}

Visually, this operation can be thought of as cutting disks out of both manifolds and connecting them with a tube, as depicted in \cref{fig:connected-sum}.

\begin{remark}
	Note that by our assumption of submanifolds preserving boundary, the point along which the manifolds are joined cannot lie on the boundary of either manifold.
\end{remark}

\begin{figure}[ht]
	\centering
	\import{diagrams}{placeholder.pdf_tex}
	\caption{A connected sum of two surfaces.}\label{fig:connected-sum}
\end{figure}

This operation is (up to orientation-preserving orientation) independent of the choice of embedding, justifying the basepoint-free notation of $M_1\+ M_2$.\footnote{As stated in Conventions, we assume that all manifolds are connected. Otherwise the connected sum would only be well-defined assuming a choice of connected component of $M_1$ and $M_2$.}
Proving the independence of connected sum from the choice of basepoint is non-trivial, and follows from a technical result by Richard Palais.

\begin{theorem}[Disk Theorem]\label{thm:disk}
	If $M^n$ is an oriented manifold and $\iota_1, \iota_2 : D^n \to M$ are orientation-preserving disk embeddings, then there is a diffeomorphism $f : M \to M$ such that $\iota_1 = f\circ \iota_2$.
\end{theorem}
\begin{proof}
	See Theorem B in \cite{palais1960diffeomorphism}.
\end{proof}

\begin{corollary}\label{cor:connected-sum-operation}
		The connected sum is well-defined, associative, and commutative up to orientation-preserving diffeomorphism.
\end{corollary}

\begin{corollary}
	For any manifold $M^n$, there is a diffeomorphism $M\+ S^n\cong M$.
\end{corollary}

\begin{corollary}
	$\partial(M_1\+M_2) = \partial M_1\sqcup \partial M_2$.
\end{corollary}

Let us now briefly discuss the effect of connected sum on homology. If we have two oriented manifolds $M_1$ and $M_2$ of dimension $n>1$, their connected sum $M_1\# M_2$ can be decomposed as a union of open sets diffeomorphic to $M_1\setminus\{p\}$ and $M_2\setminus\{p\}$. We denote these open sets $M_1^\circ$ and $M_2^\circ$ respectively. Their intersection is diffeomorphic to a tubular neighborhood of $S^{n-1}$, so by the Mayer-Vietoris sequence, we have an exact sequence
\[
	\H_k(S^{n-1})\lkxto \H_k(M_1^\circ)\oplus \H_k(M_2^\circ)\lkxto[p_k] \H_k(M_1\+M_2)\lkxto \H_{k-1}(S^{n-1})
\]
In lowest dimension $k=0$, we know that $\H_0(M_1\+ M_2)\cong \Z$ since the connected sum is connected. Next, when $k=1$, the edge term $\H_{k-1}(S^{n-1})$ is non-trivial, but the kernel is trivial by a rank argument in the $k=0$ segment and so $p_1$ is also an isomorphism.
When $k$ is not $n-1$, the boundary terms vanish completely and so $p_k$ is also an isomorphism. In the remaining cases, we get the exact sequence
\begin{equation}\label{eq:connected-sum-cohomology}
	\begin{tikzcd}
	0 & {\H_n(M_1^\circ)\oplus \H_n(M_2^\circ)} & {\H_n(M_1\+M_2)} & {\H_{n-1}(S^{n-1})} \\
	0 & {\H_{n-1}(M_1\+M_2)} & {\H_{n-1}(M_1^\circ)\oplus \H_{n-1}(M_2^\circ)}
	\arrow[from=1-1, to=1-2]
	\arrow["p_n", from=1-2, to=1-3]
	\arrow["q", from=1-3, to=1-4]
	\arrow[from=1-4, to=2-3]
	\arrow[from=2-2, to=2-1]
	\arrow["p_{n-1}"',from=2-3, to=2-2]
\end{tikzcd}
\end{equation}
If $M_1$ and $M_2$ are both closed, so is $M_1\+M_2$, and hence $q$ is an isomorphism by a fundamental class argument. This implies that $p_{n-1}$ is an isomorphism, so we conclude:

\begin{proposition}\label{prop:homology-connected-sum-closed}
	If $n>1$ and $M_1$ and $M_2$ are closed oriented manifolds, we have
	\[
		\H_k(M_1\+M_2) \cong \begin{cases}
			\H_k(M_1^\circ)\oplus \H_k(M_2^\circ) & 0 < k < n,\\
			\Z & k=0\textrm{ or }n,\\
			0 & \textrm{otherwise.}
		\end{cases}
	\]
\end{proposition}

\begin{example}
	The compact surface $X_g$ of genus $g$ can be broken down as a $g$-repeated connected sum of the torus $T^2=S^1\times S^1$, i.e. $X_g \cong \+[g] T^2$. It follows that by \cref{prop:homology-connected-sum-closed} that the Betti numbers are $\beta_0=1$, $\beta_1=2g$, and $\beta_2=1$.
\end{example}

The preceding is an example of the trivial case of a join of two manifolds. In general, the situation is not as nice. First of all, the submanifold $N$ might not have a trivial normal sphere bundle so computing the homology of the intersection of $M_1^\circ$ and $M_2^\circ$ becomes more complex. Additionally, in many applications the manifolds involved will not be closed, the homology in dimensions $n-1$ and $n$ behaves differently than in \cref{prop:homology-connected-sum-closed}.

\subsection{Joining Manifolds Along Submanifolds of the Boundary}\label{sec-joins-along-boundary-submanifolds}

There is an operation related to the join of two manifolds along a submanifold which is useful throughout geometric topology. Namely, suppose we had a manifold $M$ and a submanifold $N\subset \partial M$. Note that by our assumption that all submanifolds are neat and properly embedded, this requires $N$ to be closed. If we have two manifolds $M_1$ and $M_2$ with embeddings $\iota_1 : N \to \partial M_1$ and $\iota_2 : N \to \partial M_2$, we would like to define a join of these two manifolds along $N$. The trouble is that we do not have tubular neighborhoods since $\iota_i(N)$ does not intersect $\partial M_i$ transversally. To generalize this notion easily, we use a trick.

\begin{definition}
	The \defn{doubling}[doubling of a manifold] of a smooth manifold $M$ with non-empty boundary is the closed smooth manifold $M\cup_{\partial M} M$ obtained by identifying the collar neighborhoods of $\partial M$.
\end{definition}
\begin{figure}[ht]
	\centering
	\import{diagrams}{placeholder-small.pdf_tex}
	\caption{A doubling of a manifold.}
\end{figure}
Note that the image of $\partial M$ in the doubling $M\cup_{\partial M} M$ is a submanifold. We can glue the images of the left and right collar neighborhoods to get a tubular neighborhood for $\partial M$ in the doubling $M\cup_{\partial M} M$.
In the setup for the join operation, we can then double the manifolds $M_1$ and $M_2$ so that the embeddings $\iota_i$ embed $N$ as a submanifold of the doubled manifolds, perform the ordinary join $(M_1\cup_{\partial M_1} M_2)\cup_N (M_2\cup_{\partial M_2} M_2)$, and finally separate the doubles. See \cref{fig:boundary-connected-sum} for a visual demonstration in the case of an embedded point. We denote the resulting manifold simply by $M_1\cup_N M_2$, although we the choice of embedding matters. This is the \defn{join of two manifolds along a submanifold of the boundary}.

\begin{remark}
	To simplify the verification of technical details, we require that the tubular neighborhoods chosen for $N$ in the doubled manifolds should be restrictions of the tubular neighborhood of $\partial M_i$. Since the boundary has codimension $1$, its tubular neighborhood is a line bundle and hence the removal of the zero section gives two disconnected components. This observation is what allows us to perform the separation back to a manifold with boundary.
\end{remark}

\begin{figure}[ht]
	\centering
	\import{diagrams}{placeholder.pdf_tex}
	\caption{A boundary connected sum of two manifolds.}\label{fig:boundary-connected-sum}
\end{figure}

The main application of the operation of a join along a boundary submanifold is when we would like to take the connected sum of two manifolds with non-empty boundary while keeping their boundary connected. In this case, rather than joining along a point in the interior of the manifold like in the case of connected sum, we join along a point on the boundary.

\begin{definition}
	If two manifolds $M_1$ and $M_2$ have connected boundaries, their \defn{boundary connected sum} is their join along an embedded point of their boundaries. We denote this operation $(M_1,\partial M_1)\+ (M_2,\partial M_2)$.
\end{definition}

In fact, this operation restricts to an ordinary connected sum on the boundary itself.

\begin{proposition} We have $\partial\left[(M_1,\partial M_1)\+ (M_2,\partial M_2)\right] = \partial M_1\+\partial M_2$.
\end{proposition}

We will conclude this section with a brief remark on the resulting homology. If we do a Mayer-Vietoris decomposition of $M_1$ and $M_2$, letting $M_i^\circ=M_i\setminus D^n$\footnote{Here $D^n$ intersects the point on the boundary along which we are joining.} as in the case of the ordinary connected sum, the intersection of $M_1^\circ$ and $M_2^\circ$ will be contractible. Since the edge terms of the Mayer-Vietoris sequence vanish, we have a isomorphisms of reduced homology
\[
		\Hr_k(M_1^\circ)\oplus \Hr_k(M_2^\circ) \cong \Hr_k((M_1,\partial M_1)\+ (M_2,\partial M_2))
\]
for all $k$. Alternatively, we could collapse the intersection to a point and conclude that $(M_1,\partial M_1)\+(M_2,\partial M_2)$ has the homotopy type of $M_1\vee M_2$.

\pagebreak
\section{The Intersection Form}\label{sec:intersection-form}

One of the fundamental invariants for even-dimensional manifolds is the intersection form, a bilinear form on a lattice which captures the geometric data of submanifold intersections. The lattice in question is the free component of the middle-dimensional singular homology, and the pairing of two homology cycles by the form counts their number of ``intersections''. When the homology cycles of complementary dimension (as in the case of middle-dimensional homology cycles) are represented by smooth immersions, we can perturb them to make them transverse without changing the homology class. If the manifolds are compact, the preimage of their intersection is some finte set of points with orientation -- adding up the orientations of the points gives the oriented intersection number.

This is the geometric interpretation of intersection, and we will explore it in more depth when we construct manifolds with given intersection theory in \cref{sec:plumbing}. For now, we will stick to understanding the algebraic properties of intersections as the adage ``think in terms of intersections, prove in terms of homology'' advises. To start, let us suppose that $M$ is a compact oriented $n$-manifold.
The Poincar\'e-Lefschetz duality gives an isomorphism
\begin{equation}
	\lkxfunc{}{\H^{n-p}(M,\partial M)}{\H_p(M)}{\omega}{\omega\frown [M,\partial M]}
\end{equation}
assuming we have an orientation class $[M,\partial M]\in \H_n(M, \partial M)$ corresponding to the orientation of $M$. Under this duality, the intersection of homology classes is defined as the dual operation to the operation of cup product on cohomology classes. This operation is denoted $\alpha\cdot \beta$ for homology cycles $\alpha\in \H_p(M)$ and $\beta\in \H_q(M)$, and is the top map in \cref{eq:homology-intersection}
\begin{equation}\label{eq:homology-intersection}
	\begin{tikzcd}
		{\H_p(M)\otimes \H_q(M)} & {\H_{n-p-q}(M)} \\
		{\H^{n-p}(M, \partial M)\otimes \H^{n-q}(M,\partial M)} & {\H^{2n-p-q}(M,\partial M)}
		\arrow["\tnsv", from=1-1, to=1-2]
		\arrow[leftrightarrow, from=1-1, to=2-1]
		\arrow[leftrightarrow, from=1-2, to=2-2]
		\arrow["\smile", from=2-1, to=2-2]
	\end{tikzcd}
\end{equation}
where the vertical maps are the Poincar\'e-Lefschetz isomorphism. Again, the intuition here should be that $\alpha\cdot \beta$ is the homology class representing the intersection of $\alpha$ and $\beta$ when they are arranged in general ``transverse'' position. We discuss this interpretation and the definitions of intersection numbers in \cref{sec:submanifolds-intersection-theory}.
Done over homology classes of complementary dimension, the resulting homology intersection class is 0-dimensional and hence pairs with an integer multiple $\ell\cdot [M, \partial M]\in \H_n(M,\partial M)$ of the top-dimensional orientation class. The integer multiple $\ell\in \Z$ is the \defn{intersection number}[intersection number of homology classes] of the cycles $\alpha$ and $\beta$. Removing torsion elements and working in middle dimensional homology so that $\alpha$ and $\beta$ live in the same group, we get an integral bilinear form.

\begin{definition}
	Let $M^{2m}$ be a compact oriented even-dimensional manifold, possibly with boundary. The \defn{intersection form} on middle dimensional homology is the bilinear form
	\begin{equation}
		\lkxfunc{Q_M}{\H_m(M)_{\mathrm{free}}\otimes \H_m(M)_{\mathrm{free}}}{\Z}{a\otimes b}{a \tnsv b}
	\end{equation}
	where we identify $\H_0(M)\cong \Z$ and $\H_m(M)_{\mathrm{free}}$ denotes the free component of $\H_m(M)$ -- i.e. the quotient by the subgroup of torsion elements.
\end{definition}

\begin{remark}
	If $m$ is even then $Q_X$ is a symmetric bilinear form and if $m$ is odd then $Q_X$ is a skew-symmetric bilinear form. This follows from the graded commutativity of the cup product, and hence the intersection pairing. For brevity, we say that $Q_X$ is \defn{$(-1)^m$-symmetric} in such cases.
\end{remark}

\begin{remark}
	Note that the intersection form is defined entirely topologically, without a requirement of smooth structure. That being said, the existence of a smooth structure on a manifold can lead to noticeable effects on the smooth structure.
\end{remark}

It is often helpful to work with the dual pairing, i.e. the cup product pairing on cohomology, since it can be immediately deduced from the multiplicative structure of the cohomology ring.
\begin{definition}
	The intersection form on cohomology is the bilinear form
	\begin{equation}
		\lkxfunc{Q^M}{\H^m(M,\partial M)_{\mathrm{free}}\otimes \H^m(M,\partial M)_{\mathrm{free}}}{\Z}{\alpha\otimes \beta}{\alpha\smile \beta}
	\end{equation}
	where we identify $\H^n(M,\partial M)\cong \H_0(M)\cong \Z$.
\end{definition}

\begin{remark}
	For manifolds which do not come with an orientation, the intersection form can be extended in homology/cohomology with $\Z/2$ coefficients. We call this form the \defn{unoriented intersection form}, and denote it by $\widetilde{Q}_M$ or $\widetilde{Q}^M$ depending on if we are working with homology or cohomology. In the context of embedded submanifolds, this form captures the number of transverse intersection points modulo 2, otherwise known as the unoriented intersection number.
\end{remark}

\begin{remark} \label{rmk:dual-lattice-intersection-form}
	To see the connection between the intersection form on homology and cohomology, let us recall the universal coefficients theorem for cohomology, which gives us an exact sequence
	\[
		0 \lkxto \Ext^1(\H_{m-1}(M,\partial M)) \lkxto \H^m(M, \partial M) \lkxto \Hom(\H_m(M, \partial M), \Z) \lkxto 0.
	\]
	This is Theorem 3.2 of \cite{hatcher2002topology}. When working with the intersection form, we only care about the torsion free component of homology and cohomology. Note that the $\Ext$ term maps entirely into the torsion part of cohomology $\H^m(M,\partial M)$ since $\Ext^1(F\oplus T; \Z)\cong T$ whenever $F$ is free and $T$ is torsion. We thus get a canonical isomorphism
	\[
		\H^m(M, \partial M) \lkxisom \Hom(\H_m(M,\partial M), \Z).
	\]
	When $\H_{m}(\partial M)$ and $\H_{m-1}(\partial M)$ are trivial, the middle map in
	\[
		\H_m(\partial M) \lkxto \H_m(M) \lkxto \H_m(M,\partial M) \lkxto \H_{m-1}(\partial M)
	\]
	is an isomorphism, and so we get a canonical isomorphism
	\[
		\H^m(M, \partial M) \lkxisom \Hom(\H_m(M), \Z).
	\]
	In other words, (under suitable topological restrictions of $\partial M$) there is a canonical way to identify the lattice for the cohomology intersection form as the dual of the lattice for the homology intersection form. In particular, the matrix representations of the bilinear forms are inverses of each other.
\end{remark}

\subsection{Basic Examples of the Intersection Form}

We will see many examples of manifolds and their intersection forms throughout this thesis, so for now let's just consider the most basic examples -- complex projective spaces and tori.

\begin{proposition}\label{prop:intersection-form-complex-projective-plane}
	The intersection form for any complex projective plane $\CP^{2m}$ of even complex dimension is given by $Q=(1)$, and the intersection form for complex projective plane $\CP^{2m+1}$ of odd complex dimension is trivial.
\end{proposition}
\begin{proof}
	We will compute the intersection form in cohomology, by \cref{rmk:dual-lattice-intersection-form} we can invert the resulting matrix to get an intersection form on homology.

	Recall that the cohomology ring of complex projective space is given by
	\begin{equation}
		\H^\bullet(\CP^{n}) \approx \Z[\alpha]/(\alpha^{n+1})\quad\textrm{with}\quad |\alpha|=2.
	\end{equation}
	A proof of this can be found in any standard algebraic topology book, for instance Theorem~3.19 in \cite{hatcher2002topology}. We can assume without loss of generality that $\alpha^n\in \H^{2n}(\CP^n)$ is the fundamental class corresponding to the canonical orientation on $\CP^n$.
	Note that since the generating element has degree $2$, the middle-dimensional homology $\H^{2m+1}(\CP^{2m+1})$ is trivial and hence so is the intersection form of $\CP^{2m+1}$.

	When the complex dimension is even, the middle-dimension homology $\H^{2m}(\CP^{2m})$ is generated by $\alpha^m$. Since $\alpha^m\smile \alpha^m=\alpha^{2m}$ is a unit multiple of the fundamental class, we have $Q(\alpha^m, \alpha^m)=1$, completing the proof.
\end{proof}

\begin{figure}[ht]
	\centering
	\import{diagrams}{complex-projective-intersection.pdf_tex}
	\caption{Intersections of homology classes in a complex projective plane.}\label{fig:geometric-intersection-complex-projective}
\end{figure}

It is illuminating to interpret this result geometrically. Let's begin with the complex vector space $\C^{2m+1}$ equipped with a basis $\{e_0, e_1,\ldots, e_{2m}\}$. Consider the linear subspaces
\begin{equation}
	W = \langle e_0, e_1,\ldots, e_m\rangle \quad\textrm{and}\quad U = \langle e_0, e_{m+1},\ldots, e_{2m}\rangle
\end{equation}
in $\C^{2m+1}$. These complex hyperplanes intersect at a complex line $\langle e_0 \rangle = W\cap U$.

Now, we can pass to the projectivization $\P(\C^{2m+1})=\CP^{2m}$ and realize $W$ and $U$ as embedded submanifolds $\P(W)\approx \CP^m\subset \CP^{2m}$ and $\P(U)\approx \CP^m\subset \CP^{2m}$. Since $W$ and $U$ intersect at a line, their projectivizations $\P(W)$ and $\P(U)$ intersect at a point in $\CP^{2m}$. Furthermore, the intersection is transverse, and descending the orientation on $\C^{2m+1}$ to the embedded submanifolds gives an intersection number of $1$. Both of these embedded submanifolds represent the homology class $a^{2m}\in \H_{2m}(\CP^{2m})$ which is the Poincar\'e dual of the cohomology class $\alpha^{2m}(\CP^{2m})$. We again arrive at $Q(a^{2m}, a^{2m})=1$, although this time through homology intersections.

\begin{proposition}\label{prop:intersection-form-torus}
	The intersection form for a torus $T^{2m}=S^m\times S^m$ is given by matrices
	\begin{equation}\label{eq:hyperbolic-form-torus}
		Q = \begin{pmatrix}0 & 1 \\ 1 & 0\end{pmatrix}
		\textrm{ when }m\textrm{ is even, and }
		Q = \begin{pmatrix}0 & 1 \\ -1 & 0\end{pmatrix}
		\textrm{ when }m\textrm{ is odd.}
	\end{equation}
\end{proposition}
\begin{proof}
	Let us again begin with a cohomology computation. The cohomology of a sphere is
	\begin{equation}
		\H^\bullet(S^n) = \Z[\alpha]/(\alpha^2)\quad\textrm{with}\quad |\alpha| = n,
	\end{equation}
	and by the K\"unneth formula (see Theorem~3.15 in \cite{hatcher2002topology}), we have
	\begin{equation}
		\begin{aligned}
			\H^\bullet(T^{2m})=\H^\bullet(S^m\times S^m)\cong \H^\bullet(S^m)\otimes \H^\bullet(S^m)
			 & \cong \Z[\alpha]/(\alpha^2)\otimes \Z[\beta]/(\beta^2) \\
			 & \cong \Z[\alpha,\beta]/(\alpha^2,\beta^2).
		\end{aligned}
	\end{equation}
	Assuming $\alpha$ and $\beta$ are fundamental classes for the spheres, the fundamental class of the torus is $\alpha\smile \beta$. From this multiplicative structure and fundamental class, we clearly have
	\begin{equation}
		Q(\alpha, \alpha)=0, \quad Q(\beta,\beta)=0, \quad Q(\alpha,\beta)=1,\quad Q(\beta,\alpha)=(-1)^m Q(\alpha,\beta)=(-1)^m.
	\end{equation}
	These give exactly the matrices described in \cref{eq:hyperbolic-form-torus}.
\end{proof}

\begin{figure}[ht]
	\centering
	\import{diagrams}{torus-intersection.pdf_tex}
	\caption{Intersections of homology classes in a torus.}\label{fig:geometric-intersection-torus}
\end{figure}

The geometric proof of this claim is analogous. The Poincar\'e duals of $\alpha$ and $\beta$, denoted $a$ and $b$ in $\H_{m}(T^{2m})$, are represented
by the embedded submanifolds $S^m\times \{p\}$ and $\{p\}\times S^m$ for some basepoint $p\in S^m$. Shifting an individual embedded sphere to a disjoint embedding by a path taking $p\mapsto p'$ disjoint shows that the self-intersection numbers of $a$ and $b$ are zero. These are the zeroes along the diagonal of the matrices in \cref{eq:hyperbolic-form-torus}. However, the embedded spheres representing $a$ and $b$ intersect transversally at the point $(p,p)\in T^{2m}$. We choose orientations on $S^m\times \{p\}$ and $\{p\}\times S^m$ so that $a\cdot b=1$, then by graded-commutativity we get $b\cdot a=(-1)^m$.

\begin{remark}
	The symmetric form in \cref{eq:hyperbolic-form-torus} is known as the \defn{hyperbolic form}, denoted by
	\[
		H=\begin{pmatrix} 0 & 1\\ 1 & 0 \end{pmatrix}.
	\]
	The hyperbolic form is a fundamental building block for symmetric bilinear forms over the integers and $\Z/2$.
\end{remark}

\subsection{Properties of the Intersection Form}
We now investigate some basic properties of the intersection form.

\begin{proposition}\label{prop:orientation-intersection-form}
	For a compact manifold $M^{2m}$, we have
	$Q_{-M} \cong -Q_{M}$.
\end{proposition}
\begin{proof}
	This is immediate, since the fundamental class changes sign as orientation flips.
\end{proof}

\begin{proposition}\label{prop:connected-sum-intersection-form}
	For compact manifolds $M_1^{2m}$ and $M_2^{2m}$, we have
	$Q_{M_1\+M_2} \cong Q_{M_1}\oplus Q_{M_2}.$
\end{proposition}
\begin{proof}
	Let us assume that $m>1$, since the case $m=1$ requires a slightly modified argument. By the discussion in \cref{sec:smooth-manifold-operations}, we get a short exact sequence of cohomology
	\[
		\H^{m-1}(S^{2m-1})\lkxto \H^m(M_1\+M_2) \lkxto[p] \H^m(M_1^\circ)\oplus \H^m(M_2^\circ)\lkxto \H^m(S^{2m-1}),
	\]
	where $M_1^\circ$ and $M_2^\circ$ the manifolds with a point removed.
	Then, the middle map $p$ is an isomorphism since $m\neq 1$ so every cohomology cycle $\alpha\in \H^m(M_1\+M_2)$ can be written as a sum $p^{-1}(\alpha_1) + p^{-1}(\alpha_2)$ where $\alpha_i\in \H^m(M_i^\circ)$. \todo{finish the proof}
\end{proof}

By a similar proof, we can show that:

\begin{proposition}\label{prop:boundary-connected-sum-intersection-form}
	For compact manifolds $M_1^{2m}$ and $M_2^{2m}$ with non-empty and connected boundaries, we have $Q_{(M_1,\partial M_1)\+ (M_2,\partial M_2)} \cong Q_{M_1}\oplus Q_{M_2}$.
\end{proposition}

An immediate corollary of \cref{cor:connected-sum-operation} and \cref{prop:connected-sum-intersection-form} is
that the intersection form is a homomorphism of commutative monoids, i.e. sets with a commutative associative binary operation and identity elements. On one side, we have the monoid $\mathcal{M}^{2m}$ of oriented compact $2m$-manifolds under connected sum, and on the other side we have $\mathcal{Q}(\Z)$ of bilinear forms valued in $\Z$ under the operation of direct sum. Similarly, the unoriented intersection form maps the monoid of unoriented compact $2m$-manifolds $\widetilde{\mathcal{M}}^{2m}$ to $\mathcal{Q}(\Z)$.
\begin{equation}\label{eq:monoid-homomorphism-intersection-form}
	\lkxfunc{}{\mathcal{M}^{2m}}{\mathcal{Q}(\Z),}{M}{Q_M,}
	\quad\textrm{and}\quad
	\lkxfunc{}{\widetilde{\mathcal{M}}^{2m}}{\mathcal{Q}(\Z/2),}{M}{\widetilde{Q}_M.}
\end{equation}
The monoidal structure of the intersection form is quite useful throughout geometric topology, especially in classification problems.

\subsection{Classification of Manifolds by Intersection Form}
An illustrative case in low dimensions is the classification of compact (unoriented) surfaces up to homeomorphism. Recall that every compact surface is homeomorphic to exactly one of the following surfaces
\[
	S^2,\quad T^2\#\cdots\# T^2,\quad\textrm{or}\quad \RP^2\#\cdots\# \RP^2,
\]
i.e. it is either a sphere, a torus with some number of holes, or an unorientable surface formed by gluing together M\"obius strips. For instance, a Klein bottle is the connected sum $\RP^2\#\RP^2$.
A standard cominatorial proof of this classification by polygonal presentations can be found in Chapter 6 of \cite{lee2011topological}.

By similar arguments to \cref{prop:intersection-form-complex-projective-plane} and \cref{prop:intersection-form-torus}, the unoriented intersection forms of these generating surfaces are given by
\[
	\widetilde{Q}_{S^2}=(0),\quad \widetilde{Q}_{T^2}=\begin{pmatrix}0 & 1 \\ 1 & 0\end{pmatrix},\quad \textrm{and}\quad\widetilde{Q}_{\RP^2} = (1).
\]

\begin{example}
	There is a pretty geometric picture for the intersection form for a surface $X_g$ of genus $g$. Since $X_g=\T^2\+\cdots+\T^2$, the intersection form is given by
	\[
		\widetilde{Q}_{X_g} = \underbrace{\begin{pmatrix}0&1\\1&0\end{pmatrix}\oplus\cdots\oplus \begin{pmatrix}0&1\\1&0\end{pmatrix}}_{g\textrm{ times}}.
	\]
	Every compact connected surface has a universal cover homeomorphic to the plane -- geometrically this follows from the uniformization theorem in the theory of Riemann surfaces. The surface $X_g$ is then a quotient of the plane by a group action. For instance, the torus $T^2=X_1$ is the quotient of $\R^2$ by the action of translation by $\Z^2$. More generally, $X_g$ is the quotient of the complex upper half plane by the action of a Fuchsian group $\Gamma\subset \mathrm{PSL}_2(\R)$ which is isomorphic to the fundamental group of $X_g$. This quotient has fundamental domain a $4g$-gon, and the action generates a tiling of the projective plane by regular $4g$-gons (see \cref{fig:torus-octagon}).

	\begin{figure}[ht]
		\centering
		\import{diagrams}{torus-octagon.pdf_tex}
		\caption{Generators for $\H_1(X_g)$ when $g=1$ and $2$ (The octagonal tiling of the Poincar\'e disk was adapted from a graphic by \href{https://commons.wikimedia.org/wiki/User:Parcly_Taxel}{Parcly Taxel}).}\label{fig:torus-octagon}
	\end{figure}

	On the universal cover, we can draw lines which project down to generating cycles for $\H_1(X_g)$. For instance, see \cref{fig:torus-octagon} for examples of this in $X_1$ and $X_g$. Extending these lines by the group action to get the full group $\H_1(X_g)$ expresses the lattice $\H_1(X_g)$ as a concrete lattice of lines in hyperbolic space.
	When the lines are restricted to the fundamental domain with boundary conditions agreeing with the group action, their intersection numbers are the same as in the quotient $X_g$. For the torus $X_1$, we have two lines $a$ and $b$ which have self-intersection $0$ (perturbing either horizontally or diagonally), and cross-intersection $a\cdot b = 1$. This gives the expected matrix
	\[
		\widetilde{Q}_{X_1}=\begin{pmatrix}0 & 1\\ 1 & 0\end{pmatrix}.
	\]
	For the two-holed torus $X_2$, we have lines $a,b,c,$ and $d$ which have self-intersections $0$ and cross-intersections $1$ for any distinct pair. By symmetric row-column operations we can transform this basis to turn the intersection matrix into canonical form.
	\[
		\widetilde{Q}_{X_2} =
		\begin{pmatrix}
			0 & 1 & 1 & 1 \\
			1 & 0 & 1 & 1 \\
			1 & 1 & 0 & 1 \\
			1 & 1 & 1 & 0
		\end{pmatrix}
		\lkxto
		\begin{pmatrix}
			0 & 1 & 1 & 1 \\
			1 & 0 & 0 & 0 \\
			1 & 0 & 0 & 1 \\
			1 & 0 & 1 & 0
		\end{pmatrix}
		\lkxto
		\begin{pmatrix}
			0 & 1 & 1 & 0 \\
			1 & 0 & 0 & 0 \\
			1 & 0 & 0 & 1 \\
			0 & 0 & 1 & 0
		\end{pmatrix}
		\lkxto
		\begin{pmatrix}
			0 & 1 &   &   \\
			1 & 0 &   &   \\
			  &   & 0 & 1 \\
			  &   & 1 & 0
		\end{pmatrix}
	\]
	The resulting basis becomes $\{a+c, a+b,a+b+c,d\}$. Nice geometric pictures like this are harder to come by in dimensions beyond $4$, but much of the intuition about intersections carries over. A wonderful source for geometric pictures of intersections in 4-dimensions can be found in the books \cite{behrens2021discembedding} and \cite{scorpan2005wild}.
\end{example}


\begin{example}
	For an example which includes a projective plane, the intersection form of $T^2\+ \RP^2$ is given by
	\[
		\widetilde{Q}_{T^2\+ \RP^2} = H\oplus (1)=
		\begin{pmatrix}
			1 & 0 & 0 \\
			0 & 0 & 1 \\
			0 & 1 & 0
		\end{pmatrix}.
	\]
	This matrix represents a bilinear form, and so the transformation $Q\mapsto P^\intercal Q P$ does not affect the form. In this case, $Q^\intercal =Q$ and $Q^2=I\mod 2$, the transformation $Q\mapsto Q^\intercal Q Q$ gives the form
	\[
		\widetilde{Q}_{T^2\+ \RP^2}
		\lkxto \begin{pmatrix}
			1 & 0 & 0 \\
			0 & 1 & 0 \\
			0 & 0 & 1
		\end{pmatrix} =\oplus^3(1)= \widetilde{Q}_{\RP^2\+\RP^2\+\RP^2}.
	\]
\end{example}
As it turns out, the underlying surfaces $T^2\+\RP^2$ and $\RP^2\+\RP^2\+\RP^2$ are homoemorphic. This has an easy geometric interpretation. The operation $T^2\+$ can be thought of as adding a handle, and $\RP^2\+\RP^2\+$ being connected sum with a Klein bottle can be thought of as adding a handle in a twisted manner, i.e. one spout on one side of the surface and the other spout on the other side. Note that there might be a global notion of ``side'' if the manifold is non-orientable, but locally this picture holds.

One such non-orientable case is the projective plane $\RP^2$. If we add a torus handle to $\RP^2$ (a M\"obius band with boundary collapsed), we can move one spout around the twist of $\RP^2$ to get a twisted handle (as depicted in \cref{fig:twisted-handle-to-handle}). Thus, the surfaces $T^2\+\RP^2$ and $\RP^2\+\RP^2\+\RP^2$ are homeomorphic, a geometric fact which was detected in part by the algebraic identity of forms $H\oplus (1)=\oplus^3(1)$ in $\mathcal{Q}(\Z/2)$.

\begin{figure}[ht]
	\centering
	\import{diagrams}{handle-inversion.pdf_tex}
	\caption{Turning $T^2\+\RP^2$ into $\RP^2\+\RP^2\+\RP^2$.}\label{fig:twisted-handle-to-handle}
\end{figure}

\begin{proposition}
	Let $\mathcal{Q}_{\mathrm{skew}}(\Z/2)$ be the monoid of skew-symmetric bilinear forms over $\Z/2$. There is a presentation
	\[\mathcal{Q}_{\mathrm{skew}}(\Z/2) = \langle H, (1) \mid H\oplus (1) = \oplus^3 (1)\rangle.\]
\end{proposition}
\begin{proof}
	See Chapter III of \cite{milnorhuse1973forms} for a generalized statement and proof.
\end{proof}

Just like $H\oplus (1)= \oplus^3 (1)$ is the defining relation for skew-symmetric bilinear forms over $\Z/2$, so too is $T^2\+ \RP^2 = \RP^2\+\RP^2\+\RP^2$ the defining relation for closed surfaces. This leads to a clean restatement of the classification theorem for closed surfaces.

\begin{theorem}[Classification of Compact Surfaces]
	Let $\mathcal{S}^2\subset \widetilde{\mathcal{M}}^2$ be the monoid of \textit{closed} unoriented surfaces under connected sum. The unoriented intersection form is an isomorphism of monoids
	\[
		\lkxfunc{\widetilde{Q}}{\mathcal{S}^2}{\mathcal{Q}_{\mathrm{skew}}(\Z/2).}
	\]
\end{theorem}

The classification of compact surfaces by the intersection form is a model result of algebraic topology -- a complete algebraic classification of a class of manifolds. Better yet, simple algebraic manipulations correspond to non-trivial topological equivalences. This is part of why intersection forms are so useful -- algebraic intuition scales far better with dimension than does geometric intuition and so bilinear forms are a much more comfortable setting in which to study higher-dimensional topology. For instance, the classification theorem of Michael Freedman in his 1982 paper \cite{freedman1982manifold} is formulated entirely in terms of the intersection form and an additional $\Z/2$-valued invariant detecting the existence of a smooth structure.

\begin{theorem}[Freedman, 1982] Let $\mathcal{S}^4\subset \mathcal{M}^4$ be the monoid of simply-connected closed \emph{topological} $4$-manifolds. The intersection form
	\[
		\lkxfunc{Q}{\mathcal{S}^4}{\mathcal{Q}_{\mathrm{sym}}(\Z)}
	\]
	is at most two-to-one, i.e. a symmetric intersection form corresponds to at most two topological $4$-manifolds, one which admits a smooth structure and one which does not.
\end{theorem}
An accessible exposition to this remarkable theorem can be found in \cite{behrens2021discembedding}. We will explore this theorem and its consequences with more depth in \cref{sec:smoothing-obstructions}.

\subsection{Submanifolds and Intersection Theory}\label{sec:submanifolds-intersection-theory}

At this point, we will comment in some more detail on how the algebraically defined intersection form captures the geometric data of submanifold intersections. We will begin by reviewing some relevant concepts from differential topology.

Let $f : N^k\to M^n$ be a smooth map from a closed oriented manifold $N^k$ into some manifold $M^k$. An orientation on $N$ determines a fundamental homology class $[N]\in \H_k(N)$ which can be pushed forward along the map $f : w \to M$ to give a homology class $f_* [N]\in \H_k(M)$.
The correspondence behaves nicely with respect to homotopic perturbations, and so
the homology class associated to a map $f : N \to M$ solely depends on the homotopy type of $f$. This gives a map
\begin{equation}\label{eq:homotopy-class-to-homology-class}
	\lkxfunc{}{[N,M]}{\H_k(M)}
\end{equation}
which generalizes the Hurewicz homomorphism $\pi_k(M) \to \H_k(M)$ in the case that $N=S^k$.
As with the Hurewicz homomorphism, the correspondence in \cref{eq:homotopy-class-to-homology-class} is generally not surjective or injective.

In the case of the Hurewicz homomorphism, if $M$ is $(k-1)$-connected for $k > 1$ the Hurewicz homomorphism is in fact an isomorphism $\pi_{k}(M) \cong \H_{k}(M)$. Consequently, every homology cycle in $\H_{k}(M)$ can at least be represented by an smooth map of a sphere $S^{k}$ into $X$. However, this smooth map need not be an immersion, and even still might have unavoidable ``double-points'' -- multiple points of the sphere mapping to the same point in the image and preventing the smooth map from being an embedding.

\begin{remark}
	For a classical account of some issues which can arise when representing homology classes by smooth maps, see Chapter II of Ren\'e Thom's seminal paper \cite{thom1954}.
\end{remark}

\begin{remark}\label{rmk:homology-dimension-4-embedding}
	In some cases, the homology classes of interest to the intersection form can \emph{always} be represented by embedded submanifolds. For instance if $M$ is a simply-connected $4$-manifold, every element of $\H^2(M)$ is represented by an embedded submanifold. For some elegant proofs of this fact, see page 114 of \cite{scorpan2005wild}.
	% \begin{equation}
	% 	\H^2(M; \Z) \cong [M, K(\Z,2)] \cong [M, \CP^\infty] \cong [M,\CP^2],
	% \end{equation}
	% where the first is the representability of singular cohomology by the Eilenberg-Maclane spectrum, the second identifies $\CP^\infty$ as a $K(\Z,2)$ space, and the third uses the cellular approximation theorem to slide maps onto the $4$-skeleton. Any cohomology cycle $\omega\in \H^2(M)$ can be represented by a smooth function $f : M \to \CP^2$. If we choose this function to be transverse to $\CP^1\subset \CP^2$, then $f^{-1}(\CP^1)$ is an embedded $2$-dimensional submanifold of $M$ which corresponds to a Poincar\'e dual class to $\omega$. When $X$ is a compact manifold, Poincar\'e duality tells us that all $2$-dimensional homology cycles arise from $2$-dimensional cohomology cycles and can thus be represented by embedded submanifolds. 
	This is one example of the attractiveness of $4$-manifolds as geometric objects of study.
\end{remark}

Once in the context of differential topology, we can take the intersection number of two immersions. For an introduction to the notion of intersection number, see Chapter 2 of \cite{gp2010topology}. When homology classes are represented by immersions, the topological notion of intersection number agrees with the algebraic intersection of homology classes.
% If $M$ is a 
% \begin{equation}\label{eq:oriented-intersection-number-homotopy}
% 	\lkxfunc{I}{[N_1,M]\times [N_2,M]}{\Z}{f,g}{I(f,g)}
% \end{equation}
% This leads to a geometric version of the intersection form:

\begin{theorem}
	Suppose $i_1 : N_1 \to M$, $i_2 : N_2 \to M$ transverse immersions, with $N_1$ and $N_2$ of complementary dimension in $M$. Then we have
	\[(i_1)_*[M]\tnsv (i_2)_*[N] = I(i_1,i_2).\]
\end{theorem}

For details on the relationship between homology, intersections, and the degree of a map between manifolds, see Chapter 5 of \cite{hirsch1976differential}.

\subsection{Intersection Form Invariants}\label{sec:intersection-form-invariants}

While the complete algebraic data of an intersection form captures a lot of the topological structure of a manifold, it is useful to extract further simpler invariants from the intersection form.

We will work with a general integer lattice $\Lambda$ and $Q$ an integral bilinear form over $\Lambda$, not necessarily the intersection form of some manifold. Recall that under a change of basis matrix $P$, the matrix of the bilinear form transforms as $Q\mapsto P^\intercal QP$. We are therefore looking for quantities which are invariant under such transformations, helping us understand the structure of the monoid $\mathcal{Q}(\Z)$.

\begin{definition}
	The \defn{rank} of $Q$ is simply the dimension of the lattice $\Lambda$.
\end{definition}

When $Q$ is the intersection form of a manifold, its rank is the middle Betti number $\beta_m = \dim \H_m(X)_{\textrm{free}}$ of the manifold. This is the simplest invariant of a bilinear form.

\begin{definition}
	A form is said to be \defn{degenerate}[degenerate bilinear form] if $\det Q=0$ and \defn{non-degenerate}[non-degenerate bilinear form] otherwise.
\end{definition}

An equivalent dual way to view a bilinear form is by the homomorphism
\[
	\lkxfunc{Q^\d}{\Lambda}{\Hom(\Lambda, \Z)}{\alpha}{(\beta\mapsto Q(\alpha,\beta)).}
\]
In this context, a bilinear form non-degenerate if and only if $Q^\d$ is injective.

We can refine the notion of non-degeneracy even further.
\begin{definition}
	A bilinear form is said to be \defn{unimodular} if $\det Q=\pm 1$.
\end{definition}
A bilinear form is unimodular if and only if $Q^\d$ is an isomorphism. The notion of unimodularity for integral bilinear forms is a special case of the notion of a \defn{perfect pairing}. A perfect pairing $V\otimes W \to R$ is a bilinear map such that the dual homomorphism $V \to \Hom(W, R)$ is an isomorphism. These notions of degeneracy and unimodularity are not a terribly useful source of invariants by the following proposition.

\begin{proposition}\label{prop:unimodular-intersection-form}
	If $M$ is a compact manifold with $\H_m(\partial M)$ and $\H_{m-1}(\partial M)$ trivial, then the intersection form $Q_M$ is unimodular.
\end{proposition}
\begin{proof}
	By the universal theorem argument in \cref{rmk:dual-lattice-intersection-form}, we have an isomorphism
	\begin{equation}\label{eq:dual-lattice-isomorphism}
		\H^m(M,\partial M) \lkxisom \Hom(\H_m(M), \Z).
	\end{equation}
	The resulting pairing $\langle -, -\rangle : \H^m(M,\partial M)\otimes \H_m(M) \to \Z$, known as the Kronecker pairing, is therefore a \defn{perfect}[perfect bilinear form] pairing. Since the intersection form is given by
	\[
		Q_M(a,b) = \langle \mathrm{PD}(a), b\rangle,\quad a,b\in \H_m(M)
	\]
	where $\mathrm{PD} : \H^{m}(M,\partial M) \to \H_m(M)$ is the Poincar\'e-Lefschetz duality isomorphism, it follows that $Q_M^\d$ is the composition of $\mathrm{PD}$ with \cref{eq:dual-lattice-isomorphism} and so is an isomorphism itself.
\end{proof}

Thus far, we have only computed the intersection forms of closed manifolds and consequently all of the intersection forms we have seen have been unimodular. However, it is very easy to come up with examples of degenerate and non-unimodular intersection forms.

\begin{example}
	The handle $S^2\times D^2$ has trivial intersection form but not trivial middle dimensional homology, and so this space has a degenerate intersection form sending every homology element to $0$. Adding this onto any space with a connected sum leads to a degenerate intersection form.
	For instance, the space $M=\CP^2\+(S^2\times D^2)$ has intersection form
	\[
		Q_M = \begin{pmatrix}
			1 & 0 \\ 0 & 0
		\end{pmatrix}
	\]
	which is clearly degenerate. Note that the boundary of $M$ is $S^2\times S^2$ which has non-trivial $\H_2(S^2\times S^2)$ so \cref{prop:unimodular-intersection-form} does not apply.
\end{example}

\begin{example}
	The unit disk bundle $D(\T S^2)$ over $S^2$ has intersection form $(2)$, which is non-degenerate but not unimodular. Note that the boundary $\partial D(\T S^2)= S(\T S^2)$ is diffeomorphic to $\CP^2$, which has non-trivial homology $\H_2(\CP^2)$.
\end{example}

Aside from the hyperbolic form, another important building for intersection forms in geometric topology is the $E_8$ form, an ``exotic'' bilinear form. This form shows up ``naturally'', but is also tremendously useful in constructions, especially for homotopy spheres. We make use of it in \cref{sec:plumbing}.

\begin{definition}
	The \defn{$E_8$ form} is the bilinear form given by
	\[
		E_8=
		\begin{pmatrix}
			2 & 1 &   &   &   &   &   &   \\
			1 & 2 & 1 &   &   &   &   &   \\
			  & 1 & 2 & 1 &   &   &   &   \\
			  &   & 1 & 2 & 1 &   &   &   \\
			  &   &   & 1 & 2 & 1 &   & 1 \\
			  &   &   &   & 1 & 2 & 1 &   \\
			  &   &   &   &   & 1 & 2 &   \\
			  &   &   &   & 1 &   &   & 2 \\
		\end{pmatrix}
	\]
\end{deinition}

Note that this is a unimodular form since by a series of row/column operations of the form $Q\mapsto PQP^\intercal$ we get
	\[
		\begin{pmatrix}
			2 & 1 &   &   &   &   &   &   \\
			1 & 2 & 1 &   &   &   &   &   \\
			  & 1 & 2 & 1 &   &   &   &   \\
			  &   & 1 & 2 & 1 &   &   &   \\
			  &   &   & 1 & 2 & 1 &   & 1 \\
			  &   &   &   & 1 & 2 & 1 &   \\
			  &   &   &   &   & 1 & 2 &   \\
			  &   &   &   & 1 &   &   & 2 \\
		\end{pmatrix}
		\lkxto
		\begin{pmatrix}
			2 &             &             &             &              &             &             &   \\
			  & \frac{3}{2} &             &             &              &             &             &   \\
			  &             & \frac{4}{3} &             &              &             &             &   \\
			  &             &             & \frac{5}{4} &              &             &             &   \\
			  &             &             &             & \frac{7}{10} &             &             &   \\
			  &             &             &             &              & \frac{4}{7} &             &   \\
			  &             &             &             &              &             & \frac{1}{4} &   \\
			  &             &             &             &              &             &             & 2 \\
		\end{pmatrix}
	\]

\begin{example}\label{example:k3}
	The \defn{Fermat quartic surface} is defined by the homogeneous polynomial
	\[
		\mathcal{K} = \left\{ [z_0 : z_1 : z_2 : z_3]\in \CP^3 \mid z_0^4 + z_1^4+z_2^4 + z_3^4=0\right\}.
	\]
	This is an example of a \defn{K3 surface}, an important class of complex manifolds in algebraic geometry, representation theory, and topology.
	We will not prove this here, but it can be shown that $\H_2(\mathcal{K})\cong \Z^{22}$ and the intersection form admits the remarkable decomposition
	\[
		Q_{\mathcal{K}} =
		\begin{pmatrix}
			0 & 1 \\ 1 & 0
		\end{pmatrix}^{\oplus 3}\oplus
		-E_8^{\oplus 2}
	\]
	In particular, this matrix is unimodular since $\mathcal{K}$ is a closed manifold. 
\end{example}

\begin{remark}\label{rmk:kummer-construction}
	There is a beautiful construction of $\mathcal{K}$ from the quotient of a fourfold torus $(S^1)^{\times 4}$ under complex conjugation. The action has 16 fixed points, and so the quotient has sixteen singular points. By cutting out these singular points and replacing them with the total space of the disk bundle $\D(\T \overline{\CP}^1)$,\footnote{Complex conjugation reverses orientation, so this space has intersection form $(-2)$.} we get the simply-connected smooth manifold $\mathcal{K}$. This is known as the \defn{Kummer construction}, see Section 3.3 of \cite{scorpan2005wild} for an accessible introduction.
\end{remark}

\begin{definition}
	Write $Q=P^\intercal D P$ for a diagonal real matrix $D$ ($P$ can be a real matrix), then count the number $n^+$ of positive eigenvalues, $n^0$ the number of zero eigenvalues, and number $n^-$ of negative eigenvalues. The \defn{signature}[signature of a bilinear form] of the bilinear form is the difference between the number of positive and negative eigenvalues $n^+-n^-$.
	The triple $(n^+, n^-, n^0)$ is referred to as the \defn{inertia}[inertia of a bilinear form] of $Q$.
\end{definition}

\begin{remark}
	For non-degenerate forms, $n^0$ is always $0$.
\end{remark}

For instance, the matrix
\begin{equation}\label{eq:diagonal-matrix}
	(1)^{\oplus p}\oplus (-1)^{\oplus q} = \begin{pmatrix}
		1 &        &   &    &             \\
		  & \ddots &   &    &             \\
		  &        & 1 &    &             \\
		  &        &   & -1 &        &    \\
		  &        &   &    & \ddots &    \\
		  &        &   &    &        & -1
	\end{pmatrix}
\end{equation}
has signature $p-q$. By Sylvester's Law of Inertia (see \cite{lam2005quadratic}), any symmetric non-degenerate bilinear form over $\R$ or $\Q$ can be written as a matrix \cref{eq:diagonal-matrix}, unique up to permutation.
In fact, the rank and inertia completely classify symmetric bilinear forms on a vector space over $\R$ or $\Q$. Over the integers $\Z$, not all forms can be put into the form \cref{eq:diagonal-matrix} -- for instance, over the rationals, the hyperbolic form $H$ can be written as
\begin{equation}\label{eq:hyperbolic-form-transformation}
	\begin{pmatrix} 0 & 1\\ 1 & 0 \end{pmatrix}
	\quad\lkxto\quad
	\begin{pmatrix} 1 & 1/2 \\ 1 & -1/2 \end{pmatrix}
	\begin{pmatrix} 0 & 1 \\ 1 & 0 \end{pmatrix}
	\begin{pmatrix} 1 & 1 \\ 1/2 & -1/2 \end{pmatrix}=
	\begin{pmatrix} 1 & 0 \\ 0 & -1 \end{pmatrix}.
\end{equation}
It is an easy exercise to show that such a transformation cannot be done over a ring $R$ with $2\not\in R^\times$. While the hyperbolic form $H$ and $(1)\oplus (-1)$ have the same signature and rank, they do not represent the same integral bilinear form.
The classification of symmetric bilinear forms over the integers is thus considerably more difficult than over a field, and this complexity reflects the complexity of manifolds in $4k$-dimensions.

\begin{definition}
	The \defn{signature}[signature of a manifold] of a $4k$-dimensional manifold $M$ is the signature of its intersection form, and is denoted $\sigma(M)$.
\end{definition}

\begin{remark}
	Note that in the context of intersection forms, the signature is only a useful invariant for $4k$-manifolds, since the intersection form of a $(4k+2)$-manifold is skew-symmetric and thus has signature $0$. In the case of $(4k+2)$-manifolds, the relevant invariant is the Ar invariant, which we discuss in \cref{sec:arf-invariant}.
\end{remark}

The signature is a topological invariant of fundamental importance for $4k$-manifolds, and we will see many of its generalizations and equivalent definitions in \cref{sec:hirzebruch-signature-theorem} and \cref{sec:surgery-invariant}.

\begin{example}
	By \cref{prop:intersection-form-complex-projective-plane}, the signature of complex projective spaces is
	\[
		\sigma(\CP^k)=\begin{cases}1 & k\textrm{ even},\\ 0 & k\textrm{ odd}.\end{cases}
	\]
	Note that reversing the orientation of a manifold reverses the signature. The diagonal matrix $(1)^p\oplus (-1)^q$ is thus represented by the manifold $\+[p] \CP^{2m} \+[q] \overline{\CP}^{2m}$, where $\overline{\CP}^{2m}$ denotes the conjugate complex structure on $\CP^{2m}$ which reverses orientation.
\end{example}

\begin{example}
	The signature of the Fermat quartic surface is $\sigma(\mathcal{K})=-16$.
\end{example}

Finally, we introduce two more categories of bilinear form which split the set of bilinear forms into quadrants.

\begin{definition}
	If for all non-zero elements $a\in \Lambda$ we have $Q(a,a)>0$, then we say that $Q$ is \defn{positive-definite}[positive-definite bilinear form]. On the contrary, if we have $Q(a,a)<0$ for all non-zero elements $a\in \Lambda$, we say that $Q$ is \defn{negative-definite}[negative-definite bilinear form]. Either way, we say that it is \defn{definite}[definite bilinear form]. Otherwise, $Q$ is said to be \defn{indefinite}[indefinite bilinear form].
\end{definition}

\begin{definition}
	If for all elements $a\in \Lambda$, the diagonal entry $Q(a,a)$ is even, then $Q$ is said to be \defn{even}[even bilinear form]. Otherwise, we say that $Q$ is \defn{odd}[odd bilinear form].\footnote{Some sources refer to this property as ``type''. Odd forms are \defn{type I}[type I bilinear form] and even forms are \defn{type II}[type II bilinear form].}
\end{definition}

\begin{theorem}\label{thm:indefinite-bilinear-forms-isomorphic}
	Two unimodular indefinite bilinear forms are isomorphic if they have the same rank, parity, and signature.
\end{theorem}

For instance, the $E_8$ form is a unimodular, positive-definite, even, symmetric bilinear form. In fact, $E_8$ has the smallest size for a non-trivial bilinear form satisfying these conditions by the following theorem. We follow the proof of Serre in his classic treatise on unimodular bilinear forms 

\begin{theorem}
	The signature of an even unimodular bilinear form is divisible by $8$.
\end{theorem}
\begin{proof}
	See \cite{serre1961forms}.
\end{proof}

% \begin{figure}
% 	\centering
% 	\begin{tabular}{cc|c}
% 		& \textrm{odd} & \textrm{even}\\
% 		\textrm{definite} & $(1)^p\oplus (-1)^q\oplus$ & \todo{add examples}\\
% 		\hline
% 		\textrm{indefinite} & $\oplus^r E_8\oplus^{s>0} H$ & 
% 	\end{tabular}
% 	\caption{Table of symmetric unimodular bilinear forms.}\label{fig:unimodular-symmetric-bilinear-forms}
% \end{figure}
%
% \todo{finish this}

\pagebreak
\section{The Euler Class}\label{sec:euler-class}

A beautiful link between intersection theory and geometry comes from the Euler number, a characteristic number of oriented real vector bundles. Due to the massive number of generalizations and equivalent definitions of the Euler, we only give a brief survey of perspectives used throughout the thesis.

The first realization of the Euler number dates back to at least the 18th century, where it was noticed by Euler that any convex polyhedron satisfies $V-E+F=2$ for $V$ the number of vertices, $E$ the number of edges, and $F$ the number of faces. 
In general, the quantity $V-E+F$ is a topological invariant for any polyhedron -- convexity simply restricts the topology of the polyhedron to be spherical.
For instance, any polyhedron homeomorphic to a torus has $V-E+F=0$. 
This quantity is known as the Euler number, and is the starting point for a large tree generalizations in mathematics. 
The simplest such generalization is in terms of Betti numbers and homology.

\begin{definition}
  The \defn{Euler number} of an $n$-dimensional manifold $M$ is
  \[
    \chi(M) = \beta_0 - \beta_1 + \beta_2 - \beta_3 + \cdots + (-1)^n\beta_n = \sum_{k\geq 0} \dim \H_k(M; \Q).
  \]
\end{definition}

\begin{proposition}
  If $M_1\simeq M_2$ are homotopic, then $\chi(M_1)=\chi(M_2)$.
\end{proposition}

The standard examples are:

\begin{example}
  The Euler number of a sphere is $\chi(S^n)=1+(-1)^n$.
\end{example}

\begin{example}
  For a compact surface $X_g$ of genus $g$, we have $\chi(X_g)=2-2g$.
\end{example}

\begin{proposition}
  For any closed manifolds $M_1^n$, $M_2^n$, we have 
  \[\chi(M_1\+M_2)=\chi(M_1)+\chi(M_2)-\chi(S^n).\]
\end{proposition}

\begin{proposition}
  For any closed manifolds $M$, $N$, we have $\chi(M\times N)=\chi(M)\cdot \chi(N)$.
\end{proposition}

We will now discuss several perspectives on the Euler number which will arise throughout this thesis. Unfortunately, this short set is far from complete. Not included in this section are the perspectives of the Euler number (and Euler class) 
\begin{enumerate}[(a)]
  \item as the primary homotopic obstruction to a section on a sphere bundle,
  \item as a Lefschetz number of the identity map,
  \item as a Thom class of an oriented vector bundle,
  \item as a universal cohomology class in $\H^{2m}(\BSO_n; \Q)$,
  \item as a transgression of the \v{C}ech-de Rham or Leray-Serre spectral sequence,
  \item as the index of the Hodge-de Rham elliptic differential operator,
  \item in terms of the Riemann-Roch theorems,
\end{enumerate}
and many more. 

\begin{remark}
  For any undergraduates reading this, an idea for a senior thesis project could be to write a unified treatise on these different perspectives, filled with examples and proofs of equivalence. I would love to read such a thesis, please contact me if you end up writing it.
\end{remark}

\subsection{The Euler Number as an Intersection Number}\label{sec:euler-number-intersection}

The primary interpretation of the Euler number for us is as an intersection number, specifically as a self-intersection number. Combined with the other perspectives on the Euler number in this section, this gives us a powerful way to compute the diagonal entries in an intersection form. We begin with the following theorem: 
\begin{theorem}\label{thm:euler-number-diagonal-intersection}
  Let $M^n$ be a closed oriented manifold, and let $\Delta\subset M\times M$ be the diagonal. 
  Then we have
  \[
    \chi(M) = I(\Delta, \Delta),
  \]
  where $I(\Delta,\Delta)$ is the self-intersection number of $\Delta$ in $M\times M$.
\end{theorem}
\begin{proof}
  Let $\{\alpha^i\}\subset \H^\bullet(M; \Q)$ be a graded basis of the cohomology ring and let $\{a_i\}$ be the Poincar\'e dual basis in cohomology, i.e. $(\alpha_i\smile \alpha^j)[M]=\delta^j_i$. By the K\"unneth theorem, we have a ring isomorphism
  \[
    \H^{k}(M\times M) \cong \bigoplus_{p+q=k} \H^p(M)\otimes \H^q(M).
  \]
  By noting that the fundamental class $[M\times M]=[M]\times [M]$, we can show that the Poincar\'e dual for the homology class $[\Delta]\in \H_{n}(M\times M)$ is given by
  \[
      \mathrm{PD}([\Delta]) = \sum_{i\geq 1}(-1)^{|\alpha_i|}\alpha_i\otimes \alpha^i.
  \]
  Taking the cup product, we get
  \[
    \begin{aligned}
      \mathrm{PD}([\Delta])\smile \mathrm{PD}([\Delta] = \sum_{i\geq 1}(-1)^{|\alpha_i|}(\alpha_i\otimes \alpha^i)
    \end{aligned}
  \]
\end{proof}

This theorem gives a way to define the Euler number of any vector bundle, not just the tangent bundle of a manifold. Note that the tangent bundle of $M$ is isomorphic to the normal bundle of $\Delta$ in $M\times M$. If $M$ was embedded in an oriented $2n$-dimensional manifold $X$ instead of in $M\times M$, we might define the Euler number of its normal bundle to be
\[
    \chi(\TT X/M) = I(M,M).
\]
Such a definition would recover the original definition o the Euler number by \cref{thm:euler-number-diagonal-intersection}. In a general oriented rank $n$ vector bundle $\mathcal{E} : E \to M$, we can consider the zero section $z : M \to E$ to be an embedding. This leads us to the following definition:

\begin{definition}\label{def:euler-number-self-intersection}
  If $\mathcal{E}^{2m}$ is an oriented real vector bundle over a compact manifold $M$, 
	the Euler number $\chi(\mathcal{E})$ is the intersection
	\[
		\chi(\mathcal{E}) = I(z,z)
	\]
	where $z : B \to E$ is the zero section.\footnote{We might consider intersections done in the disk bundle associated to $\mathcal{E}$ so that all manifolds remain compact.}
\end{definition}

\subsection{The Euler Number as a Geometric Invariant}\label{sec:euler-number-geometric}

A more geometric perspective on the Euler number was first noticed (independently) by Carl Friedrich Gauss and Pierre Bonnet in the mid 19th century. In addition to the topological perspective outline above, the Euler number can be viewed as an invariant of Riemannian structure on a manifold.

\begin{theorem}[Gauss-Bonnet]
  Let $M$ be a closed orientable two-dimensional Riemannian manifold with Gaussian curvature $K$. Then we have
  \[
    \chi(M) = \frac{1}{2\pi}\int_M K\,dA
  \]
  where $dA$ is an area form on $M$.
\end{theorem}
For instance, the sphere of radius $R$ has constant Gaussian curvature $K=1/R^2$ and surface area $4\pi R^2$. This gives $\int_{S^2}K\,dA =4\pi R^2/R^2=4\pi$ which is exactly $2\pi\cdot \chi(S^2)$.

In higher dimensions, there is an elegant generalization due to Chern in the mid 1940s, where the ``existence'' of the Euler number as a topological invariant was directly linked to the existence of a ``square root'' for the determinant of a skew-symmetric matrix.\footnote{Here we mean existence as a characteristic class independent from the Pontryagin classes for oriented real vector bundles, discussed in \cref{sec:characteristic-classes}.} For a $2m\times 2m$ skew-symmetric matrix $A$, there exists a polynomial $\Pf$ in $4m^2$ variables satisfying \[\det(A) = \Pf(A)^2.\] This polynomial is known as the \defn{Pfaffian polynomial}.

\begin{example}
  Some low-dimensional examples of the Pfaffian are given below:
  \[
    \begin{aligned}
      &A=\begin{pmatrix}
        0 & a_{21}\\
        -a_{21}&0
    \end{pmatrix}&
      &\implies\quad \Pf(A) = a
    \\[1em]
      &A=\begin{pmatrix}
        0 & a_{21} & a_{31} & a_{41}\\
        -a_{21} & 0 & a_{32} & a_{42}\\
        -a_{31} & -a_{32} & 0 & a_{43}\\
        -a_{41} & -a_{42} & -a_{43} & 0\\
    \end{pmatrix}& &\implies\quad \Pf(A) = a_{21}a_{43}-a_{31}a_{42}+a_{41}a_{32}
    \end{aligned}
  \]
  More generally, we have the combinatorial formula
  \[
    \Pf(A) = \frac{1}{2^m m!}\sum_{\sigma \in S_{2m}}\sgn(\sigma)\prod^m_{i=1}a_{\sigma(2i-1),\sigma(2i)}.
  \]
  It is also invariant under an orientation-preserving orthogonal change of basis, which is an important condition in Chern-Weil theory.
  See Section 25.3 of \cite{tu2017geometry} for more information on this remarkable algebraic coincidence. 
\end{example}


\begin{theorem}[Chern-Gauss-Bonnet]
  Let $M$ be a closed orientable $2m$-dimensional Riemannian manifold, and let $F_\nabla \in \Omega^2(M; \mathfrak{so}_{2m})$ be the curvature $2$-form of the Levi-Civita connection $\nabla$. Then we have
  \begin{equation}\label{eq:chern-gauss-bonnet}
    \chi(M) = \frac{1}{(2\pi)^m}\int_M \Pf(F_\nabla).
  \end{equation}
\end{theorem}

This is an example of a characteristic class from the Chern-Weil perspective. The only data of the manifold being used on the right hand side of \cref{eq:chern-gauss-bonnet} is a curvature form on the tangent bundle $\TT M$, which is skew-symmetric since it takes values in the Lie algebra $\mathfrak{so}_{2m}$ of skew-symmetric $2m\times 2m$ matrices. 
For any oriented rank $2m$ real vector bundle $\mathcal{E} : E \to M$, we can consider the de Rham cohomology class
\[
  e(\mathcal{E}) = \frac{1}{(2\pi)^m}\Pf(F_\nabla) \quad \in \HdR^{2m}(M)
\]
where $F_\nabla$ is the curvature form of a connection $\nabla$ on the bundle $\mathcal{E}$. The form $e(\mathcal{E})$ turns out to be a well-defined cohomology class which is invariant of the connection, i.e. a change of connection transforms $e(\mathcal{E})$ exactly. This leads to another definition of the Euler number for oriented real vector bundles:
\begin{definition}
  If $\mathcal{E}^n$ is an oriented real vector bundle with connection $\nabla$ over a closed manifold $M^n$, the \defn{Euler class} of the bundle is $e(\mathcal{E})\in \HdR^{2m}(M)$.
\end{definition}

\begin{theorem}
  The Euler class is well-defined and we have $\chi(\mathcal{E})=\int_M e(\mathcal{E})$, where $\chi(\mathcal{E})$ is defined as in \cref{def:euler-number-self-intersection}.
\end{theorem}
\begin{proof}
  \todo{good citation on Chern-Weil theory}
\end{proof}

From this definition, we get some nice immediate properties. 
\begin{proposition}\label{prop:euler-class-naturality}
  If $\mathcal{E}_1 : E_1 \to M_1$ and $\mathcal{E}_2 : E_2 \to M_2$ are bundles and 
  $f : M_1 \to M_2$ is a smooth map covered by a bundle map $\mathcal{E}_1\to \mathcal{E}_2$, then we have the relation
  \[
      e(\mathcal{E}_2) = f^* e(\mathcal{E}_1).
  \]
\end{proposition}
\begin{proof}
  If we choose a connection $\nabla$ on $\mathcal{E}_1$, we get a pullback connection $f^*\nabla$ on $\mathcal{E}_2$. The curvature transforms by pullback as well, so we get
  \[
      F_{f^*\nabla} = f^* F_{\nabla} \quad\implies\quad \Pf(F_{f^*\nabla}) = f^*\Pf(F_\nabla) \quad\implies\quad e(\mathcal{E}_2) = f^* e(\mathcal{E}_1). 
  \]
  This completes the proof.
\end{proof}

\begin{proposition}
  For any vector bundles $\mathcal{E}_1$ and $\mathcal{E}_2$ over a closed compact manifold, we have $e(\mathcal{E}_1\oplus \mathcal{E}_2)=e(\mathcal{E}_1)\smile e(\mathcal{E}_2)$.
\end{proposition}
\begin{proof}
The Pfaffian is multiplicative under direct sum, i.e. for any two skew-symmetric matrices $A$ and $B$ we have
\[
  \Pf(A\oplus B)^2 = \det(A)\det(B) = \Pf(A)^2\Pf(B)^2
  \quad\implies\quad
  \Pf(A\oplus B) = \Pf(A)\Pf(B).
\]
For any two oriented real vector bundles $\mathcal{E}_1^{2m_1},\mathcal{E}_2^{2m_2}$ with connection over $M$, we have 
\[
  \begin{aligned}
  e(\mathcal{E}_1\oplus\mathcal{E}_2) 
  &= \frac{1}{(2\pi)^{m_1+m_2}} \Pf(\Omega_{\mathcal{E}_1}\oplus \Omega_{\mathcal{E}_2}) \\
  &= \frac{1}{(2\pi)^{m_1}} \Pf(\Omega_{\mathcal{E}_1}) \wedge
  \frac{1}{(2\pi)^{m_2}} \Pf(\Omega_{\mathcal{E}_2}) \\
  &= e(\mathcal{E}_1)\smile e(\mathcal{E}_2).
  \end{aligned}
\]
This completes the proof.
\end{proof}

\begin{proposition}
  For an oriented vector bundle $\mathcal{E}$, we have
  \[
    e(\overline{\mathcal{E}}) = -e(\mathcal{E}).
  \]
\end{proposition}
\begin{proof}
  This follows from \cref{prop:euler-class-naturality}, pulling back by an orientation-reversing map.
\end{proof}

These properties scratch out some common axioms for characteristic classes. We discuss this in far more detail in \cref{sec:characteristic-classes}. 
Finally, we touch upon the interpretation of the Euler class as an obstruction to a non-zero section of a bundle. 

\begin{proposition}
  A vector bundle has a non-vanishing vector field if and only if its Euler class vanishes.
\end{proposition}
\begin{proof}
If a vector bundle $\mathcal{E}$ admits a non-zero section $s : M \to E$, then $\mathcal{E}$ splits as a direct sum $\mathcal{E}=\mathcal{E}'\oplus \mathcal{L}$ where $\mathcal{L}$ is the line bundle generated by $s$. Consequently, the bundle $\mathcal{E}$ admits a reduction from $\SO_{2m}$ to $\SO_{2m-1}$ along the complement of $s$. We can modify the connection along this reduction so that it preserves the subbundle $\mathcal{L}$, and thus is non-zero only on $\so_{2n-1}\subset \so_n$. However, the Pfaffian vanishes and so does the Euler class.

The reverse direction is a bit trickier, \todo{explain or cite}
\end{proof}


Since the Euler number of a sphere vanishes only in odd dimensions, an easy consequence of this is the infamous ``Hairy Ball Theorem'', although it can be proved by simpler means. 

\begin{theorem}[Hairy Ball Theorem]
  There exists a non-vanishing vector field on $S^n$ if and only if $n$ is odd.
\end{theorem}

\subsection{The Euler Number as a Cohomology Class}\label{sec:euler-number-cohomology}

One interpretation of the Euler class is as an invariant of oriented sphere bundles, not just vector bundles. Of course, every vector bundle has an associated sphere bundle there are natural constructions of the Euler class of a sphere bundle without any reference to a vector bundle. 

For instance, from an obstruction theory perspective, a non-zero section of a vector bundle corresponds to a section of a sphere bundle. \todo{give reference to obstruction theory text}

\begin{theorem}[Gysin Sequence]\label{thm:gysin-sequence}
	For any $S^{n-1}$ fiber bundle  $p: E\to B$ over a simply-connected base $B$,
	there is a long exact sequence 
\[
	\begin{tikzcd}
	  & \cdots\rar{p^*} \snakenode{X} & \H^{k-1}(E)\snakearrow{X} \\
		{\H^{k-n}(B)}\rar{e\,\smile} & {\H^k(B)}\rar{p^*}\snakenode{Y} & {\H^k(E)}\snakearrow{Y} \\
		{\H^{k-n+1}(B)}\rar{e\,\smile} & \cdots
\end{tikzcd}
\]
	where $e\in \H^n(B)$ is the Euler class of the bundle. 
	This is the \defn{Gysin sequence} of the bundle $p$.
\end{theorem}
\begin{proof}
	See Section 4.D of \cite{hatcher2002topology} or Theorem 17.9.2 of \cite{dieck2008algebraic}.
\end{proof}

\pagebreak
\section{Cobordism}\label{sec:cobordism}

The basic principle of cobordism is to declare two manifolds equivalent if there is a manifold one dimension higher which connects the two manifolds. As an equivalence relation, cobordism is far looser than the notions of homoemorphism or diffeomorphism and so allows for a full classification of manifolds. Many notions in geometry and topology -- for instance characteristic classes -- depend solely on the cobordism type of a manifold, so understanding the structure of cobordism is immensely helpful. 
A comprehensive overview of the theory of cobordism can be found in Strong's notes on the topic \cite{strong1968cobordism}.

\begin{remark}
	The implied compactness assumption throughout the thesis is important here, otherwise any manifold $M$ is trivially the boundary of $M\times [0,\infty)$. 
\end{remark}

For now, let us begin with the two simplest types of cobordism. We now temporarily disband with the assumption made throughout the rest of the thesis that all manifolds need be connected.

\begin{definition}
	An \defn{unoriented cobordism} between closed $n$-manifolds $M_1$ and $M_2$ is an $(n+1)$-manifold $W$ with $\partial W = M_1\sqcup M_2$. We use $W : M_1\bord M_2$ to refer to the cobordism.
\end{definition}

For a simple example of a cobordism between a circle and a disjoint union of circles, see \cref{fig:pair-of-pants}. Note that this cobordism could be made much simpler by removing the handle. Simplifying cobordisms in this way is one of the major applications of surgery theory.
\begin{figure}[ht]
	\centering
	\import{diagrams}{pair-of-pants.pdf_tex}
	\caption{A cobordism $W$ between $S^1$ and $(S^1\sqcup S^1)$.}\label{fig:pair-of-pants}
\end{figure}

To change the theory, we can introduce structure on the manifolds $M_1$, $M_2$, and require the cobordism to preserve this structure. The simplest type of structure we could require is an orientation.

\begin{definition}
	An \defn{oriented cobordism} between closed oriented $n$-manifolds $M_1$ and $M_2$ is an oriented $(n+1)$-manifold $W$ with $\partial W = M_1\sqcup (-M_2)$. We use the notation to $W : M_1\sobord M_2$ refer to the cobordism.
\end{definition}

If we have two unoriented cobordisms $W : M_1 \bord M_2$ and $W' : M_1'\bord M_2'$, their disjoint union $W\sqcup W'$ is an unoriented cobordism from $M_1\sqcup M_1'$ to $M_2\sqcup M_2'$. Disjoint union is thus a well-defined commutative operation on cobordism classes of manifolds.
\begin{figure}[ht]
	\centering
	\import{diagrams}{unoriented-cobordism-Z2.pdf_tex}
	\caption{An unoriented cobordism $M\times [0,1] : M\sqcup M \bord \varnothing$.}\label{fig:unoriented-cobordism-Z2}
\end{figure}
The identity element is the empty set, and the inverse of a manifold $M$ is the manifold itself, since $M\times [0,1]$ is a cobordism from $M\sqcup M$ to the identity element $\varnothing$ (see \cref{fig:unoriented-cobordism-Z2}). This observation motivates the following definition.

\begin{definition}
The \defn{$k$-th unoriented cobordism group}[unoriented cobordism group] $\Omega^\O_k$ is the abelian group of cobordism classes of closed $k$-manifolds under the disjoint union. 
\end{definition}

Since the additive inverse of manifold $M$ in $\Omega^\O_k$ is $M$ itself, it follows that $\Omega_k^\O$ is a $\Z/2$-module. We can compute some basic examples of the group $\Omega_k^\O$ by hand. For example,

\begin{example}
	$\Omega_0^\O\cong \Z/2$. 

	An unoriented 0-dimensional manifold is just a set of points, and any pair of points is cobordant to the empty set by a path connecting them. Since adding pairs of points doesn't change the cobordism type, the number of points modulo 2 determines the cobordism class entirely.
\end{example}

\begin{example}
	$\Omega_1^\O\cong 0$.

	Any closed $1$-dimensional manifold must be a disjoint union of circles, and each circle is filled by a disk.
\end{example}

\begin{example}
	$\Omega_2^\O\cong \Z/2$.

	A connected closed $2$-dimensional manifold must either be an orientable surface of genus $g$ or a connected sum of projective planes. Every orientable genus $g$ surface can clearly be filled in, so these are null-cobordant. However, $\RP^2$ is not null-cobordant and generates $\Omega_2^\O$. This follows because the signature mod $2$ is a cobordism invariant (see \cref{prop:signature-cobordism}). Since $\RP^2$ has signature $1$, it cannot be cobordant to any genus $g$ oriented surface which has signature $0$. 
	In the case of $\Omega_2^\O$, it turns out that the signature mod $2$ is a perfect invariant, and so $\RP^2$ generates.
\end{example}

The signature is not generally a perfect invariant. For instance, $\Omega_4^\O\cong \Z/2\oplus \Z/2$ is generated by $\RP^4$ and $\RP^2\times \RP^2$ but both manifolds have signature $1$. The better unoriented cobordism invariants are the Stiefel-Whitney numbers, a set of integers in $\Z/2$ which completely determine the unoriented cobordism type of a manifold. We do not address these invariants directly, so for an overview see Section 4 of \cite{milnorstasheff1974}.

\begin{remark}
	There is a sense in which the cobordism groups can be thought of as homology groups. If we let $\mathsf{Man}^n$ be the category of (not necessarily connected) compact $n$-dimensional manifolds with boundary, the operation of disjoint union introduces an abelian group structure with identity $\emptyset$.
	The boundary of a manifold can then be interpreted as a group homomorphism
	\[
		\lkxfunc{\partial}{\mathsf{Man}^n}{\mathsf{Man}^{n-1}.}
	\]
	Furthermore, since the boundary of a manifold is closed, we get a chain complex
	\[
		\cdots\lkxto[\partial] \mathsf{Man}^n\lkxto[\partial] \mathsf{Man}^{n-1}\lkxto[\partial]\cdots\lkxto[\partial] \mathsf{Man}^1 \lkxto[\partial] \mathsf{Man}^0 \lkxto 0.
	\]
	The homology groups of this chain complex are exactly the unoriented cobordism groups $\Omega_k^\O$, i.e. it consists of the set of closed manifolds modulo manifolds which are boundaries. This interpretation of cobordism as a homology is classical, in fact Pontryagin referred to cobordism as ``homology'' as late as 1959 in \cite{pontryagin1959homotopy}.

\begin{proposition}\label{prop:cobordism-product}
	The product of manifolds is a well-defined operation on cobordism.
\end{proposition}
\begin{proof}
	If $W : M_1\bord M_2$ and $W' : M_1'\bord M_2'$ are cobordisms, consider the following join along the submanifold
	\[
			X = (W\times M_1')\cup_{M_2\times M_1'} (M_2\times W').
	\]
	Then $X : M_1\times M_1' \bord M_2\times M_2'$ so we are done.
\end{proof}

The multiplicative structure of cobordism allows us to combine the data of the cobordism groups into a cobordism ring. Many theorems about the classification of manifolds up to cobordism take a succinct description in terms of this ring.

\begin{definition}
	The \defn{unoriented cobordism ring} $\Omega^\O_\bullet$ is the set of oriented cobordism classes of closed manifolds under the operations of disjoint union and product.
\end{definition}

The unoriented cobordism ring has a grading by
\[
	\Omega_\bullet^\O = \bigoplus_{k\geq 0} \Omega^\O_k.
\]

\begin{theorem}[Dold]
	The unoriented cobordism ring is a polynomial ring with generators
	\[
		\Omega^\O_\bullet \cong \Z[x_2, x_4, x_5, x_6, x_8,\ldots].
	\]
	In even dimensions, the generator is $x_{2m}=\RP^{2m}$ and in dimensions $i=2^r(2s+1)-1$ the generator is a space known as a \defn{Dold manifold} $P(r,s)$, constructed as a total space of a fiber bundle over $\RP^{2^r-1}$ with fibers $\CP^{s2^r}$.
\end{theorem}
\begin{proof}
	The proof of by Arnold Dold constructs manifolds representing all possible Stiefel-Whitney numbers, which by the work of Thom in \cite{thom1954} is enough to classify manifolds up to unoriented cobordism.

	See \cite{dold1956} for details.
\end{proof}

We observe similar structure in the oriented case.

\begin{definition}
	The \defn{$k$-th oriented cobordism group}[oriented cobordism group] $\Omega^\SO_k$ is the abelian group of oriented cobordism classes of closed $k$-manifolds
	under disjoint union. The identity component is the empty set $\varnothing$, and negation is given by reversing orientation. 
\end{definition}

Note that the oriented cobordism group can be thought of as a $\Z$-module (or just a group), with multiplication action on a closed manifold $M$ given by
\[
	0\cdot M = \emptyset, \quad n \cdot M = \underbrace{M\sqcup \cdots \sqcup M}_{n}, \quad (-n)\cdot M = \underbrace{\overline{M}\sqcup \cdots \overline{M}}_n
\]
for all $n>0$.

\begin{example}
	$\Omega_0^\SO \cong \Z$. 

	An oriented 0-dimensional manifold is still a set of points as in the unoriented case, however the orientation now equips each point with a ``charge'', we might label as $+$ or $-$. Points of opposite ``charges'' cancel out by a path between them oriented from $-$ to $+$. After cancelling opposite charges, we are left with some integer number of points, either positive or negative by the charge of the remaining points.
\end{example}

\begin{example}
	$\Omega_1^\SO \cong 0$, $\Omega_2^\SO \cong 0$, and $\Omega_3^\SO \cong 0$.

	The first two follow immediately from the discussion in the unoriented case, and the third is non-trivial.
\end{example}

\begin{example}
	$\Omega_4^\SO \cong \Z$.

	A direct geometric proof of this is tricky, and this is usually viewed as a corollary of the complete classification \cref{thm:oriented-cobordism-structure}. 
	We will note an isomorphism is given by the signature of a manifold
	\begin{equation}\label{eq:signature-cobordism-isomorphism}
		\lkxfunc{\sigma}{\Omega_4^\SO}{\Z,}
	\end{equation}
	so $\CP^2$ generates $\Omega_4^\SO$ since it has signature $1$. 
\end{example}

\begin{remark}
	The isomorphism \cref{eq:signature-cobordism-isomorphism} has a remarkable geometric generalization to the case of Spin cobordism (for manifolds with Spin structure), where we have an isomorphism
	\[
		\lkxfunc{\alpha}{\Omega^\Spin_4}{2\Z}
	\]
	arising from the $\Ahat$ genus. In dimension $4$, this map is determined by the relation $\alpha=\sigma/8$ for Spin manifolds, and so it turns out that $\Omega^\Spin_4$ is generated by the K3 surface $\mathcal{K}$ which has signature $-16$ and thus $\alpha(\mathcal{K})=-2$.
	We discuss the $\Ahat$ genus and its relation to the signature in \cref{chap:invariants}. For a comprehensive overview see the wonderful book on spin geometry by Lawson and Michelsohn \cite{lawson1989spin}.
\end{remark}

A partial classification of manifolds up to unoriented cobordism was first accomplished in the \defn{rational cobordism ring} $\Omega_\bullet^\SO\otimes \Q$. Two manifolds $M_1$ and $M_2$ represent the same rational cobordism if there is a cobordism
\[
		n\cdot M_1 \sobord n\cdot M_2
\]
for some large enough integer $n$. The idea is that we do not consider disjoint union with a manifold whose cobordism class is torsion to have any effect on the rational cobordism class. Up to this equivalence, the classification of oriented manifolds takes the following form:

\begin{theorem}[Thom]\label{thm:oriented-cobordism-structure}
	The rational cobordism ring is a polynomial ring with generators
	\[
		\Omega_\bullet^\SO\otimes \Q \cong \Q[x_4, x_8, x_{12}, \ldots]
	\]
	where $x_{4k}$ are cobordism classes representing $\CP^{2k}$.
\end{theorem}
\begin{proof}
	The classical proof of this result by Thom involves constructing a ring isomorphism $\Omega_\bullet^\SO \cong \H^\bullet(\MSO)$, with $\MSO$ a space known as a Thom spectrum. The result then follows from the periodicity theorem of Bott \cite{bott1959stable} for the homotopy groups of the stable group $\SO$.

	For the original proof, see \cite{thom1954}.
\end{proof}

Including torsion classes into this classification is tricker. We note in passing that Dold manifolds are oriented, and give a subring
\[
	\Z/2[x_5,x_9,x_{11},\ldots] \subset \Omega_\bullet^\SO.
\]
However, the general classification of oriented manifolds up to cobordism with torsion is a bit harder. For further reading, see \cite{strong1968cobordism}, \cite{hirzebruch1966methods}, or \cite{may1999concise}.

% In higher dimensions, the classification becomes much more interesting.
%
% \begin{figure}[ht]
% 	\renewcommand{\arraystretch}{1.2}
% 	\centering
% 	\begin{tabular}{r||c|c||c|c}
% 		$k$ & $\Omega_k$          & generators                                          & $\Omega_k^\SO$ & generators                 \\
% 		\hline
% 		$0$ & $\Z/2$              & a point                                             & $\Z$           & a point                    \\
% 		$1$ & $0$                 &                                                     & $0$            &                            \\
% 		$2$ & $\Z/2$              & $\RP^2$                                             & $0$            &                            \\
% 		$3$ & $0$                 &                                                     & $0$            &                            \\
% 		$4$ & $\Z/2\oplus \Z/2$   & $\RP^4$, $\RP^2\times \RP^2$                        & $\Z$           & $\CP^2$                    \\
% 		$5$ & $\Z/2$              & $\SU_3/\SO_3$                                       & $\Z/2$         & $\SU_3/\SO_3$              \\
% 		$6$ & $(\Z/2)^{\oplus 3}$ & $\RP^6$, $\RP^2\times \RP^4$, $(\RP^2)^{\times 3}$, & $0$            &                            \\
% 		$7$ & $\Z/2$              & $(\SU_3/\SO_3) \times \RP^2$                        & $0$            &                            \\
% 		$8$ & $(\Z/2)^{\oplus 4}$ & $\RP^8, \RP^6\times \RP^2, \cdots$                  & $\Z\oplus \Z$  & $\CP^4, \CP^2\times \CP^2$ \\
% 	\end{tabular}
% 	\medskip
% 	\caption{Structure of unoriented and oriented cobordism groups.}\label{fig:cobordism-structure-table}
% \end{figure}

% The structure of \cref{fig:cobordism-structure-table} makes a lot more sense in the context of

\subsection{The Signature and Cobordism}\label{sec:signature-and-cobordism}

In our discussion of oriented cobordism, we noted that the signature of a manifold is one of its fundamental cobordism invariants. Here, we flesh out some of the details. The basic observation relating cobordism and the signature is the following proposition.

\begin{proposition}\label{prop:signature-cobordism}
	If $M$ is the boundary of a manifold, then $\sigma(M)=0$.
\end{proposition}
\begin{proof}
	Let $W$ be a coboundary of $M$, and let us assume without loss of generality that is is connected. If not, we can always take a connected sum of the components without altering the boundary. Note that suffices to consider the case that $M$ has dimension $4k$, since otherwise the signature is zero. The exact sequence of the pair $(W,M)$ for cohomology is given by
	\[
		\cdots \lkxto \H^\ell(W,M)\lkxto \H^\ell(W)\lkxto[\rho] \H^{\ell}(M)\lkxto \H^{\ell+1}(W,M)\lkxto\cdots
	\]
	In middle dimension $\ell=2k$, it follows that $\rho$ is surjective so every middle dimensional cohomology class on $M$ has an extension to $W$. (The picture in de Rham cohomology should be to multiply a form on $M$ by a bump function on a collar neighborhood of $M$) Furthermore, we have a commutative square
	\todo{finish this proof.}
\end{proof}

\begin{corollary}
	The signature is an oriented cobordism invariant. In other words, whenever $M_1\sobord M_2$ we have $\sigma(M_1)=\sigma(M_2)$.
\end{corollary}
\begin{proof}
Since $\sigma(M_1\sqcup\overline{M}_2)=\sigma(M_1)+\sigma(\overline{M}_2)=\sigma(M_1)-\sigma(M_2)$, we get $\sigma(M_1)=\sigma(M_2)$ whenever $M_1\sqcup \overline{M}_2$ bounds.
\end{proof}

\begin{remark}
	By \cref{rmk:homology-dimension-4-embedding}, every homology class $\alpha\in \H^2(M)$ of a simply-connected $4$-manifold $M$ can be represented by an embedded sphere. In this case, there is an elegant geometric proof of \cref{prop:signature-cobordism} in terms of the intersection numbers of embedded spheres lifted to the coboundary. For details, see page 121 of \cite{scorpan2005wild}.
\end{remark}

\begin{proposition}
	If $M_1$ and $M_2$ are oriented manifolds, then $\sigma(M_1\times M_2)=\sigma(M_1)\cdot \sigma(M_2)$.
\end{proposition}

\begin{proof}
	See Theorem 8.2.1 of \cite{hirzebruch1966methods} or \cite{chernhirzserre1957index}.
\end{proof}

\begin{remark}
	In the paper \cite{chernhirzserre1957index}, Chern, Hirzebruch, and Serre prove the more general result that if $F \to E \to B$ is a bundle with trivial monodromy action by $\pi_1(B)$ on $\H^\bullet(F)$, we have
	\begin{equation}\label{eq:multiplicativity-of-genera}
		\sigma(E) = \sigma(F)\cdot \sigma(B).
	\end{equation}
	This is a general pattern for many numerical invariants in topology. For instance, \cref{eq:multiplicativity-of-genera} holds if we replace signature with the Euler number of a manifold.
\end{remark}

\begin{corollary}
	The signature is a well-defined ring homomorphism
	\[
		\lkxfunc{\sigma}{\Omega^\SO_\bullet}{\Z}
	\]
	where $\Omega^\SO_\bullet$ is the oriented cobordism ring.
\end{corollary}

Since torsion elements of $\Omega^\SO_\bullet$ are sent to zero in $\Z$, the signature factors through the rational cobordism ring $\Omega^\SO_\bullet\otimes \Q$, so we actually have a ring homomorphism
\[
	\lkxfunc{\sigma}{\Omega^\SO_\bullet\otimes \Q}{\Z.}
\]
However, by \cref{thm:oriented-cobordism-structure} we know the ring structure on $\Omega^\SO_\bullet\otimes \Q$, it is polynomial in the even complex-dimensional complex projective planes. By \cref{prop:intersection-form-complex-projective-plane}, the signature of these generators is $1$. The signature thus gives a simple way to help identify the rational oriented cobordism class of a manifold. It is not a perfect invariant however, even when the dimension is taken into account since
\[
		\sigma(\CP^2\times\CP^2) = \sigma(\CP^2)\cdot \sigma(\CP^2)= 1\quad\textrm{and}\quad \sigma(\CP^4)=1,
\]
yet these are non cobordant by \cref{thm:oriented-cobordism-structure}.

\begin{example}
	Since $\Omega^\SO_4$ has no torsion and one generator, there is an oriented cobordism
	\[
		\mathcal{K} \sobord \underbrace{\overline{\CP}^2\+\cdots\+\overline{\CP}^2}_{16}
	\]
	where $\mathcal{K}$ is the Fermat quintic $4$-manifold with signature $-16$ (see \cref{example:k3}). These sixteen projective planes correspond to the sixteen singularities of the Kummer construction (see \cref{rmk:kummer-construction}).
\end{example}

\subsection{The $h$-Cobordism Theorem}

There is a strengthening of the notion of cobordism which requires ``nothing interesting'' to happen on the manifold representing the cobordism
More precisely, we define:
\begin{definition}
	A cobordism $W : M_1\bord M_2$ is said to be an \defn{$h$-cobordism} if $M_1$ and $M_2$ admit deformation retracts from $W$. We denote $h$-cobordisms by $\hbord$ when the structure is clear (e.g. $\O$ or $\SO$).
\end{definition}

Two spaces which are $h$-cobordant must thus have the same homotopy type since they are deformation retracts of the same space. Clearly, this is vastly stronger than the notions of oriented and unoriented cobordism discussed in the previous section. While this notion drastically simplifies the topology of the cobordism, we lose the simple cobordism classifications of the previous section since it is so much more restrictive. 

In the early 1960s, Stephen Smale proved \cite{smale1961generalized} a powerful characterization of $h$-cobordisms in high dimensions. Since this soon implied the generalized Poincar\'e conjecture in dimension $\geq 5$ in the topological and PL categories, he was awarded the Fields medal for his work. 
The $h$-cobordism and its generalizations are now totally fundamental to the study of smooth manifolds in geometric topology since they allow us to turn questions of diffeomorphism into questions of cobordism.
We mentioned earlier that ``nothing interesting'' should happen on an $h$-cobordism, but in simply-connected high-dimensional manifolds this holds on the nose -- all simply-connected $h$-cobordisms are the trivial cylinder cobordism. More precisely:

\begin{theorem}[$h$-cobordism]\label{thm:h-cobordism}
	Within some category of manifolds $\mathscr{C}$, if $M$ and $N$ are closed, simply-connected manifolds of dimensions $\geq 5$ and $W : M \hbord N$ is a simply-connected $h$-cobordism, then $W$ is $\mathscr{C}$-isomorphic to the cylinder $M\times [0,1]$. Furthermore, the isomorphism can be chosen to be the identity on $M\times \{0\}$.
\end{theorem}
\begin{proof}
	A full proof of this theorem would take us too far afield, so we give a sketch. The basic idea is to use a Morse function on $W$ (see \cref{sec:morse-theory}), and deform it until it has no critical points on the interior of $W$. Integrating the resulting unit gradient vector field then gives the required $\mathscr{C}$-isomorphism.

	The classic source on the $h$-cobordism theorem is due to Milnor \cite{milnor1965hcobordism}. For a more visual introduction, see Chapter 1 of \cite{scorpan2005wild}. Another excellent exposition is found in Chapter VIII of \cite{kosinski1993differential}.
\end{proof}

For us, the main application of the $h$-cobordism theorem is the following corollary:

\begin{corollary}\label{thm:h-cobordism-diffeomorphism}
	In the smooth oriented manifold category, two simply-connected closed manifolds of dimensions $\geq 5$ are $h$-cobordant if and only if they are diffeomorphic (by an orientation preserving diffeomorphism).
\end{corollary}

\begin{remark}
	The simple-connectedness assumption of the $h$-cobordism theorem is required. Let $L(p,q)$ be the quotients of the sphere $S^3$ by the $\Z/p$-action 
	\[
		(z_1,z_2) \lkxmapsto (e^{2\pi i/p}\cdot z_1, e^{2\pi i q/p}\cdot z_2)
	\]
	where we consider $S^3\subset \C^2$. The resulting manifolds are known as \defn{lens spaces}.  In particular, they have fundamental group $\Z/p$ but are not generally homeomorphic for different $q$. It can be shown then that the spaces $L(7,1)\times S^4$ and $L(7,2)\times S^4$ are $h$-cobordant but not diffeomorphic. For details, see \cite{counterexamples2022}.
\end{remark}

\begin{remark}
	In the topological category $\TOP$, the $h$-cobordism theorem is true in dimension $4$. This highly non-trivial fact was proved by Frank Quinn in \cite{quinn1982} building on the work by Freedman in \cite{freedman1982manifold}, and in some sense completing Freedman's ``revolution in topological $4$-manifolds''. In the smooth category $\DIFF$, the $4$-dimensional $h$-cobordism theorem breaks down unless more assumptions are added. See \cite{cfhs1996hcobordism} for the salvaged statement o the theorem and \cite{akbulut1991} for the idea behind constructing a counterexample.
\end{remark}

Almost immediately, the $h$-cobordism theorem implies the generalized Poincar\'e conjecture in dimensions $\geq 5$ for $\TOP$ and $\PL$. 

\begin{corollary}\label{thm:generalized-poincare-smale}
	In the manifold categories $\mathscr{C}=\TOP$ or $\PL$, any manifold $M$ of dimension $\geq 5$ which has the homotopy type of a sphere is $\mathscr{C}$-isomorphic to a sphere.	
\end{corollary}
\begin{proof}
	\begin{figure}[ht]
		\centering
		\import{diagrams}{cone-construction.pdf_tex}
		\caption{Turning a cone into an $h$-cobordism.}
	\end{figure}

	We provide an sketch of a proof here. Given a manifold $M$ with the homotopy type of a sphere, take the cylinder $M\times [0,1]$. Collapsing the set $M\times \{1\}$ gives the cone $\mathrm{C}M$ over $M$. Note that this only has a canonical manifold structure in the $\TOP$ and $\PL$ categories. By a similar argument as is used in \cref{lemma:null-h-cobordant-iff-bounds-contractible}, we remove the interior of a disk $D^{n+1}$ from $\mathrm{C}M$ to get an $h$-cobordism from $M$ to $S^{n}$. This gives a $\mathscr{C}$-isomorphim from $M$ to $S^n$, completing the proof.
\end{proof}

These results tremendously powerful. With \cref{thm:h-cobordism-diffeomorphism}, we have arrived at our first major simplification to the problem finding and classifying exotic spheres. To classify smooth structures on $S^n$, it is enough to classify the $h$-cobordism classes of smooth manifolds which are homeomorphic to $S^n$, and to find smooth manifolds which are homeomorphic to $S^n$, it suffices to consider smooth manifolds which have the homotopy type of $S^n$.

\begin{definition}
	A \defn{homotopy $n$-sphere}[homotopy sphere] is a smooth manifold which is homotopy equivalent to the sphere $S^n$.
\end{definition}

As is often the case in math, we wrap up this definition in some notation.

\begin{definition}
	Let $\Theta^n$ be the pointed set of $h$-cobordism classes of homotopy $n$-spheres (the basepoint is the ordinary sphere $S^n$).
\end{definition}

\begin{remark}
	Given the results we have proved so far, the set $\Theta^n$ is only the set of smooth structures on $S^n$ in dimensions $n\geq 5$. In lower dimensional cases, we need to work on a case by case basis. See \cref{sec:low-dimensions} for more information.
\end{remark}



\section{Groups of Homotopy Spheres}\label{sec:groups-of-homotopy-spheres}

So far, this is a fairly general setup. Instead of spheres, we could use any simply-connected base manifold and consider the set of $h$-cobordism classes of manifolds homotopy equivalent to it. This is known as the \defn{surgery structure set}, one of the central objects of study in surgery theory. In the case of spheres, there is a special bit of extra data which does not usually generalize -- a group structure under the connected sum operation defined in \cref{sec:connected-sum}. 
As we will see in \cref{chap:invariants} and \cref{chap:conclusion}, this group structure on $\Theta^n$ simplifies the work of classification tremendously.

\begin{theorem}\label{thm:group-of-homotopy-spheres}
	The connected sum turns $\Theta^n$ into a group with identity element $S^n$.
\end{theorem}

\begin{proof}
	We have already proved that the connected sum is well-defined up to diffeomorphism, associative, and commutative up to orientation preserving diffeomorphism.  
	The remaining proof splits as three lemmas -- these are Lemmas 2.2, 2.3, and 2.4 in \cite{milnorkervaire1963groups}.

	\begin{lemma}
		The connected sum is a well-defined operation on $\Theta^n$.
	\end{lemma}
	\begin{proof}
		We will prove something slightly more general. Let $M_1, M_1'$ and $M_2, M_2'$ be closed, simply-connected, and oriented manifolds, and suppose $W_1 : M_1\sohbord M_1'$ and $W_2 : M_2\sohbord M_2'$ are $h$-cobordisms. The goal is to construct an $h$-cobordism $W : (M_1\+M_2)\sohbord (M_1'\+M_2')$.

		\begin{figure}[ht]
			\centering
			\import{diagrams}{join-h-cobordism.pdf_tex}
			\caption{A join of two $h$-cobordisms along embedded arcs.}\label{fig:connected-sum-of-h-cobordisms}
		\end{figure}

		Recall that the connected sum $M_1\+M_2$ was defined as the join of $M_1$ and $M_2$ along points $p_1\in M_1$ and $p_2\in M_2$. Let us assume these were the points along which the connected sum $M_1\+M_2$ was taken, and $p_1'\in M_1'$ and $p_2'\in M_2'$ the points along which the connected sum $M_1'\+M_2'$ was taken. Choose smooth paths $\gamma_i : [0,1] \to W_i$ which send $p_i\mapsto p_i'$, and assume without loss of generality that $\gamma_i$ are embeddings which are transverse to the boundary of $\partial W_i$, making it possible to get tubular neighborhoods
		\[
			\lkxfunc{\iota_i}{\R^n\times [0,1]}{W_i.}
		\]
		Let $W$ be the join of $W_1$ and $W_2$ along the image of the arcs $\gamma_i$, as depicted in \cref{fig:connected-sum-of-h-cobordisms}. This is an oriented smooth manifold with boundary
		\[
			\partial W = (M_1\#M_2)\sqcup -(M_1'\+M_2')
		\]
		so $W$ is a cobordism $W : (M_1\+M_2)\sobord (M_1'\+M_2')$. We just need to prove that $W$ is an $h$-cobordism to complete the proof.

		\begin{lemma}\label{lemma:removing-point-homotopy-equivalence-h-cobordism}
			If $W : M \sohbord M'$ is an $h$-cobordism with $M,M'$ compact oriented and simply connected, $p\in M$ is a point, and $T$ is a tubular neighborhood of an arc from $p$ to $p'\in M'$, then the inclusion $M\setminus\{p\} \to W\setminus T$ is a homotopy equivalence.
		\end{lemma}
		\begin{proof}
			The long exact sequence of the inclusion $j : (M,M\setminus\{p\}) \to (W, W\setminus T)$ gives
			\[
				\H_{i+1}(W\setminus T, M\setminus\{p\})\lkxto
				\H_i(M\setminus\{p\}) \lkxto[j_*] \H_i(W\setminus T) \lkxto \H_i(W\setminus T, M\setminus\{p\})
			\]
			Note that by excision, $\H_i(W\setminus T, M\setminus\{p\})\cong \H_i(W,M)$, and since $W$ is an $h$-cobordism, $\H_i(W,M)\cong 0$. Consequently, $j$ induces isomorphisms on homology and since $M,M'$, and $W$ are simply connected, Whitehead's theorem and the Hurewicz isomorphism imply that $j$ is a homotopy equivalence.
		\end{proof}

		With \cref{lemma:removing-point-homotopy-equivalence-h-cobordism} in mind, we use the Mayer-Vietoris sequence
		\[\begin{tikzcd}
				{\H_i(S^{n-1})} & {\H_i(M_1\setminus\{p_1\})\oplus\H_i(M_2\setminus\{p_2\})} & {\H_{i}(M_1\+M_2)} & {\H_{i-1}(S^{n-1})} \\
				{\H_i(S^n)} & {\H_i(W_1\setminus T_1)\oplus\H_i(W_2\setminus T_2)} & {\H_{i}(W)} & {\H_{i-1}(S^n)}
				\arrow[from=1-1, to=1-2]
				\arrow[from=1-1, to=2-1]
				\arrow["f"', from=1-2, to=1-3]
				\arrow["{(j_1)_*\oplus (j_2)_*}", from=1-2, to=2-2]
				\arrow[from=1-3, to=1-4]
				\arrow["j_*", from=1-3, to=2-3]
				\arrow[from=1-4, to=2-4]
				\arrow[from=2-1, to=2-2]
				\arrow["g"', from=2-2, to=2-3]
				\arrow[from=2-3, to=2-4]
			\end{tikzcd}\]
			where $j_i : M_i\setminus \{p_i\} \to W_i\setminus T_i$ and $j : M_1\+ M_2\to W$ are inclusions. Note that $(j_1)_*\oplus (j_2)_*$ is an isomorphism by \cref{lemma:removing-point-homotopy-equivalence-h-cobordism}. When $i$ is not $1,n-1,$ or $n$, it follows immediately that $f$ and $g$ are isomorphisms and so $j$ is an isomorphism. In the remaining cases, a simple application of the  snake lemma does the trick.
			By simple connectedness, Whitehead's theorem, and the Hurewicz isomorphism, it follows that the inclusion of $M_1\+M_2 \to W$ is a homotopy equivalence, completing the proof.
	\end{proof}

	\begin{remark*}
		In the previous lemma, the simple-connectedness assumption was not strictly required -- only used to reduce the problem of homotopy equivalence to a homology problem by Whitehead's theorem. It is possible to directly construct an explicit homotopy equivalence with a more careful argument.
	\end{remark*}

	Next, we will prove a simple characterization of when a homotopy sphere is equal to the identity element in $\Theta^n$.

	\begin{lemma}\label{lemma:null-h-cobordant-iff-bounds-contractible}
		A simply-connected manifold $M$ is $h$-cobordant to $S^n$ if and only if $M$ bounds a contractible manifold.
	\end{lemma}
	\begin{proof}
		The forward direction is fairly straightforward. If $W : M\sohbord S^n$ is an $h$-cobordism, we can glue a tubular neighborhood of $D^{n+1}$ in $\R^{n+1}$ to $(-S^n)\subset \partial W$ along a collar neighborhood of $-S^n$ in $\partial W$. Any homotopy equivalence of $W$ to $S^n$ can then be extended to a contraction along $D^{n+1}$.

		In the reverse direction, let us assume that $M$ bounds a contractible manifold $W$. Pick any point in the interior of $M$ and remove the interior of an embedded disk $D^{n+1}\subset W$. This gives a simply-connected cobordism $W'=W\setminus \Int(D^{n+1}) : M\sobord (-S^n)$. By the long exact sequence of the inclusion $j : (D^{n+1}, S^n) \to (W,W')$, we have
		\[
			\H_i(W', S^n) \lkxto H_i(S^n)\lkxto[j_*] \H_i(W')\lkxto \H_{i-1}(W',S^n),
		\]
		and by contractability of $W$ the boundary terms vanish and so $S^n \to W'$ is a homotopy equivalence. Then, by the relative Poincar\'e duality isomorphism $\H_k(W',M)\cong \H^{n+1-k}(W', S^n)$ and the long exact sequence of the pair $(W',M)$, it follows that $M\to W'$ is a homotopy equivalence and so $W'$ is in fact an $h$-cobordism. 
	\end{proof}

	\begin{lemma}
		If $M$ is a homotopy sphere, then $M\+\overline{M}$ bounds a contractible manifold.
	\end{lemma}
	\begin{proof}
		There is a thorough proof of this in \cite{milnorkervaire1963groups} (see Lemma 2.4) so we only sketch a geometric picture here. The connected sum $M\+\overline{M}$ can be thought of as a union of punctured spaces $M^\circ$ and $\overline{M}^\circ$. Turning the punctured homotopy sphere $M^\circ$ inside out through the opening gives the reverse orientation $\overline{M}^\circ$.

		% \begin{figure}[ht]
		% 	\centering
		% 	\import{diagrams}{group-inverse-lemma.pdf_tex}
		% 	\caption{Turning a homotopy sphere inside out through a puncture.}\label{fig:group-inverse-lemma}
		% \end{figure}

		\noindent
		Taking the union of the intermediate stages of this inversion procedure gives a contractible manifold $W$ which bounds $M\+\overline{M}$.
	\end{proof}

	\noindent
	These lemmas complete the proof.
\end{proof}

% \subsection{The Subgroup \texorpdfstring{$\bP^{n+1}$}{bP^{n+1}}}\label{sec:framed-manifolds}

The full group $\Theta^n$ is often too complicated for direct classification. There is a subgroup, denoted $\bP^{n+1}$, of ``simpler'' homotopy spheres which are generally much easier to work with. In fact, all direct constructions of exotic spheres in this thesis are of these simpler homotopy spheres. Homotopy spheres in the complement $\Theta^n\setminus \bP^{n+1}$ of this subgroup are referred to as \defn{very exotic spheres}, and can only be detected and constructed via homotopy theoretic means. The relationship between $\bP^{n+1}$ and $\Theta^n$ will be discussed in \cref{chap:conclusion}.

\begin{definition}
	Let $\bP^{n+1}\subset \Theta^n$ be the set of homotopy spheres which have a parallelizable coboundary.
\end{definition}

Clearly, the identity $S^n$ is in this subset, and the subset is clearly closed under inversion. It is a bit tricker to see why it is closed under the group operation of connected sum. 

\begin{proposition}
	$\bP^{n+1}$ is a subgroup of $\Theta^n$.
\end{proposition}
\begin{proof}
	It suffices to show that $\bP^{n+1}$ is closed under connected sum. Suppose $M_1$ and $M_2$ are homotopy spheres bound by parallelizable manifolds $W_1$ and $W_2$. Note that the connected sum of $M_1$ and $M_2$ has a coboundary $W=(W_1,\partial W_1)\+(W_2,\partial W_2)$.

	\begin{figure}[ht]
		\centering
		\import{diagrams}{join-vector-bundle.pdf_tex}
		\caption{A join of vector bundles.}
	\end{figure}
	
	Now let us choose trivializations $\varphi_i : W_i\times \R^{n+1} \to \T W_i$ for the original coboundaries. If $p$ is the point along which $M_1$ and $M_2$ are glued, we have a fiber embeddings $\R^{n+1}\to \T_p W_i$. Then, we choose a tubular neighborhood of the fiber which agrees with the tangent bundle structure so that the join $\T W_1\cup_{\R^{n+1}}\T W_2$ is a fiber bundle over $W$. We also get a trivialization $\varphi : W\times \R^{n+1} \to \T W$ which restricts to $\varphi_i$.
\end{proof}

\smallrule

