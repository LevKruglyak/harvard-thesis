\chapter{Tools of Geometric Topology}\label{chap:fundamentals}

% \begin{epigraph}{21em}{Pythagoras}
% 	There is geometry in the humming of the strings,\\
% 	and there is music in the spacing of the spheres
% \end{epigraph}
%
%
% \begin{epigraph}{15em}{Frank Herbert}
% 	Beginnings are such delicate times.
% \end{epigraph}

\begin{epigraph}{20em}{Sir Michael Atiyah}
	Algebra is the offer made by the devil to the \\
	mathematician. The devil says: ``I will give \\
	you this powerful machine, it will answer any \\
	question you like. All you need to do is give\\
	me your soul: give up geometry.
\end{epigraph}

\noindent
To begin our journey, we review some fundamental tools of geometric topology.

We start in \cref{sec:smooth-manifold-operations} with a discussion of operations on manifolds and the technicalities which arise in the smooth case. Aside from standard operations on general topological spaces, we introduce the connected sum and boundary connected sum for manifolds with boundary. Both of these perspectives can be viewed as special cases of a more general operation of a join of two manifolds along a submanifold. Our next topic in \cref{sec:intersection-form} is the intersection form, a fundamental invariant of a manifold which encodes the data of middle-dimensional submanifold intersections. The intersection form relates this geometric data to the algebraic data of cohomology, and forms a wide bridge by which geometric problems can be transferred to algebraic ones. Related to this is the topic of the Euler class in \cref{sec:euler-class}, our first example of a characteristic class. For us, the Euler class provides a number of bridges linking intersection theory with the geometry of vector bundles, obstruction theory, characteristic classes, and others.

Finally, in section \cref{sec:cobordism} we discuss cobordism, a loose equivalence for manifolds which allows for complete classification. Cobordism serves two essential purposes in this thesis. First of all, we note that many topological invariants are invariant under cobordism. A complete determination of manifolds up to cobordism can thus help us relate two topological invariants, since we need only check the representative classes. As for the second application, there is a strengthening of the condition of cobordism known as $h$-cobordism. 
In dimensions $\geq 5$, a non-trivial result known as the $h$-cobordism theorem then turns the problem the existence problem for diffeomorphisms between two manifolds into an existence problem for a manifold with a certain homotopy type.
Such a reduction massively simplifies the problem of finding exotic spheres, and we conclude with these simplifications in \cref{sec:groups-of-homotopy-spheres}.

\subimport{/}{operations}
\subimport{/}{intersection}
\subimport{/}{euler}
\subimport{/}{cobordism}
\subimport{/}{groups}
