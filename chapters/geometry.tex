\chapter{Vector Bundles and Characteristic Classes}\label{chap:vector-bundles}

\begin{epigraph}{24em}{Albert Einstein {\normalfont to} Tullio Levi-Civita}
	I admire the elegance of your method of computation;\\
	it must be nice to ride through these fields upon the\\
	horse of true mathematics while the like of us have to\\
	make our way laboriously on foot.\\
\end{epigraph}

In this chapter, we will review some aspects of the construction and classification of vector bundles and the powerful computational methods of characteristic classes. As established in Conventions, all vector spaces will be over $\F=\R$ or $\C$, with the former assumed as default.

\section{Classifying Vector Bundles}

\subsection{Structure Groups}\label{sec:structure-groups}

Given a rank $k$ vector bundle $\mathcal{E} : E \to B$ over $\F$, there is an associated \defn{frame bundle} $\pi : \B E \to B$ with fibers $\B_p E = \{ b : \F^k \to E_p \mid b\textrm{ is an isomorphism}\}$ consisting of bases of the vector bundle $E_p$ at each point $p\in B$. Note that frame bundle is a principal $\GL_k\F$-bundle, since there is a right change of basis action given by precomposition on $\F^k$.
In the reverse direction, the action of $\GL_k\F$ on $\F^k$, the associated bundle $\B E\times_{\GL_k\F} \F^k$ is isomorphic to the original vector bundle $E$. In other words, there is a bijective correspondence
\begin{equation}\label{eq:principal-GL-bundle-vector-bundle}
	\left\{\parbox{7.5em}{$\F$-vector bundles of rank $k$ over $X$}\right\}
	\quad\lkxleftrightto \quad
	\left\{\parbox{7em}{principal $\GL_k\F$ bundles over $X$}\right\}
\end{equation}
Whenever we have a Lie group $G$ and homomorphism $\rho : G \to \GL_k\F$, there is a group action of $G$ on $\F^k$ by letting $\rho(g)$ act on $\F^k$ for each $g\in G$. As in the case of $\GL_k \F$, the associated bundle $\B E\times_G \F^k$ is a vector bundle. An isomorphism
\begin{equation}\label{eq:structure-on-a-vector-bundle}
	\lkxfunc{\varphi}{E}{\B E\times_G \F^k}
\end{equation}
is said to be a \defn{$(G,\rho)$-structure}[structure on a vector bundle] on $E$. When $\rho$ is injective, a $(G,\rho)$-structure is said to be a \defn{reduction}[reduction of the structure group] of the structure group to $G$.
Note that every vector bundle comes with a canonical $(\GL_k \F, \id)$-structure.
For real vector bundles, some common structure groups are:
\begin{itemize}
	\item A reduction to $\GL^+_k\R$ corresponds to an orientation of the bundle.
	\item A reduction to $\O_k$ corresponds to a Riemannian structure, i.e. an inner product on the bundle. Every real vector bundle admits such a structure, although not canonically.
	\item A further reduction to $\SO_k$ corresponds to an orientation and a Riemannian structure.
	\item A reduction to $\GL_{k}\C\subset \GL_{2k}\R$ corresponds to a complex structure on a rank $2k$ real vector bundle, i.e. an endomorphism $J$ with $J^2=-\id$. This is equivalent to the data of a complex vector bundle.
	\item A further reduction to $\U_k\subset \GL_k\C$ corresponds to a Hermitian structure on the complex vector bundle, i.e. a Hermitian inner product on the bundle.
\end{itemize}

\begin{remark}
	For those who prefer working with charts, there is a definition of reducing the structure group in these terms.

	Given an $\F$-vector bundle $\pi : E \to B$, and local trivializations $\varphi_\alpha : E|_{U_\alpha} \to U_{\alpha}\times \F^k$ for some open cover $\{U_\alpha\}$, we have a family of transition functions
	\[
		\lkxfunc{g_{\alpha\beta}}{U_\alpha\cap U_\beta}{\GL_k \F}{x}{\varphi_\alpha\circ \varphi_{\beta}^{-1}|_{\{x\}\times \R^k}}
	\]
	For some subgroup $G\subset \GL_k\F$, if we can choose local trivializations such that the resulting transition functions $g_{\alpha\beta}$ take values entirely in $G$, we say that the structure group of $E$ can be reduced to $G$.
\end{remark}

\begin{remark}
	\todo{Ehrlagen program}
\end{remark}

\subsection{Clutching Construction}\label{sec:clutching-construction}

When the base manifold of a vector bundle is a sphere, there is a useful construction known as the clutching construction which allows for a completely homotopy theoretic classification of vector bundles.

\begin{remark} The clutching construction is a very special case of a far more general notion of a classifying space, discussed in \cref{sec:classifying-spaces}.
\end{remark}

Suppose $\mathcal{E} : E \to S^m$ is a rank $k$ $\F$-vector bundle. We can decompose the sphere $S^m$ into hemispheral disks $S^m=D_+^m\cup D_-^m$, and these disks intersect at an equatorial sphere $D_+^m\cap D_-^m=S^{m-1}\subset S^m$ one dimension lower. The bundle $\mathcal{E}$ can then be trivialized on the hemispheres since they are contractible. Let us denote these trivializations
\[
	\lkxfunc{\varphi_+}{E|_{D^m_+}}{D^m_+\times \F^k}
	\quad\textrm{and}\quad
	\lkxfunc{\varphi_-}{E|_{D^m_-}}{D_-^m \times \F^k.}\]
Expanding their equatorial intersection by a tubular neighborhood, we get a transition function $\psi$, which fits into the following commutative diagram
$\psi$ in the commutative diagram
\[\begin{tikzcd}
		{S^{m-1}\times \F^k} && {S^{m-1}\times \F^k} \\
		& {E|_{S^{m-1}}}
		\arrow["\psi", from=1-1, to=1-3]
		\arrow["{\varphi_+|_{S^{m-1}}}", from=2-2, to=1-1]
		\arrow["{\varphi_-|_{S^{m-1}}}"', from=2-2, to=1-3]
	\end{tikzcd}\]
By definition, $\psi$ is constant on the first factor, and linear in the second factor. For each point $p\in S^{m-1}$, the diffeomorphism $\psi$ gives a linear function $\tau_p : \F^k \to \F^k$. These linear maps are the ``change of coordinate'' transformations between the fibers on the boundaries of $D_+^m$ and $D_-^m$ as depicted in \cref{fig:clutching-construction}.
\begin{figure}[ht]
	\centering
	\import{diagrams}{clutching-construction.pdf_tex}
	\caption{Getting a map $\tau : S^{m-1}\to \GL_k\F$ from a vector bundle over $S^{m}$.}\label{fig:clutching-construction}
\end{figure}

Altogether, this family of linear transformations is indexed by the equator $S^{m-1}$, and this gives us a smooth map $\tau : S^{m-1}\to \GL_k \F$. It follows that the homotopy type of $\tau$ is only dependent on the isomorphism type of the bundle $\mathcal{E}$ since any vector bundle isomorphism induces a homotopy by \cref{prop:homotopy-invariance-vector-bundle}. In other words, we have a map
\[
	\lkxfunc{}{\Vect^k_\F(S^m)}{{}[S^{m-1}, \GL_k\F]}
\]
sending an $\F$-vector bundle $\mathcal{E}$ to the its associated homotopy class $\tau\in [S^{m-1}, \GL_k\F]$. This associated map $\tau$ is called the \defn{clutching function} of the bundle $\mathcal{E}$.

\begin{remark}
	If the bundle $\mathcal{E}$ has additional structure, we can choose the clutching function to lie in some Lie subgroup $G\subset \GL_k \F$. For instance, in the real case $\F=\R$, an oriented bundle has a clutching function $\tau \in [S^{m-1},\GL_k^+\R]$ in the homotopy group of orientation preserving linear transformations. Similarly, if a bundle has an inner product structure, its clutching function can be chosen to lie in $\pi_{m-1}(\O_k)$. This is elaborated upon in \cref{sec:structure-groups}.
\end{remark}

The construction works in the opposite direction as well.

\begin{definition}
	Given a matrix Lie group $G\subset \GL_k\F$ and a clutching function $\tau\in \pi_{m-1}(G)$, the bundle associated to the function is the bundle $\mathcal{E}_\tau : E_\tau \to S^{m}$, with total space
	\[
		E_\tau = (D_+^m\times \F^k)\cup_h (D_-^{m}\times \F^k),
	\]
	where $h(x,y)=(x,\tau(x)y)$ is the glueing map.
\end{definition}

Rather than work with the homotopy groups of spaces like $\GL_k\R$, it is common to deformation retract them to compact subspaces.

\begin{proposition}
	There are deformation retracts
	\[
		\GL_k\R \cong \O_k,\quad \GL_k^+\R \cong \SO_k\quad\textrm{and}\quad \GL_k\C \cong \U_k.
	\]
\end{proposition}
\begin{proof}
	The retract is given by Gram-Schmidt orthonormalization.
\end{proof}

When the Lie group $G$ is path-connected, as in the case of $\SO_k$ and $\U_k$, we can identify the set of free homotopy classes of maps $[S^{m-1}, G]$ with the homotopy group $\pi_{m-1}(G)$.

\begin{theorem}
	The clutching construction gives bijections
	\[
		\Vect_{\R,+}^k(S^m) \cong \pi_{m-1}(\SO_k)\quad\textrm{and}\quad
		\Vect_{\C}^k(S^m) \cong \pi_{m-1}(\U_k).
	\]
\end{theorem}
\begin{proof}
	This is Proposition 1.11 and Proposition 1.14 in \cite{hatcher2003ktheory}.
\end{proof}

\begin{remark}
	The case of unoriented bundles, we do not have a bijection with the homotopy group, since $\O_m$ is generally disconnected.
\end{remark}

Let us see some examples of this in low dimensions.

\begin{example}
	Since $\SO_1 = \{e\}$, there are no non-trivial orientable real line bundles over spheres of any dimensions.
\end{example}

\begin{example}
	Since $\U_1\cong S^1$, we have
	\[
		\Vect_{\C}^1(S^m) \cong \pi_{m-1}(S^1) \cong \begin{cases} \Z & m=2,\\ 0 & \textrm{otherwise},\end{cases}
		\quad\textrm{for }m >1.
	\]
	In other words, the only sphere for which there are non-trivial complex line bundles is the $2$-dimensional sphere. The association of a complex line bundle over $S^2$ with an integer is a special case of a characteristic class. This will be explored in greater depth in \cref{sec:axiomatic-characteristic-classes}.

	It is natural to wonder what number corresponds to, say the tangent bundle, of a sphere. Recall that the $2$-dimensional sphere admits a complex structure as the complex projective plane $\CP^1$ or Riemann sphere $S^2 = \C\cup\{\infty\}$. We can cover $\CP^1$ by the two coordinate charts
	\[
		U_0 = \{[z: 1] \mid z\in \C\}
		\quad\textrm{and}\quad
		U_\infty = \{[1: w] \mid w\in \C\}.
	\]
	The coordinate change function on the overlap is then $w=1/z$, and so the transition function on the tangent bundle is the differential $dw=-1/z^2 \,dz$. When restricted to the equator $z=e^{i\theta}$, it is clear that the transition function has degree $-2\in \Z$ since $e^{i\theta}\mapsto -e^{-2i\theta}$. Note that $-2$ is, up to a sign, the Euler number of the sphere.
\end{example}


\subsection{Classifying Spaces}\label{sec:classifying-spaces}

\begin{proposition}\label{prop:homotopy-invariance-vector-bundle}
	There is a natural
\end{proposition}

\begin{theorem}\label{thm:classifying-space}
\end{theorem}
Generally, if $G$ is a Lie group there is a natural isomorphism of contravariant functors
\[
	[-, \BB G] \lkxto \Bun_G(-)
\]
where $[-, \BB G]$ is the set of homotopy classes of maps to a space $\BB G$ and $\Bun_G(-)$ is the set of isomorphism classes of principal $G$-bundles over a given space.
The space $\BB G$ is known as the \defn{classifying space}\footnote{The classifying space is rarely a manifold, and is usually an infinite dimensional CW complex.} of $G$ and this space comes equipped with a \defn{universal bundle} $\zeta : \EE G \to \BB G$. With this universal bundle, the natural isomorphism is easy to describe. Under the appropriate topological restrictions, a map $\tau : X \to \BB G$ gives us a pullback bundle $\tau^*\zeta$ over $X$. This is a principal $G$-bundle over $X$ which is entirely determined by the homotopy type of the \defn{classifying map} $\tau$.

In some sense, the universal bundle $\zeta$ is the ``most twisted $G$-bundle''. Pullbacks of bundles generally ``dilute'' the twistedness of a bundle -- for instance, the splitting principle allows any complex vector bundle to be pulled back to a direct sum of complex line bundles. It would stand to reason that every bundle is the pullback of a more twisted bundle, and the limit of this process is the universal bundle $\zeta$ over the classifying space. See Chapter IV of \cite{botttu1982differential} for a wonderful exposition on the topic.


\begin{remark}\label{rmk:clutching-construction-generalization}
	The clutching construction can be viewed as a special case of \cref{thm:classifying-space}.
	\todo{It can be shown that (under suitable topological restrictions)} there is a homotopy equivalence $\Omega \BB G \simeq G$ where $\Omega$ denotes the loop space operator in homotopy theory. Indeed from a homotopy theory perspective, the classifying space is a ``delooping'' of the group $G$. For spheres, the loop space suspension adjunction gives us natural isomorphisms
	\[
		\begin{aligned}
			\Bun_{G}(S^m) & \cong [S^m, \BB G]            \\
			              & \cong [\Sigma S^{m-1}, \BB G] \\
			              & \cong [S^{m-1}, \Omega\BB G]  \\
			              & \cong [S^{m-1}, G]
			\cong \pi_{m-1}(G).
		\end{aligned}
	\]
	This is a generalized way of understanding the clutching construction.
\end{remark}

\pagebreak
\section{Characteristic Classes}
The study of characteristic classes began with the work of Hassler Whitney and Eduard Stiefel in the mid 1930s. Since then, the fundamental idea has remained unchanged -- a vector bundle on a manifold determines certain ``characteristic'' classes in the homology or cohomology of the base manifold.
By the mid 1940s, these ideas were extended by Lev Pontryagin and Shing-Shen Chern to better capture the geometric data of oriented real and complex vector bundles respectively. In the following decades, characteristic classes quickly joined the toolboxes of mathematicians from a wide range of disciplines, finding connections to prior notions in these fields.
Applications ranged from algebraic topology, differential topology of exotic spheres, complex geometry, index theory, and may others.

We will present a few equivalent formulations of characteristic classes in this thesis, each useful in its own context. These formulations are,
\begin{itemize}
	\item as natural transformations satisfying certain axioms in \cref{sec:axiomatic-characteristic-classes},
	\item as generators of the cohomology ring of a classifying space in \cref{sec:universal-characteristic-classes},
	\item by the Chern-Weil homomorphism as images of invariant polynomials in \cref{sec:chern-weil-theory},
	\item as obstructions to problems in homotopy theory in \cref{sec:obstruction-theory}.
\end{itemize}
At this stage, we assume a basic knowledge of vector bundles, structure groups, and classifying spaces. For a brief introduction to these topics, see \cref{chap:vector-bundles}.

\subsection{Axiomatic Perspective on Characteristic Classes}\label{sec:axiomatic-characteristic-classes}

Throughout this section, a cohomology theory will refer either to singular cohomology with some PID coefficient ring such as $\Z$, $\Q$, $\Z/2$, or $\Z[1/2]$, or to de Rham cohomology. The Poincar\'e dual homology theories are then either singular homology with coefficients or compactly supported de Rham cohomology. We use $R$ to denote the coefficient ring.

\begin{definition}\label{defn:characteristic-class}
	A \defn{characteristic class} $c$, valued in a cohomology theory $h$, is a natural transformation of contravariant functors
	\[
		\lkxfunc{c}{\Vect_G}{h^\bullet,}
	\]
	given a structure group $G$. We denote the set of characteristic classes as $\Class_G^h = [\Vect_G , h^\bullet]$.
\end{definition}

Here, $h^\bullet : \Top \to \Rng$ sends a space to its  cohomology ring, and $\Vect_G : \Top \to \Set$ sends a space to the set of isomorphism classes of vector bundles over the space with structure group $G$. We assume that the natural transformation $c$ forgets the ring structure of cohomology when mapping from the set of isomorphism classes of vector bundles.

For any vector bundle $\mathcal{E} : E \to B$, a cohomology class assigns some cohomology class $c(\mathcal{E})\in h^\bullet(B)$ which is ``characteristic'' of the bundle $\mathcal{E}$.
This assignment must be done in a natural way. Given bundles $\mathcal{E}_1$ and $\mathcal{E}_2$ over bases $B_1$ and $B_2$, whenever a map $f : B_1 \to B_2$ is covered by a bundle map $\mathcal{E}_1 \to \mathcal{E}_2$, we have $f^* c(\mathcal{E}_2) = c(\mathcal{E}_1)$. In particular, isomorphic vector bundles over the same base space $B$ are sent to the same cohomology class.

\begin{convention*}
	Since every smooth manifold $M$ comes with a canonical vector bundle -- the tangent bundle -- it is common to use the notation $c(M)$ to refer to $c(\TT M)$.
\end{convention*}

\begin{remark}
	In practice, many sources define characteristic classes first as a set of homogeneous classes $c_i\in h^i(\mathcal{E})$ and then refer to the \defn{total characteristic class} $c(\mathcal{E})=\sum_i c_i(\mathcal{E})$. In this thesis, we take a somewhat unorthodox convention and assume that all characteristic classes are inhomogeneous unless otherwise stated.
\end{remark}

For any two characteristic classes $c_1$ and $c_2$ there is a natural notion of their sum $c_1+c_2$, product $c_1\smile c_2$, and scalar product $r\cdot c_1$ for any $r\in R$, simply by applying these operations in $h^\bullet(-)$. This gives the set $\Class_G^h$ of characteristic classes a ring structure.

For any base $B$, the cohomology ring $h^\bullet(B)$ is a graded ring
\[
	h^\bullet(M) = \bigoplus_{k\geq 0} h^k(B),
\]
with the cup product turning $h^\bullet(B)$ into a graded-commutative ring.
This gives the set of characteristic classes $\Class_G^h$ a grading
\[
	\Class_G^h[k] = [\Vect_G, h^k]
\]
where $h^k: \Top \to \Grp$ denotes the functor sending a space to its $k$-th cohomology group.

\begin{definition}
	A characteristic class $c$ is said to be \defn{homogeneous of degree $k$}[homogeneous characteristic class] if it lies in the graded component $[\Vect_G, h^k]$. We denote this degree by $|c|$.
\end{definition}

By naturality, characteristic classes must preserve cohomological degree since pull-backs preserve degree. We can thus write any characteristic class $c$ as an infinite sum $c=c_0+c_1+c_2+\cdots$ where each $c_k$ is a natural transformation from $\Vect_G$ to $h^k$.
In other words, the ring of characteristic classes $\Class_G^h$ admits the structure of a completion of a graded-commutative algebra (we require infinite sequences):
\[
	\Class_G^h \cong \prod_{k \geq 0} [\Vect_G, h^k].
\]
\begin{remark}
	Recall that the \defn{completion}[completion of a graded algebra] of a graded algebra $A=\bigoplus_{i\in I} A_k$ is the direct product $\overline{A}=\prod_{i\in I} A_i$. In particular, infinite sequences are allowed. If a ring is the completion of a graded algebra, we refer to it as a \defn{completed algebra}.
\end{remark}

Next, let us look at some common axioms for characteristic classes and investigate their implications. This simplifies many of the following constructions of the various species of characteristic class.

\begin{definition}
	A characteristic class $c$ is said to be \defn{multiplicative}[multiplicative characteristic class] if \begin{equation}\label{eq:whitney-product}
		c(\mathcal{E}_1\oplus \mathcal{E}_2)=c(\mathcal{E}_1)\smile c(\mathcal{E}_2)
	\end{equation}
	whenever the bundles $\mathcal{E}_1$ and $\mathcal{E}_2$ have the same base space.
\end{definition}

Equations of the form \cref{eq:whitney-product} are known as a \defn{Whitney product formula}. The homogeneous version of this equation which commonly appears is
\[
	c_n(\mathcal{E}_1\oplus \mathcal{E}_2) = \sum_{p+q=n}c_p(\mathcal{E}_1)\smile c_q(\mathcal{E}_2).
\]

\begin{remark}
	A multiplicative cohomology class is also multiplicative with respect to manifolds. Given manifolds $M_1$ and $M_2$ and a multiplicative characteristic class $c$, we have
	\[
		\begin{aligned}
			c(M_1\times M_2) & = c(\T(M_1\times M_2))      \\
			                 & = c(\T M_1\oplus \T M_2)    \\
			                 & = c(\T M_1)\smile c(\T M_2) \\
			                 & = c(M_1)\smile c(M_2).
		\end{aligned}
	\]
\end{remark}

\begin{remark}
	When the base space $B$ has finite type (as in the case of manifolds), the \defn{degree}[degree of a cohomology class] of a characteristic class $c(\mathcal{E})$ evaluated at a vector bundle $\mathcal{E} : E \to B$ is the maximal degree of $c(\mathcal{E})\in h^\bullet(B)$, and denoted $|c(\mathcal{E})|$.
\end{remark}

\begin{definition}
	A characteristic class $c$ is said to be \defn{rank-normalized} if $c_0=1$ and for any bundle $\mathcal{E}$ we have $|c(\mathcal{E})|\leq \rank(\mathcal{E})$. Note that the degree $|c(\mathcal{E})|$ is the maximal degree of a homogeneous component of $c(\mathcal{E})$.
\end{definition}

\begin{corollary}
	For any rank-normalized characteristic class $c$ and trivial bundle $\underline{\R}^k$, we have $c(\underline{\R}^k)=1$.
\end{corollary}
\begin{proof}
	A trivial bundle $\underline{\R}^k$ is the pullback of the constant bundle $\R^k\to *$ over a point. This bundle has characteristic class $1$ by the rank normalization condition, and so by naturality pulls back to $1$.
\end{proof}

\begin{corollary}\label{cor:mobius-characteristic-2-torsion}
	If $c$ is a rank-normalized multiplicative characteristic class and $\gamma_1^1$ is the canonical M\"obius bundle over $\RP^1\cong S^1$, then $c(\gamma_1^1)$ is $2$-torsion.
\end{corollary}
\begin{proof}
	Since $\gamma_1^1\oplus \gamma^1_1$ is a trivial bundle (see \cref{fig:trivial-mobius-bundle-sum}), it follows that
	\[
		(1+c_1(\gamma_1^1))(1+c_1(\gamma_1^1)) = 1+2c_1(\gamma_1^1)+c_1^2(\gamma_1^1)=1.
	\]
	However, $H^2(S^1)\cong 0$, so we have $2c_1(\gamma_1^1)=0$.
\end{proof}

\begin{figure}[ht]
	\centering
	\import{diagrams}{placeholder-small.pdf_tex}
	\caption{A Whitney sum of orthogonal M\"obius bundles.}\label{fig:trivial-mobius-bundle-sum}
\end{figure}

\begin{corollary}
	For any multiplicative and rank-normalized characteristic class, we have $c(\mathcal{E}\oplus \underline{\R}^k)=c(\mathcal{E})\smile c(\underline{\R}^k)= c(\mathcal{E})$.
\end{corollary}

For such a characteristic class, adding on trivial bundles does not affect the class. This equivalence relation on vector bundles is known as stable isomorphism.

\begin{definition}\label{def:stable-isomorphism}
	Two vector bundles $\mathcal{E}_1$ and $\mathcal{E}_2$ are said to be \defn{stably isomorphic}[stable isomorphism of vector bundles] if there is an isomorphism of vector bundles $\mathcal{E}_1\oplus \underline{\R}^{k_1} \cong \mathcal{E}_2\oplus \underline{\R}^{k_2}$.
\end{definition}

\begin{definition}
	A characteristic class $c$ is said to be \defn{stable}[stable characteristic class] if $c(\mathcal{E}\oplus \underline{\R}^k) = c(\mathcal{E})$ for any $k$. A stable characteristic class only depends on the stable isomorphism type of the vector bundle. Every multiplicative and rank-normalized characteristic class is stable.
\end{definition}

In a general completed algebra $\overline{A}=\prod_{k\geq 0}A_k$, the multiplication of two elements
\[
	(x_0+x_1+x_2+\cdots)\cdot (y_0+y_1+y_2+\cdots) = (z_0 + z_1 + z_2+\cdots)
\]
can be expanded in homogeneous components as $z_k=\sum_{p+q=k} x_py_q$, i.e.
\begin{equation}\label{eq:formal-product}
	\begin{aligned}
		z_0 & = x_0y_0,                 \\
		z_1 & = x_0y_1 + x_1y_0,        \\
		z_2 & = x_0y_2 + x_1y_1+x_2y_0, \\
		    & \;\;\vdots
	\end{aligned}
\end{equation}
To invert the series $(x_0+ x_1+\ldots)\in \overline{A}$, we would like to find some $(y_0+y_1+\cdots)$ such that the resulting product $(z_0+z_1+\cdots)=1$. Setting $z_0=1$, $z_k=1$ for $k>0$ and solving \cref{eq:formal-product} for $y_k$, we get the recursive formula
\begin{equation}\label{eq:formal-inversion}
	y_k = \begin{cases}x_0^{-1}                                        & k=0,   \\
             -x_0^{-1}(x_1y_{k-1}+x_2y_{k-2}+\cdots +x_ky_0) & k > 0.
	\end{cases}
\end{equation}
A crucial corollary of \cref{eq:formal-inversion} is that a series $(x_0+x_1+\cdots)$ can be inverted if and only if $x_0\in A_0^\times$ is a unit. This is generally not true in the non-completed case. Take for instance in the ring of formal power series $R\fps{t}$, the completion of the polynomial ring $R[t]$.a There, we have the identity
\[
	(1-t)\cdot (1+t+t^2+t^3+\cdots) = 1,
\]
which is a rearranged form of the classic equation for the infinite series of a geometric series. Even though $(1-t)$ has monic leading coefficient, it has no inverse in $R[t]$.

\begin{proposition}\label{prop:formal-inverse}
	If $A$ is a complete graded algebra, then $A^\times = A \cap A_0^\times$.
\end{proposition}

This observation has a useful consequence in the context of characteristic classes. Suppose $c$ is a characteristic class with $c_0\in R^\times$. By \cref{prop:formal-inverse}, there is a characteristic class $c^{-1}$ which is the multiplicative inverse of $c$, so that $c\smile c^{-1}=1$ is the trivial characteristic class. If the characteristic class $c$ is multiplicative, then the Whitney product formula gives
\[
	c(\mathcal{E}_1\oplus\mathcal{E}_2)	 = c(\mathcal{E}_1)\smile c(\mathcal{E}_2)
	\quad\implies\quad
	c(\mathcal{E}_1) = c(\mathcal{E}_1\oplus \mathcal{E}_2) \smile c^{-1}(\mathcal{E}_2).
\]
If $\mathcal{E}_1=\TT M$ is the tangent bundle of a manifold $M^k$ and $\mathcal{E}_2=\TT\R^{n}/M$ is the normal bundle of an embedding $M^k\to \R^{n}$, then $\mathcal{E}_1\oplus \mathcal{E}_2$ is trivial since it is the restriction of the trivial tangent bundle of $\R^n$. In the case that $c$ is rank-normalized, we get:
\begin{theorem}[Whitney Duality]\label{thm:whitney-duality}
	If $M^k$ is a submanifold of $\R^n$, and $c$ is a rank-normalized multiplicative characteristic class, we have
	\[
		c(\TT M) = c^{-1}(\TT \R^n/M).
	\]
\end{theorem}

\begin{corollary}
	If $c$ is a rank-normalized multiplicative characteristic class, then $c(S^n)=1$.
\end{corollary}
\begin{proof}
	This follows since $S^n$ embeds into $\R^{n+1}$ with a trivial normal line bundle.
\end{proof}

\begin{remark}
	As it turns out, \emph{any} homotopy sphere $M$ has $c(M)=1$ for a rank-normalized multiplicative characteristic class because the tangent bundle of a homotopy sphere is stably isomorphic to the trivial bundle. We will prove this in \cref{thm:homotopy-sphere-stably-parallelizable} using some hard theorems in homotopy theory.
\end{remark}

\subsection{Creating New Characteristic Classes out of Old}

Suppose $c$ is a characteristic class. Let $R\fps{x}$ be shorthand for the completion of the graded-commutative algebra $R[x_0,x_1,\ldots]$. Note that
\[
		R\fps{x} = \prod_{k\geq 0} R[x_0,\ldots, x_k]
\]

There is a map
\[
	\lkxfunc{}{R\fps{x}}{\Class_G^h}
\]
which sends $x_i\mapsto c_i$. Let $U_1(R\fps{x})$ denote the subset of $\R\fps{x}$ consisting of infinite sequences with 

\begin{definition}
\end{definition}

\subsection{Characteristic Numbers}

If we want to compare vector bundles on different spaces, we need a common context in which to compare their characteristic classes since the cohomology rings of the underlying spaces might not be canonically isomorphism. When the base space is a compact $R$-oriented manifold, the Poincar\'e duality isomorphism $h^{n-k}(M) \cong h_k(M)$ allows us to ``integrate'' homogeneous top-dimensional cohomology classes $\alpha\in h^{n}(M)$ along a fundamental class $[M]\in h_n(M)$ to get an element $\alpha[M]\in h_0(M)\cong R$ in the coefficient ring of a corresponding homology theory. The coefficient ring $R$ is this context in which we can compare characteristic classes over different bases.

\begin{definition}\label{def:characteristic-number}
	Given any characteristic class $c$ and closed $n$-dimensional manifold $M$ the \defn{characteristic number} of $M$ is $c[M] = c_n(\TT M)[M]$.
\end{definition}

Note that we must take the homogeneous component of $c$ to apply Poincar\'e duality. Generally, if $\Man_R$ denotes the set of closed $R$-oriented manifolds, any characteristic class $c$ can be interpreted as a map
\begin{equation}\label{eq:characteristic-number}
	\lkxfunc{c}{\Man_R}{R}{M}{c[M].}
\end{equation}
Under some extensions of \cref{def:characteristic-number} to allow for disconnected manifolds, \cref{eq:characteristic-number} is additive with respect to disjoint unions.
\begin{proposition}
	If $c$ is a multiplicative characteristic class, then $c[M_1\times M_2]=c[M_1]\cdot c[M_2]$.
\end{proposition}
\begin{proof}
	By the multiplicative property and the K\"unneth formula, we have.
	\[
		\begin{aligned}
			c[M_1\times M_2]
			 & = c(\TT (M_1\times M_2))[M_1\times M_2]                      \\
			 & = c(\TT M_1\oplus \TT M_2)[M_1\times M_2]                    \\
			 & = (c(\TT M_1)\smile c(\TT M_2))[M_1]\times [M_2]             \\
			 & = c(\TT M_1)[M_1]\cdot c(\TT M_2)[M_2] = c[M_1]\cdot c[M_2].
		\end{aligned}
	\]
	This completes the proof.
\end{proof}

\begin{remark} 
	While $\Man_R$ doesn't have a ring structure due to lack of an identity element, with the inclusion of a cobordism relation on $\Man_R$ it is possible to interpret multiplicative characteristic classes as ring homomorphisms from a cobordism ring to $R$.
\end{remark}

The set of characteristic numbers of a manifold form a topological fingerprint of the manifold. As we will see in \cref{chap:invariants}, the subtle interplay of their number theoretic properties is one of the main ways to study smooth structure on a manifold.

\subsection{Examples of Characteristic Classes}

We will now begin with some examples of characteristic classes.

We will now give the axioms for some common characteristic classes. We take the convention that $\alpha$ is the generator for the cohomology ring of projective space, either in $\Z/2$ coefficients for $\RP^n$ or in $\Z$ coefficients for $\CP^n$. In other words, we let
\[
	\begin{aligned}
		\H^\bullet(\RP^n; \Z/2) & \cong \Z/2[\alpha]/(\alpha^{n+1})\quad & |\alpha|=1, \\
		\H^\bullet(\CP^n; \Z)   & \cong \Z[\alpha]/(\alpha^{n+1})\quad   & |\alpha|=2. \\
	\end{aligned}
\]

\begin{definition}
	The \defn{Stiefel-Whitney class $w$}[Stiefel-Whitney class] is the unique characteristic class for unoriented vector bundles ($G=
		\O$) in singular cohomology with $\Z/2$ coefficients satisfying:
	\begin{enumerate}[(a)]
		\item $w$ is rank-normalized and multiplicative,
		\item $w(\gamma^1_1)=1+\alpha$.
	\end{enumerate}
	We denote by $w_i$ the degree $i$ homogeneous component of $w$ in $\H^i(-;\Z/2)$.
\end{definition}

\begin{remark}
	Note that axiom (c) does not violate \cref{cor:mobius-characteristic-2-torsion}, since we are working in $\Z/2$ and so every cohomology class has $2$-torsion.
\end{remark}

\begin{definition}
	The \defn{Chern class $\cl$}[Chern class] is the unique characteristic class for complex vector bundles ($G=\U$) in singular cohomology with $\Z$ coefficients satisfying:
	\begin{enumerate}[(a)]
		\item $\cl$ is rank-normalized and multiplicative,\footnote{Technically, we only need that $\cl_0=1$, the rank conditions follows from this and axiom (b).}
		\item $\cl(\Gamma^1_n)=1-\alpha$.
	\end{enumerate}
	We denote by $\cl_i$ the degree $2i$ homogeneous component of $c$ in $\H^{2i}(-;\Z)$.
\end{definition}

Of course, this is not a constructive definition, and it is not at all clear that a characteristic class satisfying these definitions even exists. That being said, it is instructive to explore the immediate consequences of these axioms before considering explicit constructions.

\subsection{Pontryagin Classes}

\begin{definition}
	The \defn{Pontryagin class $p$} is the unique characteristic class for oriented real vector bundles ($G=\SO$) in singular cohomology with $\Z[1/2]$ coefficients satisfying:
	\begin{enumerate}[(a)]
		\item $p$ is rank-normalized and multiplicative,
		\item
	\end{enumerate}
\end{definition}

\begin{proposition}\label{prop:pontryagin-class-complex-projective-space}
	\[
		p(\CP^n) = (1+\alpha^2)^{2n+1}
	\]
\end{proposition}

\begin{example}
	The Pontryagin classes of the first few dimensions of complex projective space are given by:
	\[
		\begin{aligned}
			p(\CP^1) & = 1,                                    \\
			p(\CP^2) & = 1+3\alpha^2,                          \\
			p(\CP^3) & = 1+4\alpha^2,                          \\
			p(\CP^3) & = 1+5\alpha^2 + 10\alpha^4,             \\
			p(\CP^4) & = 1+6\alpha^2 + 15\alpha^4,             \\
			p(\CP^4) & = 1+7\alpha^2 + 21\alpha^4+ 35\alpha^6, \\
		\end{aligned}
	\]
	Note that $p_k[\CP^{2k}]$.
\end{example}

\begin{proposition}\label{prop:pontryagin-}
	Let $K(p)$ be a Pontryagin class. If $M^{4k}$ is an oriented manifold which is the boundary of a smooth compact oriented manifold $W^{4k+1}$, then $K(p)[M]=0$.
\end{proposition}

\begin{corollary}
	A degree $k$ Pontryagin number $K(p)$ is a well-defined group homomorphism
	\[
		\lkxfunc{}{\Omega^\SO_k}{\Z}{[M]}{K(p)[M]}
	\]
\end{corollary}

\begin{corollary}
	If $K(p)$ is a multiplicative Pontryagin number, then
\end{corollary}


\subsection{Wu Classes}\label{sec:wu-classes}

\begin{theorem}
	$w=\Sq(v)$.
\end{theorem}

A refinement
For complex vector bundles, i.e.

\subsection{The Euler Class}\label{sec:euler-class}

\begin{theorem}\label{thm:euler-number-clutching-construction}
	Let $p : \SO_{2m} \to \SO_{2m}/\SO_{2m-1}= S^{2m-1}$ be the projection map of the special orthogonal group to the sphere. Then the following diagram commutes,
	\[\begin{tikzcd}
			{\pi_{2m-1}(\SO_{2m})} & \Z \\
			{\pi_{2m-1}(S^{2m-1})}
			\arrow["e", from=1-1, to=1-2]
			\arrow["{p_*}"', from=1-1, to=2-1]
			\arrow[from=2-1, to=1-2]
		\end{tikzcd}\]
	where $\pi_{2m-1}(S^{2m-1})\to \Z$ is the degree isomorphism.
\end{theorem}
\begin{proof}
	\todo{prove}
\end{proof}

\begin{corollary}\label{cor:expressible-euler-numbers-spheres}
	The image of $e : \pi_{2m-1}(\SO_{2m})\to\Z$ is $2\Z$.
\end{corollary}
\begin{proof}
	\todo{prove}
\end{proof}


\begin{definition}\label{def:euler-class}
	The \defn{Euler class}
\end{definition}

\begin{proposition}
	On a closed $n$-manifold manifold $M$, $e[\T M]=\sum_k (-1)^k \rank \H^k(M)$.
\end{proposition}

\begin{corollary}
	The Euler number of a sphere is $e[\T S^n] = (1+(-1)^n)[S^n]$.
\end{corollary}
% \begin{definition}
% 	The \defn{Euler class} $e$ is a characteristic class for oriented real vector bundles in $\Z$ singular cohomology satisfying the following axioms:
% 	\begin{enumerate}[(a)]
% 		\item For bundles $\mathcal{E}_1$ and $\mathcal{E}_2$ over a common base, we have $e(\mathcal{E}_1\oplus \mathcal{E}_2)=e(\mathcal{E}_1)\oplus e(\mathcal{E}_2)$.
% 		\item If a bundle $\mathcal{E}$ has a non-zero section, then $e(\mathcal{E})=0$. 
% 		\item If $-\mathcal{E}$ has opposite orientation to $\mathcal{E}$, then $e(-\mathcal{E})=-e(\mathcal{E})$.
% 	\end{enumerate}
% 	Note that by 
% \end{definition}

\subsection{Multiplicative Sequences}



\begin{definition}
\end{definition}

\subsection{A Universal Perspective on Characteristic Classes}\label{sec:universal-characteristic-classes}

By the work of \cref{sec:classifying-spaces}, every vector bundle $\mathcal{E} : E \to B$ with structure group $G$ is the pullback of a bundle over the classifying space $\BB G$ by a classifying map $f_{\mathcal{E}} : B \to \BB G$.
This means that homogeneous characteristic class $c$ of degree $k$ determines a universal cohomology class $u\in h^k(\BB G)$ by applying $c$ to the universal bundle over $\BB G$. Conversely, any cohomology class in $h^k(\BB G)$ gives a homogeneous characteristic class mapping
\[
	\lkxfunc{c}{\Vect_G(B)}{h^k(B)}{\mathcal{E}}{f_{\mathcal{E}}^*u}
\]

\begin{proposition}
	There is a ring isomorphism
	\[
		\Class_G^h \cong \overline{h^\bullet(\BB G)}
	\]
	where $\overline{h^\bullet(\BB G)}=\prod_{k\geq 0} h^k(\BB G)$ denotes the completion of the cohomology ring.
\end{proposition}
\begin{proof}
	\todo{Yoneda lemma}
\end{proof}

The right hand side of the  is the universal approach to characteristic classes -- und

\begin{definition}
	The \defn{infinite-dimensional Grassmannian}, denoted $\Gr_k$ or $\Gr_k(\R^\infty)$ is defined as the direct limit
	\[
		\Gr_k(\R^\infty) = \varinjlim_n \Gr_k(\R^n)
	\]
\end{definition}

\begin{theorem}
	\[
		\H^\bullet(\Gr_k; \Z/2) \cong \Z/2[w_1(\gamma), \ldots, w_n(\gamma)]
	\]
\end{theorem}

\subsection{Chern-Weil Theory}\label{sec:chern-weil-theory}

While the axiomatic and universal definitions of characteristic classes are simple and abstract, it is often useful to

