\chapter{Vector Bundles and Characteristic Classes}\label{chap:vector-bundles}

\begin{epigraph}{24em}{Albert Einstein {\normalfont to} Tullio Levi-Civita}
	I admire the elegance of your method of computation;\\
	it must be nice to ride through these fields upon the\\
	horse of true mathematics while the like of us have to\\
	make our way laboriously on foot.\\
\end{epigraph}

In this chapter, we will review some aspects of the construction and classification of vector bundles and the powerful computational methods of characteristic classes. As established in Conventions, all vector spaces will be over $\F=\R$ or $\C$, with the former assumed as default.

\section{Classifying Vector Bundles}

\subsection{Structure Groups}\label{sec:structure-groups}

Given a rank $k$ vector bundle $\mathcal{E} : E \to B$ over $\F$, there is an associated \defn{frame bundle} $\pi : \B E \to B$ with fibers $\B_p E = \{ b : \F^k \to E_p \mid b\textrm{ is an isomorphism}\}$ consisting of bases of the vector bundle $E_p$ at each point $p\in B$. Note that frame bundle is a principal $\GL_k\F$-bundle, since there is a right change of basis action given by precomposition on $\F^k$.
In the reverse direction, the action of $\GL_k\F$ on $\F^k$, the associated bundle $\B E\times_{\GL_k\F} \F^k$ is isomorphic to the original vector bundle $E$. In other words, there is a bijective correspondence
\begin{equation}\label{eq:principal-GL-bundle-vector-bundle}
	\left\{\parbox{7.5em}{$\F$-vector bundles of rank $k$ over $X$}\right\}
	\quad\lkxleftrightto \quad
	\left\{\parbox{7em}{principal $\GL_k\F$ bundles over $X$}\right\}
\end{equation}
Whenever we have a Lie group $G$ and homomorphism $\rho : G \to \GL_k\F$, there is a group action of $G$ on $\F^k$ by letting $\rho(g)$ act on $\F^k$ for each $g\in G$. As in the case of $\GL_k \F$, the associated bundle $\B E\times_G \F^k$ is a vector bundle. An isomorphism
\begin{equation}\label{eq:structure-on-a-vector-bundle}
	\lkxfunc{\varphi}{E}{\B E\times_G \F^k}
\end{equation}
is said to be a \defn{$(G,\rho)$-structure}[structure on a vector bundle] on $E$. When $\rho$ is injective, a $(G,\rho)$-structure is said to be a \defn{reduction}[reduction of the structure group] of the structure group to $G$.
Note that every vector bundle comes with a canonical $(\GL_k \F, \id)$-structure.
For real vector bundles, some common structure groups are:
\begin{itemize}
	\item A reduction to $\GL^+_k\R$ corresponds to an orientation of the bundle.
	\item A reduction to $\O_k$ corresponds to a Riemannian structure, i.e. an inner product on the bundle. Every real vector bundle admits such a structure, although not canonically.
	\item A further reduction to $\SO_k$ corresponds to an orientation and a Riemannian structure.
	\item A reduction to $\GL_{k}\C\subset \GL_{2k}\R$ corresponds to a complex structure on a rank $2k$ real vector bundle, i.e. an endomorphism $J$ with $J^2=-\id$. This is equivalent to the data of a complex vector bundle.
	\item A further reduction to $\U_k\subset \GL_k\C$ corresponds to a Hermitian structure on the complex vector bundle, i.e. a Hermitian inner product on the bundle.
\end{itemize}

\begin{remark}
	For those who prefer working with charts, there is a definition of reducing the structure group in these terms.

	Given an $\F$-vector bundle $\pi : E \to B$, and local trivializations $\varphi_\alpha : E|_{U_\alpha} \to U_{\alpha}\times \F^k$ for some open cover $\{U_\alpha\}$, we have a family of transition functions
	\[
		\lkxfunc{g_{\alpha\beta}}{U_\alpha\cap U_\beta}{\GL_k \F}{x}{\varphi_\alpha\circ \varphi_{\beta}^{-1}|_{\{x\}\times \R^k}}
	\]
	For some subgroup $G\subset \GL_k\F$, if we can choose local trivializations such that the resulting transition functions $g_{\alpha\beta}$ take values entirely in $G$, we say that the structure group of $E$ can be reduced to $G$.
\end{remark}

\begin{remark}
	\todo{Ehrlagen program}
\end{remark}

\subsection{Clutching Construction}\label{sec:clutching-construction}

When the base manifold of a vector bundle is a sphere, there is a useful construction known as the clutching construction which allows for a completely homotopy theoretic classification of vector bundles.

\begin{remark} The clutching construction is a very special case of a far more general notion of a classifying space, discussed in \cref{sec:classifying-spaces}.
\end{remark}

Suppose $\mathcal{E} : E \to S^m$ is a rank $k$ $\F$-vector bundle. We can decompose the sphere $S^m$ into hemispheral disks $S^m=D_+^m\cup D_-^m$, and these disks intersect at an equatorial sphere $D_+^m\cap D_-^m=S^{m-1}\subset S^m$ one dimension lower. The bundle $\mathcal{E}$ can then be trivialized on the hemispheres since they are contractible. Let us denote these trivializations
\[
	\lkxfunc{\varphi_+}{E|_{D^m_+}}{D^m_+\times \F^k}
	\quad\textrm{and}\quad
	\lkxfunc{\varphi_-}{E|_{D^m_-}}{D_-^m \times \F^k.}\]
Expanding their equatorial intersection by a tubular neighborhood, we get a transition function $\psi$, which fits into the following commutative diagram
$\psi$ in the commutative diagram
\[\begin{tikzcd}
		{S^{m-1}\times \F^k} && {S^{m-1}\times \F^k} \\
		& {E|_{S^{m-1}}}
		\arrow["\psi", from=1-1, to=1-3]
		\arrow["{\varphi_+|_{S^{m-1}}}", from=2-2, to=1-1]
		\arrow["{\varphi_-|_{S^{m-1}}}"', from=2-2, to=1-3]
	\end{tikzcd}\]
By definition, $\psi$ is constant on the first factor, and linear in the second factor. For each point $p\in S^{m-1}$, the diffeomorphism $\psi$ gives a linear function $\tau_p : \F^k \to \F^k$. These linear maps are the ``change of coordinate'' transformations between the fibers on the boundaries of $D_+^m$ and $D_-^m$ as depicted in \cref{fig:clutching-construction}.
\begin{figure}[ht]
	\centering
	\import{diagrams}{clutching-construction.pdf_tex}
	\caption{Getting a map $\tau : S^{m-1}\to \GL_k\F$ from a vector bundle over $S^{m}$.}\label{fig:clutching-construction}
\end{figure}

Altogether, this family of linear transformations is indexed by the equator $S^{m-1}$, and this gives us a smooth map $\tau : S^{m-1}\to \GL_k \F$. It follows that the homotopy type of $\tau$ is only dependent on the isomorphism type of the bundle $\mathcal{E}$ since any vector bundle isomorphism induces a homotopy by \cref{prop:homotopy-invariance-vector-bundle}. In other words, we have a map
\[
	\lkxfunc{}{\Vect^k_\F(S^m)}{{}[S^{m-1}, \GL_k\F]}
\]
sending an $\F$-vector bundle $\mathcal{E}$ to the its associated homotopy class $\tau\in [S^{m-1}, \GL_k\F]$. This associated map $\tau$ is called the \defn{clutching function} of the bundle $\mathcal{E}$.

\begin{remark}
	If the bundle $\mathcal{E}$ has additional structure, we can choose the clutching function to lie in some Lie subgroup $G\subset \GL_k \F$. For instance, in the real case $\F=\R$, an oriented bundle has a clutching function $\tau \in [S^{m-1},\GL_k^+\R]$ in the homotopy group of orientation preserving linear transformations. Similarly, if a bundle has an inner product structure, its clutching function can be chosen to lie in $\pi_{m-1}(\O_k)$. This is elaborated upon in \cref{sec:structure-groups}.
\end{remark}

The construction works in the opposite direction as well.

\begin{definition}
	Given a matrix Lie group $G\subset \GL_k\F$ and a clutching function $\tau\in \pi_{m-1}(G)$, the bundle associated to the function is the bundle $\mathcal{E}_\tau : E_\tau \to S^{m}$, with total space
	\[
		E_\tau = (D_+^m\times \F^k)\cup_h (D_-^{m}\times \F^k),
	\]
	where $h(x,y)=(x,\tau(x)y)$ is the glueing map.
\end{definition}

Rather than work with the homotopy groups of spaces like $\GL_k\R$, it is common to deformation retract them to compact subspaces.

\begin{proposition}
	There are deformation retracts
	\[
		\GL_k\R \cong \O_k,\quad \GL_k^+\R \cong \SO_k\quad\textrm{and}\quad \GL_k\C \cong \U_k.
	\]
\end{proposition}
\begin{proof}
	The retract is given by Gram-Schmidt orthonormalization.
\end{proof}

When the Lie group $G$ is path-connected, as in the case of $\SO_k$ and $\U_k$, we can identify the set of free homotopy classes of maps $[S^{m-1}, G]$ with the homotopy group $\pi_{m-1}(G)$.

\begin{theorem}
	The clutching construction gives bijections
	\[
		\Vect_{\R,+}^k(S^m) \cong \pi_{m-1}(\SO_k)\quad\textrm{and}\quad
		\Vect_{\C}^k(S^m) \cong \pi_{m-1}(\U_k).
	\]
\end{theorem}
\begin{proof}
	This is Proposition 1.11 and Proposition 1.14 in \cite{hatcher2003ktheory}.
\end{proof}

\begin{remark}
	The case of unoriented bundles, we do not have a bijection with the homotopy group, since $\O_m$ is generally disconnected.
\end{remark}

Let us see some examples of this in low dimensions.

\begin{example}
	Since $\SO_1 = \{e\}$, there are no non-trivial orientable real line bundles over spheres of any dimensions.
\end{example}

\begin{example}
	Since $\U_1\cong S^1$, we have
	\[
		\Vect_{\C}^1(S^m) \cong \pi_{m-1}(S^1) \cong \begin{cases} \Z & m=2,\\ 0 & \textrm{otherwise},\end{cases}
		\quad\textrm{for }m >1.
	\]
	In other words, the only sphere for which there are non-trivial complex line bundles is the $2$-dimensional sphere. The association of a complex line bundle over $S^2$ with an integer is a special case of a characteristic class. This will be explored in greater depth in \cref{sec:axiomatic-characteristic-classes}.

	It is natural to wonder what number corresponds to, say the tangent bundle, of a sphere. Recall that the $2$-dimensional sphere admits a complex structure as the complex projective plane $\CP^1$ or Riemann sphere $S^2 = \C\cup\{\infty\}$. We can cover $\CP^1$ by the two coordinate charts
	\[
		U_0 = \{[z: 1] \mid z\in \C\}
		\quad\textrm{and}\quad
		U_\infty = \{[1: w] \mid w\in \C\}.
	\]
	The coordinate change function on the overlap is then $w=1/z$, and so the transition function on the tangent bundle is the differential $dw=-1/z^2 \,dz$. When restricted to the equator $z=e^{i\theta}$, it is clear that the transition function has degree $-2\in \Z$ since $e^{i\theta}\mapsto -e^{-2i\theta}$. Note that $-2$ is, up to a sign, the Euler number of the sphere.
\end{example}


\subsection{Classifying Spaces}\label{sec:classifying-spaces}

\begin{proposition}\label{prop:homotopy-invariance-vector-bundle}
	There is a natural
\end{proposition}

\begin{theorem}\label{thm:classifying-space}
\end{theorem}
Generally, if $G$ is a Lie group there is a natural isomorphism of contravariant functors
\[
	[-, \BB G] \lkxto \Bun_G(-)
\]
where $[-, \BB G]$ is the set of homotopy classes of maps to a space $\BB G$ and $\Bun_G(-)$ is the set of isomorphism classes of principal $G$-bundles over a given space.
The space $\BB G$ is known as the \defn{classifying space}\footnote{The classifying space is rarely a manifold, and is usually an infinite dimensional CW complex.} of $G$ and this space comes equipped with a \defn{universal bundle} $\zeta : \EE G \to \BB G$. With this universal bundle, the natural isomorphism is easy to describe. Under the appropriate topological restrictions, a map $\tau : X \to \BB G$ gives us a pullback bundle $\tau^*\zeta$ over $X$. This is a principal $G$-bundle over $X$ which is entirely determined by the homotopy type of the \defn{classifying map} $\tau$.

In some sense, the universal bundle $\zeta$ is the ``most twisted $G$-bundle''. Pullbacks of bundles generally ``dilute'' the twistedness of a bundle -- for instance, the splitting principle allows any complex vector bundle to be pulled back to a direct sum of complex line bundles. It would stand to reason that every bundle is the pullback of a more twisted bundle, and the limit of this process is the universal bundle $\zeta$ over the classifying space. See Chapter IV of \cite{botttu1982differential} for a wonderful exposition on the topic.


\begin{remark}\label{rmk:clutching-construction-generalization}
	The clutching construction can be viewed as a special case of \cref{thm:classifying-space}.
	\todo{It can be shown that (under suitable topological restrictions)} there is a homotopy equivalence $\Omega \BB G \simeq G$ where $\Omega$ denotes the loop space operator in homotopy theory. Indeed from a homotopy theory perspective, the classifying space is a ``delooping'' of the group $G$. For spheres, the loop space suspension adjunction gives us natural isomorphisms
	\[
		\begin{aligned}
			\Bun_{G}(S^m) & \cong [S^m, \BB G]            \\
			              & \cong [\Sigma S^{m-1}, \BB G] \\
			              & \cong [S^{m-1}, \Omega\BB G]  \\
			              & \cong [S^{m-1}, G]
			\cong \pi_{m-1}(G).
		\end{aligned}
	\]
	This is a generalized way of understanding the clutching construction.
\end{remark}

\todo{link}

For such a characteristic class, adding on trivial bundles does not affect the class. This equivalence relation on vector bundles is known as stable isomorphism.

\begin{definition}\label{def:stable-isomorphism}
	Two vector bundles $\mathcal{E}_1$ and $\mathcal{E}_2$ are said to be \defn{stably isomorphic}[stable isomorphism of vector bundles] if there is an isomorphism of vector bundles $\mathcal{E}_1\oplus \underline{\R}^{k_1} \cong \mathcal{E}_2\oplus \underline{\R}^{k_2}$.
\end{definition}
