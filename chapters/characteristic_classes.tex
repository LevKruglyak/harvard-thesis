% \section{The Chern-Weil Homomorphism}

\begin{definition}\label{defn:poincare-pair}
	Let $X$ be an $n$-dimensional manifold and $Y$ a closed submanifold. The pair $(X,Y)$ is said to be a \defn{Poincar\'e pair} if there is a fundamental class $[X, Y]\in \H_n(X,Y)$ such that the map
	\[
		\lkxfunc{}{\H^q(X,Y)}{\H_{n-q}(X)}{\omega}{\omega \frown [X, Y]}
	\]
	given by cap product with the fundamental class is an isomorphism.
\end{definition}

\begin{remark}\label{rmk:duality-integration}
In the case of de Rham cohomology, since $X$ is connected we have $\H_0(X)\cong \R$. Under this identification, the Poincar\'e duality isomorphism for top-dimensional relative cohomology classes $\omega\in \H^n(X,Y)$ can be interpreted as integration, i.e. $\omega\frown [X,Y]$ corresponds to $\int_X \omega$. 
\end{remark}

With this notion of a Poincar\'e pair, the classical statement of Poincar\'e duality is given:

\begin{theorem}[Poincar\'e Duality]\label{thm:poincare-duality}
	If $X$ is a closed manifold, then $(X,\emptyset)$ is a Poincar\'e pair.
\end{theorem}

\begin{theorem}[Poincar\'e-Lefschetz Duality]\label{thm:poincare-lefschetz-duality}
	If $X$ is a compact manifold with boundary, then $(X,\partial X)$ is a Poincar\'e pair.
\end{theorem}

\[
	\delta I = \int_X \omega + d\mathcal{A} - \int_X \omega =\int_X d\mathcal{A} =\int_{\partial X} \mathcal{A}
\]

\begin{definition*}
	A \defn{characteristic class} is a natural transformation 
	\[
		\lkxfunc{c}{\Bun_{G}}{h^\bullet}
	\]
	from the set of principal $G$-bundles
\end{definition*}

% One of the fundamental topological invariants of a $2m$-manifold is its intersection form, a bilinear form which ``counts'' the number of intersections of submanifolds. We'll see this geometric interpretation in \cref{chap:construction_a}, but for now we'll stick to a more algebraic definition.

\begin{definition}\label{defn:intersection-form}
	Let $(X,Y)$ be a Poincar\'e pair where $X$ is a $2m$-manifold. The \defn{intersection form} of $(X,Y)$ is the bilinear form on $\H^{m}(X,Y)$ given by
	\[
		\lkxfunc{}{\H^{m}(X,Y)^{\times 2}}{\R}{\alpha, \beta}{(\alpha\smile\beta)[X,Y]}
	\]
	where $[X,Y]\in \H_{2m}(X,Y)$ is an orientation class.
\end{definition}
By the graded-commutativity of the cap product, when $m$ is odd this form is skew-symmetric and when $m$ is even this form is symmetric. For now, let's assume $m$ is even so that the form is symmetric bilinear. Following our conventions, we'll now write $2m=4k$. 


\section{Euler Class}\label{sec:euler_class}

\section{Chern and Pontryagin Classes}

\begin{proposition}\label{prop:pontryagin_classes_of_CPn}
  The Pontryagin classes of complex projective space $\CP^n$ are
  \[
    p_k(\CP^n) = \binom{n+1}{k}\quad\textrm{for}\quad 1\leq k \leq n/2.
  \]
\end{proposition}

\section{Stiefel-Whitney Classes}\label{sec:stiefel-whitney_classes}

\section{Universal Bundles}\label{sec:universal_bundles}

\todo{"the most twisted bundle" from Bott and Tu}
\cite{milnorstasheff1974characteristic}
\cite{botttu1982differential}

\begin{theorem}\label{thm:cohomology_of_BO}
  There is a ring isomorphism
  \[
    \H^\bullet(\BO_n; \Z/2) \cong \Z/2[w_1,\ldots, w_n]
  \]
  where $w_i$ are Stiefel-Whitney classes of the universal bundle over $\BO_n$. In other words, any characteristic class for unoriented real bundles with $\Z/2$-coefficients can be expressed in terms of the Stiefel-Whitney classes.
\end{theorem}

\begin{theorem}\label{thm:cohomology_of_BSO}
  Let $\Lambda$ be an integral domain containing $1/2$. There are ring isomorphisms
  \[
      \H^\bullet(\BSO_{2m+1}; \Lambda) \cong \Lambda[p_1, \ldots, p_m]
      \quad\textrm{and}\quad
      \H^\bullet(\BSO_{2m}; \Lambda) \cong \Lambda[p_1, \ldots, p_m, e]/(e^2-p_m)
  \]
  where $p_i$ and $e$ are Pontryagin and Euler classes of the universal bundle over $\BSO_n$.
  In other words, ignoring $2$-torsion, any characterstic class for oriented real bundles can be expressed in terms of Pontryagin and Euler classes.
\end{theorem}

\section{Intersection Form}
