\begin{flushleft}
	\textsl{It is a fact of sociology that topologists}\\
	\textsl{are interested in quadratic forms.}\\
	\rule[0pt]{17em}{0.5pt}\\
	\textsl{-- Serge Lang}
	\vspace{2em}
\end{flushleft}

The goal of this chapter is to construct

\begin{theorem}\label{thm:signature_8_existence_theorem}
	For any $t\in \Z$, there is a framed $4k$-manifold $M$ bounding a homotopy sphere and with signature $\sigma(M)=8t$.
\end{theorem}

Given a commutative coefficient ring $\Lambda$ and a $2m$-dimensional Poincar\'e pair $(X,Y)$, its intersection form is the bilinear form on middle dimensional cohomology
given by
\[ 
	\lkxfunc{Q_{X,Y}}{\H^m(X,Y; \Lambda)^{\times 2}_{\textrm{free}}}{\Lambda}{\omega, \eta}{(\omega\smile \eta)\frown [X,Y]}
\]
where $[X,Y]\in \H_m(X,Y; \Lambda)$ is a fundamental class and $\H^m(X,Y; \Lambda)_{\textrm{free}}$ denotes the free component of the $\Lambda$-module $\H^m(X,Y;\Lambda)$. 

\section{Intersection Theory}

\pagebreak

While the algebraic definition of the intersection form is useful for computation, there is a geometric way to interpret the data of the intersection form as actually counting the number of intersections between $m$-dimensional submanifolds in general position.

\subsection*{Compact Surfaces}

The simplest examples of the geometric data present in the intersection form occurs in the case of compact surfaces. Compact surfaces admit a succinct classification theorem:
\begin{theorem}[Classification of Compact Surfaces]
	Every closed $2$-manifold is homeomorphic to exactly one of the following 
	\[
			S^2,\quad \hash^p T^2, \quad \hash^q \RP^2,
	\]
	where $T^2=S^1\times S^1$ is the torus.
\end{theorem}

\begin{convention*}
The connected sum of a space $X$ with itself $n$ times will be denoted by $\#^n X$.
\end{convention*}

For the basic surfaces $S^2, T^2,$ and $\RP^2$, we have the cohomology rings
\[
		\begin{aligned}
			\H^\bullet(S^2; \Z_2) &= \Z_2[x]/(x^2)\quad &|x|=2,\\
			\H^\bullet(T^2; \Z_2) &= \Z_2[x,y]/(xy, x^2,y^2)\quad &|x|=|y|=1,\\
			\H^\bullet(\RP^2; \Z_2) &= \Z_2[x]/(x^3)\quad &|x|=1.\\
		\end{aligned}
\]
Correspondingly, the $\Z_2$-intersection forms can be written in matrix form as
\[
	Q_{S^2} = \begin{bmatrix}0\end{bmatrix}, \quad Q_{T^2} = \begin{bmatrix}0&1\\ 1&0\end{bmatrix}, \quad Q_{\RP^2} = \begin{bmatrix}1\end{bmatrix}.
\]
This represents 

\begin{theorem}[Classification of Compact Surfaces]
	The intersection form is an isomorphism between the monoid of homeomorphism classes of compact surfaces under connected sum and the monoid of unimodular bilinear forms over $\Z_2$.
	\[
		\mathrm{Surf} \lkxto[Q] \mathrm{Bil}(\Z_2)
	\]
	Both monoids have presentation $\langle A,B \mid A+B = 3A\rangle$.
\end{theorem}

\subsection*{Intersections}

\begin{theorem}
	Suppose $M$ and $N$ are compact oriented submanifolds of dimensions $p$ and $q$ oof a compact oriented $n$-manifold $X$ with boundary such that $p+q=n$. Suppose they intersect transversally. Letting $\omega,\eta$ be the Poincar\'e duals to the fundamental classes $[M]\in \H_p(X)$ and $[N]\in \H_q(X)$, we have
	\[
		 Q_X(\omega, \eta) = \sum_{x\in M\cap N} \sgn(x).
	\]
\end{theorem}

\section{Integral Bilinear Forms}

\subsection*{Bilinear Forms and Matrices}

Let $M_n(R)$ be the ring of $n\times n$ matrices taking values in $R$. A matrix $Q\in M_n(R)$ is invertible if it's determinant $\det Q\in R$ is a unit, and we denote the group of invertible $n\times n$ matrices by $\GL_n(R)$. Two matrices $Q$ and $P$ are said to be conjugate if $P = \Lambda^\intercal Q\Lambda$ for some invertible matrix $\Lambda\in \GL_n(R)$. If we have a free $R$-module $M$ of rank $n$ with some basis $\{e_1,\ldots, e_n\}$, any matrix $Q\in M_n(R)$ defines a bilinear form on $M$ by
\[
	Q(v,w) = v^\intercal Q w\quad \textrm{for all }v,w\in M.
\]
We abuse notation by referring to the matrix and the form interchangeably.
Note that the bilinear form only depends on the matrix conjugacy class, and the form is independent of the basis for $M$.
Similarly, if we have a bilinear form on $M$, we can define the matrix $Q$ in some basis $\{e_1,\ldots, e_n\}\subset R^n$ by setting $Q_{ij}= Q(e_i, e_j)$.

In this way, many notions pertaining to matrices are mapped to notions pertaining to bilinear forms. For instance, symmetric matrices correspond to symmetric bilinear forms and skew-symmetric matrices correspond to skew-symmetric bilinear forms.

\begin{definition}
	A bilinear form $Q$ is said to be \defn{non-degenerate}[non-degenerate bilinear form] if its matrix has non-zero determinant. Equivalently, the form $Q$ is \defn{degenerate}[degenerate bilinear form] if there is some non-zero $v\in M$ such that
	\[Q(v,w)=0 \quad\quad \textrm{for all }w\in M.\]
\end{definition}

When working with rings $R$ which aren't fields, we can get a stronger notion of non-degeneracy if we require the determinant to not only be non-zero but also invertible in $R$.

\begin{definition}
	A bilinear form $Q$ is said to be \defn{unimodular} if its matrix is $R$-invertible. Sometimes, we also refer to an $R$-invertible matrix as unimodular if it is $R$-invertible.
\end{definition}

Equivalently, a bilinear form $Q$ is unimodular if the map
\[
	\lkxfunc{}{M}{M^\d}{v}{\left\{w\mapsto Q(v,w)\right\}}
\]
is an isomorphism. The condition for non-degeneracy only requires that this map be injective.

\begin{example}
	For instance, when $M=\Z$ as a $\Z$-module, the bilinear form $B(x,y)=2xy$ is non-degenerate but not unimodular.
\end{example}

\subsection*{Hasse-Minkowski Classification}

\begin{proposition}
	There is an even positive-definite unimodular form of rank $8$ corresponding to the matrix
	\[
		E_8 = \begin{pmatrix}
			2 & 1 &   &   &   &   &   &   \\
			1 & 2 & 1 &   &   &   &   &   \\
			  & 1 & 2 & 1 &   &   &   &   \\
			  &   & 1 & 2 & 1 &   &   &   \\
			  &   &   & 1 & 2 & 1 & 0 & 1 \\
			  &   &   &   & 1 & 2 & 1 & 0 \\
			  &   &   &   & 0 & 1 & 2 & 0 \\
			  &   &   &   & 1 & 0 & 0 & 2 \\
		\end{pmatrix}
	\]
\end{proposition}

\subsection*{Dodecahedral Space}

