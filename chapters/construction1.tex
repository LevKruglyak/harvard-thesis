\chapter{Detecting Exotic Spheres}

\begin{flushleft}
	\textsl{There is geometry in the humming of the strings,}\\
	\textsl{and there is music in the spacing of the spheres.}\\
	\rule[0pt]{21em}{0.5pt}\\
	-- \textsc{Pythagoras}\\
	\vspace{2em}
\end{flushleft}

Before we begin attempts to construct an exotic sphere, we should answer the following question, without which we will get nowhere:
\begin{center}
  \textsl{How do we know if a given homotopy-sphere is exotic?}
\end{center}
Throughout this chapter, $\H^k(-)$ will mean de Rham cohomology.

This is a tricky problem -- evidenced in part by the fact that it's still open in dimension 4. We need an invariant associated to a differentiable manifold $\Sigma$ which is sensitive enough to discern smooth structure even for manifolds with the same homeomorphism type. Of course, we don't have to work in full generality, but only care about the case when $\Sigma$ is a homotopy-sphere. These requirements immediately exclude invariants such as the Euler characteristic, homotopy groups, the cohomology ring, and other objects depending solely on homeomorphism type. Since we're working with smooth structure, the natural thing to consider would be characteristic forms of the tangent bundle $\T \Sigma \to \Sigma$ such as the Stiefel-Whitney classes, the Pontryagin classes, or the Euler class.\footnote{Recall that the universal Euler and Pontryagin classes generate $\H^\bullet(\BSO_n)$ and Stiefel-Whitney classes generate $\H^\bullet(\BO_n; \Z_2)$. Thus, in a universal sense, these are all of the characteristic forms to consider.} For a brief introduction, see \cref{sec:characteristic classes}.

In order to define an invariant which can compare different manifolds, characteristic classes alone will not suffice since they live in cohomology groups which are not canonically isomorphic. A natural common setting in which invariants can take values and be compared is the coefficient ring of the cohomology theory, in this case the real numbers $\R$.
We can use the Poincar\'e isomorphism $\H^n(\Sigma) \to \R$ for closed oriented manifolds
\[
    \lkxfunc{}{\H^n(\Sigma)}{\R}{\omega}{\int_\Sigma \omega,}
\]
defined by integrating over the fundamental class $[\Sigma]\in \H_n(\Sigma)$. This integral of a characteristic form is known as a characteristic number, and many common manifold invariants can be viewed as such integrals of characteristic forms.

\begin{note*}
  If $c(\Sigma)\in \H^n(\Sigma)$ is a characteristic form, we'll use the bracket notation $c[\Sigma]=\int_\Sigma c(\Sigma)$ to denote the corresponding characteristic number.
\end{note*}

When $n$ is even, one invariant obtained in this way is the Euler characteristic, which can be shown to be equal to the integral of the Euler class $e(\Sigma)\in \H^n(\Sigma)$ over the manifold
\[
  \chi(\Sigma) = \sum_k (-1)^{\beta_k(\Sigma)} \= e[\Sigma] = \int_\Sigma e(\Sigma)\quad\textrm{where}\quad \beta_k = \dim \H_k(\Sigma),
\]
this is the Chern-Gauss-Bonnet theorem. For the standard $n$-spheres, the Euler characteristic is non-trivial in even dimensions -- generally $\chi(S^n)=1+(-1)^n$. However, the Chern-Gauss-Bonnet theorem also implies that the Euler characteristic is a purely \emph{topological} invariant, and thus cannot tell apart smooth structures.

Let's try the Pontryagin classes next. 

\begin{definition}\label{defn:pontryagin_number}
  Given a polynomial $K\in \Q[x_1,\ldots, x_k]$ with $K(x^4, x^8,\ldots, x^{4k})$ homogenous of degree $n$, there is a characteristic form $K(p_1(\Sigma), \ldots, p_k(\Sigma)) \in \H^n(\Sigma)$. The integral of this form is known as a
 \defn{Pontryagin number}:
\[
  K[p_1(\Sigma), \ldots, p_k(\Sigma)] = \int_{\Sigma} K(p_1(\Sigma), \ldots, p_k(\Sigma)).
\]
Here, we use again use a bracket notation to disambiguate the characteristic form from the characteristic number.
\end{definition}

\begin{remark}
  This definition gives a rational number because the Pontryagin classes are normalized to have integer integrals. Without additional restrictions on the polynomial $K$ or the manifold $\Sigma$, a general Pontryagin number defined in this way is only a rational number, not necessarily an integer.
\end{remark}

Just as the Euler characteristic was expressible as the characteristic number of the Euler class, there is a similar theorem expressing a classical topological invariant as a Pontryagin number. If $\Sigma$ is a $4k$-manifold, one of its fundamental invariants is the intersection form. The signature of a $4k$-manifold is the signature of this intersection form, i.e. the difference between the number of its positive and negative eigenvalues. If we let $L(x_1,\ldots, x_k)$ be the genus associated to the formal power series $\sqrt{z}/\tanh(\sqrt{z})$, the Hirzebruch signature theorem states that
\[
    \sigma(\Sigma) = \int_\Sigma L(p_1(\Sigma),\ldots, p_k(\Sigma)).
\]
For instance, in the case of $4$ and $8$ manifolds, the signature can be written as
\[
  \sigma(\Sigma^4) = \frac{p_1[\Sigma]}{3}
  \quad\textrm{and}\quad
  \sigma(\Sigma^8) = \frac{7p_2[\Sigma] - p_1^2[\Sigma]}{45}.
\]
In the case of $4k$-dimensional homotopy-spheres, the signature is zero since spheres have no middle dimensional cohomology. This is emblematic of a deeper issue.

Leveraging the Pontryagin classes to directly build a useful invariant for homotopy-spheres faces a major obstruction: while the tangent bundle of a homotopy-sphere can be far from trivial, it is \emph{stably} trivial. In other words, the sum $\T \Sigma\oplus \underline{\R}^k$ of the tangent bundle with enough trivial line bundles $\underline{\R}$ is trivial. This isn't obvious, and takes some homotopy theory to prove -- we will prove this in \cref{thm:homotopy_spheres_are_s-parallelizable} as part of the full classification of homotopy-spheres. However, the Whitney product formula implies that Pontryagin classes are only sensitive to the stable isomorphism class of a bundle. Thus, the Pontryagin classes and hence Pontryagin numbers are zero for homotopy-spheres. The same argument applies to the Stiefel-Whitney classes, so as it turns out the \emph{only} non-zero characteristic number is the Euler characteristic -- which is a purely topological invariant.

The triviality of these characteristic forms is not a cause for mourning, but rather a blessing in disguise. Recall that if all Stiefel-Whitney classes of a manifold vanish, then the manifold is null-cobordant -- it is the boundary of an compact oriented manifold $B$ one dimension higher.\footnote{In specific dimensions, there are much easier ways to prove this. For instance, in Milnor's first paper on the topic \cite{milnor1956manifolds}, the null-cobordism of homotopy-spheres is implied by the triviality of the cobordism group $\Omega^\SO_7$ in dimension $7$.} We'll refer to such a manifold $B$ as a \defn{coboundary} of the original manifold $\Sigma$. Although this extra data of a coboundary is far from unique, perhaps we could define some invariant of $B$ which only depends on its boundary $\partial B=\Sigma$. This approach will turn out to be incredibly fruitful in detecting exotic spheres, and contains some beautiful geometry.

\begin{remark} 
  This is an example of constructing a \defn{secondary invariant}. When primary invariants, in this case characteristic forms, turn out to be zero, we lift to a case where they are not zero, and use the descent data to measure ``how'' the original invariants vanished. \todo{cite}
\end{remark}

\subsection*{Smooth Invariants of Manifolds with Boundary}

To begin this journey, let's consider smooth invariants of manifolds with boundary.
Throughout let's assume $\Sigma$ is a closed $n$-manifold with oriented coboundary $B$. 
The orientation of $B$ compatible with the orientation on $\Sigma$ defines a fundamental class $[B, \Sigma]\in \H^{n+1}(B, \Sigma)$, which gives the relative Poincar\'e isomorphism, also known as the Poincar\'e-Lefschetz isomorphism
\[
  \lkxfunc{}{\H^{n+1}(B, \Sigma)}{\R}{\omega}{\int_{B} \omega.}
\]
However, characteristic forms are not a priori relative cohomology classes, so pulling them back to obtain \emph{relative} characteristic forms in order to integrate requires additional assumptions about the topology of the boundary $\Sigma$.
For any integer $\ell$, the pair $(B, \partial B) = (B, \Sigma)$ gives us a long exact sequence of cohomology groups
\begin{equation}\label{eq:relative_characteristic_classes_exact_sequence}
  \H^{\ell-1}(\Sigma) \lkxto \H^{\ell}(B, \Sigma) \lkxto[j] \H^{\ell}(B) \lkxto \H^{\ell}(\Sigma)
\end{equation}
where $j : \H^{\ell}(B, \Sigma) \to \H^{\ell}(B)$ is the induced map of the inclusion $(B,\emptyset) \to (B, \Sigma)$. This is an isomorphism if the groups on either side of \cref{eq:relative_characteristic_classes_exact_sequence} are trivial. In this case, we can pullback: 

\begin{definition}\label{defn:relative_characteristic_form}
  Suppose that the cohomology groups $\H^{\ell}(\Sigma)$ and $\H^{\ell-1}(\Sigma)$ are trivial. For a characteristic form $c_\ell(B) \in \H^{\ell}(B)$, the \defn{relative characteristic form} is the pullback
  \[
    c_\ell(B, \Sigma) = j^{-1} c_\ell(B) \quad\in \H^{\ell}(B, \Sigma).
  \]
\end{definition}

\begin{definition}\label{defn:relative_characteristic_number}
  Given a polynomial $K\in \Q[x_1,\ldots, x_k]$ satisfying the conditions of \cref{defn:pontryagin_number}, suppose that $\H^{4i}(\Sigma) = \H^{4i-1}(\Sigma)=0$ for all $i$ for which $K$ has a $x_i$ term.
  In this case, we define the \defn{relative Pontryagin number} to be the integral 
  \[
    \begin{aligned}
      K[p_1(B, \Sigma), \ldots, p_k(B,\Sigma)] 
      &= \int_B K(p_1(B, \Sigma), \ldots, p_k(B, \Sigma))\\
      &= \int_B K(j^{-1}p_1(B), \ldots, j^{-1}p_k(B)).
    \end{aligned}
  \]
\end{definition}

\begin{remark}
  Note that if $\Sigma=\emptyset$, relative characteristic forms and numbers correspond exactly to the non-relative versions since $j$ becomes the identity map.
\end{remark}

How do relative Pontryagin numbers change if we replace $B$ with another coboundary $B'$?
\todo{finish this}

\subsection*{Milnor's Invariant}

Let's see what types of invariants can be constructed out of these relative Pontryagin classes. Suppose $\Sigma$ is a $7$-dimensional homotopy-sphere with $8$-dimensional coboundary. In this case $\H^3(\Sigma)=\H^4(\Sigma)=0$ so we get the relative Pontryagin number $p_1^2(B, \Sigma)\in \Z$. However, since $\H^7(\Sigma)\neq 0$, we don't have a relative Pontryagin number $p_2(B, \Sigma)$. The two invariants of interest defined on $B$ so far are
\[
    p_1^2[B,\Sigma]
    \quad\textrm{and}\quad
    \sigma(B).
\]
Here, note that since the signature is an algebraic invariant of the cohomology ring, and so it automatically generalizes to manifolds with boundary.
Now for a closed $8$-manifold $X$, rearranging using the Hirzebruch signature theorem gives us the expression:
\[
  \sigma(X) = \frac{7p_2[X] - p_1^2[X]}{45}
  \quad\implies\quad
  p_2[X] = \frac{45\sigma(X) + p_1^2[X]}{7}.
\]
This suggests that there \emph{is} some analogue of the second Pontryagin class for $B$. For manifolds with boundary, we could define some number 
\[
  \widetilde{p_2}[B, \Sigma] = \frac{45\sigma(B) + p_1^2[B, \Sigma]}{7}
\]
which reduces to the second Pontryagin number $p_2[B]\in \Z$ when $\Sigma=\emptyset$. When $\Sigma$ is non-empty however, $\widetilde{p_2}[B,\Sigma]$ not necessarily an integer but rather a rational number.

How does the quantity change if $B$ is replaced by some other coboundary $B'$?

\todo{finish this exposition}

\subsection*{A Geometric Perspective on Smooth Invariants}

The preceding notions are subtle, so it's helpful to work in an explicit geometric context loosely following the exposition of \cite{witten1985global}. As before, let's begin with the case of a closed manifold and later generalize to a manifold with boundary.

Recall that if an $n$-manifold $X$ is equipped with a Riemannian metric and a linear connection $\Theta$, we get a curvature form $F_\Theta\in \H^2(\B X; \gl_n)$. Letting $\Inv(\gl_n)$ be the space of $\GL_n$-equivariant polynomials on $\gl_n$, we get the Chern-Weil homomorphism
\[
  \lkxfunc{}{\Inv(\gl_n)}{\H^\bullet(X)}
\]
which map sends a polynomial $h\in \Inv(\gl_n)$ to its evaluation $h(F_\Theta)\in \H^\bullet(\B X)$ on the curvature form, and finally descends this evaluation to the Chern-Weil form $\omega_\Theta(h) \in \H^\bullet(X)$ on the base. Any characteristic form can be written in terms of invariant polynomials via the Chern-Weil homomorphism. As with any characteristic form, we can integrate to form characteristic numbers.

\begin{note*}
A useful basis for the space of $\GL_n$-invariant symmetric polynomials are the even degree trace polynomials $f(x)=\Tr(x^{2k})$. 
\end{note*}

\begin{example}\label{exam:chern-weil_8-manifold}
  For a closed $8$-manifold $X$, the basic characteristic numbers of interest are
  \[
    I_1 = \int_X \Tr (F_\Theta^2)^2
    \quad\textrm{and}\quad
    I_2 = \int_X \Tr F_\Theta^4,
  \]
  these are real numbers, not necessarily rational. In terms of these characteristic numbers, the basic Pontryagin numbers can be written as
  \[
    p_1^2(X) = \frac{1}{(2\pi)^4} \left[\frac{I_1}{4}\right]
    \quad\textrm{and}\quad
    p_2(X) = \frac{1}{(2\pi)^4}\left[\frac{I_1}{8} - \frac{I_2}{4}\right].
  \]
  These numbers are now integers, although this might not be obvious when written in this form.
\end{example}

From the perspective of the Chern-Weil homomorphism, it's not clear why
characteristic numbers defined this way should be diffeomorphism invariants, at least for closed manifolds. After all, there seems to be a strong dependence on the choice of metric and connection.
Adopting a somewhat ``physicist'' approach, it's helpful to see what happens if we perturb the metric and connection -- the resulting variation in a diffeomorphism invariant should be zero.
This question can be answered by Chern-Simons theory. (see \cref{thm:chern-simons}) Firstly, if we change our connection $\Theta$ to a connection $\Theta'$, the corresponding change in the Chern-Weil form is exact, i.e. 
\[
  \delta \omega = \omega_{\Theta'}(h) - \omega_{\Theta}(h) = d\Lambda
\]
for a Chern-Simons form $\Lambda = \CS_{\Theta',\Theta}(h)$. The same conclusion can be drawn if we perturb the metric itself, though making this rigorous takes a bit more work since the frame bundle itself changes.
\begin{remark}
  \todo{explain how it makes sense to talk about the variation of a chern-weil form coming from different bundles}
\end{remark}
So how does changing 

% \begin{example}
%   For instance, an $8$-manifold $X$ has the Pontryagin classs $p_1$ and $p_2$ of an $8$-manifold $X$ can be written as
% \[
%   p_1(X) = \frac{1}{(2\pi)^2}\Tr F_\Theta^2 \in \H^4(X)
%   \quad\textrm{and}\quad
%   p_2(X) = \frac{1}{(2\pi)^4}\left[ \frac{1}{8}(\Tr F_\Theta^2)^2 - \frac{1}{4}\Tr F_\Theta^4\right] \in \H^8(X).
% \]
% \end{example}


Let's return back to our original problem of using an invariant for manifolds with boundary to construct invariants on the boundary itself.

% Considering the basis of $\GL_n$-invariant polynomials by trace polynomials $\Tr F_\Theta^{2k}$, we can write many 

\bigskip

\begin{remark}
\todo{Connection to Green-Schwartz anomaly cancellation} \cite{witten1985global}.
\end{remark}

% \begin{center}
% %   \textsl{An integrality theorem for characteristic forms of $n$-manifolds gives rise to an invariant for $(n-1)$-manifolds which bound.}
% \end{center}

% Maybe the lack of data can be amended by introducing additional degrees of geometric freedom. If we equip $M$ with a Riemannian metric, we get a tangent bundle
%
% Perhaps some appropriate formula involving characteristic classes could give us an invariant which only depends on the smooth structure of the manifold and not the geometric structure which, remember, is only there to serve as computational scaffolding. 
%
% Yet again, the topological simplicity of the spheres comes back to haunt us -- since characteristic classes live in cohomology, only top dimensional characteristic classes can hope to be non-trivial. 
%
% Discuss Stiefel-Whitney classes.
%
% \todo{what are the characteristic classes of homeomorphism spheres?}
%
% \begin{theorem}[Thom-Pontryagin]
%   The Stiefel-Whitney numbers of a closed $n$-manifold $M$ are all zero if and only if it $M$ is the boundary of a compact $(n+1)$-manifold $B$.
% \end{theorem}
%
% \todo{link}
%
% For a concrete case, let's assume for now that $M$ is a closed, oriented Riemannian $7$-manifold bounded by a $8$-manifold $B$ containing a compatible Riemannian structure.
% \[
%     I = \int_B \Tr F_\Theta^2
% \]
%
% Under a change of metric, the curvature changes 
%
% How can such an invariant be well-defined?
%
% Let's pick some group $G$ and suppose we had an invariant $\lambda(B)\in G$ associated to any compact oriented manifold that bounds $M$, i.e. $\partial B = M$. 
%
% At the very least 
%
% If $B_1$ and $B_2$ are two such manifolds which share a common boundary $M$, we can form a compact manifold \emph{without} boundary by gluing together $B_1$ and $B_2$ along $M$ in reverse orientation. 
% Let's call the resulting space $X=B_1\cup_\iota (-B_2)$, where negation denotes a reversed orientation and $\iota : \partial B_1 \to (-\partial B_2)$ is any orientation preserving diffeomorphism. Equivalently, this can be thought of as an orientation \emph{reversing} diffeomorphism $M\to M$.
%
% Now, it follows that $\lambda(X) = \lambda(B_1)-\lambda(B_2)$. We want this difference to be zero in order to get a well-defined quantity.
%
% \begin{definition}
% 	The \defn{Milnor invariant} of a $7$-manifold 
% 	\[
%     \lambda(M) = \frac{1}{7}\left[2 p_1^2(B, M) - \sigma(B)\right] \mod 1.
% 	\]
% \end{definition}
%
% \begin{definition}
%   A \defn{genus} of a cobordism theory $\Omega_*$ is a ring homomorphism
%   \[
%     \lkxfunc{\sigma}{\Omega_*}{\Z.}
%   \]
% \end{definition}
%

% Go one dimension higher,
% what kind of invariants don't depend on the 
% Relative pontryagin classes

\section{Milnor Manifolds}

What other sorts of manifolds $M$ can be constructed as total spaces of fiber bundles of the form
\[
    S^{n-1} \lkxto M^{2n-1} \lkxto S^n.
\]

\begin{historicalremark*}
  In the 1950s at Princeton, Milnor was \cite{milnor2000exotic}. \todo{finish this.}
\end{historicalremark*}


\begin{definition}
	The \defn{Milnor manifold} $\Sigma_{p,q}$ of type $(p,q)\in \pi_3(\SO_4)$ is the
\end{definition}

\subsection*{The Gromoll-Meyer Sphere}

There is another geometric way to construct an exotic $7$-sphere as a quotient of a Lie group.

\begin{definition}
	The \defn{Gromoll-Meyer sphere} $\Sigma_{\mathrm{gm}}^7$ is the 
	\[
      \Sigma_{\mathrm{gm}}^7 = \Sp_2/\Delta.
	\]
\end{definition}

\begin{theorem}
  The Gromoll-Meyer sphere has nonnegative sectional curvature.
\end{theorem}

\section{Spin Geometry and Dirac Operators}

\begin{theorem}
  Let $X^{4k}$ be a closed oriented spin $4k$-manifold. The $\Ahat$-genus $\Ahat(X)$ is an integer, and if $k$ is odd, an even integer.
\end{theorem}

\section{The Eells-Kupier Invariant}\label{sec:eels-kuper_invariant}

How do we discern the diffeomorphism type of manifolds which have the homeomorphism type of a sphere? The standard selection of invariants is currently vastly insufficient for this purpose -- characteristic numbers for instance are defined as integrals of characteristic classes of the tangent bundle.


Many manifold invariants defined thus far have required non-trivial cohomology. 

Let $M^{4k-1}$ be a closed oriented $(4k-1)$-manifold. If $M$ is the boundary of a compact oriented $4k$-manifold $W^{4k}$ which bounds $M$, i.e. $\partial W = M$, the cohomology sequence of the pair $(W,M)$ gives us a long exact sequence:
\[
  \H^{4i-1}(M;\Q) \lkxto \H^{4i}(W, M; \Q) \lkxto[j] \H^{4i}(W; \Q) \lkxto \H^{4i}(M; \Q)
\]
If we require that $\H^{4i-1}(M;\Q)$ and $\H^{4i}(M;\Q)$ are trivial, the map $j$ would be an isomorphism and so we can pull back the Pontryagin classes $p_i(W)\in \H^{4i}(W;\Q)$ to \defn{relative Pontryagin classes}[relative Pontryagin class] $p_i(W, M)\in\H^{4i}(W, M; \Q)$.

\begin{definition}
	The \defn{Eells-Kupier invariant} of a manifold $M^{4k-1}$ is
	\[
    \lambda(M) = \frac{1}{896}\left[p_1^2(B, M) - 4\sigma(B)\right] \mod 1.
	\]
\end{definition}
