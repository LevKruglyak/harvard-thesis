\chapter*{Prologue}
\addcontentsline{toc}{chapter}{Prelude} 

% Epigraph
% \begin{flushleft}
% 	\textsl{A mathematician is a blind man in }\\
% 	\textsl{a dark room looking for a black cat}\\
% 	\textsl{which isn’t there.}\\
% 	\rule[0pt]{15em}{0.5pt}\\
% 	\textsl{-- Unknown}
% \end{flushleft}

\begin{flushleft}
	\textsl{Existence plays a mischievous game with us,}\\
	\textsl{as though to tease and provoke us. In the }\\
	\textsl{midst of knowledge there yet again arises }\\
	\textsl{the mystery; in the midst of contemplation}\\
	\textsl{the riddle gains new strength.}\\
	\rule[0pt]{19.5em}{0.5pt}\\
	\textsl{-- Joseph Soloveitchik, ``Ish ha'Halakha''}
\end{flushleft}

\vspace{2em}

What is the 
% During my last summer before graduating college, I went with some friends on a mountaineering trip up Banner peak, a picturesque mountain in the eastern Sierra Nevada range of California. 
% %
% % As we descended the glacier towards our base camp, ice axes in hand, I 
% So how \emph{do} we study objects which we can't touch, see, or possibly hope to visualize in their full complexity?
%
% There are two levels to understanding.
% \begin{enumerate}
% 	\item First, we
% \end{enumerate}
%
% Understanding an object by how it behaves with respect to other objects.
% \todo{spectral lines of atoms}
%
% One of the earliest methods to ``fingerprint'' topological objects was discovered by Euler.
